
% Default to the notebook output style

    


% Inherit from the specified cell style.




    
\documentclass[11pt]{article}

    
    
    \usepackage[T1]{fontenc}
    % Nicer default font (+ math font) than Computer Modern for most use cases
    \usepackage{mathpazo}

    % Basic figure setup, for now with no caption control since it's done
    % automatically by Pandoc (which extracts ![](path) syntax from Markdown).
    \usepackage{graphicx}
    % We will generate all images so they have a width \maxwidth. This means
    % that they will get their normal width if they fit onto the page, but
    % are scaled down if they would overflow the margins.
    \makeatletter
    \def\maxwidth{\ifdim\Gin@nat@width>\linewidth\linewidth
    \else\Gin@nat@width\fi}
    \makeatother
    \let\Oldincludegraphics\includegraphics
    % Set max figure width to be 80% of text width, for now hardcoded.
    \renewcommand{\includegraphics}[1]{\Oldincludegraphics[width=.8\maxwidth]{#1}}
    % Ensure that by default, figures have no caption (until we provide a
    % proper Figure object with a Caption API and a way to capture that
    % in the conversion process - todo).
    \usepackage{caption}
    \DeclareCaptionLabelFormat{nolabel}{}
    \captionsetup{labelformat=nolabel}

    \usepackage{adjustbox} % Used to constrain images to a maximum size 
    \usepackage{xcolor} % Allow colors to be defined
    \usepackage{enumerate} % Needed for markdown enumerations to work
    \usepackage{geometry} % Used to adjust the document margins
    \usepackage{amsmath} % Equations
    \usepackage{amssymb} % Equations
    \usepackage{textcomp} % defines textquotesingle
    % Hack from http://tex.stackexchange.com/a/47451/13684:
    \AtBeginDocument{%
        \def\PYZsq{\textquotesingle}% Upright quotes in Pygmentized code
    }
    \usepackage{upquote} % Upright quotes for verbatim code
    \usepackage{eurosym} % defines \euro
    \usepackage[mathletters]{ucs} % Extended unicode (utf-8) support
    \usepackage[utf8x]{inputenc} % Allow utf-8 characters in the tex document
    \usepackage{fancyvrb} % verbatim replacement that allows latex
    \usepackage{grffile} % extends the file name processing of package graphics 
                         % to support a larger range 
    % The hyperref package gives us a pdf with properly built
    % internal navigation ('pdf bookmarks' for the table of contents,
    % internal cross-reference links, web links for URLs, etc.)
    \usepackage{hyperref}
    \usepackage{longtable} % longtable support required by pandoc >1.10
    \usepackage{booktabs}  % table support for pandoc > 1.12.2
    \usepackage[inline]{enumitem} % IRkernel/repr support (it uses the enumerate* environment)
    \usepackage[normalem]{ulem} % ulem is needed to support strikethroughs (\sout)
                                % normalem makes italics be italics, not underlines
    

    
    
    % Colors for the hyperref package
    \definecolor{urlcolor}{rgb}{0,.145,.698}
    \definecolor{linkcolor}{rgb}{.71,0.21,0.01}
    \definecolor{citecolor}{rgb}{.12,.54,.11}

    % ANSI colors
    \definecolor{ansi-black}{HTML}{3E424D}
    \definecolor{ansi-black-intense}{HTML}{282C36}
    \definecolor{ansi-red}{HTML}{E75C58}
    \definecolor{ansi-red-intense}{HTML}{B22B31}
    \definecolor{ansi-green}{HTML}{00A250}
    \definecolor{ansi-green-intense}{HTML}{007427}
    \definecolor{ansi-yellow}{HTML}{DDB62B}
    \definecolor{ansi-yellow-intense}{HTML}{B27D12}
    \definecolor{ansi-blue}{HTML}{208FFB}
    \definecolor{ansi-blue-intense}{HTML}{0065CA}
    \definecolor{ansi-magenta}{HTML}{D160C4}
    \definecolor{ansi-magenta-intense}{HTML}{A03196}
    \definecolor{ansi-cyan}{HTML}{60C6C8}
    \definecolor{ansi-cyan-intense}{HTML}{258F8F}
    \definecolor{ansi-white}{HTML}{C5C1B4}
    \definecolor{ansi-white-intense}{HTML}{A1A6B2}

    % commands and environments needed by pandoc snippets
    % extracted from the output of `pandoc -s`
    \providecommand{\tightlist}{%
      \setlength{\itemsep}{0pt}\setlength{\parskip}{0pt}}
    \DefineVerbatimEnvironment{Highlighting}{Verbatim}{commandchars=\\\{\}}
    % Add ',fontsize=\small' for more characters per line
    \newenvironment{Shaded}{}{}
    \newcommand{\KeywordTok}[1]{\textcolor[rgb]{0.00,0.44,0.13}{\textbf{{#1}}}}
    \newcommand{\DataTypeTok}[1]{\textcolor[rgb]{0.56,0.13,0.00}{{#1}}}
    \newcommand{\DecValTok}[1]{\textcolor[rgb]{0.25,0.63,0.44}{{#1}}}
    \newcommand{\BaseNTok}[1]{\textcolor[rgb]{0.25,0.63,0.44}{{#1}}}
    \newcommand{\FloatTok}[1]{\textcolor[rgb]{0.25,0.63,0.44}{{#1}}}
    \newcommand{\CharTok}[1]{\textcolor[rgb]{0.25,0.44,0.63}{{#1}}}
    \newcommand{\StringTok}[1]{\textcolor[rgb]{0.25,0.44,0.63}{{#1}}}
    \newcommand{\CommentTok}[1]{\textcolor[rgb]{0.38,0.63,0.69}{\textit{{#1}}}}
    \newcommand{\OtherTok}[1]{\textcolor[rgb]{0.00,0.44,0.13}{{#1}}}
    \newcommand{\AlertTok}[1]{\textcolor[rgb]{1.00,0.00,0.00}{\textbf{{#1}}}}
    \newcommand{\FunctionTok}[1]{\textcolor[rgb]{0.02,0.16,0.49}{{#1}}}
    \newcommand{\RegionMarkerTok}[1]{{#1}}
    \newcommand{\ErrorTok}[1]{\textcolor[rgb]{1.00,0.00,0.00}{\textbf{{#1}}}}
    \newcommand{\NormalTok}[1]{{#1}}
    
    % Additional commands for more recent versions of Pandoc
    \newcommand{\ConstantTok}[1]{\textcolor[rgb]{0.53,0.00,0.00}{{#1}}}
    \newcommand{\SpecialCharTok}[1]{\textcolor[rgb]{0.25,0.44,0.63}{{#1}}}
    \newcommand{\VerbatimStringTok}[1]{\textcolor[rgb]{0.25,0.44,0.63}{{#1}}}
    \newcommand{\SpecialStringTok}[1]{\textcolor[rgb]{0.73,0.40,0.53}{{#1}}}
    \newcommand{\ImportTok}[1]{{#1}}
    \newcommand{\DocumentationTok}[1]{\textcolor[rgb]{0.73,0.13,0.13}{\textit{{#1}}}}
    \newcommand{\AnnotationTok}[1]{\textcolor[rgb]{0.38,0.63,0.69}{\textbf{\textit{{#1}}}}}
    \newcommand{\CommentVarTok}[1]{\textcolor[rgb]{0.38,0.63,0.69}{\textbf{\textit{{#1}}}}}
    \newcommand{\VariableTok}[1]{\textcolor[rgb]{0.10,0.09,0.49}{{#1}}}
    \newcommand{\ControlFlowTok}[1]{\textcolor[rgb]{0.00,0.44,0.13}{\textbf{{#1}}}}
    \newcommand{\OperatorTok}[1]{\textcolor[rgb]{0.40,0.40,0.40}{{#1}}}
    \newcommand{\BuiltInTok}[1]{{#1}}
    \newcommand{\ExtensionTok}[1]{{#1}}
    \newcommand{\PreprocessorTok}[1]{\textcolor[rgb]{0.74,0.48,0.00}{{#1}}}
    \newcommand{\AttributeTok}[1]{\textcolor[rgb]{0.49,0.56,0.16}{{#1}}}
    \newcommand{\InformationTok}[1]{\textcolor[rgb]{0.38,0.63,0.69}{\textbf{\textit{{#1}}}}}
    \newcommand{\WarningTok}[1]{\textcolor[rgb]{0.38,0.63,0.69}{\textbf{\textit{{#1}}}}}
    
    
    % Define a nice break command that doesn't care if a line doesn't already
    % exist.
    \def\br{\hspace*{\fill} \\* }
    % Math Jax compatability definitions
    \def\gt{>}
    \def\lt{<}
    % Document parameters
    \title{0.1 CRSP}
    
    
    

    % Pygments definitions
    
\makeatletter
\def\PY@reset{\let\PY@it=\relax \let\PY@bf=\relax%
    \let\PY@ul=\relax \let\PY@tc=\relax%
    \let\PY@bc=\relax \let\PY@ff=\relax}
\def\PY@tok#1{\csname PY@tok@#1\endcsname}
\def\PY@toks#1+{\ifx\relax#1\empty\else%
    \PY@tok{#1}\expandafter\PY@toks\fi}
\def\PY@do#1{\PY@bc{\PY@tc{\PY@ul{%
    \PY@it{\PY@bf{\PY@ff{#1}}}}}}}
\def\PY#1#2{\PY@reset\PY@toks#1+\relax+\PY@do{#2}}

\expandafter\def\csname PY@tok@go\endcsname{\def\PY@tc##1{\textcolor[rgb]{0.53,0.53,0.53}{##1}}}
\expandafter\def\csname PY@tok@s2\endcsname{\def\PY@tc##1{\textcolor[rgb]{0.73,0.13,0.13}{##1}}}
\expandafter\def\csname PY@tok@ne\endcsname{\let\PY@bf=\textbf\def\PY@tc##1{\textcolor[rgb]{0.82,0.25,0.23}{##1}}}
\expandafter\def\csname PY@tok@sh\endcsname{\def\PY@tc##1{\textcolor[rgb]{0.73,0.13,0.13}{##1}}}
\expandafter\def\csname PY@tok@nb\endcsname{\def\PY@tc##1{\textcolor[rgb]{0.00,0.50,0.00}{##1}}}
\expandafter\def\csname PY@tok@mi\endcsname{\def\PY@tc##1{\textcolor[rgb]{0.40,0.40,0.40}{##1}}}
\expandafter\def\csname PY@tok@dl\endcsname{\def\PY@tc##1{\textcolor[rgb]{0.73,0.13,0.13}{##1}}}
\expandafter\def\csname PY@tok@kr\endcsname{\let\PY@bf=\textbf\def\PY@tc##1{\textcolor[rgb]{0.00,0.50,0.00}{##1}}}
\expandafter\def\csname PY@tok@w\endcsname{\def\PY@tc##1{\textcolor[rgb]{0.73,0.73,0.73}{##1}}}
\expandafter\def\csname PY@tok@c\endcsname{\let\PY@it=\textit\def\PY@tc##1{\textcolor[rgb]{0.25,0.50,0.50}{##1}}}
\expandafter\def\csname PY@tok@ss\endcsname{\def\PY@tc##1{\textcolor[rgb]{0.10,0.09,0.49}{##1}}}
\expandafter\def\csname PY@tok@cm\endcsname{\let\PY@it=\textit\def\PY@tc##1{\textcolor[rgb]{0.25,0.50,0.50}{##1}}}
\expandafter\def\csname PY@tok@kp\endcsname{\def\PY@tc##1{\textcolor[rgb]{0.00,0.50,0.00}{##1}}}
\expandafter\def\csname PY@tok@mb\endcsname{\def\PY@tc##1{\textcolor[rgb]{0.40,0.40,0.40}{##1}}}
\expandafter\def\csname PY@tok@cs\endcsname{\let\PY@it=\textit\def\PY@tc##1{\textcolor[rgb]{0.25,0.50,0.50}{##1}}}
\expandafter\def\csname PY@tok@gd\endcsname{\def\PY@tc##1{\textcolor[rgb]{0.63,0.00,0.00}{##1}}}
\expandafter\def\csname PY@tok@kn\endcsname{\let\PY@bf=\textbf\def\PY@tc##1{\textcolor[rgb]{0.00,0.50,0.00}{##1}}}
\expandafter\def\csname PY@tok@nv\endcsname{\def\PY@tc##1{\textcolor[rgb]{0.10,0.09,0.49}{##1}}}
\expandafter\def\csname PY@tok@err\endcsname{\def\PY@bc##1{\setlength{\fboxsep}{0pt}\fcolorbox[rgb]{1.00,0.00,0.00}{1,1,1}{\strut ##1}}}
\expandafter\def\csname PY@tok@vi\endcsname{\def\PY@tc##1{\textcolor[rgb]{0.10,0.09,0.49}{##1}}}
\expandafter\def\csname PY@tok@nc\endcsname{\let\PY@bf=\textbf\def\PY@tc##1{\textcolor[rgb]{0.00,0.00,1.00}{##1}}}
\expandafter\def\csname PY@tok@mh\endcsname{\def\PY@tc##1{\textcolor[rgb]{0.40,0.40,0.40}{##1}}}
\expandafter\def\csname PY@tok@na\endcsname{\def\PY@tc##1{\textcolor[rgb]{0.49,0.56,0.16}{##1}}}
\expandafter\def\csname PY@tok@mf\endcsname{\def\PY@tc##1{\textcolor[rgb]{0.40,0.40,0.40}{##1}}}
\expandafter\def\csname PY@tok@bp\endcsname{\def\PY@tc##1{\textcolor[rgb]{0.00,0.50,0.00}{##1}}}
\expandafter\def\csname PY@tok@ow\endcsname{\let\PY@bf=\textbf\def\PY@tc##1{\textcolor[rgb]{0.67,0.13,1.00}{##1}}}
\expandafter\def\csname PY@tok@c1\endcsname{\let\PY@it=\textit\def\PY@tc##1{\textcolor[rgb]{0.25,0.50,0.50}{##1}}}
\expandafter\def\csname PY@tok@sb\endcsname{\def\PY@tc##1{\textcolor[rgb]{0.73,0.13,0.13}{##1}}}
\expandafter\def\csname PY@tok@nd\endcsname{\def\PY@tc##1{\textcolor[rgb]{0.67,0.13,1.00}{##1}}}
\expandafter\def\csname PY@tok@sx\endcsname{\def\PY@tc##1{\textcolor[rgb]{0.00,0.50,0.00}{##1}}}
\expandafter\def\csname PY@tok@nl\endcsname{\def\PY@tc##1{\textcolor[rgb]{0.63,0.63,0.00}{##1}}}
\expandafter\def\csname PY@tok@ni\endcsname{\let\PY@bf=\textbf\def\PY@tc##1{\textcolor[rgb]{0.60,0.60,0.60}{##1}}}
\expandafter\def\csname PY@tok@sa\endcsname{\def\PY@tc##1{\textcolor[rgb]{0.73,0.13,0.13}{##1}}}
\expandafter\def\csname PY@tok@mo\endcsname{\def\PY@tc##1{\textcolor[rgb]{0.40,0.40,0.40}{##1}}}
\expandafter\def\csname PY@tok@gu\endcsname{\let\PY@bf=\textbf\def\PY@tc##1{\textcolor[rgb]{0.50,0.00,0.50}{##1}}}
\expandafter\def\csname PY@tok@ge\endcsname{\let\PY@it=\textit}
\expandafter\def\csname PY@tok@sd\endcsname{\let\PY@it=\textit\def\PY@tc##1{\textcolor[rgb]{0.73,0.13,0.13}{##1}}}
\expandafter\def\csname PY@tok@s\endcsname{\def\PY@tc##1{\textcolor[rgb]{0.73,0.13,0.13}{##1}}}
\expandafter\def\csname PY@tok@no\endcsname{\def\PY@tc##1{\textcolor[rgb]{0.53,0.00,0.00}{##1}}}
\expandafter\def\csname PY@tok@nt\endcsname{\let\PY@bf=\textbf\def\PY@tc##1{\textcolor[rgb]{0.00,0.50,0.00}{##1}}}
\expandafter\def\csname PY@tok@s1\endcsname{\def\PY@tc##1{\textcolor[rgb]{0.73,0.13,0.13}{##1}}}
\expandafter\def\csname PY@tok@se\endcsname{\let\PY@bf=\textbf\def\PY@tc##1{\textcolor[rgb]{0.73,0.40,0.13}{##1}}}
\expandafter\def\csname PY@tok@gr\endcsname{\def\PY@tc##1{\textcolor[rgb]{1.00,0.00,0.00}{##1}}}
\expandafter\def\csname PY@tok@nn\endcsname{\let\PY@bf=\textbf\def\PY@tc##1{\textcolor[rgb]{0.00,0.00,1.00}{##1}}}
\expandafter\def\csname PY@tok@m\endcsname{\def\PY@tc##1{\textcolor[rgb]{0.40,0.40,0.40}{##1}}}
\expandafter\def\csname PY@tok@sc\endcsname{\def\PY@tc##1{\textcolor[rgb]{0.73,0.13,0.13}{##1}}}
\expandafter\def\csname PY@tok@o\endcsname{\def\PY@tc##1{\textcolor[rgb]{0.40,0.40,0.40}{##1}}}
\expandafter\def\csname PY@tok@vc\endcsname{\def\PY@tc##1{\textcolor[rgb]{0.10,0.09,0.49}{##1}}}
\expandafter\def\csname PY@tok@gp\endcsname{\let\PY@bf=\textbf\def\PY@tc##1{\textcolor[rgb]{0.00,0.00,0.50}{##1}}}
\expandafter\def\csname PY@tok@nf\endcsname{\def\PY@tc##1{\textcolor[rgb]{0.00,0.00,1.00}{##1}}}
\expandafter\def\csname PY@tok@vm\endcsname{\def\PY@tc##1{\textcolor[rgb]{0.10,0.09,0.49}{##1}}}
\expandafter\def\csname PY@tok@ch\endcsname{\let\PY@it=\textit\def\PY@tc##1{\textcolor[rgb]{0.25,0.50,0.50}{##1}}}
\expandafter\def\csname PY@tok@fm\endcsname{\def\PY@tc##1{\textcolor[rgb]{0.00,0.00,1.00}{##1}}}
\expandafter\def\csname PY@tok@sr\endcsname{\def\PY@tc##1{\textcolor[rgb]{0.73,0.40,0.53}{##1}}}
\expandafter\def\csname PY@tok@k\endcsname{\let\PY@bf=\textbf\def\PY@tc##1{\textcolor[rgb]{0.00,0.50,0.00}{##1}}}
\expandafter\def\csname PY@tok@kt\endcsname{\def\PY@tc##1{\textcolor[rgb]{0.69,0.00,0.25}{##1}}}
\expandafter\def\csname PY@tok@kd\endcsname{\let\PY@bf=\textbf\def\PY@tc##1{\textcolor[rgb]{0.00,0.50,0.00}{##1}}}
\expandafter\def\csname PY@tok@kc\endcsname{\let\PY@bf=\textbf\def\PY@tc##1{\textcolor[rgb]{0.00,0.50,0.00}{##1}}}
\expandafter\def\csname PY@tok@gs\endcsname{\let\PY@bf=\textbf}
\expandafter\def\csname PY@tok@si\endcsname{\let\PY@bf=\textbf\def\PY@tc##1{\textcolor[rgb]{0.73,0.40,0.53}{##1}}}
\expandafter\def\csname PY@tok@vg\endcsname{\def\PY@tc##1{\textcolor[rgb]{0.10,0.09,0.49}{##1}}}
\expandafter\def\csname PY@tok@gt\endcsname{\def\PY@tc##1{\textcolor[rgb]{0.00,0.27,0.87}{##1}}}
\expandafter\def\csname PY@tok@gh\endcsname{\let\PY@bf=\textbf\def\PY@tc##1{\textcolor[rgb]{0.00,0.00,0.50}{##1}}}
\expandafter\def\csname PY@tok@il\endcsname{\def\PY@tc##1{\textcolor[rgb]{0.40,0.40,0.40}{##1}}}
\expandafter\def\csname PY@tok@gi\endcsname{\def\PY@tc##1{\textcolor[rgb]{0.00,0.63,0.00}{##1}}}
\expandafter\def\csname PY@tok@cp\endcsname{\def\PY@tc##1{\textcolor[rgb]{0.74,0.48,0.00}{##1}}}
\expandafter\def\csname PY@tok@cpf\endcsname{\let\PY@it=\textit\def\PY@tc##1{\textcolor[rgb]{0.25,0.50,0.50}{##1}}}

\def\PYZbs{\char`\\}
\def\PYZus{\char`\_}
\def\PYZob{\char`\{}
\def\PYZcb{\char`\}}
\def\PYZca{\char`\^}
\def\PYZam{\char`\&}
\def\PYZlt{\char`\<}
\def\PYZgt{\char`\>}
\def\PYZsh{\char`\#}
\def\PYZpc{\char`\%}
\def\PYZdl{\char`\$}
\def\PYZhy{\char`\-}
\def\PYZsq{\char`\'}
\def\PYZdq{\char`\"}
\def\PYZti{\char`\~}
% for compatibility with earlier versions
\def\PYZat{@}
\def\PYZlb{[}
\def\PYZrb{]}
\makeatother


    % Exact colors from NB
    \definecolor{incolor}{rgb}{0.0, 0.0, 0.5}
    \definecolor{outcolor}{rgb}{0.545, 0.0, 0.0}



    
    % Prevent overflowing lines due to hard-to-break entities
    \sloppy 
    % Setup hyperref package
    \hypersetup{
      breaklinks=true,  % so long urls are correctly broken across lines
      colorlinks=true,
      urlcolor=urlcolor,
      linkcolor=linkcolor,
      citecolor=citecolor,
      }
    % Slightly bigger margins than the latex defaults
    
    \geometry{verbose,tmargin=1in,bmargin=1in,lmargin=1in,rmargin=1in}
    
    

    \begin{document}
    
    
    \maketitle
    
    

    
    \section{CRSP}\label{crsp}

\begin{itemize}
\tightlist
\item
  Load CRSP data
\item
  Tickle the data
\item
  Create variables

  \begin{itemize}
  \tightlist
  \item
    ME
  \item
    Prior
  \end{itemize}
\end{itemize}

    \begin{Verbatim}[commandchars=\\\{\}]
{\color{incolor}In [{\color{incolor}47}]:} \PY{k+kn}{library}\PY{p}{(}data.table\PY{p}{)}    \PY{c+c1}{\PYZsh{} read csv much faster than standard function}
         \PY{k+kn}{library}\PY{p}{(}dplyr\PY{p}{)}         \PY{c+c1}{\PYZsh{} infinitely nicer grouping operations}
         \PY{k+kn}{library}\PY{p}{(}ggplot2\PY{p}{)}       \PY{c+c1}{\PYZsh{} sexy plots}
\end{Verbatim}


    \subsection{Load Data}\label{load-data}

    \begin{Verbatim}[commandchars=\\\{\}]
{\color{incolor}In [{\color{incolor}48}]:} crsp.path \PY{o}{=} \PY{l+s}{\PYZsq{}}\PY{l+s}{C:/Data/CRSP/20171123\PYZus{}CRSP\PYZus{}196001\PYZus{}201612.csv\PYZsq{}}
         \PY{c+c1}{\PYZsh{} colClasses=\PYZsq{}character\PYZsq{} because fread is upset that our columns}
         \PY{c+c1}{\PYZsh{}   contain multiple variable types}
         \PY{c+c1}{\PYZsh{} CRSP uses \PYZsq{}C\PYZsq{} and \PYZsq{}Z\PYZsq{} and some other characters to mean stuff}
         \PY{c+c1}{\PYZsh{}   whatever they mean, fread is upset}
         \PY{c+c1}{\PYZsh{} fread rather smugly reports how quickly it has read our file}
         \PY{c+c1}{\PYZsh{}   if we do not set showProgress=FALSE}
         crsp \PY{o}{=} fread\PY{p}{(}crsp.path\PY{p}{,} colClasses\PY{o}{=}\PY{l+s}{\PYZsq{}}\PY{l+s}{character\PYZsq{}}\PY{p}{,} showProgress\PY{o}{=}\PY{k+kc}{FALSE}\PY{p}{)}
\end{Verbatim}


    Let's see what we have.

    \begin{Verbatim}[commandchars=\\\{\}]
{\color{incolor}In [{\color{incolor}49}]:} \PY{k+kp}{colnames}\PY{p}{(}crsp\PY{p}{)}
\end{Verbatim}


    \begin{enumerate*}
\item 'PERMNO'
\item 'date'
\item 'SHRCD'
\item 'EXCHCD'
\item 'SICCD'
\item 'TICKER'
\item 'COMNAM'
\item 'CUSIP'
\item 'DLSTCD'
\item 'DLRET'
\item 'PRC'
\item 'RET'
\item 'SHROUT'
\item 'SPREAD'
\item 'vwretd'
\end{enumerate*}


    
    There should be no duplicate \texttt{PERMNO}-\texttt{date} pairs.

    \begin{Verbatim}[commandchars=\\\{\}]
{\color{incolor}In [{\color{incolor}65}]:} pairs \PY{o}{=} \PY{k+kp}{duplicated}\PY{p}{(}crsp\PY{p}{[}\PY{p}{,}\PY{k+kt}{c}\PY{p}{(}\PY{l+s}{\PYZdq{}}\PY{l+s}{PERMNO\PYZdq{}}\PY{p}{,} \PY{l+s}{\PYZdq{}}\PY{l+s}{date\PYZdq{}}\PY{p}{)}\PY{p}{]}\PY{p}{)}
         \PY{k+kp}{sum}\PY{p}{(}pairs\PY{p}{)}
\end{Verbatim}


    0

    
    \begin{Verbatim}[commandchars=\\\{\}]
{\color{incolor}In [{\color{incolor}66}]:} \PY{k+kr}{if} \PY{p}{(}\PY{k+kp}{sum}\PY{p}{(}pairs\PY{p}{)}\PY{o}{\PYZgt{}}\PY{l+m}{0}\PY{p}{)}\PY{p}{\PYZob{}}
             crsp \PY{o}{=} crsp\PY{p}{[}\PY{o}{!}\PY{k+kp}{duplicated}\PY{p}{(}crsp\PY{p}{[}\PY{p}{,}\PY{k+kt}{c}\PY{p}{(}\PY{l+s}{\PYZdq{}}\PY{l+s}{PERMNO\PYZdq{}}\PY{p}{,} \PY{l+s}{\PYZdq{}}\PY{l+s}{date\PYZdq{}}\PY{p}{)}\PY{p}{]}\PY{p}{)}\PY{p}{,}\PY{p}{]}
         \PY{p}{\PYZcb{}}
\end{Verbatim}


    We know we only want firms with a \texttt{SHRCD} of 10 or 11 and an
\texttt{EXCHCD} of 1, 2, or 3.

    \begin{Verbatim}[commandchars=\\\{\}]
{\color{incolor}In [{\color{incolor}50}]:} crsp \PY{o}{=} crsp\PY{p}{[}crsp\PY{o}{\PYZdl{}}SHRCD\PY{o}{==}\PY{l+s}{\PYZdq{}}\PY{l+s}{10\PYZdq{}} \PY{o}{|} crsp\PY{o}{\PYZdl{}}SHRCD\PY{o}{==}\PY{l+s}{\PYZdq{}}\PY{l+s}{11\PYZdq{}}\PY{p}{,} \PY{p}{]}
         crsp \PY{o}{=} crsp\PY{p}{[}crsp\PY{o}{\PYZdl{}}EXCHCD\PY{o}{==}\PY{l+s}{\PYZdq{}}\PY{l+s}{1\PYZdq{}}\PY{o}{|} crsp\PY{o}{\PYZdl{}}EXCHCD\PY{o}{==}\PY{l+s}{\PYZdq{}}\PY{l+s}{2\PYZdq{}} \PY{o}{|} crsp\PY{o}{\PYZdl{}}EXCHCD\PY{o}{==}\PY{l+s}{\PYZdq{}}\PY{l+s}{3\PYZdq{}}\PY{p}{,}\PY{p}{]}
\end{Verbatim}


    For now, we are not bothered about the \texttt{TICKER}, \texttt{COMNAM},
\texttt{CUSIP}, \texttt{DLSTCD}, \texttt{SPREAD}, or \texttt{vwretd}. We
will borrow the value-weighted return from Dr. Ken French.

    \begin{Verbatim}[commandchars=\\\{\}]
{\color{incolor}In [{\color{incolor}51}]:} \PY{c+c1}{\PYZsh{} The funny \PYZsq{}\PYZpc{}\PYZgt{}\PYZpc{}\PYZsq{} operator belongs to dplyr. It has all sorts of}
         \PY{c+c1}{\PYZsh{}   magical properties but basically, we can perform grouping and}
         \PY{c+c1}{\PYZsh{}   operations on groups in a sane and readable way.}
         \PY{c+c1}{\PYZsh{}   Unfortunately, dplyr is so magic it begins to get in the way of}
         \PY{c+c1}{\PYZsh{}   automating some of the more interesting procedures.}
         crsp \PY{o}{=} crsp \PY{o}{\PYZpc{}\PYZgt{}\PYZpc{}}
             select\PY{p}{(}\PY{o}{\PYZhy{}}\PY{k+kt}{c}\PY{p}{(}SHRCD\PY{p}{,} TICKER\PY{p}{,} COMNAM\PY{p}{,} CUSIP\PY{p}{,} DLSTCD\PY{p}{,} SPREAD\PY{p}{,} vwretd\PY{p}{)}\PY{p}{)}
\end{Verbatim}


    \subsubsection{How many firms do we
have?}\label{how-many-firms-do-we-have}

There are roughly 24K unique \texttt{PERMNO} codes. But by 2016 there
are less than 4K firms. We should be mindful of firms dying. We also see
two important dates regarding the exchanges used in the data. In 1962 we
see tha addition of AMEX stocks and in 1973 we see the addition of
NASDAQ stocks. It is also important to note that NASDAQ stocks are the
most common.

    \begin{Verbatim}[commandchars=\\\{\}]
{\color{incolor}In [{\color{incolor}88}]:} \PY{k+kp}{length}\PY{p}{(}\PY{k+kp}{unique}\PY{p}{(}crsp\PY{o}{\PYZdl{}}PERMNO\PY{p}{)}\PY{p}{)}
\end{Verbatim}


    24010

    
    \begin{Verbatim}[commandchars=\\\{\}]
{\color{incolor}In [{\color{incolor}116}]:} dt \PY{o}{=} crsp \PY{o}{\PYZpc{}\PYZgt{}\PYZpc{}} group\PYZus{}by\PY{p}{(}\PY{k+kp}{date}\PY{p}{)} \PY{o}{\PYZpc{}\PYZgt{}\PYZpc{}} summarise\PY{p}{(}N\PY{o}{=}n\PY{p}{(}\PY{p}{)}\PY{p}{)}
          dt\PY{o}{\PYZdl{}}date \PY{o}{=} \PY{k+kp}{as.Date}\PY{p}{(}dt\PY{o}{\PYZdl{}}\PY{k+kp}{date}\PY{p}{,} format\PY{o}{=}\PY{l+s}{\PYZsq{}}\PY{l+s}{\PYZpc{}Y\PYZpc{}m\PYZpc{}d\PYZsq{}}\PY{p}{)}
          p \PY{o}{=} ggplot\PY{p}{(}dt\PY{p}{,} aes\PY{p}{(}x\PY{o}{=}\PY{k+kp}{date}\PY{p}{,} y\PY{o}{=}N\PY{p}{)}\PY{p}{)} \PY{o}{+} geom\PYZus{}line\PY{p}{(}\PY{p}{)}
          p \PY{o}{=} p \PY{o}{+} scale\PYZus{}x\PYZus{}date\PY{p}{(}date\PYZus{}breaks\PY{o}{=}\PY{l+s}{\PYZdq{}}\PY{l+s}{2 year\PYZdq{}}\PY{p}{,} date\PYZus{}labels\PY{o}{=}\PY{l+s}{\PYZdq{}}\PY{l+s}{\PYZpc{}Y\PYZpc{}m\PYZdq{}}\PY{p}{)}
          p \PY{o}{+} theme\PY{p}{(}axis.text.x\PY{o}{=}element\PYZus{}text\PY{p}{(}angle\PY{o}{=}\PY{l+m}{90}\PY{p}{,} vjust\PY{o}{=}\PY{l+m}{0.5}\PY{p}{)}\PY{p}{)}
\end{Verbatim}


    
    
    \begin{center}
    \adjustimage{max size={0.9\linewidth}{0.9\paperheight}}{output_15_1.png}
    \end{center}
    { \hspace*{\fill} \\}
    
    \begin{Verbatim}[commandchars=\\\{\}]
{\color{incolor}In [{\color{incolor}117}]:} dt \PY{o}{=} crsp \PY{o}{\PYZpc{}\PYZgt{}\PYZpc{}} group\PYZus{}by\PY{p}{(}\PY{k+kp}{date}\PY{p}{,} EXCHCD\PY{p}{)} \PY{o}{\PYZpc{}\PYZgt{}\PYZpc{}} summarise\PY{p}{(}N\PY{o}{=}n\PY{p}{(}\PY{p}{)}\PY{p}{)}
          dt\PY{o}{\PYZdl{}}date \PY{o}{=} \PY{k+kp}{as.Date}\PY{p}{(}dt\PY{o}{\PYZdl{}}\PY{k+kp}{date}\PY{p}{,} format\PY{o}{=}\PY{l+s}{\PYZsq{}}\PY{l+s}{\PYZpc{}Y\PYZpc{}m\PYZpc{}d\PYZsq{}}\PY{p}{)}
          p \PY{o}{=} ggplot\PY{p}{(}dt\PY{p}{,} aes\PY{p}{(}x\PY{o}{=}\PY{k+kp}{date}\PY{p}{,} y\PY{o}{=}N\PY{p}{,} color\PY{o}{=}EXCHCD\PY{p}{)}\PY{p}{)} \PY{o}{+} geom\PYZus{}line\PY{p}{(}\PY{p}{)}
          p \PY{o}{=} p \PY{o}{+} scale\PYZus{}x\PYZus{}date\PY{p}{(}date\PYZus{}breaks\PY{o}{=}\PY{l+s}{\PYZdq{}}\PY{l+s}{2 year\PYZdq{}}\PY{p}{,} date\PYZus{}labels\PY{o}{=}\PY{l+s}{\PYZdq{}}\PY{l+s}{\PYZpc{}Y\PYZpc{}m\PYZdq{}}\PY{p}{)}
          p \PY{o}{+} theme\PY{p}{(}axis.text.x\PY{o}{=}element\PYZus{}text\PY{p}{(}angle\PY{o}{=}\PY{l+m}{90}\PY{p}{,} vjust\PY{o}{=}\PY{l+m}{0.5}\PY{p}{)}\PY{p}{)}
\end{Verbatim}


    
    
    \begin{center}
    \adjustimage{max size={0.9\linewidth}{0.9\paperheight}}{output_16_1.png}
    \end{center}
    { \hspace*{\fill} \\}
    
    On average, there are 4.5K firms each month. This has implications for
when we make buckets. Fine sorts would be great but we see already there
are some problems with extremely fine sorts.

For example, if we want to make 10x10 unconditional sorts, we will be
left with, on average, 45 firms in each bucket.

    \begin{Verbatim}[commandchars=\\\{\}]
{\color{incolor}In [{\color{incolor}92}]:} \PY{k+kp}{round}\PY{p}{(}\PY{k+kp}{mean}\PY{p}{(}dt\PY{o}{\PYZdl{}}N\PY{p}{)}\PY{p}{,}\PY{l+m}{0}\PY{p}{)}
\end{Verbatim}


    4558

    
    \begin{Verbatim}[commandchars=\\\{\}]
{\color{incolor}In [{\color{incolor}94}]:} dt \PY{o}{\PYZpc{}\PYZgt{}\PYZpc{}} arrange\PY{p}{(}\PY{o}{\PYZhy{}}N\PY{p}{)} \PY{o}{\PYZpc{}\PYZgt{}\PYZpc{}} \PY{k+kp}{head}
\end{Verbatim}


    \begin{tabular}{r|ll}
 date & N\\
\hline
	 1997-07-31 & 7525      \\
	 1997-12-31 & 7520      \\
	 1997-11-28 & 7517      \\
	 1997-06-30 & 7515      \\
	 1997-08-29 & 7505      \\
	 1997-10-31 & 7501      \\
\end{tabular}


    
    All our columns are characters because we told \texttt{fread} to do this
to prevent it from whining. We can change them into numeric values and
remove the annoying letters at the same time (creating the ugly
\texttt{Warning} messages). \texttt{SHROUT} is divided by 1,000.

    \begin{Verbatim}[commandchars=\\\{\}]
{\color{incolor}In [{\color{incolor}52}]:} str\PY{p}{(}crsp\PY{p}{)}
\end{Verbatim}


    \begin{Verbatim}[commandchars=\\\{\}]
Classes 'data.table' and 'data.frame':	3117344 obs. of  8 variables:
 \$ PERMNO: chr  "10000" "10000" "10000" "10000" {\ldots}
 \$ date  : chr  "19860131" "19860228" "19860331" "19860430" {\ldots}
 \$ EXCHCD: chr  "3" "3" "3" "3" {\ldots}
 \$ SICCD : chr  "3990" "3990" "3990" "3990" {\ldots}
 \$ DLRET : chr  "" "" "" "" {\ldots}
 \$ PRC   : chr  "-4.37500" "-3.25000" "-4.43750" "-4.00000" {\ldots}
 \$ RET   : chr  "C" "-0.257143" "0.365385" "-0.098592" {\ldots}
 \$ SHROUT: chr  "3680" "3680" "3680" "3793" {\ldots}
 - attr(*, ".internal.selfref")=<externalptr> 

    \end{Verbatim}

    \begin{Verbatim}[commandchars=\\\{\}]
{\color{incolor}In [{\color{incolor}60}]:} crsp\PY{o}{\PYZdl{}}PRC \PY{o}{=} \PY{k+kp}{as.numeric}\PY{p}{(}crsp\PY{o}{\PYZdl{}}PRC\PY{p}{)}
         crsp\PY{o}{\PYZdl{}}RET \PY{o}{=} \PY{k+kp}{as.numeric}\PY{p}{(}crsp\PY{o}{\PYZdl{}}RET\PY{p}{)}
         crsp\PY{o}{\PYZdl{}}DLRET \PY{o}{=} \PY{k+kp}{as.numeric}\PY{p}{(}crsp\PY{o}{\PYZdl{}}DLRET\PY{p}{)}
         crsp\PY{o}{\PYZdl{}}SHROUT \PY{o}{=} \PY{k+kp}{as.numeric}\PY{p}{(}crsp\PY{o}{\PYZdl{}}SHROUT\PY{p}{)} \PY{o}{/} \PY{l+m}{1000}
\end{Verbatim}


    Some of our columns only contain numbers but they are descrete values.

\begin{itemize}
\tightlist
\item
  \texttt{EXCHCD} is only 1, 2 or 3
\item
  \texttt{SICCD} is only from 1000 to 9000 or something
\item
  \texttt{PERMNO} is unique to each firm
\end{itemize}

So we will leave these values as characters.

    We need to grab the year and month and from the date.

    \begin{Verbatim}[commandchars=\\\{\}]
{\color{incolor}In [{\color{incolor}123}]:} \PY{c+c1}{\PYZsh{} Rather annoyingly, this method seems to take unnecessarily long}
          \PY{c+c1}{\PYZsh{} crsp\PYZdl{}date = as.Date(crsp\PYZdl{}date, \PYZdq{}\PYZpc{}Y\PYZpc{}m\PYZpc{}d\PYZdq{})}
          \PY{c+c1}{\PYZsh{} crsp\PYZdl{}Year = as.numeric(format(crsp\PYZdl{}date, \PYZdq{}\PYZpc{}Y\PYZdq{}))}
          \PY{c+c1}{\PYZsh{} crsp\PYZdl{}Month = as.numeric(format(crsp\PYZdl{}date, \PYZdq{}\PYZpc{}m\PYZdq{}))}
          
          \PY{c+c1}{\PYZsh{} This is faster, I guess \PYZsq{}format\PYZsq{} and \PYZsq{}as.Date\PYZsq{} take a while}
          crsp\PY{o}{\PYZdl{}}Year \PY{o}{=} \PY{k+kp}{as.numeric}\PY{p}{(}\PY{k+kp}{substr}\PY{p}{(}crsp\PY{o}{\PYZdl{}}\PY{k+kp}{date}\PY{p}{,} \PY{l+m}{1}\PY{p}{,} \PY{l+m}{4}\PY{p}{)}\PY{p}{)}
          crsp\PY{o}{\PYZdl{}}Month \PY{o}{=} \PY{k+kp}{as.numeric}\PY{p}{(}\PY{k+kp}{substr}\PY{p}{(}crsp\PY{o}{\PYZdl{}}\PY{k+kp}{date}\PY{p}{,} \PY{l+m}{5}\PY{p}{,} \PY{l+m}{6}\PY{p}{)}\PY{p}{)}
\end{Verbatim}


    We want to adjust our returns for any delisting returns. Where the
return is missing but we have a delisting return we will just drop the
delisting return in place of the return. Where we have a return and
delisting return we will multiply the return by the delisting return.
Where we have neither we will just leave it.

    \begin{Verbatim}[commandchars=\\\{\}]
{\color{incolor}In [{\color{incolor}62}]:} ix \PY{o}{=} \PY{k+kp}{is.na}\PY{p}{(}crsp\PY{o}{\PYZdl{}}RET\PY{p}{)}
         crsp\PY{o}{\PYZdl{}}RET\PY{p}{[}ix\PY{p}{]} \PY{o}{=} crsp\PY{o}{\PYZdl{}}DLRET\PY{p}{[}ix\PY{p}{]}
         ix \PY{o}{=} \PY{o}{!}\PY{k+kp}{is.na}\PY{p}{(}crsp\PY{o}{\PYZdl{}}RET\PY{p}{)} \PY{o}{\PYZam{}} \PY{o}{!}\PY{k+kp}{is.na}\PY{p}{(}crsp\PY{o}{\PYZdl{}}DLRET\PY{p}{)}
         crsp\PY{o}{\PYZdl{}}RET\PY{p}{[}ix\PY{p}{]} \PY{o}{=} \PY{p}{(}\PY{l+m}{1}\PY{o}{+}crsp\PY{o}{\PYZdl{}}RET\PY{p}{[}ix\PY{p}{]}\PY{p}{)} \PY{o}{*} \PY{p}{(}\PY{l+m}{1}\PY{o}{+}crsp\PY{o}{\PYZdl{}}DLRET\PY{p}{[}ix\PY{p}{]}\PY{p}{)} \PY{o}{\PYZhy{}} \PY{l+m}{1}
\end{Verbatim}


    We will frequently rebalance portfolios every 12 months. We hold
portfolios from July to June.

    \begin{Verbatim}[commandchars=\\\{\}]
{\color{incolor}In [{\color{incolor}124}]:} crsp\PY{o}{\PYZdl{}}HP \PY{o}{=} crsp\PY{o}{\PYZdl{}}Year
          crsp\PY{o}{\PYZdl{}}HP\PY{p}{[}crsp\PY{o}{\PYZdl{}}Month\PY{o}{\PYZlt{}}\PY{l+m}{7}\PY{p}{]} \PY{o}{=} crsp\PY{o}{\PYZdl{}}HP\PY{p}{[}crsp\PY{o}{\PYZdl{}}Month\PY{o}{\PYZlt{}}\PY{l+m}{7}\PY{p}{]} \PY{o}{\PYZhy{}} \PY{l+m}{1}
\end{Verbatim}


    \subsection{Tickle Data}\label{tickle-data}

    Where did we end up?

    \begin{Verbatim}[commandchars=\\\{\}]
{\color{incolor}In [{\color{incolor}126}]:} \PY{k+kp}{head}\PY{p}{(}crsp\PY{p}{,} n\PY{o}{=}\PY{l+m}{10}\PY{p}{)}
\end{Verbatim}


    \begin{tabular}{r|llllllllllll}
 PERMNO & date & EXCHCD & SICCD & DLRET & PRC & RET & SHROUT & Year & Month & ME & HP\\
\hline
	 10000     & 19860131  & 3         & 3990      & NA        & -4.37500  &        NA & 3.680     & 1986      &  1        & 16.100000 & 1985     \\
	 10000     & 19860228  & 3         & 3990      & NA        & -3.25000  & -0.257143 & 3.680     & 1986      &  2        & 11.960000 & 1985     \\
	 10000     & 19860331  & 3         & 3990      & NA        & -4.43750  &  0.365385 & 3.680     & 1986      &  3        & 16.330000 & 1985     \\
	 10000     & 19860430  & 3         & 3990      & NA        & -4.00000  & -0.098592 & 3.793     & 1986      &  4        & 15.172000 & 1985     \\
	 10000     & 19860530  & 3         & 3990      & NA        & -3.10938  & -0.222656 & 3.793     & 1986      &  5        & 11.793878 & 1985     \\
	 10000     & 19860630  & 3         & 3990      & NA        & -3.09375  & -0.005025 & 3.793     & 1986      &  6        & 11.734594 & 1985     \\
	 10000     & 19860731  & 3         & 3990      & NA        & -2.84375  & -0.080808 & 3.793     & 1986      &  7        & 10.786344 & 1986     \\
	 10000     & 19860829  & 3         & 3990      & NA        & -1.09375  & -0.615385 & 3.793     & 1986      &  8        &  4.148594 & 1986     \\
	 10000     & 19860930  & 3         & 3990      & NA        & -1.03125  & -0.057143 & 3.793     & 1986      &  9        &  3.911531 & 1986     \\
	 10000     & 19861031  & 3         & 3990      & NA        & -0.78125  & -0.242424 & 3.843     & 1986      & 10        &  3.002344 & 1986     \\
\end{tabular}


    
    We need to create some variables;

\begin{itemize}
\tightlist
\item
  Market equity
\item
  Prior return
\end{itemize}

    Market equity is price times shares outstanding.
\[abs(PRC) \cdot SHROUT\] We need to use the absolute price because CRSP
uses a negative sign to indicate that the price was not directly
available but we infer it from the bid-ask spread.

    \begin{Verbatim}[commandchars=\\\{\}]
{\color{incolor}In [{\color{incolor}119}]:} crsp\PY{o}{\PYZdl{}}ME \PY{o}{=} \PY{k+kp}{abs}\PY{p}{(}crsp\PY{o}{\PYZdl{}}PRC\PY{p}{)} \PY{o}{*} crsp\PY{o}{\PYZdl{}}SHROUT
\end{Verbatim}


    \subsubsection{ME Quantiles}\label{me-quantiles}

We will frequently use the breakpoints for ME from NYSE stocks at
certain months (December, June, lagged).

    \begin{Verbatim}[commandchars=\\\{\}]
{\color{incolor}In [{\color{incolor}168}]:} q \PY{o}{=} \PY{k+kt}{c}\PY{p}{(}\PY{l+m}{.05}\PY{p}{,} \PY{l+m}{.1}\PY{p}{,} \PY{l+m}{.15}\PY{p}{,} \PY{l+m}{.2}\PY{p}{,} \PY{l+m}{.25}\PY{p}{,} \PY{l+m}{.3}\PY{p}{,} \PY{l+m}{.35}\PY{p}{,} \PY{l+m}{.4}\PY{p}{,} \PY{l+m}{.45}\PY{p}{,} \PY{l+m}{.5}\PY{p}{,}
                \PY{l+m}{.55}\PY{p}{,} \PY{l+m}{.6}\PY{p}{,} \PY{l+m}{.65}\PY{p}{,} \PY{l+m}{.7}\PY{p}{,} \PY{l+m}{.75}\PY{p}{,} \PY{l+m}{.8}\PY{p}{,} \PY{l+m}{.85}\PY{p}{,} \PY{l+m}{.9}\PY{p}{,} \PY{l+m}{.95}\PY{p}{,} \PY{l+m}{1}\PY{l+m}{.}\PY{p}{)}
          
          dt \PY{o}{=} crsp \PY{o}{\PYZpc{}\PYZgt{}\PYZpc{}} filter\PY{p}{(}EXCHCD\PY{o}{==}\PY{l+s}{\PYZdq{}}\PY{l+s}{1\PYZdq{}}\PY{p}{,} \PY{o}{!}\PY{k+kp}{is.na}\PY{p}{(}ME\PY{p}{)}\PY{p}{)}
          
          N \PY{o}{=} dt \PY{o}{\PYZpc{}\PYZgt{}\PYZpc{}} group\PYZus{}by\PY{p}{(}\PY{k+kp}{date}\PY{p}{)} \PY{o}{\PYZpc{}\PYZgt{}\PYZpc{}} summarise\PY{p}{(}N\PY{o}{=}n\PY{p}{(}\PY{p}{)}\PY{p}{)} \PY{o}{\PYZpc{}\PYZgt{}\PYZpc{}} \PY{k+kp}{as.data.frame}
          \PY{k+kp}{rownames}\PY{p}{(}N\PY{p}{)} \PY{o}{=} N\PY{o}{\PYZdl{}}\PY{k+kp}{date}
          N\PY{o}{\PYZdl{}}date \PY{o}{=} \PY{k+kc}{NULL}
          
          quantiles \PY{o}{=} \PY{k+kp}{do.call}\PY{p}{(}\PY{l+s}{\PYZdq{}}\PY{l+s}{rbind\PYZdq{}}\PY{p}{,} \PY{k+kp}{tapply}\PY{p}{(}dt\PY{o}{\PYZdl{}}ME\PY{p}{,} dt\PY{o}{\PYZdl{}}\PY{k+kp}{date}\PY{p}{,} quantile\PY{p}{,} \PY{k+kp}{q}\PY{p}{)}\PY{p}{)}
          quantiles \PY{o}{=} tibble\PY{o}{::}rownames\PYZus{}to\PYZus{}column\PY{p}{(}\PY{k+kp}{cbind}\PY{p}{(}quantiles\PY{p}{,} N\PY{p}{)}\PY{p}{,} \PY{l+s}{\PYZdq{}}\PY{l+s}{date\PYZdq{}}\PY{p}{)}
          
          write.csv\PY{p}{(}quantiles\PY{p}{,} \PY{l+s}{\PYZdq{}}\PY{l+s}{C:/Data/Thesis/ME\PYZus{}20\PYZus{}quantiles.csv\PYZdq{}}\PY{p}{)}
\end{Verbatim}


    Only about 1\% are missing in total. As many as 8.5\% are missing in
August, 1976. The months with the highest percentages of missing values
are all in the mid-1970s. This is when NASDAQ stocks were introduced.

    \begin{Verbatim}[commandchars=\\\{\}]
{\color{incolor}In [{\color{incolor}74}]:} \PY{k+kp}{round}\PY{p}{(}\PY{k+kp}{summary}\PY{p}{(}crsp\PY{o}{\PYZdl{}}ME\PY{p}{)}\PY{p}{[}\PY{l+s}{\PYZdq{}}\PY{l+s}{NA\PYZsq{}s\PYZdq{}}\PY{p}{]} \PY{o}{/} \PY{k+kp}{nrow}\PY{p}{(}crsp\PY{p}{)} \PY{o}{*} \PY{l+m}{100}\PY{p}{,} \PY{l+m}{2}\PY{p}{)}
\end{Verbatim}


    
    \begin{verbatim}
NA's 
1.21 
    \end{verbatim}

    
    \begin{Verbatim}[commandchars=\\\{\}]
{\color{incolor}In [{\color{incolor}115}]:} dt \PY{o}{=} crsp \PY{o}{\PYZpc{}\PYZgt{}\PYZpc{}} group\PYZus{}by\PY{p}{(}\PY{k+kp}{date}\PY{p}{)} \PY{o}{\PYZpc{}\PYZgt{}\PYZpc{}} summarise\PY{p}{(}mssg\PY{o}{=}\PY{k+kp}{sum}\PY{p}{(}\PY{k+kp}{is.na}\PY{p}{(}ME\PY{p}{)}\PY{p}{)}\PY{o}{/}n\PY{p}{(}\PY{p}{)}\PY{p}{)}
          dt\PY{o}{\PYZdl{}}date \PY{o}{=} \PY{k+kp}{as.Date}\PY{p}{(}dt\PY{o}{\PYZdl{}}\PY{k+kp}{date}\PY{p}{,} format\PY{o}{=}\PY{l+s}{\PYZsq{}}\PY{l+s}{\PYZpc{}Y\PYZpc{}m\PYZpc{}d\PYZsq{}}\PY{p}{)}
          \PY{c+c1}{\PYZsh{}qplot(date, mssg*100, data=dt, geom=\PYZsq{}line\PYZsq{})}
          p \PY{o}{=} ggplot\PY{p}{(}dt\PY{p}{,} aes\PY{p}{(}x\PY{o}{=}\PY{k+kp}{date}\PY{p}{,} y\PY{o}{=}mssg\PY{o}{*}\PY{l+m}{100}\PY{p}{)}\PY{p}{)} \PY{o}{+} geom\PYZus{}line\PY{p}{(}\PY{p}{)}
          p \PY{o}{=} p \PY{o}{+} scale\PYZus{}x\PYZus{}date\PY{p}{(}date\PYZus{}breaks\PY{o}{=}\PY{l+s}{\PYZdq{}}\PY{l+s}{2 year\PYZdq{}}\PY{p}{,} date\PYZus{}labels\PY{o}{=}\PY{l+s}{\PYZdq{}}\PY{l+s}{\PYZpc{}Y\PYZpc{}m\PYZdq{}}\PY{p}{)}
          p \PY{o}{+} theme\PY{p}{(}axis.text.x\PY{o}{=}element\PYZus{}text\PY{p}{(}angle\PY{o}{=}\PY{l+m}{90}\PY{p}{,} vjust\PY{o}{=}\PY{l+m}{0.5}\PY{p}{)}\PY{p}{)}
\end{Verbatim}


    
    
    \begin{center}
    \adjustimage{max size={0.9\linewidth}{0.9\paperheight}}{output_40_1.png}
    \end{center}
    { \hspace*{\fill} \\}
    
    \begin{Verbatim}[commandchars=\\\{\}]
{\color{incolor}In [{\color{incolor}87}]:} dt \PY{o}{\PYZpc{}\PYZgt{}\PYZpc{}} arrange\PY{p}{(}\PY{o}{\PYZhy{}}mssg\PY{p}{)} \PY{o}{\PYZpc{}\PYZgt{}\PYZpc{}} \PY{k+kp}{head}
\end{Verbatim}


    \begin{tabular}{r|ll}
 date & mssg\\
\hline
	 1976-08-31 & 0.08732453\\
	 1975-01-31 & 0.05919836\\
	 1974-12-31 & 0.05803938\\
	 1976-01-30 & 0.05555556\\
	 1974-11-29 & 0.05358969\\
	 1975-02-28 & 0.05226193\\
\end{tabular}


    
    Given the plots above about the number of firms on each exchange and the
results below, we see results will be driven by NYSE and NASDAQ stocks.
There are more NASDAQ stocks but they are small. There are few NYSE
stocks but they are large.

    \begin{Verbatim}[commandchars=\\\{\}]
{\color{incolor}In [{\color{incolor}122}]:} dt \PY{o}{=} crsp \PY{o}{\PYZpc{}\PYZgt{}\PYZpc{}} group\PYZus{}by\PY{p}{(}\PY{k+kp}{date}\PY{p}{,} EXCHCD\PY{p}{)} \PY{o}{\PYZpc{}\PYZgt{}\PYZpc{}}
              summarise\PY{p}{(}ME\PY{o}{=}\PY{k+kp}{mean}\PY{p}{(}ME\PY{p}{,} na.rm\PY{o}{=}\PY{k+kc}{TRUE}\PY{p}{)}\PY{p}{)}
          dt\PY{o}{\PYZdl{}}date \PY{o}{=} \PY{k+kp}{as.Date}\PY{p}{(}dt\PY{o}{\PYZdl{}}\PY{k+kp}{date}\PY{p}{,} format\PY{o}{=}\PY{l+s}{\PYZsq{}}\PY{l+s}{\PYZpc{}Y\PYZpc{}m\PYZpc{}d\PYZsq{}}\PY{p}{)}
          p \PY{o}{=} ggplot\PY{p}{(}dt\PY{p}{,} aes\PY{p}{(}x\PY{o}{=}\PY{k+kp}{date}\PY{p}{,} y\PY{o}{=}ME\PY{p}{,} color\PY{o}{=}EXCHCD\PY{p}{)}\PY{p}{)} \PY{o}{+} geom\PYZus{}line\PY{p}{(}\PY{p}{)}
          p \PY{o}{=} p \PY{o}{+} scale\PYZus{}x\PYZus{}date\PY{p}{(}date\PYZus{}breaks\PY{o}{=}\PY{l+s}{\PYZdq{}}\PY{l+s}{2 year\PYZdq{}}\PY{p}{,} date\PYZus{}labels\PY{o}{=}\PY{l+s}{\PYZdq{}}\PY{l+s}{\PYZpc{}Y\PYZpc{}m\PYZdq{}}\PY{p}{)}
          p \PY{o}{+} theme\PY{p}{(}axis.text.x\PY{o}{=}element\PYZus{}text\PY{p}{(}angle\PY{o}{=}\PY{l+m}{90}\PY{p}{,} vjust\PY{o}{=}\PY{l+m}{0.5}\PY{p}{)}\PY{p}{)}
\end{Verbatim}


    \begin{Verbatim}[commandchars=\\\{\}]
Warning message:
"Removed 47 rows containing missing values (geom\_path)."
    \end{Verbatim}

    
    
    \begin{center}
    \adjustimage{max size={0.9\linewidth}{0.9\paperheight}}{output_43_2.png}
    \end{center}
    { \hspace*{\fill} \\}
    

    % Add a bibliography block to the postdoc
    
    
    
    \end{document}

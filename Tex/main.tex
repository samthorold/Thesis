\documentclass[a4paper, 12pt]{article}

\usepackage{afterpage}
\usepackage{amsmath, amsthm}
\usepackage[backend=biber, style=authoryear]{biblatex}
\usepackage{bookmark}       % links w/in PDF
\usepackage{booktabs}       % align tables by decimal point -- regression t-stats
\usepackage{dcolumn}        % table cells alignment
\usepackage[margin=2.5cm]{geometry}
%\usepackage{hyperref}      % links w/in PDF
\usepackage{import}         % nested files
\usepackage[utf8]{inputenc} % unsure
\usepackage{pdflscape}      % landscape pages
\usepackage{setspace}       % double line spacing but not in tables etc.
\usepackage{subfig}         % tables next to each other

\addbibresource{references.bib}

\title{
  {A top-down approach to factor models}\\
  {\large Norwegian School of Economics}\\
  {\large NOVA School of Business and Economics}
}
\author{Sam Thorold}
\date{\today}

\begin{document}

\maketitle

\setstretch{1.5}

\begin{abstract}
  A five-factor model of
  market, size, value, momentum and profitability factors
  has a maximum Sharpe ratio of 0.56.
  The largest marginal contributions are from value and profitability.
  Value and momentum, through their relationship with book equity,
  interact to make investment unnecessary.
  Omitting the investment factor does not hinder performance when describing
  sorts on investment.
  The model struggles most with sorts on volatility and momentum.
  The returns on small, unprofitable stocks that invest aggressively behave
  like the returns on small, unprofitable stocks with poor recent performance.
\end{abstract}

\pagebreak

\import{./}{intro}

\import{./}{literature}

%\import{./}{background_group}

\import{./}{data}

\import{./}{analysis_group}

\import{./}{conclusions}

\setstretch{1.0}

\printbibliography

\appendix

\import{./}{tables_group}

%\import{./}{vars}

\end{document}

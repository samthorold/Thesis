\documentclass[a4paper, 12pt]{article}

\usepackage{afterpage}      % sensible text around landscape pages
\usepackage{amsmath, amsthm}
\usepackage{authblk}        % handle author nonsense on titlepage
\usepackage[backend=biber, style=authoryear]{biblatex}
\usepackage{booktabs}       % align tables by decimal point, regression t-stats
\usepackage{dcolumn}        % table cells alignment
\usepackage{float}          % tables where I put them
\usepackage{fontspec}
\usepackage[margin=2.5cm]{geometry}
\usepackage{graphicx}       % rotated column headers
\usepackage{import}         % sane treatment of nested files
\usepackage[utf8]{inputenc} % unsure
\usepackage{pdflscape}      % landscape pages
\usepackage{setspace}       % double line spacing but not in tables etc.
\usepackage[sf]{titlesec}   % section headers to sans-serif

\addbibresource{references.bib}

\setmainfont[Ligatures=TeX]{Times New Roman}

\title{
  {A top-down approach to factor models}\\[1cm]
  {\Large Sam Thorold}\\%\thanks{sam.thorold@gmail.com}}\\
  {\large Finance Program}\\
}

\author[1]{Advisors: Francisco Santos}
\affil[1]{Norwegian School of Economics}

\author[2]{Andr\'e Silva}
\affil[2]{NOVA School of Business and Economics}

\date{\today}

\begin{document}

\maketitle

\begin{abstract}
  A five-factor model of
  market, size, value, momentum and profitability factors
  has a maximum squared Sharpe ratio of 0.316.
  Value and momentum, through their relationship with book equity,
  interact to make investment unnecessary.
  Omitting the investment factor does not hinder performance when describing
  sorts on investment.
  The model struggles most with sorts on volatility and momentum.
  The returns on small, unprofitable stocks that invest aggressively behave
  like the returns on small, unprofitable stocks with poor recent performance.
  Problems are driven by firm size rather than liquidity constraints.
\end{abstract}

\pagebreak

% \doublespacing  % equivalent to \setstretch{1.667} (tex.stackexchange 84260)
\setstretch{2}

\import{./}{intro}
\import{./}{literature}
\import{./}{data}
\import{./}{analysis_group}
\import{./}{conclusions}

\setstretch{1.0}

\printbibliography

\appendix
%\import{./}{tables_group}
\import{./}{vars}
\import{./}{replication}

\end{document}


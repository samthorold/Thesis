\documentclass[a4paper, 12pt]{article}

\usepackage{afterpage}
\usepackage{amsmath, amsthm}
\usepackage[backend=biber, style=authoryear]{biblatex}
\usepackage{bookmark}       % links w/in PDF
\usepackage{booktabs}       % align tables by decimal point -- regression t-stats
\usepackage{dcolumn}        % table cells alignment
\usepackage[margin=2.5cm]{geometry}
%\usepackage{hyperref}      % links w/in PDF
\usepackage{import}         % nested files
\usepackage[utf8]{inputenc} % unsure
\usepackage{pdflscape}      % landscape pages
\usepackage{setspace}       % double line spaceing but not in tables etc.
\usepackage{subfig}         % tables next to each other

\addbibresource{references.bib}

\setstretch{2.0}

\title{
  {Asset-pricing models: Turtles all the way down}\\
  {\large Norwegian School of Economics}\\
  {\large NOVA School of Business and Economics}
}
\author{Sam Thorold}
\date{\today}

\begin{document}

\maketitle

\begin{abstract}
  A five-factor model of
  market, size, value, momentum and profitability factors
  has a maximum Sharpe ratio of 0.56.
  The largest marginal contributions are from value and profitability.
  Value and momentum interact to make investment unnecessary.
  Omitting investment does not hinder performance when explaining sorts on investment.
  The model struggles most with sorts on volatility and momentum.
\end{abstract}

\import{./}{intro}

\import{./}{literature}

\import{./}{data}

%\import{./}{method}

\import{./}{results}

\import{./}{conclusions}

\printbibliography

\appendix

\import{./}{tables}

%\import{./}{addl_results}

\import{./}{vars}

%\import{./}{replication}

\end{document}

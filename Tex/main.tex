\documentclass[a4paper, 12pt]{article}

\usepackage{afterpage}      % sensible text around landscape pages
\usepackage{amsmath, amsthm}
\usepackage{authblk}        % handle author nonsense on titlepage
\usepackage[backend=biber, style=authoryear]{biblatex}
\usepackage{booktabs}       % align tables by decimal point, regression t-stats
\usepackage{dcolumn}        % table cells alignment
\usepackage{float}          % tables where I put them
\usepackage{fontspec}
\usepackage[margin=2.5cm]{geometry}
\usepackage{graphicx}       % rotated column headers
\usepackage{import}         % sane treatment of nested files
\usepackage[utf8]{inputenc} % unsure
\usepackage{pdflscape}      % landscape pages
\usepackage{setspace}       % double line spacing but not in tables etc.
% \usepackage[sf]{titlesec}   % section headers to sans-serif

\addbibresource{references.bib}

\setmainfont[Ligatures=TeX]{Times New Roman}

\title{
  {A top-down approach to factor models}\\[1cm]
  {\Large Sam Thorold}\\%\thanks{sam.thorold@gmail.com}}\\
  {\large Finance Program}\\
}

\author[1]{Advisors: Francisco Santos}
\affil[1]{Norwegian School of Economics}

\author[2]{Andr\'e Silva}
\affil[2]{NOVA School of Business and Economics}

\date{\today}

\begin{document}

\maketitle

\begin{abstract}
I propose choosing factors based on the maximum squared Sharpe ratio
($\text{Sh}^2$) of the model.
The model then provides a description of anomalies, but anomalies do not drive
the choice of factors.
I introduce a five-factor model of market, size, value, momentum, and
profitability factors with a $\text{Sh}^2$ of 0.316.
The $\text{Sh}^2$ is higher than competing models and mispricing is reduced
for common anomalies.
Value and momentum subsume the popular investment factor through their ability
to forecast changes in book equity.
The model struggles to price sorts on value and momentum.
The model's description of these anomalies, small, unprofitable stocks with
poor recent returns, point to problems with firm size beyond the usual
explanation of illiquidity.
In particular, I find that problems arise from the interaction of size,
volatility, and value.
\end{abstract}

\pagebreak

% \doublespacing  % equivalent to \setstretch{1.667} (tex.stackexchange 84260)
\setstretch{2}

\import{./}{intro}
\import{./}{literature}
\import{./}{data}
\import{./}{analysis_group}
\import{./}{conclusions}

\setstretch{1.0}

\printbibliography

\appendix
\import{./}{vars}
\import{./}{replication}

\end{document}


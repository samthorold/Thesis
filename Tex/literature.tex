% !TeX root=./main.tex

\section*{Prior Research}

\textcite{ball2016accruals} show that the Sharpe ratio of a five-factor
model created by adding a cash profitability factor to the four factor model
of \textcite{carhart1997persistence} cannot be improved by an accruals factor.
They test a model similar to mine except they rebalance value annually.
I take advantage of the interaction between monthly value and momentum and
their abililty to forecast changes in BE.
\textcite{fama2016choosing} show that adjusting profitability for accruals in
the style of Ball et al. improves the Sharpe ratio of the five-factor model of
\textcite{fama2015five}.
Their model has the time series regression form
\begin{equation} \label{eq:F16}
R_t^i = a^i+b^iR^M+s^iSMB_t+v^iHML_t+p^iPMU_t+i^iCMA_t
\end{equation}
where $CMA$ is the return on a portfolio long conservatively investing stocks
and short aggressively investing stocks mimicking the investment premium.
Fama and French use the intuition of \textcite{barillas2016alpha} who explain
we can compare the ability of models to price the returns on all
portfolios by comparing the Sharpe ratio of the factors.
Ideally, the level of mispricing for test portfolios is given by the
quadratic form of the intercepts, ``alphas", from time series regressions.
This measure gives the amount the squared Sharpe ratio ($Sh^2$) can be
improved by investing in the test portfolios as well as the factors.
The relationship is given by
$a'V_ea=Sh^2\left(R,f\right)-Sh^2\left(f\right)$
where $a$ is the vector of alphas, $V_e$ is the covariance matrix of the
residuals, $Sh^2\left(\cdot\right)$ is the maximum squared Sharpe ratio, $R$ is
the matrix of test asset excess returns and $f$ is the matrix of factor
returns.
We cannot identify all portfolios so an absolute measure of mispricing is not
possible for all portfolios.
The key insight from Barillas and Shanken is that
$Sh^2\left(R,f\right)=Sh^2\left(R\right)$ because our factors, $f$, are
contained in all assets, $R$.
The level of mispricing becomes $a'V_ea=Sh^2\left(R\right)-Sh^2\left(f\right)$.
Maximizing $Sh^2\left(f\right)$ minimizes $a'V_ea$ without
identifying all portfolios.
I use a bootstrap procedure to show the investment factor does not improve the
Sharpe ratio of model \ref{eq:B16} and that model \ref{eq:B16} has a higher
Sharpe ratio than model \ref{eq:F16}.

\textcite{fama2015five} test model \ref{eq:F16}'s ability
to price the returns on sorts constructed from the same variables
used to construct the factors.
They find that value is subsumed by profitability and investment but
acknowledge this may be due to the sample period.
Fama and French find small stocks with negative profitability and investment
slopes cause the most problems for their model.
In contrast, I find that value, when combined with momentum, is not subsumed by
profitability and investment.
Further, I find that investment is subsumed by value and momentum.
% \textcite{lewellen2010skeptical} point out that a model should also be able to
% price the returns on sorts constructed from variables not used to construct the
% factors.
\textcite{fama2016dissecting} test model \ref{eq:F16}'s ability
to price the returns on sorts constructed from different variables than those
used to construct the factors.
Positive profitability and investment slopes help to price the returns to
low-beta and low-volatility stocks.
Negative profitability and investment slopes continue to be a problem with
high-beta and high-volatility stocks causing the most problems.
My different factors give a different description of problem portfolios.
I find the highest mispricing occurs for small, unprofitable stocks with poor
recent returns.
Further, momentum helps to reduce the mispricing for low-beta and high-volatilty
stocks.

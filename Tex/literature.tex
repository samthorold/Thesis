% !TeX root=./main.tex

\section{Literature Review}

\import{./Literature/}{joint}

\import{./Literature/}{errors}

\subsection{Value and Momentum}

\textcite{kok2017facts} argue  that fundamental value measures, such as the popular 
book-to-market ratio, are mean-reverting.
\[
B_{t+1}/p_{t+1} = B_t/p_t \cdot B_{t+1}/B_t \cdot p_t/p_{t+1}
\]
taking logs
\[
log(B_{t+1}/p_{t+1}) = log(B_{t}/p_{t}) + log(B_{t+1}/B_{t}) - log(p_{t+1}/p_{t})
\]
where; $B_t$ is a measure of book value at time $t$ and $p$ is market price.
A high book-to-market ratio can revert to its mean through an increase price or
a decrease in book equity.
Investors wish to identify those stocks that will increase in price because
$R_{t+1}=p_{t+1}/p_{t}$.
Book-price measures are polluted by stocks that revert to mean values through
decreases in book equity.
Book-to-market ratios are useful, so long as we can separate stocks that will
experience a decrease in book values from those that will increase in price.
\textcite{kok2017facts} show that decreases in book-equity are lower for high
momentum stocks.

\import{./Literature/}{kok2017facts}

\textcite{asness2013devil} show that momentum can forecast future changes in
book equity.
Combining value and momentum identifies high book-to-market stocks that will
revert to their mean fundamental value through an increase in price.
This is true of the monthly, current measure of book-to-market without the
presence of momentum in a multi-factor regression.
\textbf{Why include momentum then?}


\import{./Literature/}{barillas2016alpha}



\import{./Literature/}{similar}


\subsection{The ``Lethal" Combination}

\textcite{fama2015five, fama2016choosing, fama2017international} show that portfolios with
returns that behave like those of small, low profit but somehow aggressively investing
stocks are a problem.

\import{./Literature/}{intuition}

\subsection{Assorted Ramblings}

The consumption models performs poorly during significant downturns.
Momentum experiences crashes \parencite{daniel2016momentum}.


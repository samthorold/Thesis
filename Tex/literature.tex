% !TeX root=./main.tex

\section*{Prior Research}

\textcite{fama2016choosing} show that adjusting profitability for accruals in
the style of \textcite{ball2016accruals} improves the Sharpe ratio of the five-
factor model of \textcite{fama2015five}.
The model has the time series regression form
\begin{equation} \label{eq:F16}
R_t^i = a^i+b^iR^M+s^iSMB_t+v^iHML_t+p^iPMU_t+i^iCMA_t
\end{equation}
where $CMA$ is the return on a portfolio long conservatively investing stocks
and short aggressively investing stocks mimicking the investment premium.
The investment factor is rebalanced annually.
The remaining factors are as described in model \ref{eq:B16} except value is
rebalanced annually.
Fama and French rely heavily on the intuition of \textcite{barillas2016alpha}
who explain we can compare the ability of models to describe the returns on all
assets by comparing the Sharpe ratio of their factors.
The level of mis-pricing for a set of test assets is given by the quadratic
form of the intercepts from time series regressions (alphas).
This measure gives the amount the squared Sharpe ratio ($\text{Sh}^2$) can be
improved by investing in the test assets as well as the factors.
The relationship is given by $a'V_ea=Sh^2\left(R,f\right)-Sh^2\left(f\right)$
where $a$ is the vector of alphas, $V_e$ is the covariance matrix of the
residuals, $Sh^2\left(\cdot\right)$ is the maximum squared Sharpe ratio, $R$ is
the matrix of test asset excess returns and $f$ is the matrix of factor
returns.
The key insight from Barillas and Shanken is that
$Sh^2\left(R,f\right)=Sh^2\left(R\right)$ if $R$ is the returns on all assets.
We can write the level of mis-pricing for all assets as
$a'V_ea=Sh^2\left(R\right)-Sh^2\left(f\right)$ because our factors, $f$, are
contained in all assets, $R$.
We do not need to identify all the components of $R$ to minimize mis-pricing.
We only need to maximize the squared Sharpe ratio of the factors.
I follow the bootstrap procedure of \textcite{fama2016choosing} to show the
investment factor does not improve the Sharpe ratio of model \ref{eq:B16} and
that model \ref{eq:B16} has a higher Sharpe ratio than model \ref{eq:F16}.

\textcite{fama2016dissecting}, following the advice of
\textcite{lewellen2010skeptical}, test model \ref{eq:F16}'s ability to describe
the returns on anomalies created from sorts on variables not used to create the
factors. They find that many anomalies
behave like the returns on small, unprofitable stocks that
somehow invest aggressively. I find that such sorts behave like the returns on
small, unprofitable stocks with poor recent performance.

\textcite{fama2006profitability} use the dividend discount model to link the
profitability and investment factors to firm value.
With clean surplus accounting, the dividend discount model says that market
equity equals discounted expected future earnings less the change in book
equity,
$ME_t = \sum_{s=1}^\infty E(Y_{t+s}-\Delta BE_{t+s})/R^s$
where $Y$ is earnings and $R$ is the gross discount rate.
Fama and French use change in assets in place of change in book equity.
With sorts on value and investment, I show that change in assets and change in
book equity move in opposite directions for small, growth stocks.
Investment is a poor proxy for changes in book equity for small, growth stocks.

I use the value factor specification of \textcite{asness2013devil} who propose
updating BM monthly.
Consider a stock that has just had a positive return, all else the same, we
expect the BM to descrease.
Annually rebalanced value will stay the same while monthly rebalanced value
will reflect this change.
The power of using a more timely measure is not simply more frequent updating.
\textcite{kok2017facts} find that monthly value has greater power to identify
those stocks that will revert to their mean BM through a change in market
equity, driving returns, than book equity.
Monthly value has particular strength when combined with momentum.
Momentum is the tendency of stocks with higher (lower) recent returns than
their peers to continue to outperform (underperform) in the short term
\parencite{jegadeesh1993returns} and has been found in markets all over
the world \parencite{asness2013value}.
% I measure momentum as the previous year's return not including the most recent
% month \parencite{fama2016dissecting}.
Value and momentum enhance each other's slopes in time series regressions.
Any two variables that are negatively correlated with each other but positively
correlated with the independent variable will exhibit this effect
\parencite{fama2015incremental}.
I use sorts on value and momentum to show their strength is not just
statistical but also economic through their ability to forecast changes in BE.
Sorts on value and momentum create variation in investment and book equity
characteristics.
Value and momentum factors describe the returns on the investment anomaly
better than value and investment factors.

The specification of profitability is important as \textcite{ball2015deflating}
show that the scaling variable in the denominator can influence a ratio's
predictive power as much as the profitability level in the numerator.
\textcite{fama2006profitability} suggest net income scaled by
book equity (OP).
Their measure struggles to explain the returns on the accruals anomaly of
\textcite{sloan1996stock}.
\textcite{ball2016accruals} explain that investors are concerned with cash
flows to a firm.
They favour OP adjusted for accruals as these are the non-cash components of
financial statements.
They show that the Sharpe ratio of a five-factor model created by including CP
in the four factor model of \textcite{carhart1997persistence} cannot be
improved by an accruals factor.
% I use their balance sheet adjusted CP specification rather than the cash flow
% statement adjustment as balance sheet data is available for the entire sample.
Ball et al. test a model similar to mine except they rebalance value annually.
I take advantage of the interaction between monthly value and momentum and
their abililty to forecast changes in BE.


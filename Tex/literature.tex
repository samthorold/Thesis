% !TeX root=./main.tex

\section{Literature Review}

\import{./Literature/}{barillas2016alpha}

\import{./Literature/}{kok2017facts}

\subsection{Assorted Ramblings}

Consumption-based models typically specify $m$ as the marginal rate of 
substitution
\begin{equation}
\label{eq:consumption}
m=\beta\frac{u'(c_{t+1})}{u'(c_t)}
\end{equation}
where $u'(c)$ is the marginal utility from one more unit of consumption.
In this model, $m$ is the rate at which an investor will give up consumption 
today for more consumption tomorrow.
Equation \ref{eq:consumption} says that an investor will buy more of an asset 
until their discounted marginal utility

The consumption models performs poorly during significant downturns.
Momentum experiences crashes (\parencite{daniel2016momentum}).

``Anomalous" portfolios are important.
If nothing else, investors make gazillions of dollars each year from trading 
on them.
Nevertheless, ``anomalous" portfolios should not be the driving force behind 
discovering or updating factors.
The most effective defense against data-mining (since finance resolutely 
refuses to embrace replication studies) is intuition-based models.

How $1=E(mR^i)$ let's us say things about a model's performance pricing all 
assets.
Hansen, $a'Ia$ is nice because it preserves stuff but
$a'\Omega a$ can be nicer for asset pricing purposes (covariances of errors 
are important for asset-pricing models because this influences the maximum 
Sharpe ratio that can be obtained from the errors).

Motivated by the dividend discount model, \textcite{fama2016choosing} 
suggest a five-factor model (FF2016) of; the market, size, value, 
profitability.
They find a maximum Sharpe ratio of 0.24 for their model.
\textcite{hou2015digesting} put forward a four-factor model of; the market, 
size, 
investment and return on equity.
Their focus is not on maximizing the Sharpe ratio, although they report a 
value in excess of 0.4.
They concentrate on minimizing the absolute size of the intercepts in time 
series regressions of anomaly portfolios on their factors.


\begin{table}[ht] \centering
\caption{Max $Sh^2(f)$ and factor contributions, $\left( \frac{\alpha_i}{\sigma_i}\right) ^2$}
\label{tbl:Sh2}
\begin{tabular}{l D{.}{.}{1.3} D{.}{.}{1.3} D{.}{.}{1.3} D{.}{.}{1.3} D{.}{.}{1.3} D{.}{.}{1.3} D{.}{.}{1.3}}
\toprule
               & \multicolumn{1}{c}{$Sh^2$} & \multicolumn{1}{c}{Mkt} & \multicolumn{1}{c}{Size} & \multicolumn{1}{c}{Val} & \multicolumn{1}{c}{Prof} & \multicolumn{1}{c}{Inv} & \multicolumn{1}{c}{Mom} \\
\midrule
My Model       &  0.348  &  0.086  &  0.018  &  0.206  &  0.200  &         &  0.131 \\
My Model + Inv &  0.351  &  0.087  &  0.018  &  0.115  &  0.201  &  0.003  &  0.096 \\
FF2016         &  0.246  &  0.074  &  0.021  &  0.019  &  0.185  &  0.021  &        \\
FF2016  + Mom  &  0.260  &  0.076  &  0.017  &  0.025  &  0.161  &  0.017  &  0.014 \\
\bottomrule
\multicolumn{8}{r}{\small{Returns are from July, 1963 to December, 2015 (630 months)}}
\end{tabular} 
\end{table} 


Table \ref{tbl:Sh2} shows the maximum squared Sharpe ratios and marginal 
contributions from each of the factors for my model and that of 
\textcite{fama2016choosing}.
I include the missing investment and momentum factors where applicable to 
highlight marginal contributions to the maximum Sharpe ratio.
My model's maximum squared Sharpe ratio, 0.348, exceeds that of FF2016, 0.246, 
by 0.102.
For my model, value and cash profitability have similar marginal 
contributions to the Sharpe ratio of 0.2.
The market and momentum contribute 0.1 while size contributes 0.02.
The investment factor contributes very little, 0.003, as investment's effects 
are absorbed by the combination of value and momentum.
Popular asset-pricing models\footnote{Such as the three-factor model of 
  \textcite{fama1993common}, the four factor model of 
  \textcite{hou2015digesting}, the 
  five-factor model of \textcite{fama2015five} and the cash profitability 
  five-factor model of \textcite{fama2016choosing}} omit the momentum factor, 
despite 
momentum's persistence as an anomaly (\parencite{fama2016dissecting, 
  fama2017international}).
Including momentum in model FF2016 increases the Sharpe ratio from 0.246 
to 0.26.
Momentum contributes a comparable amount, 0.014, to the Sharpe ratio 
of FF2016 as the size factor, 0.017.
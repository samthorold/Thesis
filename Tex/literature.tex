% !TeX root=./main.tex

\section{Literature Review}

\subsection{Value and Momentum}

\textcite{kok2017facts} argue  that fundamental value measures, such as the popular 
book-to-market ratio, are mean-reverting.
\[
B_{t+1}/p_{t+1} = B_t/p_t \cdot B_{t+1}/B_t \cdot p_t/p_{t+1}
\]
taking logs
\[
log(B_{t+1}/p_{t+1}) = log(B_{t}/p_{t}) + log(B_{t+1}/B_{t}) - log(p_{t+1}/p_{t})
\]
where; $B_t$ is a measure of book value at time $t$ and $p$ is market price.
A high book-to-market ratio can revert to its mean through an increase price or
a decrease in book equity.
Investors wish to identify those stocks that will increase in price because
$R_{t+1}=p_{t+1}/p_{t}$.
Book-price measures are polluted by stocks that revert to mean values through
decreases in book equity.
Book-to-market ratios are useful, so long as we can separate stocks that will
experience a decrease in book values from those that will increase in price.
\textcite{kok2017facts} show that decreases in book-equity are lower for high
momentum stocks.

\import{./Literature/}{kok2017facts}

\textcite{asness2013devil} show that momentum can forecast future changes in
book equity.
Combining value and momentum identifies high book-to-market stocks that will
revert to their mean fundamental value through an increase in price.
This is true of the monthly, current measure of book-to-market without the
presence of momentum in a multi-factor regression.
\textbf{Why include momentum then?}



\subsection{Comparing Models with Sharpe Ratio}

Following \textcite{jensen1968performance}, most tests of factor models include
time series regressions of the form
\[
r_t^i = a^i + b\cdot f + e^i
\]
where;$r_t^i$ is the return on asset $i$ at time $t$ in excess of the risk free
rate, $a$ is the regression intercept, $f$ is a set of factors and $e$ is the
error terms.
If our factors explain all of the variation in the returns on asset $i$, the
regression intercept, $a$, should be 0.
\textcite{gibbons1989test} show that for a set of assets, our null hypothesis
that all intercepts are 0 follows an F-distribution.
This provides a test statistic for whether a set of factors completely
describes the returns on a set of test assets.
In the literature, this is the ``GRS" statistic.
They show that the squared Sharpe ratio of the intercepts is
\[
a'V_ea = Sh^2(r,f) - Sh^2(f)
\]
where $V_e$ is the covariance matrix of $e$ and $Sh^2(\cdot)$ denotes squared
Sharpe ratio.
The Sharpe ratio of the intercepts is the increase in the Sharpe ratio when
adding the test assets to the set of factors.

\textcite{barillas2016alpha} make the brilliant and intuitive point that
\[
Sh^2(r,f) = Sh^2(r)
\]
if $r$ is the set of returns on all assets because the set of factors $f$ is
included in $r$.
We can minimize the Sharpe ratio of the unknown intercepts by maximizing the
Sharpe ratio of the factors.

\import{./Literature/}{gibbons1989test}

\import{./Literature/}{barillas2016alpha}



\subsection{Today's ``state of the art"}

The factor model closest to mine is that of \textcite{fama2016choosing}.
They suggest a five-factor model with the time series regression form of
\begin{equation}
\label{eq:tsFF2016}
r_t^i - r_t^f= \alpha^i + \beta^i \cdot Mkt + s^i \cdot Size + v^i \cdot Val + p^i \cdot Prof 
+ i^i \cdot Inv
\end{equation}
where; $Val$ is annually rebalanced using $ME$ from December of last year.
The other factors are as specified in \ref{eq:tsmodel}.
The dividend discount model, $p_t = \sum E(d_t)/R_t$, says that stock prices are the expected 
discounted sum of their dividends.
\textcite{fama2006profitability} explain that by assuming clean surplus accounting and 
dividing each side by book-equity we find
\begin{equation}
\label{eq:ddm}
\frac{ME_t}{BE_t} = \frac{\sum \frac{E(Prof_t-\Delta BE_t)}{R_t}}{BE_t}
\end{equation}
They use this relationship to make three intuitive points.
All else the same;
a lower $ME$ (and higher $BE/ME$) leads to a higher return,
a higher profit leads to a higher return and
a lower change in book-equity leads to a higher return.
While (\ref{eq:ddm}) calls for change in book-equity, asset-growth is used as a proxy in 
empirical work (See \textcite{fama2015five, fama2016choosing, fama2017international}).
The choice to use asset-growth is not arbitrary, Fama and French argue this is a better 
``picture" of what the dividend discount model implies $\Delta BE$ represents, 
investment.

\textcite{ball2016accruals} suggest a similar model with the time series regression form
\begin{equation}
\label{eq:tsBall2016}
r_t^i - r_t^f= \alpha^i + \beta^i \cdot Mkt + s^i \cdot Size + v^i \cdot Val + p^i \cdot Prof 
+ m^i \cdot Mom
\end{equation}

\subsubsection{Three-factor model of \textcite{fama1993common}}

Some intuition.


\subsection{The ``Lethal" Combination}

\textcite{fama2015five, fama2016choosing, fama2017international} show that portfolios with 
returns that behave like those of small, low profit but somehow aggressively investing stocks 
are a problem.

\subsection{Intuition vs. Statistics}

\textcite{kozak2017interpreting} show that factor models composed of a small number of 
principal-component (PC) factors performs similarly to our intuition-based factors in 
regressions on the returns used to extract the components.
This is a novel approach but it is harder to give PC-factor loadings intuitive meaning.
\textcite{kozak2017interpreting} remark that the second and third components extracted from 25 
size-value portfolios are similar to the size and value factors.
We can only make this comparison because we have the intuition-based factors to compare with.
Furthermore, it would be hard to improve the maximum Sharpe ratio of a set of factors in this 
fashion.
Completely statistics-based factors encourage data-mining.

\subsection{Assorted Ramblings}

The consumption models performs poorly during significant downturns.
Momentum experiences crashes (\parencite{daniel2016momentum}).

``Anomalous" portfolios are important.
If nothing else, investors make gazillions of dollars each year from trading 
on them.
Nevertheless, ``anomalous" portfolios should not be the driving force behind 
discovering or updating factors.
The most effective defense against data-mining (since finance resolutely 
refuses to embrace replication studies) is intuition-based models.

How $1=E(mR^i)$ let's us say things about a model's performance pricing all 
assets.
Hansen, $a'Ia$ is nice because it preserves stuff but
$a'\Omega a$ can be nicer for asset pricing purposes (covariances of errors 
are important for asset-pricing models because this influences the maximum 
Sharpe ratio that can be obtained from the errors).

Table \ref{tbl:Sh2} shows the maximum squared Sharpe ratios and marginal 
contributions from each of the factors for my model and that of 
\textcite{fama2016choosing}.
I include the missing investment and momentum factors where applicable to 
highlight marginal contributions to the maximum Sharpe ratio.
My model's maximum squared Sharpe ratio, 0.348, exceeds that of FF2016, 0.246, 
by 0.102.
For my model, value and cash profitability have similar marginal 
contributions to the Sharpe ratio of 0.2.
The market and momentum contribute 0.1 while size contributes 0.02.
The investment factor contributes very little, 0.003, as investment's effects 
are absorbed by the combination of value and momentum.
Popular asset-pricing models\footnote{Such as the three-factor model of 
  \textcite{fama1993common}, the four factor model of 
  \textcite{hou2015digesting}, the 
  five-factor model of \textcite{fama2015five} and the cash profitability 
  five-factor model of \textcite{fama2016choosing}} omit the momentum factor, 
despite 
momentum's persistence as an anomaly (\parencite{fama2016dissecting, 
  fama2017international}).
Including momentum in model FF2016 increases the Sharpe ratio from 0.246 
to 0.26.
Momentum contributes a comparable amount, 0.014, to the Sharpe ratio 
of FF2016 as the size factor, 0.017.

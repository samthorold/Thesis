% !TeX root=../main.tex

\subsection{Investment explained by value and momentum}

\import{./Tables/}{Inv_fmt}

Table \ref{tbl:Inv} shows the investment factor regressed on other factors. Regression 1
shows the combination of value and momentum explain the average return on the investment
factor. The intercept, $a$, is insignificant with a t-statistic of -0.66. Both value and
momentum slopes are positive and highly significant with coefficients of 0.5 and 0.25
respectively. Regression 2 tells a similar story for annually rebalanced value and
momentum. An insignificant intercept means that including an investment factor does not
improve our B2016 model \parencite{fama1998determining, barillas2016alpha,
fama2016choosing}. Using the logic from table \ref{tbl:Table1}, an insignificant intercept
means the contribution, $(a/\sigma)^2$, to the maximum squared Sharpe ratio of a model can
only be zero.

Regressions 3, 4 and 5 show that none of monthly value, annual value or momentum explain
the returns on the investment factor on their own. Monthly value is highly significant
with a slope of 0.31 but still leaves 0.2\% average monthly return unexplained. Annual
value is also highly significant with a slope of 0.49 and leaving only 0.13\% unexplained
return. Momentum is insignificant and leaves the total 0.31\% average monthly return from
table \ref{tbl:Table1} unexplained. Regression 6 shows that the market and profitability
factors increase the unexplained average return of the investment factor. The market and
profitability are negatively correlated with investment and positively correlated with
each other resulting in an increase in the unexplained average return. Regression 7 shows
that annual value and momentum are not robust to including other factors. Investment's
negative correlation with the market and profitability means the intercept, 0.12, is
significant in regression 7. The t-statistic of 1.99 means this result is far from the
higher thresholds that \textcite{harvey2016and} call for. Nevertheless, the power of
monthly value and momentum to explain the returns on the investment factor is greater than
the power of annual value and momentum. Monthly value and momentum have similar slopes
when used in conjunction with all of the factors from B2016 to explain the average return
on the investment factor. In a statistical sense, the combination of monthly value and
momentum is a slam dunk because of the slope enhancement described in
\textcite{fama2015incremental}.

We are more interested in the economics behind the combination of value and momentum
explaining the returns on the investment factor because of the problems with data-mining
outlined in section \ref{sec:Literature}. Equation \ref{eq:ddm} uses the dividend discount
model to show that changes in book equity (BE) may be related to stock returns. Beginning
with \textcite{fama2006profitability}, changes in total assets are used in place of
changes in book equity to proxy for ``investment". \textcite{kok2017facts} and
\textcite{asness2013devil} show that value and momentum combine to identify stocks that
will exhibit changes in BE in the future. Value stocks often exhibit decreases in book
equity while high momentum stocks typically exhibit fewer of these decreases as there is
less bad news to incorporate into book values in the future. High value, past winners
correspond with conservatively investing stocks -- they indicate lower changes in BE.

The slope on the annual value factor provides some resistence to this theory. If high
value is an indicator of future changes in BE then the slope in regression 4 should be
negative. The returns on a portfolio long low investment stocks should have a negative
correlation with a portfolio long high value stocks. The summary characteristics in sorts
on size, value and momentum in table \ref{tbl:32_Size_BM_Prior_Sorts} also suggest this
relationship may not be true -- future changes in BE are lowest for low-value stocks.

\newgeometry{margin=.25in}
\afterpage{
\clearpage
\begin{landscape}
\import{./Tables/}{ValMomSort_fmt}
\end{landscape}
\clearpage
}
\restoregeometry

Table \ref{tbl:32_Size_BM_Prior_Sorts} shows summary statistics from sorts on size, value
and momentum. The left hand side uses the monthly rebalanced value factor, $HML^m$ while
the right hand side uses the annually rebalanced value factor, $HML$. Comparing the two
sorts highlights the differences in other characteristics caused by the alternative
specifications of value.

For monthly value, returns increase across value and momentum quartiles. Although, for
high-momentum big stocks the value effect is less convincing. While for annual value,
returns are more flat across value quartiles in the high momentum quartile and, for big
stocks, more flat across momentum quartiles in the high value quartile. In sorts on
monthly value, annual value is increasing with increasing momentum. The spread in annual
value increases as monthly value increases. Holding the monthly value quartile constant, a
stock must have much higher relative value to move through the momentum quartiles. As
momentum increases, the monthly value measure will decrease due to the increasing market
equity in the denominator of B/M. Annual value is not updated each month and so those
stocks in the high-value and high-momentum portfolio have much higher annual value than
those in the same value quartile with lower momentum. For annual value we see the opposite
effect. Those stocks in the high-value and high-momentum portfolio are held in place by
the annually rebalanced value. Even though we observe the decrease in monthly value driven
by high-momentum, the same stocks are held in the intersection of the high portfolios by
their previously high value.

In both specifications of value, momentum is flat across value quartiles. With the
exception of the highest momentum quartile, momentum characteristics are similar for small
and big stocks.

Investment characteristics are similar across both specifications of value for small and
big stocks. Investment is decreasing as value increases with the exception of the low
momentum quartile for big stocks which is increasing. Investment does not vary much with
momentum although the lowest momentum quartile tends to have slightly higher levels of
investment. The similar investment characteristics between specifications of value is
consistent with the ability to explain returns on investment of both monthly and annual
value combined with momentum. The strong interaction of monthly value and momentum means
they remain robust to including other factors.

Table \ref{tbl:32_Size_BM_Prior_Sorts} includes summaries for past and future changes in
BE. This is necessary because I pin much of the economic intuition behind value and
momentum explaining the returns on the investment factor on their ability to forecast
changes in BE. The relationship is not as described by \textcite{kok2017facts}. I find
that small, low-value stocks exhibit decreases in future BE, while they find high-value
stocks exhibit decreases in BE. There is no clear pattern across momentum quartiles. This
is not wholly unsurprising as the relationship between momentum and future changes in BE
reported in \textcite{asness2013devil} significant but weak in magnitude. The summary
statistics for future BE are not a total disaster. For big stocks, increases in future BE
decrease as value increases. This is in-line with the returns on high-value stocks
corresponding with the returns on conservatively investing stocks, if we continue to
accept that both factors are proxying for changes in BE.

In light of the evidence in tables \ref{tbl:Inv} and \ref{tbl:32_Size_BM_Prior_Sorts}, we
should keep in mind that the dividend discount model is a lense through which to view
statistical evidence from regressions. Factors are proxies for some true priced state
variable. More than this, the factors I use are all \emph{mimicking} portfolios for the
unknown state variables. It is possible that changes in BE have nothing to do with
investment, value or momentum. We cannot get too carried away when trying to fit economic
stories to statistical evidence. The economic intuition of value and momentum providing a
better proxy for an unknown state variable, changes in BE in the dividend discount model,
coupled with statistical evidence is strong enough to not be thrown off by a positive
regression slope between investment and annual value. Furthermore, characteristic
summaries from unconditional sorts on other characteristics are not the same as
conditional regression slopes. It is not a contradiction to say that changes in BE are
increasing in sorts on value while the returns on high-value stocks behave like the
returns on stocks with low investment.

% !TeX root=../main.tex

\subsection{Value and Momentum}

\import{./Tables/}{Inv_fmt}

Table \ref{tbl:Inv} shows the investment factor regressed on other factors. Regression 1
shows the combination of value and momentum explain the average return on the investment
facor. The intercept, $a$, is insignificant with a t-statistic of -0.66. Both value and
momentum slopes are positive and highly significant with coefficients of 0.5 and 0.25
respectively. An insignificant intercept means that including an investment factor does
not improve our BS2015 model \parencite{fama1998determining, barillas2016alpha,
fama2016choosing}. Using the logic from table \ref{tbl:Table1}, an insignificant intercept
means the contribution, $(a/\sigma)^2$, to the maximum squared Sharpe ratio of a model can
only be zero.

Regressions 2 and 3 show that neither value nor momentum alone explain the returns on the
investment factor. Value is highly significant with a slope of 0.31 but still leaves 0.2\%
average monthly return unexplained. Momentum is insignificant and leaves the total 0.31\%
average monthly return from table \ref{tbl:Table1} unexplained. Regression 4 shows that
the market and profitability factors increase the unexplained average return of the
investment factor. The market and profitability are negatively correlated with investment
and positively correlated with each other resulting in an increase in the unexplained
average return. Regressions 5 and 6 show that neither value nor investment alone combine
with other factors to explain the average return on the investment factor. This highlights
the unique relationship between value and momentum. The two factors are positively
correlated with many other factors and anomalies but negatively correlated with each
other. Regression 7 shows that value and momentum have similar slopes when used in
conjunction with all of the factors from BS2015 to explain the average return on the
investment factor. In a statistical sense, the combination of value and momentum is a slam
dunk because of the slope enhancement described in \textcite{fama2015incremental}.

We are more interested in the economics behind the combination of value and momentum
explaining the returns on the investment factor because of the problems with data-mining
outlined in section \ref{sec:Literature}. Equation \ref{eq:ddm} uses the dividend discount
model to show that changes in book equity (BE) may be related to stock returns. Typically,
changes in total assets are used in place of changes in book equity, beginning with
\parencite{fama2006profitability}. \textcite{kok2017facts} and \textcite{asness2013devil}
show that value and momentum combine to identify stocks that will exhibit changes in BE in
the future. Value stocks often exhibit decreases in book equity while high momentum stocks
typically exhibit fewer of these decreases as there is less bad news to incorporate into
book values in the future. Despite this, both regression slopes are positive. The returns
on the investment factor behaves like the returns on high momentum, value stocks. Rather
than being contradictory, we are identifying stocks that will likely have decreases in the
future by using value and momentum to identify those that will not - high momentum, value
stocks. This is similar to finding out if a coin is showing heads by asking if it is
showing tails.



Value-weighted characteristics in sorts. Average characteristics in size sorts. Spanning
regressions showing it is value and momentum that kill investment. Investment is change in
$BE$ from equation (\ref{eq:ddm}). Economics behind correlation/slope attenuation:
\textcite{asness2013devil} show that momentum  can forecast changes in $BE$.

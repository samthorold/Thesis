% !TeX root=../main.tex

\subsection{Investment explained by value and momentum}

\import{./Tables/}{Inv_fmt}

Table \ref{tbl:Inv} shows the investment factor regressed on other factors. Regression 1
shows the combination of value and momentum explain the average return on the investment
factor. The intercept, $a$, is insignificant with a t-statistic of -0.66. Both value and
momentum slopes are positive and highly significant with coefficients of 0.5 and 0.25
respectively. Regression 2 tells a similar story for annually rebalanced value and
momentum. An insignificant intercept means that including an investment factor does not
improve our B2016 model \parencite{fama1998determining, barillas2016alpha,
fama2016choosing}. Using the logic from table \ref{tbl:Table1}, an insignificant intercept
means the contribution, $(a/\sigma)^2$, to the maximum squared Sharpe ratio of a model can
only be zero.

Regressions 3, 4 and 5 show that none of monthly value, annual value or momentum explain
the returns on the investment factor on their own. Monthly value is highly significant
with a slope of 0.31 but still leaves 0.2\% average monthly return unexplained. Annual
value is also highly significant with a slope of 0.49 and leaving only 0.13\% unexplained
return. Momentum is insignificant and leaves the total 0.31\% average monthly return from
table \ref{tbl:Table1} unexplained. Regression 6 shows that the market and profitability
factors increase the unexplained average return of the investment factor. The market and
profitability are negatively correlated with investment and positively correlated with
each other resulting in an increase in the unexplained average return. Regression 7 shows
that annual value and momentum are not robust to including other factors. Investment's
negative correlation with the market and profitability means the intercept, 0.12, is
significant in regression 7. The t-statistic of 1.99 means this result is far from the
higher thresholds that \textcite{harvey2016and} call for. Nevertheless, the power of
monthly value and momentum to explain the returns on the investment factor is greater than
the power of annual value and momentum. Monthly value and momentum have similar slopes
when used in conjunction with all of the factors from B2016 to explain the average return
on the investment factor. In a statistical sense, the combination of monthly value and
momentum is a slam dunk because of the slope enhancement described in
\textcite{fama2015incremental}.

We are more interested in the economics behind the combination of value and momentum
explaining the returns on the investment factor because of the problems with data-mining
outlined in section \ref{sec:Literature}. Equation \ref{eq:ddm} uses the dividend discount
model to show that changes in book equity (BE) may be related to stock returns. Beginning
with \textcite{fama2006profitability}, changes in total assets are used in place of
changes in book equity to proxy for ``investment". \textcite{kok2017facts} and
\textcite{asness2013devil} show that value and momentum combine to identify stocks that
will exhibit changes in BE in the future. Value stocks often exhibit decreases in book
equity while high momentum stocks typically exhibit fewer of these decreases as there is
less bad news to incorporate into book values in the future. High value, past winners
correspond with conservatively investing stocks -- they indicate lower changes in BE.

The slope on the annual value factor provides some resistence to this theory. If high
value is an indicator of future changes in BE then the slope in regression 4 should be
negative. The returns on a portfolio long low investment stocks should have a negative
correlation with a portfolio long high value stocks. The summary characteristics in sorts
on size, value and momentum in table \ref{tbl:32_Size_BM_Prior_Sorts} also suggest this
relationship may not be true -- future changes in BE are lowest for low-value stocks.

\afterpage{\clearpage
\begin{landscape}

% https://tex.stackexchange.com/questions/47311/include-table-as-a-subfigure
\begin{table}%
  \label{tbl:chars}%
  \scriptsize
  \centering
  \caption{Average value-weighted characteristics -- 1963-07 through 2016-12}%
  \subfloat[][$Size$-$BM^m$-$Prior$\label{tbl:valmomsort}]{\import{./Tables/}{ValMomSort}}%
  \qquad
  \subfloat[][$Size$-$BM$-$Inv$\label{tbl:valinvsort}]{\import{./Tables/}{AnnualValInvSort}}%
\end{table}

\end{landscape}
\clearpage}

Table \ref{tbl:valmomsort} shows value-weighted characteristics from sorts on size,
monthly rebalanced value and momentum. Returns are increasing across value and momentum
quartiles. The largest spreads across value are in the low momentum quartile while the
largest spreads across momentum are in the low value quartile. This matches results from
\textcite{asness1997interaction} who finds that the momentum effect is strongest among
stocks with low value and vice versa. The value spreads are smaller among big stocks than
small stocks. This matches results from \textcite{asness2015fact} and
\textcite{kok2017facts}. Monthly value increases across value quartiles but remains flat
across momentum quartiles. The spreads in monthly value characteristics are smaller for
big stocks. Annual value is increasing across value and momentum quartiles. Consider a
stock that is in the high-value, low-momentum bucket in July, this is when the monthly and
annual value measures are most similar in terms of measurement date and therefore probably
the most similar during the holding period. Assume the stock now exhibits high returns
relative to other stocks. The stock will move up through the momentum quartiles each
month. Monthly value will decrease as the high returns lead to an increase in market
equity while annual value will remain constant. Those stocks that remain in the high-value
quartile must have had very high value at the beginning of the period to have high value
relative to other stocks after an increase in market equity. This very high beginning
value is seen in annual value measure. Momentum increases across momentum quartiles and
remains flat across value quartiles, except for the high-momentum quartile of small stocks
which decreases slightly across value quartiles. Investment, Fama and French's proxy for
increases in BE, decreases slightly across momentum quartiles. This matches
\textcite{fama2016dissecting} but offers some resistence to \textcite{asness2013devil} who
find that momentum from up to three years ago forecasts increases in BE. Investment
decreases across value quartiles. This makes sense given the positive correlation between
$CMA$ and $HML^m$. Future changes in BE do not vary across momentum quartiles. For small
stocks, the lowest value quartile is interesting because the future changes in BE for
these stocks is strongly negative. Compared to the lowest value quartile, the other
quartiles show litle variation. For big stocks the relationship is reversed, low-value
stocks have higher future changes in BE.

Table \ref{tbl:valinvsort} shows value-weighted characteristics from sorts on size,
annually rebalanced value and investment. We can compare the characteristics of stocks
sorted on value and momentum with those sorted on value and investment. Returns are
increasing across value quartiles. The value spreads are smaller among big stocks than
small stocks but spreads do not change across investment quartiles. Returns decrease
across investment quartiles. Those stocks with relatively lower asset growth, on average,
make higher returns. The spreads in returns across investment quartiles are smaller than
the spreads in returns across value quartiles. Both monthly and annual value increase
across value quartiles but remain flat across investment quartiles, except for small
stocks with low or high value. The spreads in monthly value characteristics are similar
for small and big stocks. For small and big stocks, momentum exhibits a slight decrease
across value quartiles but otherwise is flat across value and investment. Investment
increases across investment quartiles. For big stocks, investment characteristics in the
high-investment quartile increase across value quartiles. There are differences in future
changes in BE compared to the sorts on size, value and momentum. For small stocks, future
changes in BE are decreasing across value quartiles. This relationship is the opposite to
that seen in table \ref{tbl:valmomsort}. For big stocks, this relationship is only
noticeable in the low-investment quartile.

In light of the evidence in tables \ref{tbl:Inv}, \ref{tbl:valmomsort} and
\ref{tbl:valinvsort}, we should keep in mind that the dividend discount model is a lense
through which to view statistical evidence from regressions. Factors are proxies for some
true priced state variable. More than this, the factors I use are all \emph{mimicking}
portfolios for the unknown state variables. It is possible that changes in BE have nothing
to do with investment, value or momentum. We cannot get too carried away when trying to
fit economic stories to statistical evidence. The economic intuition of value and momentum
providing a better proxy for an unknown state variable, changes in BE in the dividend
discount model, coupled with statistical evidence is strong enough to not be thrown off by
a positive regression slope between investment and annual value nor counterintuitive
results from crude sorts on characteristics. Furthermore, characteristic summaries from
unconditional sorts on other characteristics are not the same as conditional regression
slopes. It is not a contradiction to say that changes in BE are increasing in sorts on
value while the returns on high-value stocks behave like the returns on stocks with low
investment.

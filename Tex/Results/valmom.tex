% !TeX root=../main.tex

\subsection{Value and Momentum}

\textbf{VW Characteristics from sorts on Size-Value-Momentum, esp. $\delta BE$.}

\import{./Tables/}{Inv_fmt}

Table \ref{tbl:Inv} shows the investment factor regressed on other factors. Regression 1
shows the combination of value and momentum explain the average return on the investment
facor. The intercept, $a$, is insignificant with a t-statistic of -0.66. Both value and
momentum slopes are positive and highly significant with coefficients of 0.5 and 0.25
respectively. Regression 2 tells a similar story for annually rebalanced value and
momentum. An insignificant intercept means that including an investment factor does not
improve our B2016 model \parencite{fama1998determining, barillas2016alpha,
fama2016choosing}. Using the logic from table \ref{tbl:Table1}, an insignificant intercept
means the contribution, $(a/\sigma)^2$, to the maximum squared Sharpe ratio of a model can
only be zero.

Regressions 3, 4 and 5 show that none of monthly value, annual value or momentum explain
the returns on the investment factor on their own. Monthly value is highly significant
with a slope of 0.31 but still leaves 0.2\% average monthly return unexplained. Annual
value is also highly significant with a slope of 0.49 and leaving only 0.13\% unexplained
return. Momentum is insignificant and leaves the total 0.31\% average monthly return from
table \ref{tbl:Table1} unexplained. Regression 6 shows that the market and profitability
factors increase the unexplained average return of the investment factor. The market and
profitability are negatively correlated with investment and positively correlated with
each other resulting in an increase in the unexplained average return. Regression 7 shows
that annual value and momentum are not robust to including other factors. Investments
negative correlation with the market and profitability means the intercept, 0.12, is
significant in regression 7. The t-statistic of 1.99 means this result is far from the
higher thresholds that \textcite{harvey2016and} call for. Nevertheless, the power of
monthly value and momentum to explain the returns on the investment factor is greater than
the power of annual value and momentum. Monthly value and momentum have similar slopes
when used in conjunction with all of the factors from B2016 to explain the average return
on the investment factor. In a statistical sense, the combination of monthly value and
momentum is a slam dunk because of the slope enhancement described in
\textcite{fama2015incremental}.

We are more interested in the economics behind the combination of value and momentum
explaining the returns on the investment factor because of the problems with data-mining
outlined in section \ref{sec:Literature}. Equation \ref{eq:ddm} uses the dividend discount
model to show that changes in book equity (BE) may be related to stock returns. Beginning
with \parencite{fama2006profitability}, changes in total assets are used in place of
changes in book equity. \textcite{kok2017facts} and \textcite{asness2013devil} show that
value and momentum combine to identify stocks that will exhibit changes in BE in the
future. Value stocks often exhibit decreases in book equity while high momentum stocks
typically exhibit fewer of these decreases as there is less bad news to incorporate into
book values in the future. High value, past winners correspond with conservatively
investing stocks -- they indicate lower changes in BE. The slope on the annual value
factor provides some resistence to this theory. If high value is an indicator of future
changes in BE then the slope in regression 4 should be negative. The returns on a
portfolio long low investment stocks should have a negative correlation with a portfolio
long high value stocks. We should keep in mind that the dividend discount model is a lense
through which to view statistical evidence from regressions. Factors are proxies for some
true priced state variable. More than this, the factors I use are all \emph{mimicking}
portfolios for the unknown state variables. It is possible that changes in BE have nothing
to do with investment, value or momentum. We cannot get too carried away when trying to
fit economic stories to statistical evidence. The economic intuition of value and momentum
providing a better proxy for an unknown stat variable, changes in BE in the dividend
discount model, coupled with statistical evidence is strong enough to not be thrown off by
a positive regression slope between investment and annual value.


\textbf{Value-weighted characteristics in sorts. Average characteristics in size sorts.}

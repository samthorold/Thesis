% !TeX root=../main.tex

\subsection{Sharpe Ratio}

\import{./Tables/}{Table1}
Table \ref{tbl:Table1} panel A shows summary statistics for individual factors as well as
the  B2016 and FF2016 models. The market has a high average monthly return, 0.51\%,
offset by a high standard deviation,  4.42, giving a modest Sharpe ratio of 0.12. Size,
value and investment factors had similar average monthly returns and standard deviations
giving Sharpe ratios from 0.09 to 0.15. Momentum had the highest average monthly return at
0.66\%. This does not translate to a significantly higher Sharpe ratio, 0.16, as the
standard  deviation is also high at 4.23. Cash profitability has the highest Sharpe ratio
of the individual factors at 0.27. This is driven by the low standard deviation of 1.4
rather than a high average return, which  is modest at 0.37\%. The models do not improve
the average monthly return, 0.42\% for B2016 and 0.37\% for  FF2016, but do lower the
standard deviation of returns to 0.69 and 0.75. This results in significantly improved
Sharpe ratios of 0.6 and 0.5. Given the unimpressive monthly returns, we should keep in
mind that models most  appropriate for pricing may not be the most appropriate for
investing  \parencite{pastor2000comparing}. We are concerned with the way our models co-
vary with all assets, not whether they  generate superior returns.

Table \ref{tbl:Table1} panel B shows the intercepts, standard deviation of the residuals,
contribution to the maximum squared Sharpe ratio and the weight in the models for each
factor. For B2016, value and profit factors contribute similar amounts, 0.22 and 0.2, to
the Sharpe  ratio. This is in-line with our intuition that fairly-priced, profitable
stocks will generate higher  returns those relatively more expensive or less profitable.
Value's large contribution is driven primarily by a large intercept, 1.11, while
profitability's contribution is driven by a low standard deviation of residuals, 1.16. The
market and momentum also have large intercepts, 1.32 and  1.11, but the higher standard
deviation of residuals, 3.79 and 3.11, dampen the factors' contributions to the B2016
model,  0.11 and 0.13. Size contributes very little to the Sharpe ratio, 0.04. Despite
value and profitability's similar contributions to the Sharpe ratio of the B2016  model,
the profit factor has a much higher weight than value, 0.45 compared to 0.23, in the
portfolio which generates the maximum Sharpe ratio. Other factor weights are in-line with
contributions to Sharpe ratio. For FF2016, the market intercept is 1.09 but the high
standard deviation of residuals, 3.76,  drives a low contribution of 0.08. Despite
investment's low standard deviation of residuals, 1.38, its low intercept, 0.14,  drives a
negligible contribution of 0.01. Size and value contribute similar amounts, 0.03 and 0.04,
thanks to modest intercepts, 0.47  and 0.36, coupled with modest standard deviations of
residuals of 2.83 and 1.9. Profitability contributes the most at 0.19 driven by a low
standard deviation of residuals at  1.19. The profitability factor has the largest weight,
0.54, in the portfolio which maximizes the  Sharpe ratio of FF2016. Remaining factors are
all weighted between 0.09 and 0.15.

Factor contributions given the other four factors need not sum to the total model squared
Sharpe ratio because $(a/\sigma)^2$ is the amount the squared Sharpe ratio of the four
other  factors is improved when the fifth is included. A factor's contribution depends on
the amount of its mean return, $a$, and variation,  $\sigma$, that is left unexplained by
the other factors. Unexplained mean return is the traditional alpha from a time series
regression while the  unexplained variation is the standard deviation of the residuals
from the same regression. The unexplained mean return represents the potential to add
information to the other factors  in a model while the unexplained variation dampens the
effect of unexplained factor return on  the model's Sharpe ratio
\parencite{fama2016choosing}.

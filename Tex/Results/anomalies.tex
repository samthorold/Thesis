% !TeX root=../main.tex

\subsection{Anomalies}

Once we have used the Sharpe ratio to identify our preferred model, anomaly regressions
can provide useful insights into specific cases where the model performs poorly. The
p-value of the GRS statistic of \textcite{gibbons1989test} gives a probability that all of
the test assets jointly have zero alphas. Almost always, the GRS p-value rejects the
hypothesis that all alphas are jointly zero. I focus on the GRS itself rather than the
p-value because the GRS gives a measure of how much the factors can be improved by
including the test assets in the model - this is more useful than finding every model is
wrong. B2016 omits the investment factor so first we turn to sorts on size and investment
to make sure we have not lost any performance compared with the FF2016 model. To check
that the poor perfomance of value in large stocks \parencite{asness2015fact}, I sort on
size-value and size-B/M-investment. Sorts on volatility \parencite{ang2006cross} and
momentum \parencite{jegadeesh1993returns} have been shown to cause problems for the FF2016
model \parencite{fama2016dissecting}. I check the performance of the B2016 model when
explaining returns on these two anomalies. As well as highlighting deficiencies in
individual portfolios, anomaly regressions will show where value and momentum slopes of
B2016 differ from the investment slopes of FF2016.

\import{./Results/Anomalies/}{GRS}

\import{./Results/Anomalies/}{Inv}

%\import{./Results/Anomalies/}{Beta}

\import{./Results/Anomalies/}{Var}

\import{./Results/Anomalies/}{Prior}

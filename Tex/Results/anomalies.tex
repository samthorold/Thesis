% !TeX root=../main.tex

\subsection{Anomalies}

\subsubsection{Investment}

\import{./Tables/}{25_Size_Inv_196307_201612}
We have established that the BS2015 model has a higher Sharpe ratio than the FF2016b model
but  it is informative to see if we lose any performance in sorts on the excluded
investment factor. Table \ref{tbl:25_Size_Inv} shows the performance of the FF2016b and
BS2015 models when  explaining returns on size and investment sorts. BS2015 leaves fewer
alphas than FF2016b\footnote{BS2015 also leaves fewer alphas than the  FF2015 and FF2016a
models (see table \ref{tbl:25_Size_Inv_b} in appendix  \ref{sec:addl_results}).}. FF2016b
struggles with the third investment quintile leaving alphas in the first and second  size
quintiles, both 0.14. BS2015 struggles where the highest investment quintile meets the
fourth highest size quintile,  leaving an alpha of 0.18. Both models leave an alpha the
small-high investment portfolio. The values are -0.28 for FF2016b and -0.26 for BS2015.

Table \ref{tbl:25_Size_Inv} also shows the regression slopes on the value and investment
or  value and momentum factors. For the FF2016b model, the investment slopes, $cma$,
decrease as the investment quintiles  increase. We expect the returns on portfolios with
higher investment characteristics to behave like the  returns on the factor constructed
from a more coarse sort on the same variable. The spreads between the investment slopes
increase as the size quintiles increase. The small quintile sees the slope fall from 0.47
for the lowest investment quintile to -0.24  for the highest. The big size quintile has
investment slopes from 0.77 for the low quintile to -0.7 for the  highest. Value is fairly
uniform across size and investment quintiles except for the negative slopes in  the
highest investment quintile. For the BS2015 model, the value slopes are uniform for the
smallest three investment  quintiles, except for the small-low investment intersection
which is not significant. The slope becomes negative in the largest investment quintile.
For the larger size quintiles this includes a gradual shift through the fourth largest
investment quintile. This suggests that returns on firms with high investment behave like
those with a low  book-to-market ratio, growth firms. Despite momentum's small slopes, the
factor is highly significant is almost all portfolios. Mirroring value, slopes are uniform
for all investment quintiles except for the largest, where  slopes become negative.
Negative momentum slopes suggest the returns on firms with large investments behave like
those  with lower returns over the previous year.

\subsubsection{Volatility}

\import{./Tables/}{25_Size_Vol_BS2015_196307_201612}
Table \ref{tbl:25_Size_Vol_BS2015} shows the performance of the BS2015 model when
explaining returns on size and volatility sorts. Alphas are left in all of the small stock
portfolios except for the fourth. For the smaller portfolios the alphas range from 0.26 to
0.18. Value loadings are positive for these portfolios while momentum and profit
contribute very little. The highest volatility portfolio has an unexplained average
monthly return of -0.79. This tells us the BS2015 model considerably overestimates the
returns to small-high volatility stocks. Value is insignificant while momentum and profit
loadings are negative. This suggests the returns on small-high volatility stocks behave
like those of low profit stocks with poor relative returns over the previous year.

In general, value decreases with volatility quintile. The biggest size quintile has the
smallest spread in slopes. This makes sense as the value effect is weakest among larger
stocks \parencite{asness2015fact}. Momentum and profit slopes are similar. Magnitude is
low or insignificant for all volatility quintiles except the highest, which has negative
slopes for both factors. The momentum slope in the small-high volatility portfolio is
-0.41. Typically, the magnitude of momentum slopes is not this high and suggests strong
comovement between the returns of small-high volatility stocks and past losers. The
momentum slope, -0.41, is much higher than that of the investment factor of the FF2016b
model (See table \ref{tbl:25_Size_Vol_FF2016b} in appendix \ref{sec:addl_results}). The
profit slope, -0.65, in the FF2016b model is higher the profit slope in the BS2015 model,
-0.3. Value is insignificant in both models. Momentum alone seems to capture the
information in investment and profitability. This is puzzling because value is
insignificant in both models suggesting it is not the interaction between value and
momentum, as in the case of the investment sorts, that is causing the differences in
slopes. This could be a topic for future research.

While the BS2015 model improves on the performance of the FF2016b model (See table
\ref{tbl:25_Size_Vol_FF2016b} in appendix \ref{sec:addl_results}) when explaining returns
to stocks sorted by size and volatility, the alpha of -0.79 for small-high volatility
stocks is a large.


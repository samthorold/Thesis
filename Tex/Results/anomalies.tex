% !TeX root=../main.tex

\subsection{Anomalies}

\subsubsection{Investment}

\import{./Tables/}{25_Size_Inv_196307_201612}
We have established that the BS2015 model has a higher Sharpe ratio than the FF2016b model but 
it is informative to see if we lose any performance in sorts on the excluded investment factor.
Table \ref{tbl:25_Size_Inv} shows the performance of the FF2016b and BS2015 models when 
explaining returns on size and investment sorts.
BS2015 leaves fewer alphas than FF2016b\footnote{BS2015 also leaves fewer alphas than the 
FF2015 and FF2016a models (see table \ref{tbl:25_Size_Inv_b} in appendix 
\ref{sec:addl_results}).}.
FF2016b struggles with the third investment quintile leaving alphas in the first and second 
size quintiles, both 0.14.
BS2015 struggles where the highest investment quintile meets the fourth highest size quintile, 
leaving an alpha of 0.18.
Both models leave an alpha the small-high investment portfolio.
The values are -0.28 for FF2016b and -0.26 for BS2015.

Table \ref{tbl:25_Size_Inv} also shows the regression slopes on the value and investment or 
value and momentum factors.
For the FF2016b model, the investment slopes, $cma$, decrease as the investment quintiles 
increase.
We expect the returns on portfolios with higher investment characteristics to behave like the 
returns on the factor constructed from a more coarse sort on the same variable.
The spreads between the investment slopes increase as the size quintiles increase.
The small quintile sees the slope fall from 0.47 for the lowest investment quintile to -0.24 
for the highest.
The big size quintile has investment slopes from 0.77 for the low quintile to -0.7 for the 
highest.
Value is fairly uniform across size and investment quintiles except for the negative slopes in 
the highest investment quintile.
For the BS2015 model, the value slopes are uniform for the smallest three investment 
quintiles, except for the small-low investment intersection which is not significant.
The slope becomes negative in the largest investment quintile.
For the larger size quintiles this includes a gradual shift through the fourth largest 
investment quintile.
This suggests that returns on firms with high investment behave like those with a low 
book-to-market ratio, growth firms.
Despite momentum's small slopes, the factor is highly significant is almost all portfolios.
Mirroring value, slopes are uniform for all investment quintiles except for the largest, where 
slopes become negative.
Negative momentum slopes suggest the returns on firms with large investments behave like those 
with lower returns over the previous year.

\subsubsection{Volatility}

\import{./Tables/}{25_Size_Vol_BS2015_196307_201612}
Table \ref{tbl:25_Size_Vol_BS2015} shows the performance of the FF2016b and BS2015 models
when explaining returns on size and investment sorts.


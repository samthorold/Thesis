% !TeX root=../main.tex

\subsection{Anomalies}

Once we have used the Sharpe ratio to identify our preferred model, anomaly regressions
can provide useful insights into specific cases where the model performs poorly. I have
omitted the investment factor so first we turn to sorts on size and investment to make
sure we have not lost any performance compared with the FF2016b model. Sorts on volatility
\parencite{ang2006cross} and momentum \parencite{jegadeesh1993returns} have been show to
cause problems for the FF2016b model \parencite{fama2016dissecting}. With this in mind, we
also check the performance of the BS2015 model when explaining returns on these two
anomalies. As well as highlighting deficiencies in individual portfolios, anomaly
regressions will show where value and momentum slopes of BS2015 differ from the investment
slopes of FF2016b.

\import{./Tables/}{GRS_196307_201612}

\subsubsection{Investment}

\import{./Tables/}{25_Size_Inv_BS2015_196307_201612}

Table \ref{tbl:25_Size_Inv_BS2015} shows the performance of the BS2015
model when  explaining returns on size and investment sorts. BS2015 leaves fewer alphas
than FF2016b, see table \ref{tbl:25_Size_Inv_FF2016b} in appendix \ref{sec:addl_results}),
supporting the view that value and momentum are a better proxy for changes in book equity.
BS2015 struggles where the highest investment quintile meets the fourth highest size
quintile, leaving an alpha of 0.18. Both models leave an alpha the small-high investment
portfolio. The values are -0.28 for FF2016b and -0.26 for BS2015. FF2016b struggles with
the third investment quintile leaving alphas in the first and second  size quintiles, both
0.14.

Table \ref{tbl:25_Size_Inv_BS2015} also shows the regression slopes on value and momentum
factors. For the BS2015 model, the value slopes are uniform for the smallest three
investment  quintiles, except for the small-low investment intersection which is not
significant. The slope becomes negative in the largest investment quintile. For the larger
size quintiles this includes a gradual shift through the fourth largest investment
quintile. This suggests that returns on firms with high investment behave like those with
a low  book-to-market ratio, growth firms. Despite momentum's small slopes, the factor is
highly significant is almost all portfolios. Mirroring value, slopes are uniform for all
investment quintiles except for the largest, where  slopes become negative. Negative
momentum slopes suggest the returns on firms with large investments behave like those with
lower returns over the previous year.

\begin{landscape}
\import{./Tables/}{32_Size_OP_Inv_BS2015_196307_201612}
\end{landscape}

Table \ref{tbl:32_Size_OP_Inv_BS2015} shows the performance of the BS2015 model when
explaining returns on size, operating profitability and investment sorts. Small stocks
that invest aggresively despite weak profits remain a problem with an alpha of -0.33\%.
For small stocks, the only other alpha, 0.14\%, occurs in the second profitability and
third investment portfolio. For big stocks, both alphas are in the lowest profitability
quartile in the second, -0.26\%, and highest, 0.27\%, investment quartiles. The model
performs similarly for small and big stocks. This is somewhat unexpected given value's
poor performance in big stocks. Since value is taking up much of the slack left by
investment, we would expect omitting a value factor to be detrimental to the model. As in
the sorts on size and investment, BS2015 leaves fewer alphas than FF2016b, see table
\ref{tbl:25_Size_OP_Inv_FF2016b} in appendix \ref{sec:addl_results}).

For small stocks, the value slopes decrease with higher investment and increase with
profitability. Within profitability quartiles, the returns on high investment stocks
behave like those of stocks with lower B/M than low investment stocks. While the spread in
the value slopes remain close to 0.5, the lowest profiability quartile has much lower
value slopes than the three higher profitability quartiles.
Given that we are replacing investment with value and momentum we would expect the 

In the context of the divident
discount model (eq. \ref{eq:ddm}), there are a couple of possible explanations for the
negative relationship between the returns on high investment stocks and returns on high
B/M stocks. First,  high B/M stocks, on average, exhibit decreases in book equity
\parencite{kok2017facts} while investment is a proxy for increases in book equity. Second,
high B/M stocks, when combined with momentum, will not exhibit changes in book equity.

\subsubsection{Beta}

\import{./Tables/}{25_Size_Beta_BS2015_196307_201612}
\import{./Tables/}{25_Size_Beta_FF2015_196307_201612}
\import{./Tables/}{25_Size_Beta_FF2016a_196307_201612}
\import{./Tables/}{25_Size_Beta_FF2016b_196307_201612}

\subsubsection{Volatility}

\import{./Tables/}{25_Size_Vol_BS2015_196307_201612}
Table \ref{tbl:25_Size_Vol_BS2015} shows the performance of the BS2015 model when
explaining returns on size and volatility sorts. Alphas are left in all of the small stock
portfolios except for the fourth. For the smaller portfolios the alphas range from 0.26 to
0.18. Value loadings are positive for these portfolios while momentum and profit
contribute very little. The highest volatility portfolio has an unexplained average
monthly return of -0.79. This tells us the BS2015 model considerably overestimates the
returns to small-high volatility stocks. Value is insignificant while momentum and profit
loadings are negative. This suggests the returns on small-high volatility stocks behave
like those of low profit stocks with poor relative returns over the previous year.

In general, value decreases with volatility quintile. The biggest size quintile has the
smallest spread in slopes. This makes sense as the value effect is weakest among larger
stocks \parencite{asness2015fact}. Momentum and profit slopes are similar. Magnitude is
low or insignificant for all volatility quintiles except the highest, which has negative
slopes for both factors. The momentum slope in the small-high volatility portfolio is
-0.41. Typically, the magnitude of momentum slopes is not this high and suggests strong
comovement between the returns of small-high volatility stocks and past losers. The
momentum slope, -0.41, is much higher than that of the investment factor of the FF2016b
model (See table \ref{tbl:25_Size_Vol_FF2016b} in appendix \ref{sec:addl_results}). The
profit slope, -0.65, in the FF2016b model is higher the profit slope in the BS2015 model,
-0.3. Value is insignificant in both models. Momentum alone seems to capture the
information in investment and profitability. This is puzzling because value is
insignificant in both models suggesting it is not the interaction between value and
momentum, as in the case of the investment sorts, that is causing the differences in
slopes. This could be a topic for future research.

While the BS2015 model improves on the performance of the FF2016b model (See table
\ref{tbl:25_Size_Vol_FF2016b} in appendix \ref{sec:addl_results}) when explaining returns
to stocks sorted by size and volatility, the alpha of -0.79 for small-high volatility
stocks is a large.


\subsubsection{Momentum}

\import{./Tables/}{25_Size_Mom_BS2015_196307_201612}

Table \ref{tbl:25_Size_Mom_BS2015} shows the performance of the BS2015 model when
explaining returns on size and volatility sorts. The model struggles to explain the
average returns on the small quintile with alphas ranging from -0.41 in the lowest
momentum portfolio to 0.37 in the highest. The big-second lowest momentum portfolio also
poses a problem with an alpha of 0.24. This is an improvement over the FF2016b model (See
table \ref{tbl:25_Size_Mom_FF2016b} in appendix \ref{sec:addl_results}). While it is not
surprising that the BS2015 model leaves fewer alphas than the FF2016b model that does not
include a momentum factor, it is puzzling that the model performs poorly playing ``at
home." We would expect a model explaining finer sorts on one of the included factors to
explain more of the average returns.

The profitability slopes are small, rarely above 0.2 in magnitude, but significant in the
extreme momentum quintiles. All significant intercepts are negative suggesting the returns
on both low and high momentum stocks behave like the returns of stocks with low cash
profits. Unsurprisingly, the momentum slopes increase with the momentum quintiles. There
is a small increase in the maginitude of the slopes as the size quintile increases,
although the increase is around 0.2 in magnitude for all momentum quintiles. Momentum
slopes are most extreme in the lowest momentum quintile, -0.65 to -0.78. Value slopes are
low in the low and high momentum quintiles as well as the big size quintile. This is
puzzling because of the slope enhancement described in \textcite{fama2015incremental}.
When momentum slopes are extreme we would expect value slopes to be enhanced. Most
intriguingly, value is insignificant in the small-high momentum portfolio. Other value
slopes in the small size quintile are around 0.3. Value's insignificance seems to
contribute to the alpha of 0.37 in the small-high momentum portfolio. As high returns
increase the price of a stock, our monthly book-to-market ratio will decrease. High
momentum will flatten the value slopes in sorts on size and momentum limiting the
effectivesness of value. This is the variable attentuation effect described in
\textcite{fama2015incremental}.

While the BS2015 model may improve the overall performance in sorts on size and momentum,
the model's greatest strength may be its greatest weakness as the variable attentuation
from the interaction between value and momentum outweighs the slope enhancement.


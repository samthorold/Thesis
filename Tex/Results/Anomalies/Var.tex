% !TeX root=.../main.tex

\subsubsection{Volatility}

% \newgeometry{margin=.25in}
% \afterpage{\clearpage
% \begin{landscape}

% % https://tex.stackexchange.com/questions/47311/include-table-as-a-subfigure
% \begin{table}%
%   \label{tbl:vol}%
%   \scriptsize
%   \centering
%   \caption{25-Size-Var Portfolios -- 1963-07 through 2016-12}%
%   \subfloat[][B2016\label{tbl:25_Size_Var_B2016}]{\import{./Tables/}{25_Size_Var_B2016_fmt}}%
%   \qquad
%   \subfloat[][FF2016\label{tbl:25_Size_Var_FF2016}]{\import{./Tables/}{25_Size_Var_FF2016_fmt}}%
% \end{table}

% \end{landscape}
% \clearpage}
% \restoregeometry

The poor returns to high volatility stocks are related to the flatter than aniticipated
relationship between market beta and returns. The flat-beta anomaly was first noted by
\textcite{jensen1972capital} and futher documented by \textcite{fama1973risk}. Returns
with lower volatility have less covariance with the market. Since market beta is driven by
covariance with the market, the volatility and flat-beta anomalies are often used together
as part of a ``defensive equity" strategy \parencite{frazzini2014betting}. The volatility
portfolios from Dr. Kenneth French's website use conditional sorts on size and volatility.
NYSE volatility breakpoints are set for each size group.

Table \ref{tbl:vol} shows the performance of B2016 and FF2016 when explaining
returns on size and volatility sorts. These sorts feature the highest intercept in
absolute value for B2016, -0.79 in the small-high volatility portfolio. Small-high
volatility stocks underperform their peers and this is not captured by B2016 nor FF2016.
In short, value and momentum explain most of the returns in all of the portfolios except
for the high-volatility quintile. Value, momentum and profitability share honours for
explaining returns in the high-volatility quintile. Where value disappears, in the small-
high volatility portfolio, B2016 performs poorly.

Beta slopes are included for the sorts on volatility because of the link between market
beta and volatility. Betas increase as volatility increases with little sign of a size
effect. Large beta spreads, 0.7 to 1.3, are unusual for anomaly sorts once other factors
have been controlled for. For both models, value slopes are flat for the three lowest
volatility quintiles, perhaps showing a small size effect in the largest size quintile.
Value slopes decrease across the highest two volatility quintiles, becoming negative in
the high-volatility quintile. Momentum slopes tell a similar story except negative slopes
in the high-volatility quintile are much larger in magnitude than positive slopes in lower
volatility quintiles. Unusually for momentum, smaller stocks exhibit larger spreads in
momentum slopes. The momentum factor appears to capture the returns to low-volatility
stocks as those alphas drop out in the B2016 model. Slopes on the momentum factor, 0.16 in
the second size portfolio and 0.18 in the third, correspond to the reductions in alphas.
For both models, profitability slopes are insignificant except for the high-volatility
quintile. The profitability slopes of B2016 are consistently lower in magnitude than those
of FF2016. Nevertheless the power of profitability to explain the poor returns on
volatility supports evidence in \textcite{novy2014understanding}. Novy-Marx finds that the
returns on defensive equity are reduced when profitability and size are correctly
accounted for.

Small, high-volatility stocks clearly pose a problem. There is no obvious difference
between the slopes for B2016 and FF2016 besides FF2016 favouring profitability a little
more than B2016. Noticeably, value is nowhere to be found when either model needs it most.
We should keep in mind the implications of the sorts. Table \ref{tbl:vol} says that
selling small, high-volatility stocks makes a 0.8\% returns a month, on average. To
actually trade this strategy, we are reliant on people buying small, unprofitable stocks
with poor recent returns. This is a tough sell for even the best salespeople. Perhaps
volatility captures some of the liquidity risk described in \textcite{nagel2005short}.
Short-selling constraints would account for the asymmetry in the alphas. ``Smart money"
does not need to short-sell to take advantage of under-pricing by ``noise" traders.
Whereas sophisticated investors must be able to sell stocks they do not have to take
advantage of over-pricing.

% !TeX root=.../main.tex

\subsubsection{Volatility}

\import{./Tables/}{25_Size_Var_B2016_196307_201612}
Table \ref{tbl:25_Size_Var_B2016} shows the performance of the BS2015 model when
explaining returns on size and volatility sorts. Alphas are left in all of the small stock
portfolios except for the fourth. For the smaller portfolios the alphas range from 0.26 to
0.18. Value loadings are positive for these portfolios while momentum and profit
contribute very little. The highest volatility portfolio has an unexplained average
monthly return of -0.79. This tells us the BS2015 model considerably overestimates the
returns to small-high volatility stocks. Value is insignificant while momentum and profit
loadings are negative. This suggests the returns on small-high volatility stocks behave
like those of low profit stocks with poor relative returns over the previous year.

In general, value decreases with volatility quintile. The biggest size quintile has the
smallest spread in slopes. This makes sense as the value effect is weakest among larger
stocks \parencite{asness2015fact}. Momentum and profit slopes are similar. Magnitude is
low or insignificant for all volatility quintiles except the highest, which has negative
slopes for both factors. The momentum slope in the small-high volatility portfolio is
-0.41. Typically, the magnitude of momentum slopes is not this high and suggests strong
comovement between the returns of small-high volatility stocks and past losers. The
momentum slope, -0.41, is much higher than that of the investment factor of the FF2016b
model (See table \ref{tbl:25_Size_Var_FF2016} in appendix \ref{sec:addl_results}). The
profit slope, -0.65, in the FF2016b model is higher the profit slope in the BS2015 model,
-0.3. Value is insignificant in both models. Momentum alone seems to capture the
information in investment and profitability. This is puzzling because value is
insignificant in both models suggesting it is not the interaction between value and
momentum, as in the case of the investment sorts, that is causing the differences in
slopes. This could be a topic for future research.

While the BS2015 model improves on the performance of the FF2016b model (See table
\ref{tbl:25_Size_Var_FF2016} in appendix \ref{sec:addl_results}) when explaining returns
to stocks sorted by size and volatility, the alpha of -0.79 for small-high volatility
stocks is a large.
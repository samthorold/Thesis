% !TeX root=.../main.tex

\subsubsection{Investment}

\import{./Tables/}{25_Size_Inv_B2016_196307_201612}

Table \ref{tbl:25_Size_Inv_B2016} shows the performance of the B2016 model when explaining
returns on size and investment sorts. B2016 leaves fewer alphas than FF2016, see table
\ref{tbl:25_Size_Inv_FF2016} in appendix \ref{sec:addl_results}, supporting the view that
value and momentum are a better proxy for changes in book equity. B2016 struggles where
the highest investment quintile meets the fourth highest size quintile, leaving an alpha
of 0.18. Both models leave an alpha the small-high investment portfolio. The values are
-0.28 for FF2016 and -0.26 for B2016. FF2016 struggles with the third investment quintile
leaving alphas in the first and second  size quintiles, both 0.14. For the B2016 model,
the value slopes are uniform for the smallest two investment quintiles, except for the
small-low investment intersection which is not significant. In the larger size quintiles,
the slope decreases as the investment quintile increases. This suggests that returns on
firms with high investment behave like those with a low B/M. This is consistent with our
story that, controlling for momentum, higher B/M stocks exhibit fewer changes in BE.
Despite momentum's small slopes, the factor is highly significant is almost all
portfolios. Momentum slopes are uniform for all investment quintiles except for the
largest, where slopes become negative. Negative momentum slopes suggest the returns on
firms with large investments behave like those of past losers.

\afterpage{
\clearpage
\begin{landscape}
\import{./Tables/}{32_Size_OP_Inv_B2016_196307_201612}
\end{landscape}
\clearpage
}

Table \ref{tbl:32_Size_OP_Inv_B2016} shows the performance of the B2016 model when
explaining returns on size, operating profitability and investment sorts. Small stocks
that invest aggresively despite weak profits remain a problem with an alpha of -0.33\%.
For small stocks, the only other alpha, 0.14\%, occurs in the second profitability and
third investment portfolio. For big stocks, both alphas are in the lowest profitability
quartile in the second, -0.26\%, and highest, 0.27\%, investment quartiles. The model
performs similarly for small and big stocks. This is somewhat unexpected given value's
poor performance in big stocks. Since value is taking up much of the slack left by
investment, we would expect omitting a value factor to be detrimental to the model. As in
the sorts on size and investment, B2016 leaves fewer alphas than FF2016, see table
\ref{tbl:32_Size_OP_Inv_FF2016} in appendix \ref{sec:addl_results}).



For small and big stocks, the value slopes decrease with higher investment and increase
with profitability. The shared story for small and big stocks is somewhat surprising given
value's weakness in big stocks. Within profitability quartiles, the returns on high
investment stocks behave like those of stocks with lower B/M than low investment stocks.
While the spread in the value slopes remain close to 0.5, the lowest profiability quartile
has much lower value slopes than the three higher profitability quartiles. As investment
increases returns are behaving more like those of low value past losers. This makes sense
given that high value, past winners exhibit fewer changes in BE.

\afterpage{
\clearpage
\begin{landscape}
\import{./Tables/}{32_Size_BM_Inv_B2016_196307_201612}
\end{landscape}
\clearpage
}

Table \ref{tbl:32_Size_BM_Inv_B2016} shows the performance of the B2016 model when
explaining returns on size, B/M and investment sorts. There are more alphas than in the
sorts on size, operating profitability and investment sorts. This is because the value
factor does not vary as much across investment portfolios when we control for value.
Alphas are in the low B/M and investment quartiles for both the small and big sorts.

The value slope decreases slightly in the lowest value quartile from -0.27 in the lowest
investment portfolio to -0.46 in the highest investment portfolio. The highest value
quartile sees a similarly small decrease from 0.66 to 0.46. Value slopes increase as the
value quartiles increase. Momentum slopes are also largely flat within value quartiles. By
controlling for value we have limited B2016's ability to explain variations in investment.
This is highlighted by the large decreases in value slopes as investment increases that we
see in table \ref{tbl:32_Size_OP_Inv_B2016}.

\textbf{Does annual value change with investment?}

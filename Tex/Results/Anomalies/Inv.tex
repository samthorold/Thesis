% !TeX root=.../main.tex

\subsubsection{Investment}

\import{./Tables/}{25_Size_Inv_fmt}

Table \ref{tbl:25_Size_Inv} shows the performance of B2016 and FF2016 when explaining
returns on size and investment sorts. This sort was not one of the problem sorts
identified by a high GRS in table \ref{tbl:GRS}. In fact, B2016 outperforms FF2016 with a
GRS of 1.97 compared with 2.46. The improved GRS performance is seen in the fewer alphas
left by B2016 than FF2016. Nevertheless, we want to compare the slopes on value and
mometum in B2016 with value and investment in FF2016.

For B2016, value slopes decrease as investment quintiles increase. The returns on
aggressively investing stocks behave like the returns on growth stocks. Spreads in value
slopes across investment quintiles for B2016 are greater than spreads in value slopes for
FF2016. Without the investment factor, value does the heavy lifting when explaining
returns across investment quintiles. Spreads in value slopes in B2016 are smaller than
spreads in investment slopes in FF2016. The remaining spread is picked up by momentum in
B2016. While value and momentum are negatively correlated, they have the same sign in
sorts on size and investment.

Both value and momentum have positive slopes except for the highest investment quintile.
This drop-off in slope suggest the returns to the investment factor are more to do with
the poor performance of aggressively investing stocks than the strong performance of
conservatively investing stocks. Profitability slopes support this view as they are
negative for the highest investment quintile. The returns on aggressively investing stocks
behave like the returns on unprofitable stocks. The poor performance of aggressively
investing stocks is seen in the alpha of -0.26 for small-aggressively investing portfolio
of B2016. This alpha is almost identical to that left by FF2016 of -0.28. Both models
predict higher returns than actually observed. Table \ref{tbl:25_Size_Inv} confirms that
performance in sorts on size and investment is not sacrificed by replacing the investment
factor with a momentum factor. Value and momentum slopes in B2016 are similar to value and
investment slopes in FF2016.

Table \ref{tbl:32_Size_BM_Inv} shows the performance of the B2016 model when explaining
returns on size, annual value and investment sorts. FF2016 explains the returns on big
stocks better than B2016. This greater performance is seen in the single alpha of 0.18 in
big stocks for FF2016 compared to three for B2016 ranging from 0.28 to -0.3. In short, the
inability of the value factor to vary across investment quartiles within value quartiles
results in poor performance for B2016.

For B2016, value slopes in the lowest and highest value quartiles decrease as investment
quartiles increase. Value slopes in the middle value quartiles remain flat. The effect is
similar for small and big stocks. Decreasing value slopes within low value quartiles of
small and big stocks is opposite to the increasing value characteristics in table
\ref{tbl:valinvsort}. For small stocks, slopes decrease from -0.27 to -0.46 while
characteristics increase from 0.08 to 0.27. For big stocks, slopes decrease from -0.12 to
-0.66 while characteristics remain flat between 0.28 to 0.24. For FF2016, value slopes in
the lowest value quartiles decrease as investment quartiles increase while value slopes in
the highest value quartiles increase as investment quartiles increase. The effect is
stronger for small stocks than big stocks. Average value-weighted annual value exhibits
similar patterns in table \ref{tbl:valinvsort}. The difference between value
characteristics and slopes for B2016 could be to do with momentum's increased power in
low-value stocks \parencite{asness1997interaction}. Value slopes could be pulled around by
momentum slopes due to the strong interaction between the variables. A weak value effect
explanation is supported by an even larger slope decrease in big, low-value stocks. The
value effect is weak among big stocks \parencite{asness2015fact}.

Momentum slopes are similar for small and big stocks. Slopes are flat across investment
quartiles except for the lowest value quartile where the momentum slope drops in the
highest investment quartile. The returns on stocks with high investment and low value
behave like the returns on stocks with recent poor returns. This drop in slope could be
the driver behind value's reversed relationship with investment in the lowest value
quartile for small and big stocks. Largely flat momentum slopes across investment
quartiles matches the average value-weighted prior returns in table \ref{tbl:valinvsort}.
Momentum slopes increase as value quartiles increase. Decreasing slopes do not match
average value-weighted prior returns. The disparity between momentum slopes and
characteristics could be driven by value and momentum's strong interaction.

For FF2016, value slopes are similar to those in B2016 with the exception of the low value
quartile mentioned above. Slopes match annual value characteristics except for small, high
value stocks where slopes are increasing but characteristics are decreasing. Slopes
increase from 0.54 to 0.76 while characteristics decrease from 1.87 to 1.61. Investment
slopes decrease as investment increases because the investment factor is long
conservatively investing stocks. The spreads in slopes is much larger than that of value
or momentum in B2016. For example, investment slopes drop from 0.6 to -0.38 in the small,
low-value quartile and 0.26 to -0.3 in the big, high-value quartile. Compare this with
value slopes of -0.27 to -0.46 and 0.66 to 0.45 for B2016 in the same quartiles.
Investment slopes decrease as value quartiles increase, except for the two highest
investment quartiles in small stocks. This does not match investment characteristics in
table \ref{tbl:valinvsort} where investment only varies across value quartiles in the big,
high-investment quartile.

The returns on the investment factor are a noisy proxy for changes in BE as shown by the
divergence of characteristics and regression slopes. That said, by controlling for value
we have limited B2016's ability to explain variations in investment.

\newgeometry{margin=.25in}
\afterpage{
\clearpage
\begin{landscape}
\import{./Tables/}{32_Size_BM_Inv_fmt}
\end{landscape}
\clearpage
}
\restoregeometry

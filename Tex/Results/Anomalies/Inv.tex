% !TeX root=.../main.tex

\subsubsection{Investment}

\import{./Tables/}{25_Size_Inv_B2016_196307_201612}

Table \ref{tbl:25_Size_Inv_B2016} shows the performance of the BS2015
model when  explaining returns on size and investment sorts. BS2015 leaves fewer alphas
than FF2016b, see table \ref{tbl:25_Size_Inv_FF2016} in appendix \ref{sec:addl_results}),
supporting the view that value and momentum are a better proxy for changes in book equity.
BS2015 struggles where the highest investment quintile meets the fourth highest size
quintile, leaving an alpha of 0.18. Both models leave an alpha the small-high investment
portfolio. The values are -0.28 for FF2016b and -0.26 for BS2015. FF2016b struggles with
the third investment quintile leaving alphas in the first and second  size quintiles, both
0.14.

Table \ref{tbl:25_Size_Inv_B2016} also shows the regression slopes on value and momentum
factors. For the BS2015 model, the value slopes are uniform for the smallest three
investment  quintiles, except for the small-low investment intersection which is not
significant. The slope becomes negative in the largest investment quintile. For the larger
size quintiles this includes a gradual shift through the fourth largest investment
quintile. This suggests that returns on firms with high investment behave like those with
a low  book-to-market ratio, growth firms. Despite momentum's small slopes, the factor is
highly significant is almost all portfolios. Mirroring value, slopes are uniform for all
investment quintiles except for the largest, where  slopes become negative. Negative
momentum slopes suggest the returns on firms with large investments behave like those with
lower returns over the previous year.

\begin{landscape}
\import{./Tables/}{32_Size_OP_Inv_B2016_196307_201612}
\end{landscape}

Table \ref{tbl:32_Size_OP_Inv_B2016} shows the performance of the BS2015 model when
explaining returns on size, operating profitability and investment sorts. Small stocks
that invest aggresively despite weak profits remain a problem with an alpha of -0.33\%.
For small stocks, the only other alpha, 0.14\%, occurs in the second profitability and
third investment portfolio. For big stocks, both alphas are in the lowest profitability
quartile in the second, -0.26\%, and highest, 0.27\%, investment quartiles. The model
performs similarly for small and big stocks. This is somewhat unexpected given value's
poor performance in big stocks. Since value is taking up much of the slack left by
investment, we would expect omitting a value factor to be detrimental to the model. As in
the sorts on size and investment, BS2015 leaves fewer alphas than FF2016b, see table
\ref{tbl:25_Size_OP_Inv_FF2016} in appendix \ref{sec:addl_results}).

For small stocks, the value slopes decrease with higher investment and increase with
profitability. Within profitability quartiles, the returns on high investment stocks
behave like those of stocks with lower B/M than low investment stocks. While the spread in
the value slopes remain close to 0.5, the lowest profiability quartile has much lower
value slopes than the three higher profitability quartiles.
Given that we are replacing investment with value and momentum we would expect the 

In the context of the divident
discount model (eq. \ref{eq:ddm}), there are a couple of possible explanations for the
negative relationship between the returns on high investment stocks and returns on high
B/M stocks. First,  high B/M stocks, on average, exhibit decreases in book equity
\parencite{kok2017facts} while investment is a proxy for increases in book equity. Second,
high B/M stocks, when combined with momentum, will not exhibit changes in book equity.

\begin{landscape}
\import{./Tables/}{32_Size_BM_Inv_B2016_196307_201612}
\end{landscape}

Table \ref{tbl:32_Size_BM_Inv_B2016} shows the performance of the BS2015 model when
explaining returns on size, operating profitability and investment sorts. 

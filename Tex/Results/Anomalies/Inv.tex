% !TeX root=.../main.tex

\subsubsection{Investment}

\import{./Tables/}{25_Size_Inv_fmt}

Table \ref{tbl:25_Size_Inv} shows the performance of B2016 and FF2016 when explaining
returns on size and investment sorts. This sort was not one of the problem sorts
identified by a high GRS in table \ref{tbl:GRS}. In fact, B2016 outperforms FF2016 with a
GRS of 1.97 compared with 2.46. The improved GRS performance is seen in the fewer alphas
left by B2016 than FF2016. Nevertheless, we want to compare the slopes on value and
mometum in B2016 with value and investment in FF2016.

For B2016, value slopes decrease as investment quintiles increase. The returns on
aggressively investing stocks behave like the returns on growth stocks. Spreads in value
slopes across investment quintiles for B2016 are greater than spreads in value slopes for
FF2016. Without the investment factor, value does the heavy lifting when explaining
returns across investment quintiles. Spreads in value slopes in B2016 are smaller than
spreads in investment slopes in FF2016. The remaining spread is picked up by momentum in
B2016. While value and momentum are negatively correlated, they have the same sign in
sorts on size and investment.

Both value and momentum have positive slopes except for the highest investment quintile.
This drop-off in slope suggest the returns to the investment factor are more to do with
the poor performance of aggressively investing stocks than the strong performance of
conservatively investing stocks. Profitability slopes support this view as they are
negative for the highest investment quintile. The returns on aggressively investing stocks
behave like the returns on unprofitable stocks. The poor performance of aggressively
investing stocks is seen in the alpha of -0.26 for small-aggressively investing portfolio
of B2016. This alpha is almost identical to that left by FF2016 of -0.28. Both models
predict higher returns than actually observed. Table \ref{tbl:25_Size_Inv} confirms that
performance in sorts on size and investment is not sacrificed by replacing the investment
factor with a momentum factor. Value and momentum slopes in B2016 are similar to value and
investment slopes in FF2016.

Table \ref{tbl:32_Size_BM_Inv} shows the performance of the B2016 model when explaining
returns on size, annual value and investment sorts. FF2016 explains the returns on big
stocks better than B2016. This greater performance is seen in the single alpha of 0.18 in
big stocks for FF2016 compared to three for B2016 ranging from 0.28 to -0.3. In short, the
inability of the value factor to vary across investment quartiles within value quartiles
results in poor performance for B2016.

Value slopes increase as value quartiles increase for small and big stocks for both
models. For B2016, value slopes across investment quartiles are flat in lower value
quartiles of small stocks. For example, the lowest value quartile slopes are -0.27 for the
lowest investment quartile and -0.46 for the highest investment quartile. Compare this to
-0.55 to -0.3 for FF2016 in the same portfolios. Furthermore, we can see in the highest
value quartile that the value slope relationship with invesment quartiles is reversed for
the different models. B2016 value slopes are decreasing as investment quartiles increase
while FF2016 value slopes are increasing. This is true of small and big stocks. Value
slopes for small stocks do not match average value-weighted value characteristics in table
\ref{tbl:valinvsort}. Monthly value characteristics within the lowest value quartile
increase as investment increases while the value slope decreases. For big stocks, value
characteristics are flatter than value slopes for the lowest and highest value quartiles.

Momentum slopes are similar for small and big stocks. Slopes are flat across investment
quartiles except for the lowest value quartile where the momentum slope drops in the
highest investment quartile. The returns on stocks with high investment and low value
behave like the returns on stocks with recent poor returns. Largely flat momentum slopes
across investment quartiles matches the average value-weighted prior returns in table
\ref{tbl:valinvsort}. Momentum slopes increase as value quartiles increase.

There are more alphas than in the
sorts on size, operating profitability and investment sorts. This is because the value
factor does not vary as much across investment portfolios when we control for value.
Alphas are in the low B/M and investment quartiles for both the small and big sorts.

The value slope decreases slightly in the lowest value quartile from -0.27 in the lowest
investment portfolio to -0.46 in the highest investment portfolio. The highest value
quartile sees a similarly small decrease from 0.66 to 0.46. Value slopes increase as the
value quartiles increase. Momentum slopes are also largely flat within value quartiles. By
controlling for value we have limited B2016's ability to explain variations in investment.

\newgeometry{margin=.25in}
\afterpage{
\clearpage
\begin{landscape}
\import{./Tables/}{32_Size_BM_Inv_fmt}
\end{landscape}
\clearpage
}
\restoregeometry

% !TeX root=.../main.tex

\subsubsection{GRS}

%\import{./Tables/}{GRS_fmt}

Table \ref{tbl:GRS} shows the GRS statistic for common anomaly sorts and popular factor
models as well as the max $Sh^2$ that can be obtained from each model's factors. Comparing
the GRS statistic of different models shows where certain factors help or hinder when
explaining returns on subsets of assets. C1997 is the four-factor model of
\textcite{carhart1997persistence}. C1997b replaces annually rebalanced value with the
monthly version. C1997b is the same as removing the profitability factor from B2016 and
the $Sh^2$ of C1997b is lacking profitability's marginal contribution of 0.2 (see table
\ref{tbl:Table1}). B2016b replaces monthly value with the annual version -- B2016b can
also be thought of as replacing investment in FF2016 with momentum. B2016c removes the
momentum factor from B2016 and the $Sh^2$ of B2016c is lacking momentum's marginal
contribution of 0.13 (see table \ref{tbl:Table1}). C1997b and B2016b show where simply
using monthly value over annual value improves the power to explain returns. Similarly,
C1997b shows how much help we get from the profitability factor.

The GRS statistic, in general, lines up with the $Sh^2$. Those models with lower GRS
statistics have higher $Sh^2$. B2016 has the highest $Sh^2$ of 0.36 and the lowest GRS for
all assets of 1.76. Reading down the columns for each model shows that the GRS is
sensitive to which set of test assets are used. The GRS of B2016 fluctuates between 1.1
for $Size$-$Beta$ portfolios to 4.64 for $Size$-Residual Variance portfolios. Reading
across different rows, we see that the GRS does not always give a consistent ranking for
models. C1997b has the lowest GRS for $Size$-$Beta$ sorts but the highest GRS for
$Size$-$BM$-$Inv$ sorts. This is not to say that the GRS is flawed. The GRS is a useful
and intuitive way of measuring how well a model prices a subset of test assets. We can use
the statistic as a way to highlight those portfolios that pose particular problems for a
given model.

In relative terms, sorts on annual value are a problem for B2016. The model is
consistently outperformed by FF2016. The GRS for all BM sorts, excluding
$Size$-$BM$-$Prior$, is 1.31 for B2016 and 1.23 for FF2016. The $Sh^2$ of B2016 can be
improved more than that of FF2016 by including in the factors portfolios formed by sorts
on annual value. In absolute terms, sorts on momentum and volatility are the biggest
problem for B2016 but these sorts are where B2016 makes up the most ground on the other
models. In sorts on size and residual variance, B2016 has a GRS of 4.64 while FF2016 has a
GRS of 5.31. In sorts on size and momentum, B2016 has a GRS of 3.7 while FF2016 has a GRS
of 4.22. B2016 does a better job of pricing some of the most puzzling anomalies
\parencite{fama2016dissecting} than other models.

We can see where some of this performance increase comes from by comparing the GRS of
competing models across different sorts. For example, the profitability factor has a high
marginal contribution to the $Sh^2$ of B2016 and FF2016, see table \ref{tbl:Table1}. This
contribution is reflected in table \ref{tbl:GRS} with the two models that omit
profitability, C1997 and C1997b, having the lowest $Sh^2$ and highest GRS for all test
assets. C1997 is not outperformed in all sorts. For example, in sorts on size and
profitability, The GRS of C1997, 2.32, is comparable to that of FF2016, 2.29. This strong
performance is achieved without a profitability factor. Furthermore, the performance of
``defensive equity" portfolios, those formed on sorts on size and volatility and/or beta
has been attributed to profitability \parencite{novy2014understanding}. For sorts on
$Size$-$Var$, C1997b has an almost identical GRS, 4.83, to FF2016 and B2016b that both
contain a profitability factor. The comparable GRS statistics show the similar marginal
contributions to $Sh^2$ of value and profitability from table \ref{tbl:Table1}. Omitting
either the combination of value and momentum or profitability increases the GRS to an
identical level. While this makes sense from a statistical perspective -- identical
marginal contributions lead to identical increases in the amount of average return left
unexplained by models that omit these factors -- it is less clear how to interpret the
role of value and momentum in sorts on volatility. Individual regression slopes will shed
some light on the relationship.

C1997b's performance in sorts on size and volatility and beta show the benefits of the
interaction between value and momentum. Comparing B2016 and B2016b shows the downsides.
B2016 is outperformed by B2016b in sorts on annual value. For all sorts on annual value,
B2016b is comparable to FF2016 with an almost identical GRS of 1.24. Meanwhile, B2016 lags
behind with a GRS of 1.31. B2016c, containing monthly value and omitting momentum,
performs even worse with a GRS of 1.4 for all annual value sorts. Capturing price changes
in our measure of value clearly adds information about the cross-section of stock returns,
but this additional information comes at the expense of information in an annual measure
of value. Overall, B2016 outperforms other models in tests of subsets of assets.

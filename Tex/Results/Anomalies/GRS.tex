% !TeX root=.../main.tex

\subsubsection{GRS}

\import{./Tables/}{GRS_196307_201612}

Table \ref{tbl:GRS} shows the GRS statistic for common anomaly sorts and popular factor
models as well as the max $Sh^2$ that can be obtained from each model's factors. C1997 is
the four-factor model of \textcite{carhart1997persistence}. C1997b replaces annually
rebalanced value with the monthly version. C1997b is the same as removing the
profitability factor from B2016 and we can see the $Sh^2$ of C1997b is lacking
profitability's marginal contribution of 0.2 (see table \ref{tbl:Table1}). B2016b replaces
the monthly rebalanced value with the annual version. B2016c removes the momentum factor
from B2016. The GRS is sensitive to the choice of test assets -- the statistic varies
significantly across anomalies for all models. For size-beta sorts, the GRS says that
C1997b is improved the least by adding the test assets to the factors. Yet, for size-value
sorts C1997b is woeful. This flip-flopping is not desirable for a measure of a model's
ability to price all assets.

C1997b has a higher $Sh^2$, 0.157, than C1997, 0.092. The GRS statistic using all anomaly
sorts reports that the monthly value version can be improved more, 2.196, than the annual
value version, 2.15. In this instance, the $Sh^2$ and GRS present conflicting evidence.
With the exception of C1997 and C1997b, I find, I find the GRS and $Sh^2$ line up. The
alignment of $Sh^2$ and GRS mirrors findings in \textcite{fama2016choosing}. The GRS
provides an intuitive way to express mis-pricing but the statistic is not the best way to
judge a model's ability to price all assets. I favour the $Sh^2$ since we want to price
all assets and not just subsets of assets. I use the GRS to show where certain models fall
down. Then I use individual test asset regression slopes to explore why the model falls
down in the sorts identified by the GRS.

B2016 performs poorly in sorts on B/M, despite having the highest $Sh^2$. The ``All-BM"
row reports the GRS for all of the anomalies sorting on B/M. B2016 can be improved more,
1.315, than FF2016, 1.228, for sorts on B/M. The performance of B2016 is particularly poor
in sorts on size-B/M-investment. The GRS of 2.166 is worse than that of FF2016, 1.691, as
well as the two variants of B2016 that either use annual value, B2016b with 1.750, or omit
the momentum factor, B2016c with 2.052. Examining regression slopes for this sort, table
\ref{tbl:32_Size_BM_Inv_B2016}, shows the model fails to explain the returns across
investment quartiles when controlling for value.

B2016's problems with sorts on B/M may stem more from the monthly specification of value
than omitting the investment factor or including the momentum factor. B2016b replaces the
monthly value factor with an annual one and outperforms FF2016 in sorts on B/M, except for
size-B/M-investment. By changing the specification of value we have found a better proxy
for changes in BE but have lost some of the ability to proxy for value itself.

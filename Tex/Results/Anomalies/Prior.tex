% !TeX root=.../main.tex

\subsubsection{Momentum}

% \import{./Tables/}{25_Size_Prior_fmt}

Table \ref{tbl:25_Size_Prior} shows the performance of BS2016 and FF2016 when explaining
returns on size and momentum sorts. B2016 struggles to explain the average returns on the
small quintile, with alphas ranging from -0.41 in the lowest momentum portfolio to 0.37 in
the highest, and the high-momentum quintile, with alphas ranging from 0.37 to 0.26. For
all of the sorts that leave alphas, value works against momentum. Value and momentum's
strong interaction comes back to bite in sorts on momentum.

For B2016, value slopes show a frown shape across momentum quintiles and little spread
across size quintiles. Value offers little towards explaining returns at the extremes of
momentum with -0.1 being the slope largest in maginitude. In fact, value works against
momentum in the highest momentum quintile -- value slopes in the second and third size
quintile are -0.1 while momentum slopes are 0.32 and 0.37. This divergence in slopes could
be because we have not controlled for value and value and momentum are negatively
correlated. Intuitively, those stocks with higher recent returns will have a lower measure
of B/M as the market value is pushed up by higher returns. Those stocks with lower recent
returns exhibit more decreases in book equity, pulling B/M down. Nevertheless, typically
value and momentum slopes work together. The interaction of value and momentum is a
disadvantage when playing ``at home" in sorts on momentum. Momentum slopes increase as
momentum quintiles increase. The spread in slopes increases as size increase, -0.65 to
0.31 for the small quintile and -0.78 to 0.47 for the large. Momentum's greater
performance in large stocks partly explains the alphas in the small quintile.
Profitability is only significant in the extremes of momentum and always negative. The
returns on stocks with very low or very high returns behave like the returns on
unprofitable stocks.

For FF2016, value slopes no longer show a frown but exhibit a drop in slope for the
highest momentum quintile. The slopes on annual value do not drop in the lowest momentum
quintile perhaps because the less timely version of value does not weed out those stocks
that will exhibit a drop in BE in the future. In the absence of the momentum factor,
profitability does most of the heavy lifting in the lower momentum quintiles. Slopes are
large in magnitude ranging from -0.7 to -.54. Profitability slopes are positive but
smaller in magnitude for high momentum stocks. Investment has little to offer except in
the lowest momentum quintile where slopes range from -0.3 to -0.35. The returns on stocks
with poor recent returns behave like the returns on aggressively investing stocks.

Utlimately, the performance of B2016 is dissappointing given that the model is on home
field explaining sorts on one of the factors included in the model. The relationship
between value and momentum comes back to bite us as monthly value's negative correlation
with momentum renders value useless in sorts on momentum. To explore the problem further I
sort on size, value and momentum. This sort shows the spread in value across momentum
quartiles and vice versa. Intuition says that high momentum will inhibit the positive slopes
of value and low momentum will inhibit the negative slopes of value.

% \afterpage{
% \clearpage
% \begin{landscape}
% \import{./Tables/}{32_Size_BM_Prior_B2016_fmt}
% \end{landscape}
% \clearpage
% }

Table \ref{tbl:sizevalmom} shows the performance of BS2016 and FF2016 when explaining
returns on size, monthly value and momentum sorts. Alphas are left in the extremes of
momentum. Both models over-estimate the returns on past losers and under-estimate the
returns on past winners. In short, the strong momentum effect in low-value stocks
overrides the value effect preventing value from capturing returns in the extremes of
momentum.

For B2016, small-stock alphas are particularly large for the low-value quartile of the
momentum extremes. The low-value, low-momentum portfolio has an alpha of -0.51 while the
low-value, high-momentum has an alpha of 0.4. This symmetry in the alphas differs from the
asymmetry seen in table \ref{tbl:vol}. We cannot turn to the short-selling constraint
theory as readily as sorts on size and volatility because there are few barriers
preventing smart money profiting from underpriced stocks. Big stock alphas are smaller in
magnitude than small stock alphas, ranging from -0.25 in the second value, high-momentum
portfolio to 0.27 in the third value, low-momentum portfolio. Unlike small stocks, the
low-value, low- momentum alpha is insignificant. Value slopes are highly significant and
exhibit large spreads. For small stocks, the spread across value quartiles is -0.41 for
low-value to 0.5 for high- value in the low- momentum quartile rising to -0.42 to 1.03 for
the high-momentum quartile. The increase in spread highlights the increasing role momentum
plays as the momentum quartile increases. Just as in table \ref{tbl:valmomsort}, even
though we are controlling for value when we look across the value rows, the value slopes
become more extreme due to the increasing momentum. For low-value, moving from the third
to the highest momentum quartile decreases the value slopes while for high-value, the
opposite is true. This depressing of value by momentum could account for the 0.42\% that
B2016 under- estimates the returns on low-value, high-momentum small stocks. This frown in
value slopes for the low-value quartile mirrors the slopes in table \ref{tbl:vol}. For big
stocks we see a similar effect although the implications for alpha are not a pronounced.
Table \ref{tbl:valmomsort} shows that the monthly value characteristics exhibit a frown
for small stocks but not for big stocks. This difference in characteristics could drive
the difference in the importance of value slopes across big and small stocks. Momentum
slopes increase across value quartiles for small and large stocks. For small stocks, the
effect is less pronounced in the low-value quartile. The spread is largely consistent. For
example, the momentum slopes in the small, high-momentum quartile rises from 0.13 to 0.41
while the slopes in the big, low-momentum quartile rise from -0.73 to -0.48. Momentum
slopes increase as momentum quintiles increase. Profitability does not have much to add in
the presence of value and momentum.

For FF2016, small-stock alphas are also particularly large for the low-value quartile of
the momentum extremes. The low-value, low-momentum portfolio has an alpha of -0.86 while
the low-value, high-momentum has an identical alpha to that of B2016 at 0.4. Big stock
alphas resemble small stocks alphas much more closely than in B2016 with a low-value, low-
momentum alpha of -0.32 and a low-value, high-momentum alpha of 0.34. Most notably, FF2016
over-estimates low-value, low-momentum returns more than B2016. Value slopes are highly
significant and exhibit large spreads, although not as large as monthly value in B2016.
For small stocks, the spread across value quartiles is -0.08 for low-value to 0.6 for
high-value in the low-momentum quartile rising to -0.46 to 0.7 for the high-momentum
quartile. The increase in spread highlights the increasing role momentum plays as the
momentum quartile increases. Characteristics do not completely match regression slopes as
the spreads in the annual value characteristics are larger than the spreads in the monthly
value characteristic as seen in table \ref{tbl:valmomsort}. We should keep in mind that
unconditional characteristic sorts do not always match regression slopes, especially when
the regression specifications are different across the two models. Value disappears in the
low-value, low-momentum portfolio. In contrast, the value slope for B2016 is strongly
negative at -0.41. Profitability plays a much larger role in explaining the returns on
low-momentum stocks in the absence of the momentum factor. For small and big stocks,
profitability slopes for past losers are strongly negative ranging from -0.61 to -0.79 for
small stocks to -0.18 to -0.79 for big stocks. The returns on past losers behave like the
returns on unprofitable stocks. For small stocks, the spread in profitability slopes is
flat across value quartiles while for big stocks, profitability is decreasing as value
quartiles increase. Investment also picsks up some of the slack in the absense of the
momentum factor. Investment plays a role in the low-momentum quartiles with strongly
negative slopes. The troublesome small, low-value and low-momentum portfolio shares the
``lethal combination" described in \textcite{fama2015five} of negative profitability and
investment slopes.

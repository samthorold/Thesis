% !TeX root=.../main.tex

\subsubsection{Momentum}

\import{./Tables/}{25_Size_Prior_fmt}

Table \ref{tbl:25_Size_Prior} shows the performance of the BS2015 model when
explaining returns on size and volatility sorts. The model struggles to explain the
average returns on the small quintile with alphas ranging from -0.41 in the lowest
momentum portfolio to 0.37 in the highest. The big-second lowest momentum portfolio also
poses a problem with an alpha of 0.24. This is an improvement over the FF2016b model (See
table \ref{tbl:25_Size_Prior_FF2016} in appendix \ref{sec:addl_results}). While it is not
surprising that the BS2015 model leaves fewer alphas than the FF2016b model that does not
include a momentum factor, it is puzzling that the model performs poorly playing ``at
home." We would expect a model explaining finer sorts on one of the included factors to
explain more of the average returns.

The profitability slopes are small, rarely above 0.2 in magnitude, but significant in the
extreme momentum quintiles. All significant intercepts are negative suggesting the returns
on both low and high momentum stocks behave like the returns of stocks with low cash
profits. Unsurprisingly, the momentum slopes increase with the momentum quintiles. There
is a small increase in the maginitude of the slopes as the size quintile increases,
although the increase is around 0.2 in magnitude for all momentum quintiles. Momentum
slopes are most extreme in the lowest momentum quintile, -0.65 to -0.78. Value slopes are
low in the low and high momentum quintiles as well as the big size quintile. This is
puzzling because of the slope enhancement described in \textcite{fama2015incremental}.
When momentum slopes are extreme we would expect value slopes to be enhanced. Most
intriguingly, value is insignificant in the small-high momentum portfolio. Other value
slopes in the small size quintile are around 0.3. Value's insignificance seems to
contribute to the alpha of 0.37 in the small-high momentum portfolio. As high returns
increase the price of a stock, our monthly book-to-market ratio will decrease. High
momentum will flatten the value slopes in sorts on size and momentum limiting the
effectivesness of value. This is the variable attentuation effect described in
\textcite{fama2015incremental}.

While the BS2015 model may improve the overall performance in sorts on size and momentum,
the model's greatest strength may be its greatest weakness as the variable attentuation
from the interaction between value and momentum outweighs the slope enhancement.

\afterpage{
\clearpage
\begin{landscape}
\import{./Tables/}{32_Size_BM_Prior_B2016_196307_201612}
\end{landscape}
\clearpage
}

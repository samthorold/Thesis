\begin{itemize}
  \item From asset pricing to obsession with risk story with returns
  \item Characteristics vs. risk
  \item Compensation for risk vs. tail risk
  \item Varying factor loadings over time | ``dynamic" models
  \item Problems for risk story | quality and momentum
  \item F\&F's refusal to consider momentum until 2016. Not considered because a $WML$ 
  factor does not add explanatory power to factor models. Sorts on other characteristics
  do not produce variation in momentum. By construction we have made a $WML$ factor
  rubbish but this does not mean momentum is not important. If we constructed portfolios
  to have no variation in size then a size factor would be rubbish but that does not mean
  size is not important
  \item Does basing everything on Sharpe-Lintner mean we are in a risk-story world before
  we even consider the validity of behavioural stories?
  \item Gibbons, Ross and Shanken test $H_0$: Excess returns are jointly zero

\end{itemize}

\section{Story so far}

How would an investor allocate their money between risky assets?
\begin{itemize}
  \item They would minimize the portfolio variance for a given level of return.
  Markowitz, 1952\cite{markowitz1952portfolio}
  \item Two stage process, they would choose risky a risky portfolio then allocate between
  the risky portfolio and a risk free asset. Tobin, 1958\cite{tobin1958liquidity}
\end{itemize}
How would equilibrium prices work if everyone was a child of Markowitz and Tobin?
\begin{itemize}
  \item Assuming all investors share beliefs about the expected future returns and future
  covariances, everyone would hold a portfolio on the efficient frontier.
  Sharpe, 1964\cite{sharpe1964capital} and Lintner, 1965\cite{lintner1965valuation}
  \item The efficient frontier can be replicated with any two portfolios in opposite
  ``hemispheres." The market is a linear combination of all portfolios and all
  portfolios are efficient under our assumptions. Therefore, the market is the optimum
  risky portfolio. Black, 1972\cite{black1972capital}
  \item An asset's expected return is a linear function of the asset's covariance with the
  market portfolio
  \begin{equation}
    E(R_i) = \alpha + \beta_M E(R_M)
  \end{equation}
\end{itemize}

Analysis found that actual compensation for risk was too flat.
Fama and MacBeth, 1973\cite{fama1973risk} and Jensen, 1978\cite{jensen1978some}.
Further \emph{anomalies}, trading strategies that could not be explained by market
covariance, emerged.
\begin{itemize}
  \item Size (Banz, 1981\cite{banz1981relationship})
  \item Value
  \item Quality (Graham and Dodd, 1934\cite{graham1934security})
\end{itemize}

Fama and French, 1993\cite{fama1993common}, suggested a three factor model with three
betas; the market, size, value.
\begin{equation}
  E(R_i) = \alpha + \beta_M E(R_M) + \beta_S SMB + \beta_V HML
\end{equation}

Analysis and intuition continued to throw up more unexplained anomalies that resulted in
significant and tradeable alpha in the three factor model.
\begin{itemize}
  \item Momentum (Jegadeesh and Titman, 1993\cite{jegadeesh1993returns} and
  Carhart, 1997\cite{carhart1997persistence})
  \item Accruals (Sloan, 1996\cite{sloan1996stock})
  \item Liquidity
  \item Volatility (Ang et. al., 2006\cite{ang2006cross})
  \item Earnings
  \item Dividends
  \item Net Share Issues
\end{itemize}

Motivated by the dividend discount model (gordon growth), Fama and French,
2015\cite{fama2015five}, suggest the inclusion of profitability and investment factors.
\begin{equation}
  E(R_i) = \alpha + \beta_mE(R_m) + \beta_SSMB + \beta_VHML + \beta_PRMW + \beta_ICMA
\end{equation}
The new factors decreased the number of anomalies unexplained by the asset pricing model.
Furthermore, the anomalies that remain can grouped together as those portfolios whose
returns behave like those of small firms who invest aggresively despite weak profits.
Interestingly, the five factor model underperforms a model which omits the profitibility
factor when attempting to explain the accruals anomaly.

Operating profit includes information about management's ability to use earnings and their
decisions regarding firm structure (through the level of interest payments). This is an
important metric but does not reflect \emph{true} profitability. Novy-Marx,
2013\cite{novy2013other} suggests using gross profits as a proxy for true profitability.
This specification of factor has the attractive qualities of reviving the value factor and
providing a hedge for value. Moreover, much of the analysis of asset pricing models uses
the intercept from regressions as the measure of a model's performance. Barillas and
Shanken, 2016\cite{barillas2016alpha}, suggest a more general method. Given the implied
problems with operating profitability and the attractive qualities of newer tests of a
model's marginal power when explaining anomaly returns, Fama and French,
2016\cite{fama2016dissecting}, consider a five factor model with operating profit replaced
with a cash measure of profitability. This is our current ``state of the art."

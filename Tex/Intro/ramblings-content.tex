\begin{itemize}
  \item \sout{From asset pricing to obsession with risk story with returns}
  Samuelson, 1965 and Mandelbrot, 1966 suggest that price \emph{changes} are i.i.d. -
  sub-martingales
  \item Characteristics vs. risk
  \item Compensation for risk vs. tail risk
  \item Varying factor loadings over time | ``dynamic" models
  \item Problems for risk story | quality and momentum
  \item F\&F's refusal to consider momentum until 2016. Not considered because a $WML$ 
  factor does not add explanatory power to factor models. Sorts on other characteristics
  do not produce variation in momentum. By construction we have made a $WML$ factor
  rubbish but this does not mean momentum is not important. If we constructed portfolios
  to have no variation in size then a size factor would be rubbish but that does not mean
  size is not important
  \item Does basing everything on Sharpe-Lintner mean we are in a risk-story world before
  we even consider the validity of behavioural stories?
  \item Gibbons, Ross and Shanken test $H_0$: Excess returns are jointly zero
  \item If $TERM$ and $DEF$ were so instrumental in moving from time series to
  cross-section of returns, where have they gone?
  \item \sout{Who said the market is the optimal risky portfolio} Black
  \item Momentum only has power in sorts on momentum because sorts on momentum do not
  produce variation in the other variables - fine but this does not make momentum useless.
  It is traded on in industry for example
  \item NASDAQ stocks added after 1973
  \item Cash profitability better captures returns on sorts on accruals (but not
  necessarily a better proxy for true profitability, Penman)
  \item $OP_{Cash}$ better than $GP$ Ball et al 2016
  \item Dividends in sorts on investment - is ``plowback" a better proxy for true
  investment
  \item Why is $a'\sum^{-1}a$ attractive weighting for asset pricing tests (GRS)
  \item Check not data-mining by using in other countries
  \item Industry constraints e.g. short selling

\end{itemize}

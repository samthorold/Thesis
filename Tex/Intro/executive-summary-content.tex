
Markowitz\cite{markowitz1952portfolio} demonstrated how a risk averse investor 
would allocate their assets across a portfolio of risky assets by minimizing 
the variance for a given level of return.
Tobin\cite{tobin1958liquidity} showed that each investor will separate their 
investment decision into two stages.
First, the investor will decide how to allocate resources between risky assets.
Second, the investor will decide how much to allocate to the risk free asset 
and the total risky portfolio.
Sharpe\cite{sharpe1964capital} and Lintner\cite{lintner1965valuation} used 
mean-variance individual investors to motivate a model of equilibrium, the 
Capital Asset Pricing Model (CAPM).
Assuming all investors shared beliefs about the expected returns and 
covariances of all assets, they would all hold a portfolio on the efficient 
frontier.
Black\cite{black1972capital} explained that, since the market consists of all 
portfolios and all portfolios are efficient, the market is the optimum risky 
portfolio.
An asset's return is a linear function of its covariance with the market.
\begin{equation} \label{eq:mktmodel}
  E(r_i) = \alpha + \beta_M E(r_m)
\end{equation}

Empirical work showed the CAPM did not match reality.
Fama and MacBeth, 1973\cite{fama1973risk} and Jensen, 
1978\cite{jensen1978some} found that returns are flatter than predicted by the 
CAPM.
Furthermore, Roll\cite{roll1977critique} questioned the nature of tests of the 
CAPM. Roll points out that tests of the CAPM are not tests of market 
efficiency 
since we cannot know the true market portfolio of all assets.
Any test of the CAPM is really a test of a joint hypothesis of market 
efficiency and some asset pricing model.
We cannot, with the tools and tests available at present, reject market 
efficiency.
Sorts on; size, value, quality and dividends could all create returns 
unexplainable by the CAPM.

Fama and French\cite{fama1993common}, suggested a three factor model.
\begin{equation} \label{eq:FF3F}
  E(R_i) = \alpha + \beta_M E(R_M) + \beta_S SMB + \beta_V HML
\end{equation}
where; $R$ stands for return in excess of the risk free rate, $SMB$ is a time 
series of returns on a portfolio long small stocks and short big stocks and 
$HML$ is a time series of the returns on a portfolio long high book-to-market 
stocks and short low book-to-market stocks.
The three factor model struggled to explain returns on portfolios constructed 
from sorts on size and; net share issues, momentum, accruals, volatility, 
liquidity and  book-to-market.

Motivated by the dividend discount model (gordon growth), Fama and 
French\cite{fama2015five}, suggest the inclusion of profitability and 
investment factors.
\begin{equation} \label{eq:FF5F}
  E(R_i)=\alpha+\beta_mE(R_m)+\beta_SSMB+\beta_VHML+\beta_PRMW+\beta_ICMA
\end{equation}
The new factors decreased the number of anomalies unexplained by the asset 
pricing model.
Furthermore, the anomalies that remain can grouped together as those 
portfolios whose returns behave like those of small firms who invest 
aggressively despite weak profits.
Interestingly, the five factor model under performs a model which omits the 
profitability factor when attempting to explain the accruals anomaly.

Operating profit includes information about management's ability to use 
earnings and their decisions regarding firm structure (through the level of 
interest payments).
This is an important metric but does not reflect \emph{true} profitability.
Moreover, operating profit is a big source of accruals. 
Novy-Marx\cite{novy2013other} suggests using gross profits as a proxy for true 
profitability.
Ball et al.\cite{ball2016accruals} suggest a cash measure of operating 
profitability.
They find that the cash component of earnings is more persistent than the 
accrual component.
Much of the analysis of asset pricing models uses the intercept from 
regressions as the measure of a model's performance.
Barillas and Shanken\cite{barillas2016alpha}, suggest a more general method.
Strong asset pricing models will leave residuals with small sharpe ratios.
We cannot know the true market portfolio but we know that it must contain all 
assets.
This means the true market portfolio contains the factors we use in our models.
The factors cannot improve the sharpe ratio of the true market.
With this in mind, the sharpe ratio of the residuals is made smaller by 
increasing the sharpe ratio of the factors. This provides a simple method 
comparing factors without knowin the true market portfolio.

Given the implied problems with operating profitability and the attractive 
qualities of newer tests of a model's marginal power when explaining anomaly 
returns, Fama and French\cite{fama2016dissecting} consider a five factor model 
with operating profit replaced with a cash measure of profitability.
This is our current ``state of the art."

I consider two extensions to Fama and French's ``Choosing Factors" paper.
Firstly, Ansess et al.\cite{asness2013devil} and Fama and 
French\cite{fama2015incremental} suggests a $HML$ factor that adjusts the 
market value of equity monthly. This specification of factor captures some of 
the momentum effects of changing stock returns. As explained by Fama and 
French\cite{fama2015incremental}, this is because $HML$ and momentum are 
negatively correlated. Fama and French\cite{fama2016choosing} do not consider 
and momentum factor in their tests. Momentum is a successful strategy in 
industry.
Secondly, I consider competing models under typical industry constraints such 
as 130:30 funds. Fama and French find that the loadings on factors with 
maximum sharpe ratios can become unrealistic to be put into practice.

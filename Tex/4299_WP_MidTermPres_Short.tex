\documentclass{beamer}

\usepackage{booktabs}
\usepackage{dcolumn}
\usepackage{graphicx}
\usepackage[utf8]{inputenc}

\usetheme{Madrid}

\title[Turtles]{Asset pricing models: turtles all the way down}
\author[Thorold]{4299 Sam Thorold\\
                 Supervisors: Francisco Santos and Andr\'e Silva}
\institute{NHH and NOVA}
\date[April 2018]{\today}

\begin{document}

\frame{\titlepage}

\section{Background}

\begin{frame}
\frametitle{What makes a good model?}
\begin{itemize}
    \item<1-> We want our asset pricing models to minimize pricing errors for
    all assets
    \item<1-> When we price returns, the pricing error is given by
    \[
    \text{Sh}^2(a) = \text{Sh}^2(R, f) - \text{Sh}^2(f)
    \]
    \item<2-> If $R$ is all assets then $\text{Sh}^2(R, f) = \text{Sh}^2(R)$
    \[
    \text{Sh}^2(a) = \text{Sh}^2(R) - \text{Sh}^2(f)
    \]
    \item<2->Model with the highest Sharpe ratio of the factors is best
\end{itemize}
\end{frame}

\begin{frame}
\frametitle{Which factors?}
\begin{itemize}
    \item<1-> ICAPM says investors care about their payoff in the future but
    also what they can do with it
    \item<1-> Market + state variables
    \item<1-> We do not know these state variables but we can make proxies
    \item<1-> Investors want to identify \emph{cheap} and \emph{profitable} stocks
\end{itemize}
\end{frame}

\begin{frame}
\frametitle{Proxies for cheap and profitable}
\begin{itemize}
    \item<1-> Cheap $\rightarrow$ Book-to-Market (BM)
    \item<1-> Some problems here as we want to identify changes in BM due to
    changes in market equity rather than book equity
    \item<1-> Monthly value + momentum
    \item<1-> Profitable $\rightarrow$ Operating profit adjusted for accruals
\end{itemize}
\end{frame}

\begin{frame}
\frametitle{Testing models}
\begin{itemize}
    \item<1-> $R_t^i = a^i + b^iR_t^M + s^iSMB_t + v^iHML_t^m + m^iWML_t + p^iPMU_t$
    \item<2-> Max Sharpe ratio of factors
    \item<2-> Gibbons, Ross, Shanken (GRS) statistic that the Sharpe ratio
    can be improved by investing in the test assets
    \item<2-> Individual time series regressions of test assets on the factors
\end{itemize}
\end{frame}

\section{Results}

\begin{frame}
\frametitle{Sharpe ratio results: Which model is best? (Why?)}
\begin{itemize}
    \item<1-> $\text{Sh}^2(R^M + SMB + HML^m + WML + PMU)=0.316$
    \item<1-> $\text{Sh}^2(R^M + SMB + HML   + CMA + PMU)=0.225$
    \item<2-> Investment factor is subsumed by value and momentum
    \item<3-> $ME_t = \sum_{s=1}^\infty E\left( \text{Profit}_{t+s} - \Delta\text{BE}_{t+s}\right) /R^s$
    \item<3-> Investment does not describe future $\Delta$BE for small, growth
    stocks while sorts on value and momentum describe investment and future
    $\Delta$BE
\end{itemize}
\end{frame}

\begin{frame}
\frametitle{Momentum and variation sort problems}
\begin{itemize}
    \item<1-> Momentum factor helps, but not much, in sorts on momentum
    \item<1-> Something going on with momentum returns that is not to do with
    momentum characteristics
    \item<2-> Profit slopes are negative in the extremes of momentum
    \item<2-> Defensive equity $\rightarrow$ beta and volatility
    \item<3-> Sorts on beta pose the least problems
    \item<4-> Sorts on variance are a big problem
    \item<4-> Small, high-variance alpha of -0.66\% per month
\end{itemize}
\end{frame}

\section{Conclusions}

\begin{frame}
\frametitle{Conclusions}
\begin{itemize}
    \item<1-> Mkt, size, value, momentum and profitability factors have the
    highest Sharpe ratio
    \item<2-> Value and momentum subsume investment
    \item<3-> ``Lethal combination" of small, unprofitable stocks that somehow
    invest aggressively $\rightarrow$ small, unprofitable firms with poor
    recent returns
\end{itemize}
\end{frame}

% \begin{frame}
% \frametitle{Momentum and variation sorts: Why problems?}
% \begin{itemize}
%     \item<1-> To take advantage of this alpha we need to sell small,
%     unprofitable stocks with poor recent returns
%     \item<2-> That's a tough sell $\rightarrow$ Liquidity
%     \item<3-> But, liquidity hard to approach from a cross-section perspective
%     \item<3-> Institutional ownership, Short-term reversal factor as a proxy
%     for market-maker's returns $\rightarrow$ market microstructure
% \end{itemize}
% \end{frame}

% \begin{frame}
% \frametitle{Momentum and variation sorts: Why problems?}
% \begin{itemize}
%     \item<1-> Cross-section perspective
%     \item<1-> High-volatility firms are much smaller
%     \item<1-> Small says high returns while high-volatility says low returns
%     \item<2-> Value disappears in the problem momentum and variance sorts
%     \item<2-> Normally, value and profitability do most of the heavy lifting
% \end{itemize}
% \end{frame}

\end{document}

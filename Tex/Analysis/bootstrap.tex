% !TeX root=../main.tex

Table \ref{tbl:bootstrap} shows the maximum $\text{Sh}^2$ for seven models as well as the
90\% confidence intervals for 100,000 bootstrap simulations of the difference in
$\text{Sh}^2$ between the model in the row and the model in the column. The main message
of the table is that adding the investment factor to market, size, monthly value, momentum
and cash profitability factors, B16 model, does not reliably improve the Sharpe ratio of
the factors (the 90\textsuperscript{th} percentile of
$\text{Sh}^2(\text{B16b})-\text{Sh}^2(\text{B16})$ is 0). The table also shows the
benefits of combining monthly value, momentum and cash profitability as both monthly value
and momentum combined and cash profitability alone dramatically increase the $\text{Sh}^2$
of the factors.

Panel A shows the maximum $\text{Sh}^2$ of the modified \textcite{ball2016accruals} model
I propose, B16, is 0.316. The $\text{Sh}^2$ is not improved by adding the investment
factor, B16b, as it remains 0.316. Adding the momentum factor, F17, to the five-factor
model of \textcite{fama2016choosing}, F16, improves the $\text{Sh}^2$ from 0.225 to 0.24.
Modifying profitability to account for accruals as in the F16 model icreases the
$\text{Sh}^2$ of the five- factor model of \textcite{fama2015five}, F15, from 0.099. To
compare to non-nested models I also show the four-factor model of
\textcite{carhart1997persistence}. The four-factor model combining annual value with
momentum, C97, has a $\text{Sh}^2$ of 0.09. The four- factor model that combines monthly
value with momentum has a $\text{Sh}^2$ of 0.136.

Panel B shows the results in panel A are, for the most part, believable. Those models with
higher $\text{Sh}^2$s than their peers in panel A have 90\% confidence intervals that are
positive in panel B. By randomly drawing months with replacement in a bootstrap procedure
I check that the time series of returns we have observed is not drastically different from
those we might have observed in another universe. This is important because when analysing
returns we do not have the benefit of seeing the distribution at each point in time, we
see only the distribution across time. Running many bootstrap procedures gives a
distribution of $\text{Sh}^2$s. The importance of the procedure is shown by any 90\%
confidence intervals that cross zero. These intervals imply that $\text{Sh}^2(\text{Row})$
is not reliably better than $\text{Sh}^2(\text{Column})$.

F15 is not reliably better than either specification of the Carhart model. The confidence
interval for $\text{Sh}^2(\text{F15})-\text{Sh}^2(\text{C97(b)})$ is not strictly
positive. Value and momentum combine to perform similarly to profitability and investment.
This is striking because profitability and investment are credited with reducing the
number of anomalies and reducing the defining characteristics of many anomalies to small,
unprofitable stocks that somehow invest aggressively \parencite{fama2015five,
fama2016dissecting}.

We can eyeball where the main contributions to model $\text{Sh}^2$ come from by comparing
the $\text{Sh}^2$ of nested models, or close to nested. The biggest contributors to
maximum $\text{Sh}^2$ appear to be cash profitability alone and monthly value coupled with
momentum. Cash profitability pushes the squared Sharpe ratio for F15 from 0.099 to 0.225
for F16. Monthly value and momentum push the squared Sharpe ratio for F17 from 0.24 to
0.316 for B16(b). We can pull out the exact contributions each factor makes to the
$\text{Sh}^2$s in panel A of table \ref{tbl:bootstrap} by examining individual spanning
regressions.

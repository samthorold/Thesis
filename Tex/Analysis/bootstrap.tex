% !TeX root=../main.tex

\afterpage{
\begin{landscape}
\import{./Tables/}{bootstrap_tbl}
\end{landscape}
}

Table \ref{tbl:bootstrap} shows the maximum $\text{Sh}^2$ for models
\ref{eq:B16} and \ref{eq:F16} as well as five other popular models.
Additional models are based on the five-factor model of \textcite{fama2015five}
or the four-factor model of \textcite{carhart1997persistence}.
Nested models show incremental benefits to adding factors.
I include non-nested models to highlight the importance of value and momentum
and to provide contrast in Sharpe and GRS evidence.
The main messages of table \ref{tbl:bootstrap} are that the Sharpe ratio of
model \ref{eq:B16} is higher than that of model \ref{eq:F16} and
cannot be improved by adding the investment factor.
Model \ref{eq:B16} has the lowest mispricing for all portfolios.

Panel A shows the $\text{Sh}^2$ of model \ref{eq:B16} is 0.316 and this
is not increased by adding the investment factor, CMA.
The $\text{Sh}^2$ of model \ref{eq:F16} is 0.225 although this can be
increased to 0.24 by including the momentum factor.
The $\text{Sh}^2$ of the original Fama and French five-factor model is 0.099.
The original Carhart four-factor model has a $\text{Sh}^2$ of 0.09 although
this is increased to 0.136 by updating the value factor monthly.
The combination of monthly value and momentum in the modified four-factor
model gives a higher Sharpe ratio than the combination of value, operating
profitability and investment in the original five-factor model.
This is striking because profitability and investment are credited with
reducing the number and maginitude of many common anomalies
\parencite{fama2015five}.
Furthermore, profitability and investment slopes provide a unifying description
of anomalies not constructed from sorts on the factor variables.
The returns on problem sorts behave like the returns on small,
unprofitable stocks that somehow invest aggressively
\parencite{fama2016dissecting}.
The combination of monthly value and momentum has lower mispricing for
the returns on all portfolios than profitability and investment.

Panel B shows, generally, those models with a higher $\text{Sh}^2$ than another
model in panel A have 90\% confidence intervals that are strictly positive.
For example, the 5\textsuperscript{th} percentile of
$\text{Sh}^2(\text{model 1})-\text{Sh}^2(\text{model 2})$ is 0.055 and the
95\textsuperscript{th} is 0.138.
This means 90\% of values are within this range and we
can conclude the $\text{Sh}^2$ of model \ref{eq:B16} is reliably higher
than that of model \ref{eq:F16}.
Any confidence intervals that cross zero imply that neither model is reliably
better than the other.
The time series of returns that are observed from July 1963 to December 2017
and used to create the $\text{Sh}^2$s in panel A may not represent the ``true"
ranking of the models.
For example, the four-factor model that rebalances value monthly may not be
truly better than the original five-factor model, despite a higher
$\text{Sh}^2$ in panel A.
The bootstrap procedure produced a significant number of combinations of months
where the five-factor model outperformed the four-factor model.
Momentum goes through ``crashes" or periods of very poor performance
\parencite{daniel2016momentum, barroso2015momentum}.
The bootstrap procedure will produce combinations of months that include
many of the crash months for momentum.
The presence of enough momentum crashes, despite the four-factor model with
monthly value's higher $\text{Sh}^2$ 
in panel A, could drive the inconclusive confidence interval with the original
five-factor model in panel B.
There is no such uncertainty for models \ref{eq:B16} and \ref{eq:F16}
even though model \ref{eq:B16} utilizes the same monthly value and momentum
relationship as the four-factor model with monthly value.
Despite crashes in the momentum factor, model reliably \ref{eq:B16} provides
the highest $\text{Sh}^2$.


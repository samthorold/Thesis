% !TeX root=../main.tex


Table \ref{tbl:contributions} shows the contributions of each factor to models
\ref{eq:B16}, \ref{eq:B16}b and \ref{eq:F16} including momentum.
The main message of the table is that the largest contributions to the
$\text{Sh}^2$ of model \ref{eq:B16} are from the value and profitability
factors. The large contributions support the intuition that
cheapness and profitability proxy for state variables priced by investors.
The average unexplained return of a factor, $a$, left by the other factors is
given by the 
intercept from the ``spanning" regression of the factor in
question on the remaining
factors in a model.
The unexplained variation in returns of a factor, $\sigma_e$, left by the other
factors is given by the standard deviation of the residuals from the same
regression.
The unexplained average return and variation in returns drive the contribution,
$(a/\sigma_e)^2$, of a factor to the $\text{Sh}^2$ of a model.
Large unexplained average return represents the potential to contribute
to the $\text{Sh}^2$ of a model.
The potential to contribute is dampened by the variation in returns that cannot
be explained by the other factors.
Just as the quadratic form of the alphas, $a'V_ea$, emphasises the importance
of covariance of alphas,
the reduction in contribution to Sharpe ratio given by the unexplained
variation of a factor emphasises the importance of the covariance of factors.


The data for model \ref{eq:B16} shows the largest contributions to
$\text{Sh}^2$ are from value and momentum at 0.18.
The market and momentum have similar contributions, 0.1 and 0.11 respectively.
Size makes a small contribution of 0.03.
The contribution from value is driven by unexplained average return of 1\% per
month dampened by unexplained variation of 2.36.
The contribution from profitability is driven by low unexplained variation,
1.17, rather than the modest unexplained average return of 0.5\% per month.
The data for model \ref{eq:B16}b shows the largest contribution is
from profitability at 0.18.
In the presence of investment, value, at 0.11, has a similar contribution to
the market, 0.1, and momentum, 0.09.
The reduction in value's contribution is driven by the lower unexplained
average return, 0.59\% per month, but mitigated somewhat by the reduced
unexplained variation, 1.83.
Both reductions are driven by value's strong correlation with investment.
The slope attentuation described in \textcite{fama2015incremental} causes the
spillover effect for momentum seen in the small reductions to unexplained
average returns and variation.
The negligible contribution from investment is driven by the small unexplained
average return of 0.01\% per month.
The data for model \ref{eq:F16} including momentum shows the largest
contribution is from profitability at 0.14.
The market and size maintain their contributions, at 0.08 and 0.03
respectively, but value and momentum see large reductions in contribution to
0.03 or below.
Value and investment are strongly correlated and this drives the low
unexplained average return of 0.33\% per month.
Without the interaction between monthly value and momentum, the variables
cannot benefit from slope enhancement that helps the contributions in
models \ref{eq:B16} and \ref{eq:B16}b.
The investment factor adds nothing to the Sharpe ratio of model \ref{eq:B16}.
Next I investigate which factors specifically subsume investment.

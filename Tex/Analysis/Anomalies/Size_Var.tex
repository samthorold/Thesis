% !TeX root=../../main.tex

Table \ref{tbl:25_Size_Var_B16} shows 25 time series regressions from
size-variance sorts for model \ref{eq:B16} while table
\ref{tbl:25_Size_Var_F16} shows the same regressions for model \ref{eq:F16}.
The main message of the table is that stocks that share the
``lethal combination" of negative profitability and investment slopes behave
like the returns on small, unprofitable stocks with poor recent performance.
What is in this new description of problem sorts is just as important as what
is not -- value.
Those sorts that create problems for model \ref{eq:B16} have returns that do
not covary with the returns on the value factor.

I include the market and size slopes, despite their similarity across models,
because they show volatility's relationship with market beta and size.
Beta increases as volatility buckets increase and remains flat across size
buckets.
This relationship with beta makes sense given individual stock, or portfolio,
variance drives covariance with the market.
Size slopes increase with volatility buckets.
In the smallest two size buckets, the variation in size slopes is extreme.
For model \ref{eq:B16}, the smallest size bucket slopes rise from 0.63 to 1.5
and the second size bucket slopes rise from 0.51 to 1.18.
Value slopes are also similar across models.
For both models, low-variance stocks have positive value slopes so
the returns on low-variance stocks behave like the returns on value stocks.
There is a drop in value slopes for the high-variance bucket so
the returns on high-variance stocks behave like the returns on growth stocks.
The value slope is insignificant in only the small, high-variance bucket for
both models.
Profitability slopes tell a similar story across both models,
although the profitability factor does much more of the heavy lifting for model
\ref{eq:F16} in the absence of momentum.
Profitability slopes are positive for the low-variance bucket and negative for
the high-variance bucket.
The returns on low-variance stocks behave like those of profitable stocks.
This relationship is consistent with findings in
\textcite{novy2014understanding} and earlier findings that momentum sorts
may be identifying volatile stocks through their profitability slopes.
For model \ref{eq:B16}, momentum slopes are negative for the high-variance
bucket.
The returns on high-variance stocks behave like the returns on stocks with poor
recent returns. 
For model \ref{eq:F16}, investment has very little to add for sorts on
variance.
The small, high-variance intercept is large for both models.
The intercept is -0.66\% per month for model \ref{eq:B16} and -0.8\% per month
for model \ref{eq:F16}.
The improvements for model \ref{eq:B16} are from the negative momentum factor
slopes where the investment factor has nothing to add to model \ref{eq:F16}.

To take advantage of the alpha in the small, high-variance bucket of table
\ref{tbl:25_Size_Var_B16} an investor has to sell stocks whose returns behave
like those of small, unprofitable firms with poor recent performance.
Intuitively, this is a tough sell.
Such stocks are likely to be more illiquid than than big, profitable stocks
with strong recent performance.
Small stocks are likely to be held by fewer institutional investors or
``market makers".
\textcite{nagel2012evaporating} uses the short-term reversal strategy to proxy
for the returns to market makers who represent the largest group of buyers for
these ``problem" stocks.
Furthermore, Nagel finds liquidity measured in this way is related to the
aggregate level of volatility as proxied for by the VIX.
%\textcite{nagel2005short} "for mis-pricing to persist in the in the presence
%of [smart money], limits to arbtrage must exist"
Liquidity explanations are not easy to tackle from a cross-section
perspective and often focus on market microstructure.
Furthermore, \textcite{ang2006cross} test many possible liquidity
explanations\footnote{for example, bid-ask spreads, trading volume and
turnover.} for the volatility anomaly and find none help greatly to describe
the returns on highly volatile stocks.
They also find that the aggregate level of volatility, proxied for by the VIX,
does not drive out alphas in sorts on volatility.

\textcite{asness2018deep} 


\begin{itemize}
  \item AQR -- level of surprise in the VIX paper, ``deep value" paper
  \item Consumption and SDF
  \item If I plot my SDF from optimal factor weightings what does it do in
  downturns or periods of deep value?
  \item If I plot the residuals from small, high-variance regression, what are
  they doing in downturns or periods of deep value?
\end{itemize}

% !TeX root=../../main.tex

Table \ref{tbl:25_Size_Var_B16} shows 25 time series regressions
from size-variance sorts using B16 factors while table
\ref{tbl:25_Size_Var_F16} shows the same regressions using F16 factors.
The main message of the table is that the ``lethal combination" of small,
unprofitable stocks that somehow invest aggressively is described by stocks
whose returns behave like those of small, unprofitable stocks with recent poor
performance.
What is in this new description of problem sorts is just as important as what
is not -- there is no value.
Those sorts that create problems for B16 have returns that do not covary with
the returns on the value factor.

I include the market and size slopes, despite their similarity across models,
because they show volatility's relationship with market beta and size.
Beta increases as volatility buckets increase and remains flat across size
buckets.
This relationship with beta makes sense given individual stock, or portfolio,
variance drives covariance with the market.
Size slopes increase with volatility buckets.
In the smallest two size buckets, the variation in size slopes is extreme.
The smallest size bucket slopes rise from 0.63 to 1.5.
The second size bucket slopes rise from 0.51 to 1.18.
Value slopes are also similar across models.
Low-variation stocks have positive value slopes.
The returns on low-variation stocks behave like the returns on value-stocks.
There is a drop in value slopes for the high-variation bucket.
The returns on high-variation stocks behave like the returns on growth stocks.
The value slope is insignificant in only one bucket for both models --
small, high-variation.
Profitability slopes tell a similar story across both models, although the
profitability factor does much more of the heavy lifting for F16 in the absence
of momentum.
Profitability slopes are positive for the low-variation bucket and negative
for the high-variation bucket.
The returns on low-variation stocks behave like the returns on low-profit
stocks.
This relationship is consistent with findings in
\textcite{novy2014understanding} and our earlier findings that momentum sorts
may be identifying volatile stocks through their profitability slopes.
For B16, momentum slopes are negative for the high-variation bucket.
The returns on high-variation stocks behave like the returns on stocks with
poor recent returns.
For F16, investment has very little to add for sorts on variation.
The small, high-variation intercept is large for both models.
The intercept is -0.66\% per month for B16 and -0.8\% per month for F16.
The improvements for B16 are from the negative momentum factor slopes where the
investment factor has nothing to add (t-stat=-1.25).

To take advantage of the alpha in the small, low-value, low-prior bucket of
table \ref{tbl:Size_BMm_Prior} and the small, high-variance bucket of table
\ref{tbl:25_Size_Var_B16} an investor has to sell stocks whose returns behave
like those of small, unprofitable firms with poor recent performance.
Such stocks are likely to be more illiquid than than big, profitable stocks
with strong recent performance.
\textcite{nagel2012evaporating} relates the short-term reversal premium to the
VIX while \textcite{nagel2005short} ...
Liquidity explanations often focus on market microstructure.

Cross-section explanation -- size and variance slopes pull in opposite
directions.

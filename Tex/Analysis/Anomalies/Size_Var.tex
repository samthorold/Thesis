% !TeX root=../../main.tex

\import{./Tables/}{Size_Var_tbl}

Table \ref{tbl:Size_Var} shows 25 time series regressions from
size-variance sorts for models \ref{eq:B16} and \ref{eq:F16}.
The main message of the table is that stocks that share the
``lethal combination" of negative profitability and investment slopes behave
like unprofitable stocks with poor recent returns.
Like momentum sorts, the portfolios with the most mispricing coincide with
value's disappearance.
Those sorts that create problems for model \ref{eq:B16} have returns that do
not covary with the returns on the value factor.

In contrast to momentum and beta sorts, there is no significant improvement in
the magnitude of the alphas.
The small, high-Var portfolio has an alpha of -0.75 for model \ref{eq:B16} and
-0.8 for model \ref{eq:F16}.
Beta increases as volatility buckets increase and remains flat across size
buckets.
The relationship with beta makes sense given individual portfolio
variance drives covariance with the market.
Despite controlling for size slopes increase with volatility buckets.
In the smallest two size buckets, the variation in size slopes is extreme.
The smallest size bucket slopes rise from 0.63 to 1.5
and the second size bucket slopes rise from 0.51 to 1.18.
The importance of size slopes corresponds with characteristic evidence in
table \ref{tbl:Size_BMm_Prior_chars} that shows greater variation in the
variance characteristic for small stocks.
Positive value slopes show that low-Var stocks behave like the returns on value stocks.
Negative value slopes show that high-Var stocks behave like the returns on growth stocks.
The value slope is insignificant in the small, high-Var bucket.
Momentum slopes are negative for the high-Var bucket, $\sim$-0.3, but
otherwise remain low, $\sim$0.1.
The returns on high-Var stocks behave like the returns on stocks with poor
recent returns. 
Profitability slopes are negative for the high-Var bucket and increase with
size buckets from -0.5 for small stocks to -0.17 for big stocks.
The returns on low-Var stocks behave like unprofitable stocks.
The remaining profitability slopes are positive and $\sim$0.15.
Sorts on variance are a major problem, as shown by the -0.75\% alpha per month
in the small, high-Var portfolio.
Variance is related to beta but the component of variance associated with
returns on the market does not cause problems for model \ref{eq:B16}, as shown
by the insignificant alphas in table \ref{tbl:Size_Beta}.

% not liquidity
To take advantage of the alpha in the small, high-Var bucket of table
\ref{tbl:Size_Var}, an investor has to sell stocks whose returns behave
like those of small, unprofitable firms with poor recent returns.
% Intuitively, this is a tough sell.
\textcite{nagel2005short} writes ``for mis-pricing to persist in the in the
presence of [smart money], limits to arbtrage must exist".
small, unprofitable firms with poor recent returns are likely more illiquid
than than big, profitable stocks with strong recent returns.
Furthermore, small stocks are likely to be held by fewer institutional
investors or ``market makers" willing to provide the other side of a trade for
arbitrageurs.
\textcite{nagel2012evaporating} uses the short-term reversal strategy to proxy
for the returns to market makers who represent the largest group of buyers for
problem stocks.
Nagel finds liquidity measured in this way is related to the aggregate level
of volatility, proxied by the VIX.
There may be a large alpha for the small, high-Var portfolio because noone is
willing to buy from arbitrageurs trying to sell small, high-Var stocks.

Corresponding with my findings that value disappears in the small, high-Var
portfolio, \textcite{asness2018deep} find that volatility increases during
periods of ``deep value".
Deep value means the spread between the value characteristic of cheap and
expensive stocks is large.
During a deep value event, the returns on the spread between cheap and
expensive stocks is higher, volatility is higher and the earnings of cheap
stocks are particularly low compared to normal times.
Asness et al. also find illiquidity indicators such as short interest and
analysts revisions of forecasts are increased during periods of deep value.
Value's disappearance seems to coincide with an increase in illiquidity that
makes selling small, high-Var stocks difficult.
Without arbitrageurs selling small, high-Var stocks, there is no price
correction and inflated prices remain.

Nevertheless, liquidity explanations are not easy to tackle from a
cross-section perspective and often focus on market microstructure such as
market makers and analysts' forecasts revisions.
Furthermore, \textcite{ang2006cross} test many possible liquidity
explanations\footnote{Bid-ask spreads, trading volume, and
turnover, among others.} for the volatility anomaly from a cross-section perspective
and find none help greatly to price the returns on highly volatile stocks.
They also find that the aggregate level of volatility, proxied for by the VIX,
does not drive out alphas in sorts on volatility.
Liquidity, through aggregate volatility or deep value, is not the cause of
alpha in small, high-volatility stocks.

% maybe size
I use value's disappearance as a starting point for a cross-section
explanation for the problems with small, high-Var portfolios.
Value is a noisy proxy because BE and ME are imprecise ways to describe firms.
Furthermore, BE and ME are explicitly part of other factors, for example
profitability and size.
Value can be ``cleaned up" with momentum but prior research finds there is more
to value than simply BE and ME\footnote{
\textcite{cohen2003value, daniel2006market, fama2008average}}.
\textcite{gerakos2017decomposing} note that much of the value factor relates to
changes in firm size and split the factor into size-driven value and
non-size-driven, ``orthogonal" value.
They find that not all high-BM firms make the value premium and some low-BM
firms do make the value premium.
This observation does not render the value factor useless, but it does help to
illustrate that the value premium is not the same as the ``empirical regularity
that average returns increase in BM ratios".
Further, size-driven value is more related to variance than orthogonal value.
I find the alpha in the small, high-Var bucket disappears for both models
when the size factor is omitted (not shown).
Omitting the size factor results in negative value slopes suggesting that,
without size clouding the issue, small, high-Var stocks behave like
unprofitable, growth stocks with poor recent returns just as table
\ref{tbl:Size_Beta} shows for small, high-beta stocks.
Small, high-Var firms have poor returns but the size factor pushes up
the predicted return from the model.
Furthermore, the size factor subsumes the value factor negating the value
slopes that could pull the predicted return from the model down and
reduce mispricing.
In keeping with the top-down approach, this does not mean the size factor
should be removed from the model.
Rather, the model provides a description of the problem.
The returns on small, high-Var stocks behave like the returns on small,
unprofitable stocks with poor recent performance.
Mispricing stems from firm size, through the relationship between size and
volatility, and not illiquidity, as has been the focus of prior research.
This is evident directly through the size slope inflating the predicted return
from model \ref{eq:B16} as well as indirectly through value's disappearance
in problem buckets.

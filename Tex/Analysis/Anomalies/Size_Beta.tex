% !TeX root=../../main.tex

\import{./Tables/}{Size_Beta_tbl}

Table \ref{tbl:Size_Beta} shows 25 time series regressions from size-beta
portfolios for models \ref{eq:B16} and \ref{eq:F16}.
The main messages of the table are that sorts on beta are not a problem for model
\ref{eq:B16} and value disappears in the small, high-beta bucket.
The beta characteristics, and volatility characteristics tied to beta, do not drive
mispricing in momentum sorts.
Positive momentum slopes for model \ref{eq:B16} capture the mispricing for 
model \ref{eq:F16} in the low-beta bucket.
Where \textcite{fama2016dissecting} find that low-beta stocks behave like
profitable, conservatively investing stocks, I find that low-beta stocks behave
like unprofitable, value stocks with strong recent returns.

Beta slopes increase with size and beta buckets.
Big stocks behave more like the return on the market than small stocks.
The spread in beta slopes is similar for small and big stocks,
$\sim$0.7 to $\sim$1.3.
Value slopes are flat across beta buckets except for a drop in the high-beta
bucket.
High-beta stocks behave like the returns on growth stocks.
Like the small, high-momentum portfolio, value disappears in the
small, high-beta portfolio although, unlike momentum,
this does not result in mispricing.
There is a size effect in value slopes for the high-beta bucket with slopes
decreasing from -0.06 for small stocks to -0.28 for big stocks.
Momentum slopes decrease as beta buckets increase.
Low-beta stocks behave like stocks with strong recent returns, while
high-beta stocks behave like stocks with poor recent returns.
This contradicts table \ref{tbl:Size_Prior} that shows that low \emph{and}
high-momentum stocks behave like the returns on high-beta stocks.
Like value, there is a size effect in momentum slopes for the high-beta bucket
with slopes decreasing from -0.1 for small stocks to -0.24 for big stocks.
Profitability slopes are largely insignificant except for the low-beta bucket
and the small, high-beta portfolio.
Profitability slopes are negative for the low-beta bucket and this makes sense
given profitability's relationship with defensive equity.

Beta and profitability slopes in table \ref{tbl:Size_Prior} point to a
shared story for momentum and volatility.
Table \ref{tbl:Size_Beta} shows that beta itself, and volatility tied to beta,
is not the problem.
In fact, table \ref{tbl:GRS} shows that sorts on beta cause the least problems
of all sorts tested, GRS of 1.15 for model \ref{eq:B16}.
Beta's lack of mischief is important because of the historical stubbornness of
the anomaly \parencite{jensen1972capital}.
Model \ref{eq:B16} drives out alphas in low-beta portfolios.
Next, I investigate mispricing and factor slopes for sorts on variance.

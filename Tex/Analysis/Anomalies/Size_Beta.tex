% !TeX root=../../main.tex

Table \ref{tbl:Size_Beta} shows 25 time series regressions from size-beta
portfolios for models \ref{eq:B16} and \ref{eq:F16}. The main message of the
table is that sorts on beta are not a problem for model \ref{eq:B16}. Momentum
slopes pick up the slack left in model \ref{eq:F16} to drive out the alphas in
the low-beta bucket. The shared story between momentum and beta points to a
shared story between momentum and volatility. There is only one intercept in
the 25 portfolios, in the big, third beta bucket. This does not have any
apparent economic meaning and may be purely statistical. Looking at table
\ref{tbl:GRS}, we see that the size-beta sorts cause the least problems of all
the sorts. Beta's lack of mischief is important because of the historical
stubbornness of the anomaly and beta's relationship with volatility. The
``flatness" of the relationship between beta and return
\parencite{jensen1972capital} is what kicked off the search for factors beyond
the market risk premium. Shortly afterwards, we had the size anomaly of
\textcite{banz1981relationship} and many more followed as detailed earlier in
the paper. Despite the explosion of factors, the flat-beta anomaly stuck around
including up to a recent analysis in \textcite{fama2016dissecting}.



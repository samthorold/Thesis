% !TeX root=../../main.tex

Table \ref{tbl:Size_Beta} shows 25 time series regressions from size-beta
portfolios for models \ref{eq:B16} and \ref{eq:F16}. The main message of the
table is that sorts on beta are not a problem for model \ref{eq:B16}. Momentum
slopes pick up the slack left in model \ref{eq:F16} to drive out the alphas in
the low-beta bucket. The shared story between momentum and beta points to a
shared story between momentum and volatility. There is only one intercept in
the 25 portfolios, in the big, third beta bucket. This does not have any
apparent economic meaning and may be purely statistical. Looking at table
\ref{tbl:GRS}, we see that the size-beta sorts cause the least problems of all
the sorts. Beta's lack of mischief is important because of the historical
stubbornness of the anomaly and beta's relationship with volatility. The
``flatness" of the relationship between beta and return
\parencite{jensen1972capital} is what kicked off the search for factors beyond
the market risk premium. Shortly afterwards, we had the size anomaly of
\textcite{banz1981relationship} and many more followed as detailed earlier in
the paper. Despite the explosion of factors, the flat-beta anomaly stuck around
including up to a recent analysis in \textcite{fama2016dissecting}.

For model \ref{eq:B16}, value slopes decrease as beta buckets increase and are
largely flat as size buckets increase except for a drop in slopes for the
biggest size bucket.
The returns on high-beta stocks behave like the returns on growth firms.
Momentum slopes exhibit a similar pattern.
High-beta stocks behave like firms with poor recent returns.
Momentum slopes do not exhibit smile across beta buckets as beta slopes did
across momentum buckets.
Low and high-momentum stocks behave like high-beta stocks but only low-beta
stocks behave like high-momentum stocks.
Profitability slopes are largely insignificant in the high-beta bucket and
negative for the low-beta bucket.
This relationship makes sense given profitability's role in describing the
returns to defensive equity stocks.

Low-beta stocks are a problem for model \ref{eq:F16} with three intercepts of
$\sim$0.2\% per month in the small, third and fourth size buckets.
The investment factor does not have much to add here as the slopes are
insignificant.
Model \ref{eq:B16} remedies the low-beta problems and shows that, while
low-beta stocks are generally unprofitable, they have strong past returns as
shown by the positive momentum slopes.
Overall, beta sorts do not help to explain the poor performance in momentum
sorts.

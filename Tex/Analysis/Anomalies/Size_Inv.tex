% !TeX root=../../main.tex

Table \ref{tbl:Size_Inv} shows 25 time series regressions from size-investment
portfolios for models \ref{eq:B16} and \ref{eq:F16}.
The main message of the table is that value picks up the slack left by
omitting investment.
For model \ref{eq:B16},
value and momentum slopes are positive apart from the high-Inv bucket
with little variation across size buckets except for a drop in value slope and
increase in momentum slope in the smallest bucket.
For aggressively investing stocks, the slopes for value, momentum and
profitability (not shown) are all negative.
The negative slopes show that the returns on aggressively investing stocks
behave like unprofitable growth firms with poor recent returns.
Importantly, value slopes vary across investment buckets ranging from ~0.2 in
the low-Inv bucket to ~-0.3 in the high-Inv bucket.
Model \ref{eq:B16} leaves three alphas in the high-Inv bucket,
although their absolute value does not exceed 0.25\% per month.
%Model \ref{eq:F16} performs poorly for small stocks.
Value slopes are smaller for model \ref{eq:F16} but this is expected given the
presence of the investment factor.
Given that value does most of the work in sorts on investment, sorting on value
and investment should result in poor performance due to a locak variation in
value slopes across investment buckets within value buckets.


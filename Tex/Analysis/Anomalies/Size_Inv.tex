% !TeX root=../../main.tex

Table \ref{tbl:Size_Inv} shows 25 time series regressions from size-investment
portfolios using B16 and F16 factors.
The main message of the table is that value picks up most of the slack left by
omitting investment.
For B16, value and momentum slopes are positive apart from the high-Inv bucket
with little variation across size buckets except for a drop in value slope and
increase in momentum slope in the smallest bucket.
For aggressively investing stocks, the slopes for value, momentum and
profitability (not shown) are all negative.
The negative slopes show that the returns on aggressively investing stocks
behave like unprofitable growth firms with poor recent returns.
Importantly, value slopes vary across investment buckets ranging from ~0.2 in
the low-Inv bucket to ~-0.3 in the high-Inv bucket.
The B16 model leaves three alphas in the high-Inv bucket,
although their absolute value does not exceed 0.25\% per month.
Comparing to the F16 model, B16 performs better among small stocks where F16
leaves more significant unexplained average returns.
Value slopes are smaller for F16 but this is expected given the presence of the
investment factor.


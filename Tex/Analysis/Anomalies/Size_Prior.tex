% !TeX root=.../main.tex

Table \ref{tbl:Size_Prior} shows 25 time series regressions from size-prior
portfolios using B16 and F17 factors.
I compare to F17, rather than F16, here to explore momentum slopes when
combined with monthly or annual value.
The main message of the table is that value is insignificant in the extremes of
momentum.
Performance is poor where the model cannot leverage the interaction between
value and momentum.
Value's disappearance makes some sense for the high-Prior bucket as value is
weakest among high-Prior stocks.
It is more of a puzzle why value is insignificant in the low-Prior bucket.

Slopes for the two models are very similar.
Value exhibits a frown with coefficients remaining ~0.3 in the middle momentum
buckets but largely insignificant in the extreme momentum buckets.
Momentum slopes increase uniformly from ~-0.7 to ~0.4 as momentum buckets
increase.
Profitability slopes also exhibit a frown but, in contrast to value, are
negative and significant in the extremes of momentum but largely insignificant
in the inner momentum buckets.
The investment factor in F17 offers little with largely insignificant slopes.
B16 struggles most with the high-Prior bucket with three significant intercepts
out of five.
F17 has similar problems in the high-Prior bucket but also struggles with the
small bucket with four significant intercepts out of five.

Profitability helps to describe the returns to ``defensive equity" strategies
\parencite{novy2014understanding}.
Such strategies often focus on low-beta, low-volatility stocks.
The negative profitability slopes may be an indicator of high volatility in the
extremes of momentum.
Momentum's interaction with volatility, rather than momentum itself, could
drive some the returns in the extremes of momentum.
Value's weakness in the low-momentum bucket may stem from returns behaving like
the returns on low-beta and/or low-volatility stocks.


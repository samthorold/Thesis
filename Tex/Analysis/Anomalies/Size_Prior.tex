% !TeX root=.../main.tex

\import{./Tables/}{Size_Prior_tbl}

Tables \ref{tbl:Size_Prior} shows 25
time series regressions from size-momentum portfolios for models
\ref{eq:B16} and \ref{eq:F16}.
The main message of the table is that value is insignificant in the extremes
of momentum.
Value's disappearance makes some sense for the high-Prior bucket as the value
effect is weakest among high-Prior stocks as shown in table
\ref{tbl:Size_BMm_Prior_chars}.
It is more of a puzzle why value is insignificant in the low-Prior bucket.

Slopes for the two models are very similar.
Value exhibits a frown with coefficients remaining $\sim$0.3 in the middle
momentum buckets but insignificant in the extreme momentum buckets.
Momentum slopes increase uniformly from $\sim$-0.7 to $\sim$0.4 as momentum
buckets increase.
Profitability slopes also exhibit a frown but, in contrast to value, are
negative and significant in the extremes of momentum but insignificant
in the inner momentum buckets.
The investment factor offers little to model \ref{eq:F16} with insignificant
slopes.
Model \ref{eq:B16} struggles most with the high-Prior bucket with three
significant intercepts out of five.
Model \ref{eq:F16} has similar problems in the high-Prior bucket but also
struggles with the small bucket with four significant intercepts out of five.
Beta slopes exhibit a smile across momentum buckets. Beta and variance are
linked as a stock's variance drives its covariance with the market.
Furthermore, profitability helps to describe the returns on ``defensive equity"
strategies \parencite{novy2014understanding}.
Such strategies often focus on low-beta, low-volatility stocks.
The negative profitability slopes may be an indicator of high beta and/or
high volatility in the extremes of momentum.
Momentum's interaction with volatility, rather than momentum itself, could
drive at least a portion of the returns in the extremes of momentum.
Value's puzzling weakness in the low-momentum bucket may stem from returns
behaving like the returns on high-beta and/or high-volatility stocks,
rather than low-momentum stocks.


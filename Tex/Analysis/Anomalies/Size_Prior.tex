% !TeX root=.../main.tex

\import{./Tables/}{Size_Prior_tbl}

Tables \ref{tbl:Size_Prior} shows 25 time series regressions from size-momentum
portfolios for models \ref{eq:B16} and \ref{eq:F16}.
Momentum is the tendency for stocks with higher recent returns than
their peers to continue to earn higher returns in the short term
\parencite{jegadeesh1993returns} and has been found in markets and
asset classes throughout the world \parencite{asness2013value}.
The main message of the table is that in the extremes of momentum,
value is insignificant and profitability slopes are negative.

Model \ref{eq:B16} struggles most with the high-Prior bucket with three
significant intercepts out of five.
Model \ref{eq:F16} has similar problems in the high-Prior bucket but also
struggles with the small bucket with four significant intercepts out of five.
The small, low-Prior, -0.55, and small, high-Prior, 0.54, intercepts for
model \ref{eq:F16} are reduced to -0.37 and 0.38 for model \ref{eq:B16}.
Improvements in mispricing for model \ref{eq:B16} are due to
the negative momentum slopes for low-Prior and positive momentum slopes for
high-Prior.

Beta slopes exhibit a smile across momentum buckets.
The small bucket has a drop from 1.06 in the low-Prior bucket to 0.86 in the
middle Prior bucket before returning to 1.02 in the high-Prior bucket.
Similarly, the big bucket has a drop from 1.16 to 0.94 before rising again to 1.09.
The extremes of momentum behave like the returns on high-beta stocks.
Value slopes exhibit a frown with coefficients remaining $\sim$0.3 in the
middle momentum buckets but becoming insignificant in the extreme momentum
buckets.
The extremes of momentum do not covary with the returns on the value factor.
Value does most of the heavy lifting along with profitability and
there is mispricing in the buckets where value disappears.
Momentum slopes increase uniformly from $\sim$-0.7 to $\sim$0.4 as momentum
buckets increase, although the spread in slopes is slightly larger for big
stocks than small stocks.
Profitability slopes also exhibit a frown but, in contrast to value, are
negative and significant in the extremes of momentum but insignificant
in the inner momentum buckets.

Beta and variance are linked as a stock's variance drives its covariance with
the market.
The returns on the extremes of momentum, and subsequent mispricing, may be due
to low and high-momentum identifying stocks with high volatility.
Furthermore, profitability slopes also point to high volatility stocks
because profitability helps to describe the returns on ``defensive equity"
strategies \parencite{novy2014understanding}.
Such strategies focus on low-beta, low-volatility stocks.
Momentum's interaction with volatility characteristics, rather than momentum
itself, could drive returns in the extremes of momentum.
Value's weakness in the extremes of momentum may stem from returns
behaving like the returns on high-beta and/or high-volatility stocks,
rather than low or high-momentum stocks.
Next, I investigate mispricing and factor slopes for sorts on beta.

% !TeX root=../../main.tex

Table \ref{tbl:Size_BMm_Prior} shows 32 time series regressions from
Size-$\text{BM}^{m}$-Inv portfolios for models \ref{eq:B16} and \ref{eq:F16}.
The main message of the table is that small, unprofitable growth firms have a
sizeable alpha of -0.66\% per month for \ref{eq:B16} and -0.91\% per month
for model \ref{eq:F16}.
This portfolio shares the negative slope coefficients on profitability and
investment in F16 dubbed the ``lethal" combination by Fama and French.

For model \ref{eq:B16}, value slopes increase with value buckets and prior
buckets.
This is different from the characteristic sorts in table
\ref{tbl:Size_BMm_Prior_chars} where value decreased with prior buckets.
Characteristic sorts are not bound to multivariate regression coefficients, but
a model will struggle to describe the returns on sorts where characteristics do
not match regression slopes.
Sorting on value as well as prior returns revives the value factor in the
small, low-prior bucket that disappeared in the sorts on size and prior returns
alone.

For model \ref{eq:F16}, the profitability factor does most of the work
describing the returns
on the small, low-value, low-prior bucket, although a very large alpha of -0.91
is left unexplained.
Profitability's power to describe returns on this bucket, even in the presence
of value and momentum in B16, suggests a shared story for the low returns to
small, growth firms with poor recent results.
Following from the findings in the size-prior sorts in table
\ref{tbl:25_Size_Prior_B16}, profitability slopes suggest returns on extreme
buckets of momentum behave like volatile stocks.
Table \ref{tbl:Size_BMm_Prior} shows that this is a particular problem for
growth stocks.

Prior research finds low-value stocks tend to overperform and high-value stocks
tend to underperform the returns predicted from factor models\footnote{
\textcite{fama1992cross, fama1993common, loughran1997book, fama2006value}
among others.}
For size-BM-Inv sorts, low-value alphas are positive and
high-value slopes are negative.
This mirrors findings in prior research.
For the size-$\text{BM}^m$-Prior sorts, low-value alphas are negative and
high-value alphas are positive.
This is contradictory to conventional wisdom but a possible explanation may lie
in the summary characteristics from sorts on size, value and prior returns.
The small, low-value and low-prior bucket in table
\ref{tbl:Size_BMm_Prior_chars} has a significantly lower return, 0.09\% per
month, than its peers.
The next highest return in either value or prior bucket is 0.76 and 0.69,
respectively.
The low alpha in the small, low-value, low-prior bucket for B16 may reflect
the crude sorts used to create the factors.
The factors are made from independent 2$\times$3 sorts on size and one other
variable while the test assets are made from 2$\times$4$\times$4 sorts
dependent on size.
\textcite{fama2015five} address this question but find similar alphas in finer
sorts on their own factor variables.
The magnitude of the unexplained return in the small, low-value, low-prior
bucket coupled with the significance of the profitability factor suggests an
economic reason rather than a problem with the way the factors are constructed.
We can examine the individual regression slopes for sorts on size-beta and
size-volatility to pick out where the problems may lie with momentum.


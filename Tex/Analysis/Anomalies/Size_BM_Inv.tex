% !TeX root=../../main.tex

Table \ref{tbl:Size_BM_Inv} shows 32 time series regressions from Size-BM-Inv
portfolios for models \ref{eq:B16} and \ref{eq:F16}.
The main message of the table is that constraining value decreases performance
in sorts on investment because value slopes cannot vary across investment
buckets within value buckets.
For F16, investment slopes vary considerably across investment buckets, ranging
from 0.61 to -0.37 for the small, low-BM bucket and from 0.27 to -0.31 for the
big, high-BM bucket.
Value varies similarly for both models but for model \ref{eq:B16} the variation
in momentum or profitability does not make up for the missing investment
factor.
Model \ref{eq:B16}'s inferior performance to model \ref{eq:F16} in sorts on
annual value and investment is not surprising because value is weak among
big stocks and by sorting on value we constrain the variation in value slopes
across investment buckets.
For model \ref{eq:F16}, value slopes increase as investment buckets increase
across small and big stocks.
This is counterinuitive given the positive correlation between value and
investment factors.
If the returns on high minus low value behave like the returns on conservative
minus aggressive investment, we would expact value slopes to be decreasing as
investment buckets.
For model \ref{eq:B16}, value slopes behave more as expected and they decrease
or exhibit a frown as investment buckets increase.
Value took up most of the slack in sorts on size and investment and by
isolating big stocks with independent sorts and constraining the variation in
value slopes with value sorts we limit value's ability to capture average
return across investment buckets.

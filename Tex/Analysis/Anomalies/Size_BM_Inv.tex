% !TeX root=../../main.tex

Table \ref{tbl:Size_BM_Inv} shows 32 time series regressions from Size-BM-Inv
portfolios using B16 and F16 factors.
The main message of the tables is that monthly value in B16 does not do enough
to capture average returns in the big half of the portfolios.
For F16, investment slopes vary considerably across investment buckets, ranging
from 0.61 to -0.37 for the small, low-BM bucket and from 0.27 to -0.31 for the
big, high-BM bucket.
Value varies similarly for both models but for B16 the variation in momentum or
profitability does not make up for the missing investment factor.
The B16 model's inferior performance to F16 in sorts on annual value and
investment is not surprising because value is weak among big stocks and by
sorting on value we constrain the variation in value slopes across investment
buckets.
For F16, value slopes increase as investment buckets increase across small and
big stocks.
This is counterinuitive given the positive correlation between value and
investment factors.
If the returns on high minus low value behave like the returns on conservative
minus aggressive investment, we would expact value slopes to be decreasing as
investment buckets.
For B16, value slopes behave more as expected and they decrease or exhibit a
frown as investment buckets increase.
Value took up most of the slack in sorts on size and investment/value and by
isolating big stocks with independent sorts and constraining the variation in
value slopes with value sorts we limit value's ability to capture average
return.

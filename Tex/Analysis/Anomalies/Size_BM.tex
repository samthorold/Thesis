% !TeX root=../../main.tex

Table \ref{tbl:Size_BM} shows 25 time series regressions from size-value
portfolios using B16 and F16 factors.
The main message of the table is that momentum slopes turn out to be the
achilles heel of B16. While monthly value slopes in B16 track annual value
slopes in F16 very closely, the additional negative slopes of momentum create
positive alphas in the big, low-Value buckets.
Value has been shown to be weaker among larger stocks, so it is not entirely
surprising that as B16 relies heavily on value to explain the cross-section of
stock returns that big stocks will pose a challange.
In tests of model performance pricing all stocks, the relationship between
value and momentum resulted in a lot of power.
Now that we have restricted stocks to subsets sorted on characteristics, value
and momentum's greatest strength turns out to be their greatest weakness.
Momentum getting in the way of value slopes is also visible in the differences
between momentum slopes in B16 and investment slopes in F16.
Momentum slopes track value slopes while investment slopes are largely
insignificant.


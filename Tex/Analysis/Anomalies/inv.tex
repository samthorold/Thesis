% !TeX root=.../main.tex

Tables \ref{tbl:25_Size_Inv_B16} and \ref{tbl:25_Size_Inv_F16} show 25 time series
regressions from size-investment portfolios using B16 and F16 factors, respectively. The
main message of the tables is that value picks up most of the slack left by omitting
investment. Value slopes are positive apart from the high-Inv bucket. Momentum slopes are
smaller in magnitude but behave similarly. For aggressively investing stocks, the slopes
for value, momentum and profitability are all negative. The negative slopes show that the
returns on aggressively investing stocks behave like unprofitable growth firms with poor
recent returns. The B16 model leaves three alphas in the high-Inv bucket, although their
absolute value does not exceed 0.25\% per month. Comparing to the F16 model, B16 performs
better among small stocks where F16 leaves more significant unexplained average returns.
Value slopes are smaller for F16 but this is expected given the presence of the investment
factor. Profitability slopes are similar across models and this reflects the lack of
correlation between value and momentum with profitability. Swapping factors with similar
explanatory power that are unrelated to profitability does not change profitability
slopes.

Tables \ref{tbl:32_Size_BM_Inv_B16} and \ref{tbl:32_Size_BM_Inv_F16} show 32 time series
regressions from Size-BM-Inv portfolios using B16 and F16 factors, respectively. The main
message of the tables is that monthly value in B16 does not do enough to capture average
returns in the big half of the portfolios. For F16, investment slopes vary considerably
across investment buckets, ranging from 0.61 to -0.37 for the small, low-BM bucket and
from 0.27 to -0.31 for the big, high-BM bucket. Value varies similarly for both models but
for B16 the variation in momentum or profitability does not make up for the missing
investment factor. The B16 model's inferior performance to F16 in sorts on annual value
and investment is not surprising because value is weak among big stocks and by sorting on
value we constrain the variation in value slopes across investment buckets. Value took up
most of the slack in sorts on size and investment and by isolating big stocks with
independent sorts and constraining the variation in value slopes with value sorts we limit
value's ability to capture average return.

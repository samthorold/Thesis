% !TeX root=../main.tex

Table \ref{tbl:bootstrap} shows the maximum squared Sharpe ratio for seven models as well
as the 90\% confidence intervals for 100,000 bootstrap simulations of the difference in
squared Sharpe ratio between the model in the row and the model in the column. The main
message of the table is that adding the investment factor to market, size, monthly value,
momentum and cash profitability factors, B16 model, does not reliably improve the Sharpe
ratio of the factors (the 90\textsuperscript{th} percentile of $Sh^2(B16b)-Sh^2(B16)$ is
0). The table also shows the benefits of combining monthly value, momentum and cash
profitability as both monthly value and momentum combined and cash profitability alone
dramatically increase the squared Sharpe ratio of the factors.

In panel A we see the maximum squared Sharpe ratio of the modified
\textcite{ball2016accruals} model I propose, B16, is 0.316. The squared Sharpe ratio is
not improved by adding the investment factor, B16b, as it remains 0.316. Adding the
momentum factor, F17, to the five-factor model of \textcite{fama2016choosing}, F16,
improves the squared Sharpe ratio from 0.225 to 0.24. Modifying the profitability to
account for accruals as in the F16 model icreases the squared Sharpe ratio of the five-
factor model of \textcite{fama2015five}, F15, from 0.099. To compare to non-nested models
I also show the four-factor model of \textcite{carhart1997persistence}. The four-factor
model combining annual value with momentum, C97, has a squared Sharpe ratio of 0.09. The
four-factor model that combines monthly value with momentum has a squared Sharpe ratio of
0.136.

In panel B we see the results in panel A are, for the most part, believable. Those models
with higher squared Sharpe ratios than their peers in panel A have 90\% confidence
intervals that are positive in panel B. By randomly drawing months with replacement in a
bootstrap procedure I check that the time series of returns we have observed is not
drastically different from those we might have observed in another universe. This is
important because when analysing returns we do not have the benefit of seeing the
distribution at each point in time, we see only the distribution across time. Running many
bootstrap procedures gives a distribution of squared Sharpe ratios.

Panel B also shows that F15 is not reliably better than either specification of the
Carhart model. The confidence interval for $Sh^2(F15)-Sh^2(C97(b))$ is not strictly
positive. Value and momentum combine to perform similarly to profitability and investment.
This is striking because profitability and investment are credited with reducing the
number of anomalies and reducing the defining characteristics of many to small,
unprofitable stocks that somehow invest aggressively \parencite{fama2015five,
fama2016dissecting}.

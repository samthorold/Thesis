% !TeX root=../main.tex

\import{./Tables/}{Size_BMm_Prior_chars_tbl}
\import{./Tables/}{Size_BM_Inv_chars_tbl}

Table \ref{tbl:Size_BMm_Prior_chars} shows average value-weighted
characteristics for 32 size-$\text{BM}^m$-Prior portfolios and table
\ref{tbl:Size_BM_Inv_chars} shows average value-weighted characteristics for 32
size-BM-Inv portfolios. The main message of the tables is that monthly value
and momentum's relationship with small, growth stocks' future growth in BE is
the opposite that of annual value and investment. For monthly value and
momentum, future growth in BE increases from 0.31 to 0.61 as momentum buckets
increase. For annual value and investment, this relationship is reversed as the
future growth in BE decreases from 0.59 to 0.4 as investment buckets increase.

The variation in characteristics in table \ref{tbl:Size_BMm_Prior_chars} drive
many of the results for subsets of stocks.
A model has the potential to describe the
returns on a given sort where the characteristics used to construct factors are
able to create variation in the characteristic(s) used to construct the subset
of stocks. If increasing one characteristic increases another, this will drive
a positive and significant regression slope in time series regressions.
Nevertheless, summary characteristics are not mathematically bound to
regression slopes for returns constructed with those characteristics.
A model will struggle where returns do not behave like the characteristics they
are constructed from.
These deviations are easy to identify because the change in
regression slopes across buckets will not match the changes in the underlying
characteristics across the same buckets. It is more difficult to identify
\emph{why} returns do not follow the underlying characteristics.

Returns across $\text{BM}^m$ and Prior buckets range from 0.09 in the small,
low-$\text{BM}^m$ and low-Prior bucket to 2.04 in the small, high-$\text{BM}^m$
and high-Prior bucket. Returns across BM and Inv buckets range from 1.01 in the
big, low-BM and low-Inv bucket to 1.46 in the small, high-BM and low-Inv
bucket. The greater variation in returns for size-$\text{BM}^m$-Prior
portfolios is expected given the interaction between value and momentum. Their
interaction is also visible in the variation in returns at each
characteristic's extreme. For example, in low-$\text{BM}^m$ the variation in
returns across Prior buckets is much larger, 0.09 to 1.47 for small stocks and
0.55 to 0.96 for big stocks, than in high-$\text{BM}^m$ buckets, 1.01 to 2.04
for small stocks and 1.21 to 1.42 for big stocks. The same is true of the
variation in returns across $\text{BM}^m$ buckets in the extreme Prior buckets.
Lower variation of returns across one characteristic in the extreme buckets of
the other mirrors findinings in \textcite{asness2013devil}. The higher
variation in returns for size, value and momentum sorts provides potential to
describe the returns on more subsets of stocks than size, value, investment
sorts.

For both size-$\text{BM}^m$-Prior and size-BM-Inv portfolios, $\text{BM}^m$ is
flat across Prior/Inv buckets and increases uniformly as $\text{BM}^m$/BM
buckets increase. $\text{BM}^m$ is negatively related to Prior because strong
recent returns will push ME up. This is not reflected in the Prior buckets
because portfolios are rebalanced monthly. $\text{BM}^m$ is positively related
to Inv as shown by the positive regression slopes in table
\ref{tbl:redundant_CMA}. This is not reflected in the Inv buckets because value
is controlled for by sorting on BM. For size-$\text{BM}^m$-Prior portfolios, BM
increases as Prior buckets increase. For example, in the high-$\text{BM}^m$
bucket, BM increases from 1.48 to 2.88 for small stocks and from 1.07 to 1.54
for big stocks. This makes sense if we consider a stock that is in the
high-$\text{BM}^m$, high-Prior bucket in July i.e. just after annual BM is
recalculated. At this point in the year, $\text{BM}^m$ and BM are probably at
their most similar because the ME values
used differ by six months as opposed to the
maximum of eighteen. If a stock has relatively high returns in the following
months, high Prior, its $\text{BM}^m$ will decrease as market equity increases
while book equity is held constant by the lack of new financial statements. For
a stock to remain in the high-$\text{BM}^m$, high-Prior bucket, it must have
had particularly high BM in order for high returns not to drag the
$\text{BM}^m$ down sufficiently to take the stock out of the
high-$\text{BM}^m$, high- Prior bucket. This effect is not seen in the
size-BM-Inv sorts.
For the size-$\text{BM}^m$-Prior sorts, Prior increases uniformly
with Prior buckets and is flat across $\text{BM}^m$ buckets.
For the size-BM-Inv sorts, there is little variation in Prior across either
BM or Inv buckets.
The lack of variation in momentum characteristics means annual value and
investment will struggle to describe the returns on momentum
sorts.

For both size-$\text{BM}^m$-Prior and size-BM-Inv portfolios,
growth firms are more profitable than value firms.
Within value buckets, aggressively investing firms are less profitable than
conservatively investing firms.
The small, low-value, low-investment bucket is much more profitable than any
other bucket with an average value-weighted profitability measure of 2.51.
Importantly, there is very little variation in CP characteristics across
momentum buckets within value buckets for size-$\text{BM}^m$-Prior sorts.
Controlling for value, momentum and profitability characteristics are unrelated
except for a frown in the small, growth bucket.

Future variance is strongly related to size. Small stocks are more volatile
than big stocks with the average level and variation in variance
characteristics greater for small stocks. Variance decreases marginally, from
$\sim$0.2 to $\sim$0.1, with momentum/investment buckets but only in small
stocks. For the most part, sorts on value and momentum/investment do not create
variation in variance characteristics. Monthly value and momentum as well as
annual value and investment will struggle to describe the returns on subsets of
stocks constructed from sorts on variance.

Future investment decreases across both $\text{BM}^m$ and Prior buckets for
both small and large stocks. In contrast to returns, the effects of each
characteristic are similar in the extreme buckets of the other. For example, in
the low-Prior bucket, Inv decreases from 0.39 to 0.2 for small stocks and
remains flat for big stocks. In the high-Prior bucket, Inv decreases from 0.25
to 0.06 for small stocks and from 0.22 to 0.09 for big stocks. In the
low-$\text{BM}^m$ bucket, Inv decreases from 0.39 to 0.25 for small stocks and
remains flat for big stocks. In the high-$\text{BM}^m$ bucket, Inv decreases
from 0.2 to 0.06 for small stocks and from 0.2 to 0.09 for big stocks. The
interaction of monthly value and momentum accounts for the variation in Inv in
small and big stocks. In size-BM-Inv sorts, Inv increases uniformly across Inv
buckets and remains flat across BM buckets. Controlling for investment means we
do not observe the positive relationship in table \ref{tbl:redundant_CMA}.

In size-$\text{BM}^m$-Prior sorts, future growth in BE in the low-$\text{BM}^m$
bucket increases from 0.31 to 0.61 as Prior increases. In contrast, future
growth in BE in the low-BM bucket increases as Inv decreases from 0.59 to 0.4
in size-BM-Inv sorts. For both sorts, future growth in BE moves in the opposite
direction to the Inv characteristic for small, growth stocks.
For size-$\text{BM}^m$-Prior sorts, Inv
increases while future growth in BE decreases while for size-BM-Inv sorts, Inv
decreases while future growth in BE increases. Using the dividend discount
model logic of Fama and French, Inv is used as a proxy for changes in BE. This
divergence of logic and data suggests asset growth, Inv, is not the best proxy
for changes in BE.

Other future growth in BE buckets are largely flat for both sorts. For the
size-$\text{BM}^m$-Prior sorts, the high-$\text{BM}^m$ bucket has negative,
future growth in BE, -0.1, for the low-Prior bucket rising to -0.03 for the
high-Prior bucket. This mirrors findings in \textcite{kok2017facts} that
high-$\text{BM}^m$ stocks are mean reverting through a future decrease in BE.
Furthermore, the increase to -0.03 for high-Prior stocks from -0.1
for low-Prior mirrors their findings that momentum somewhat weeds out
those high-$\text{BM}^m$ stocks that will exhibit decreases in BE in the
future.


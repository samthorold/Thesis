% !TeX root=../main.tex

Table \ref{tbl:BMm_Prior_future_BE} shows average value-weighted characteristics for 32
size-$\text{BM}^m$-Prior portfolios while table \ref{tbl:BM_Inv_future_BE} shows average
value-weighted characteristics for 32 size-BM-Inv portfolios. The main message of the
tables is that monthly value and momentum's relationship with small, growth stocks' future
growth in BE is the opposite that of annual value and investment. For monthly value and
momentum, future growth in BE increases from 0.42 to 0.62 as momentum buckets increase.
For annual value and investment, this relationship is reversed as the future growth in BE
decreases from 0.6 to 0.42 as investment buckets increase. This change is large in the
context of the other buckets, where future growth in BE rarely exceeds 0.15.

Returns across $\text{BM}^m$ and Prior buckets range from -0.01 in the small,
low-$\text{HML}^m$ and low-Prior bucket to 1.88 in the small, high-$\text{BM}^m$ and high-
Prior bucket. Returns across HML and Inv buckets range from 0.96 in the big, low-HML and
low-Inv bucket to 1.43 in the small, high-HML and low-Inv bucket. The greater variation in
returns for size-$\text{BM}^m$-Prior portfolios is expected given the interaction between
value and momentum. The interaction between value and momentum is also visible in the
variation in returns at each characteristic's extreme. For example, in low-$\text{BM}^m$
the variation in returns across Prior buckets is much larger, -0.01 to 1.47 for small
stocks and 0.53 to 1.2 for big stocks, than in high-$\text{BM}^m$ buckets, 1 to 1.88 for
small stocks and 1.01 to 1.41 for big stocks. The same is true of the variation in returns
across $\text{BM}^m$ buckets in the extreme Prior buckets. Lower variation of returns
across one characteristic in the extreme buckets of the other mirrors findinings in
\textcite{asness2013devil}.

For both size-$\text{BM}^m$-Prior and size-BM-Inv portfolios, $\text{BM}^m$ is flat across
Prior/Inv buckets and increases uniformly as $\text{BM}^m$/BM buckets increase.For
size-$\text{BM}^m$-Prior portfolios, BM increases as Prior buckets increase. For example,
in the high-$\text{BM}^m$ bucket, BM increases from 1.51 to 2.91 for small stocks and from
1.07 to 1.53 for big stocks. This makes sense if we consider a stock that is in the
high-$\text{BM}^m$, high- Prior bucket in July i.e. just after annual BM is recalculated.
At this point in the year, $\text{BM}^m$ and BM are probably at their most similar. If a
stock has relatively high returns in the following months, high Prior, its $\text{BM}^m$
will decrease as market equity increases while book equity is held constant by the lack of
new financial statements. For a stock to remain in the high-$\text{BM}^m$, high-Prior
bucket, it must have had particularly high BM in order for high returns not to drag the
$\text{BM}^m$ down sufficiently to take the stock out of the high-$\text{BM}^m$, high-
Prior bucket. This effect is not seen in the size-BM-Inv sorts. For the
size-$\text{BM}^m$-Prior sorts, Prior increases uniformly with Prior buckets and is flat
across $\text{BM}^m$ buckets. For the size-BM-Inv sorts, there is little variation in
Prior across either BM or Inv buckets.

Investment decreases across both $\text{BM}^m$ and Prior buckets for both small and large
stocks. In contrast to returns, the effects of each characteristic are strongest in the
extreme buckets of the other. For example, in the low-Prior bucket, Inv decreases from
0.37 to 0.19 for small stocks and from 0.22 to 0.21 for big stocks. In the high-Prior
bucket, Inv decreases from 0.24 to 0.06 for small stocks and from 0.22 to 0.09 for big
stocks. In the low-$\text{BM}^m$ bucket, Inv decreases from 0.37 to 0.24 for small stocks
and stays flat 0.22 for big stocks. In the high-$\text{BM}^m$ bucket, Inv decreases from
0.19 to 0.06 for small stocks and from 0.21 to 0.09 for big stocks. The interaction of
monthly value and momentum accounts for the variation in Inv in small and big stocks. In
size-BM-Inv sorts, Inv increases uniformly across Inv buckets and remains flat across BM
buckets.

In size-$\text{BM}^m$-Prior sorts, future growth in BE in the low-$\text{BM}^m$ bucket
increases from 0.42 to 0.62 as Prior increases. In contrast, future growth in BE in the
low-BM bucket increases as Inv decreases from 0.6 to 0.42 in size-BM-Inv sorts. For both
sorts, future growth in BE moves in the opposite direction to the Inv characteristic. For
size-$\text{BM}^m$-Prior sorts, Inv increases while future growth in BE decreases while
for size-BM-Inv sorts, Inv decreases while future growth in BE increases. Using the
dividend discount model logic of Fama and French, Inv is used as a proxy for changes in
BE. This divergence of logic and data suggests either the dividend discount model is wrong
or Inv is not the best proxy for changes in BE.

Other future growth in BE buckets are largely flat for both sorts. For the
size-$\text{BM}^m$-Prior sorts, the high-$\text{BM}^m$ bucket has negative, -0.09, future
growth in BE for the low-Prior bucket rising to -0.02 for the high-Prior bucket. This
mirrors findings in \textcite{kok2017facts} that high-$\text{BM}^m$ identifies stocks that
will exhibit decreases in BE in the future. Furthermore, the increase to -0.02 for high-
Prior mirrors their findings that momentum weeds out those high-$\text{BM}^m$ stocks that
will exhibit decreases in BE in the future.

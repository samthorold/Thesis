% !TeX root=../main.tex

Using a top-down approach, I find model \ref{eq:B16} performs best for all
stocks.
This does not necessarily mean the model performs best for all subsets of
stocks.
%The GRS statistic gives a measure of how much the Sharpe ratio of a model can
%be improved by investing in the test assets as well as the factors.
Table \ref{tbl:GRS} shows the GRS statistic for selected anomaly sorts across
the seven models from table \ref{tbl:bootstrap}.
A lower GRS statistic indicates a better description of the returns on the
subset of stocks.
%The anomalies are from Dr Kenneth French's website except for the size,
%monthly value and momentum  portfolios which are constructed from CRSP data.
As the number of anomaly portfolios increases, the subset of stocks converges
on all stocks.
Consequently, we would expect the GRS statistic evidence for all of the
anomalies to match the Sharpe ratio evidence and this is largely the case.
The main messages of table \ref{tbl:GRS} are that the GRS evidence matches the
Sharpe ratio evidence and that momentum and volatility sorts are a problem for
all models.
Table \ref{tbl:GRS} includes the value, profitability and investment anomalies
\parencite{fama2006profitability, fama2015five},
the accruals, beta, net issues, momentum and volatility anomalues
\parencite{fama2016dissecting} as well as a sort on size, monthly value and
momentum.
The anomalies are grouped to make three points;
constraining annual value hinders the ability of model \ref{eq:B16} to describe
returns,
constraining monthly value is far less disastrous for model \ref{eq:B16}
than other models and
in sorts on all variables that are not annual value
model \ref{eq:B16} outperforms model \ref{eq:F16} (+ WML).
The variation in characteristics seen in tables \ref{tbl:Size_BMm_Prior_chars}
and \ref{tbl:Size_BM_Inv_chars} drive the results in table \ref{tbl:GRS}
for models \ref{eq:B16} and \ref{eq:F16}.

For sorts on annual value (BM), the variation in the BM characteristic within
monthly value ($\text{BM}^m$) buckets drives the disappointing GRS results for
model \ref{eq:B16}.
%By sorting on value we are in effect controlling value.
%The sorts on monthly value show that the two specifications of value are
%sufficiently different that controlling for $\text{BM}^m$ is not the same as
%controlling for BM and there is undesired variation in BM within $\text{BM}^m$
%buckets.
In contrast, BM is flat within BM buckets and model \ref{eq:F16}, GRS of 1.62,
provides a better description of the returns to sorts on BM than model
\ref{eq:B16}, GRS of 1.81.
For sorts on $\text{BM}^m$ and momentum (Prior), model \ref{eq:B16} is
``playing at home" in that the model contains factors for all sort variables.
This creates variation in $\text{BM}^m$ and Prior not present in sorts on BM
and investment (Inv). Model \ref{eq:B16}, GRS of 4.85, outperforms model
\ref{eq:F16}, GRS of 6.52, for sorts on $\text{BM}^m$ and Prior.
The improvement in sorts on $\text{BM}^m$ and Prior provided by model
\ref{eq:B16} is enough to overcome the problems in sorts on BM.
Model \ref{eq:B16} performs best in sorts on value with a GRS of 2.83 compared
to the next best of 3.21 for model \ref{eq:F16}.

Model \ref{eq:B16} describes sorts on
variables not used to contruct the factors better than model \ref{eq:F16}.
For example, accruals, beta, net issues, and variance anomalies all
increase the Sharpe ratio of model \ref{eq:B16} less than model \ref{eq:F16}.
Part of the improvement can be attributed to statistics alone.
Sorts on $\text{BM}^m$ and Prior create much greater variation in returns than
sorts on BM and Inv.
By construction, model \ref{eq:B16} will capture more variation in returns than
model \ref{eq:F16}.
Tables \ref{tbl:Size_BMm_Prior_chars} and \ref{tbl:Size_BM_Inv_chars} show that
value and momentum provide a better proxy for changes in BE than value and
investment.
Value and momentum help to identify those portfolios constructed from sorts on
accruals, beta, net issues and variance that will exhibit changes in BE in the
future better than value and investment.
Momentum sorts are more puzzling because model \ref{eq:B16}, GRS of 3.62,
and model \ref{eq:F16}, GRS of 3.59, with the momentum factor both perform
poorly.

GRS evidence across models adds colour to the Sharpe evidence in table
\ref{tbl:bootstrap}.
Two of he biggest improvements for the four-factor model with monthly value
over the original five-factor model come from sorts on accruals,
GRS of 3.1 and 3.89 respectively, and beta, GRS of 1.22 and 1.83.
The four-factor model also describes sorts on momentum better than the original
five-factor model and the five-factor model with cash profitability but this is
expected given the four-factor model contains the momentum factor.
The original four-factor model ourperforms the four-factor model with monthly
momentum for sorts on annual value, constrained BM GRS of 2.31 and 2.69
respectively.
This supports the idea that undesired variation in $\text{BM}^m$ within BM
buckets results in poor performance since the only difference in the models is
the value specification.

The largest problems for all models are caused by momentum and volatility
sorts.
Individual regression sorts can help explore why this may be the case by
showing which particular sorts leave significant alpha.
The sign of factor regression slopes will show which end of the factor return
spread problem portfolios behave like.
For example, a negative (postive) profitability slope suggests returns behave
like the returns on unprofitable (profitable) stocks.
%Performance in sorts on investment is improved despite omitting the investment
%factor.
%Individual regression slopes will show which factors pick up the slack
%in investment's absence.
%Furthermore, since we have already seen the relationship between value,
%momentum and investment and that monthly value misbehaves within annual value
%buckets, we can confirm that model \ref{eq:B16} fails predictably.
%If value and investment are controlled for, the comparatively poor performance
%observed in sorts on value and investment should be due to a lack of variation
%in value slopes across investment buckets within value buckets.
I focus on models \ref{eq:B16} and \ref{eq:F16} individual regression slopes
because they have two of the highest Sharpe ratios and are not nested.


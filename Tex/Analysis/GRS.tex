% !TeX root=../main.tex

\import{./Tables/}{GRS_tbl}

Tables \ref{tbl:Size_BMm_Prior_chars} and \ref{tbl:Size_BM_Inv_chars} show
model \ref{eq:B16} has greater potential to price the returns on many
characteristic sorts.
Summary characteristics are not bound to regression slopes and a model will
struggle where returns do not behave like the characteristics they are
constructed from.
Table \ref{tbl:GRS} shows the GRS statistic for selected anomalies across
the seven models from table \ref{tbl:bootstrap}.
A lower GRS statistic indicates less mispricing.
As the number of anomaly portfolios increases, the subset of portfolios converges
on all portfolios.
Consequently, we would expect the GRS statistic evidence for all
anomalies to match the Sharpe ratio evidence and this is the case.
The main messages of table \ref{tbl:GRS} are that the GRS evidence matches the
Sharpe ratio evidence and that momentum and volatility sorts are a problem for
all models.

Table \ref{tbl:GRS} includes; value, profitability and investment anomalies
\parencite{fama2006profitability, fama2015five};
accruals, beta, net issues, momentum and volatility anomalues
\parencite{fama2016dissecting}; and a sort on size, monthly value and
momentum.
The anomalies are grouped to illustrate that;
constraining annual value hinders the ability of model \ref{eq:B16} to price returns,
constraining monthly value is the least disastrous for model \ref{eq:B16}, and
in sorts on all variables that are not annual,
value model \ref{eq:B16} outperforms model \ref{eq:F16} (+ WML).
% For sorts on annual value (BM), the variation in the BM characteristic within
% monthly value ($\text{BM}^m$) buckets drives the disappointing GRS results for
% model \ref{eq:B16}.

Value compensates for the missing investment factor in model \ref{eq:B16}.
Constraining value limits the variation in value across other characteristics.
Model \ref{eq:B16}, GRS of 1.81, has higher mispricing than model \ref{eq:F16},
GRS of 1.58, for BM constrained sorts.
This impact is particularly evident in size-BM-Inv sorts where the GRS increases
from 1.49 for model \ref{eq:F16} to 2.13 for model \ref{eq:B16}.
Comparing the GRS for the four-factor models confirms the specification of value,
rather than including momentum, is the problem.
The GRS for the original four-factor model with annual value, 2.31, is lower
than the GRS for the four-factor model with monthly value, 2.69, for all sorts on BM.
The only difference between the two models is the specification of value and
this must drive any differences in mispricing.

The size-$\text{BM}^m$-Prior sort shows that monthly value and momentum are
difficult to price for all models.
The GRS for all models is highest for this sort for all models.
Model \ref{eq:B16} struggles for the size-$\text{BM}^m$-Prior sort in an
absolute sense, GRS of 4.85.
Relative to other models, model \ref{eq:B16} performs well.
Model \ref{eq:F16} has a GRS of 6.52 and the original five-factor model has
a GRS of 7.55.
Updating the measure of value monthly reduces mispricing more than including
the momentum factor.
This is shown by the improvement in the GRS, 7.26 to 5.91, from the original
four-factor model to the modified four-factor model with monthly value.
In contrast, adding momentum to model \ref{eq:F16} helps very little.
The GRS drops from 6.52 to 6.51.

For all sorts not involving BM, model \ref{eq:B16} has the lowest mispricing,
GRS of 4.85 for size-$\text{BM}^m$-Prior and 2.23 for selected anomalies.
This supports Sharpe evidence in table \ref{tbl:bootstrap} that model \ref{eq:B16}
minimizes mispricing for all assets.
The relatively poor performance of model \ref{eq:B16} compared to model \ref{eq:F16}
for sorts on BM is not enough to cancel out the gains made in all other sorts.
The GRS for all sorts is 2.23 for model \ref{eq:B16} and 2.44 for model \ref{eq:F16}.
GRS evidence adds colour to the Sharpe ratio bootstrap evidence.
For example, table \ref{tbl:bootstrap} shows Sharpe ratio of the modified
four-factor model is not reliably higher than the Sharpe ratio of the
original five-factor model but does not indicate why.
Table \ref{tbl:GRS} shows Sharpe evidence is not clear beacuse the modified
four-factor model has lower mispricing for accruals, beta, and momentum but
otherwise underperforms the original five-factor model.

The largest problems for all models are caused by momentum and volatility
sorts.
Individual regression sorts can help explore why this may be the case by
showing which particular portfolios leave significant alpha.
The sign of factor regression slopes will show which end of the factor return
spread problem portfolios behave like.
For example, a negative profitability slope suggests returns behave
like the returns on unprofitable stocks.
I focus individual regression slopes for models \ref{eq:B16} and \ref{eq:F16}
because they have two of the highest Sharpe ratios and are non-nested.

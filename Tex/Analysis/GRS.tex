% !TeX root=../main.tex


Model \ref{eq:B16} performs best for all stocks but this does not necessarily
mean the model performs best for all subsets of stocks.
The GRS statistic gives a measure of how much the Sharpe ratio of a model can
be improved by investing in the test assets as well as the factors.
A lower GRS statistic indicates a better description of the returns on the
test assets.
Table \ref{tbl:GRS} shows the GRS statistic for selected anomaly sorts across
the seven models from table \ref{tbl:bootstrap}.
The anomalies are from Dr Kenneth French's website except for the size, monthly
value and momentum  portfolios which are constructed from CRSP data.
The anomalies represent a broad selection of historically troublesome sorts and
include variables used to construct the factors as well as variables that do
not feature in any of the seven models.
As the number of anomaly portfolios increases, the test assets converge on
all assets.
Consequently, we would expect the GRS statistic evidence for all of the
anomalies to match the Sharpe ratio evidence and this is largely the case.
The main messages of table \ref{tbl:GRS} are that the GRS evidence matches the
Sharpe ratio evidence and that momentum and volatility sorts are a problem for
all models.

Table \ref{tbl:GRS} includes the value, profitability and investment anomalies
\parencite{fama2006profitability, fama2015five},
the accruals, beta, net issues, momentum and volatility anomalues
\parencite{fama2016dissecting} as well as a sort on size, monthly value and
momentum.
The anomalies are grouped to make three points;
constraining annual value hinders the ability of model \ref{eq:B16} to describe
returns,
constraining monthly value is far least disastrous for model \ref{eq:B16} and
in sorts on all variables that are not annual value
model \ref{eq:B16} outperforms model \ref{eq:F16} (+ WML).

The variation in characteristics seen in tables \ref{tbl:Size_BMm_Prior_chars}
and \ref{tbl:Size_BM_Inv_chars} drive the results in table \ref{tbl:GRS}.
For sorts on annual value (BM), the variation in the BM characteristic within
monthly value ($\text{BM}^m$) buckets drives the disappointing GRS results for
model \ref{eq:B16}.
By sorting on value we are in effect controlling value.
The sorts on monthly value show that the two specifications of value are
sufficiently different that controlling for $\text{BM}^m$ is not the same as
controlling for BM and there is undesired variation in BM within $\text{BM}^m$
buckets.
In contrast, BM is flat within BM buckets and model \ref{eq:F16}, GRS of 1.62,
provides a better description of the returns to sorts on BM than model
\ref{eq:B16}, GRS of 1.81.
For sorts on $\text{BM}^m$ and momentum (Prior), model \ref{eq:B16} is
``playing at home" in that the model contains factors for all sort variables.
This creates variation in $\text{BM}^m$ and Prior not present in sorts on BM
and investment (Inv). Model \ref{eq:B16}, GRS of 4.85, outperforms model
\ref{eq:F16}, GRS of 6.52, for sorts on $\text{BM}^m$ and Prior.
Model \ref{eq:B16} describes sorts on
variables not used to contruct the factors better than model \ref{eq:F16}.
For example, accruals, beta, net issues, and variance anomalies all
increase the Sharpe ratio of model \ref{eq:B16} less than model \ref{eq:F16}.
Part of the improvement can be attributed to statistics alone.
Sorts on $\text{BM}^m$ and Prior create much greater variation in returns than
sorts on BM and Inv.
By construction, model \ref{eq:B16} will capture more variation in returns than
model \ref{eq:F16}.



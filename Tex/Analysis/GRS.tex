% !TeX root=../main.tex

Table \ref{tbl:GRS} shows the GRS statistic of \textcite{gibbons1989test} for the seven
models in table \ref{tbl:bootstrap} across selected anomaly sorts. The main message of the
table is that GRS evidence supports the $\text{Sh}^2$ evidence that B16, with a GRS
statistic of 1.91, better describes the cross-section of stock returns than F17,  with a
GRS statistic of 2.05. The table also shows that B16 struggles in sorts on annual value,
BM, where the GRS statistic is above that of models that include annual value in place of
monthly value. Furthermore, the sensitivity of the GRS test itself to the choice of test
assets is highlighted by the impact of the size-$\text{BM}^m$-Prior sort that tilts the
GRS evidence for value sorts in favour of B16.

The GRS statistic gives a measure of how far from an ``efficient" portfolio the factors
are in the context of the test assets. The GRS \emph{test} answers the question: do the
factors span the portfolio that includes the test assets and factors? While the GRS
\emph{statistic} gives a measure of \emph{how far} the factors are from spanning the
portfolio that includes the test assets and factors. This means we can compare different
models' ability to explain the returns on subsets of stocks. In general, the ability to
price the returns on subsets of stocks matches the ability to price the returns on all
stocks given by the maximum squared Sharpe ratio.

B16 performs poorly in absolute terms in sorts on Size-Var, Size-RVar,
Size-$\text{BM}^m$-Prior (monthly value and momentum) and Size-Prior (momentum). While the
GRS statistic can identify sorts that cause problems for factors models, it does not help
with why sorts cause problems. I examine individual intercepts and slopes from time series
regressions for the portfolios from these sorts. Since I omit the investment factor, I
include time series regressions for the Size-Inv (investment) and Size-OP-Inv
(profitability from \textcite{fama2006profitability} and investment) to investigate how
value and momentum slopes describe the returns on investment sorts.

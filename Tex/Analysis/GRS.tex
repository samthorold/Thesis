% !TeX root=../main.tex

B16 performs poorly in absolute terms in sorts on Size-Var, Size-RVar,
Size-$\text{BM}^m$-Prior (monthly value and momentum) and Size-Prior (momentum). While the
GRS statistic can identify sorts that cause problems for factors models, it does not help
with why sorts cause problems. I examine individual intercepts and slopes from time series
regressions for the portfolios from these sorts. Since I omit the investment factor, I
include time series regressions for the Size-Inv (investment) and Size-OP-Inv
(profitability from \textcite{fama2006profitability} and investment) to investigate how
value and momentum slopes describe the returns on investment sorts.

Table \ref{tbl:GRS} shows the GRS statistic for selected anomaly sorts across
the seven models from table \ref{tbl:bootstrap}.
The main message of the table is that GRS evidence broadly lines up with
$\text{Sh}^2$ evidence from table \ref{tbl:bootstrap} but my proposed model,
B16, struggles where value is constrained by sorts on the value characteristic.
Anomalies are chosen based on prior research.
\textcite{fama2015five} test the five-factor model, F15, with finer sorts on
the characteristics used to create the factors.
\textcite{fama2016dissecting}, following intuion in
\textcite{lewellen2010skeptical}, test the five-factor model with sorts on
charatceristics not used to create the factors.
The two-way sorts are independent 5$\times$5 sorts on size and one other
characteristic giving 25 test portfolios.
The three-way sorts first split the data in half based on size before making
independent 4$\times$4 sorts on two other characteristics giving 32 test
portfolios.
The GRS statistics are a measure of how much the $\text{Sh}^2$ of a model can
be improved by investing in the test portfolios as well as the factors.
If the $\text{Sh}^2$ cannot be improved, the factors are said to ``span" the
test portfolios and the factors describe the returns on the portfolios.

The table is split to make three points;
constraining annual value hinders the ability of B16 to describe returns,
constraining monthly value is far less disastrous for B16 than other models and
in sorts on all variables that are not annual value, B16 outperforms F17.
The three sorts involving $BM$ at the top of table \ref{tbl:GRS} have a GRS of
1.23 for F17 and 1.38 for B16.
The smaller GRS for F17 than B16 means the $\text{Sh}^2$ of F17 can be improved
less than that of B16 by investing in the portfolios constructed from sorts on
$BM$.
F17 describes the returns on sorts on $BM$ better than B16.


% !TeX root=../main.tex

Table \ref{tbl:redundant_CMA} shows the investment factor regressed on other
factors.
The idea is similar to the spanning regressions from table
\ref{tbl:contributions} except here I omit some factors to narrow down which
specific factors subsume investment.
The main message of the table is that value and momentum desribe the returns on
the investment factor and that monthly value and momentum are more robust to
including other factors.

Regression 1 uses monthly value, $HML^m$, and momentum, $WML$, as independent
variables.
The intercept, -0.04, is insignificant with a t-statistic of -0.7.
Monthly value and momentum describe returns on the investment factor.
The value slope is 0.52 with a t-statistic of 23.24.
Value is highly significant when describing returns on investment and this is
to be expected given value and investment's correlation.
The momentum slope is 0.26 with a t-statistic of 14.45.
Momentum is also highly significant when describing returns on investment
although this is probably more to do with the slope enhancement from value than
momentum's relationship with investment.
Regression 2 uses annual value, $HML$, and momentum, $WML$, as independent
variables.
The intercept, 0.08, is insignificant with a t-statistic of 1.32.
Annual value and momentum also describe returns on the investment factor.
The value slope is 0.51 with a t-statistic of 25.32.
The momentum slope is 0.05 with a t-statistic of 3.9.
While momentum is still significant when paired with annual value,
the slope and significance are drastically reduced.
This supports the idea that momentum's earlier large slope and high
significance were due to the slope enhancement from monthly value.

Regression 3 shows that investment regressed on the factors from the B16 model
leaves very little unexplained return, 0.01\% per month.
This is the same regression used to find investment's contribution the B16b
model in table \ref{tbl:contributions}.
The value slope is very similar to the slope in regression 1 at 0.49 and highly
significant.
The momentum slope also remains similar to regression 1 with a slope of 0.22.
The market has a significant negative slope of -0.09 and
profitability has a significant positive slope of 0.1.
Size is not significant when describing the returns on investment.
Regression 4 shows that investment regressed on the factors from the F17 model,
without investment as this is the dependent variable,
leaves 0.12\% per month unexplained return with a t-statistic of 2.01.
This is a change from regression 2 where only annual value and momentum were
used as independent variables and the intercept was insignificant.
Monthly value and momentum are more robust to including other factors in the
regression.
This is the same regression used to find investment's contribution the F17
model in table \ref{tbl:contributions}.
The value slope is very similar to the slope in regression 2 at 0.48 and highly
significant.
The momentum slope also remains similar to regression 2 with a slope of 0.03.
The market has a significant negative slope of -0.09 but
profitability is insignificant.
As with regression 3, we find size is not significant when describing the
returns on investment.

To say that value and momentum describe the returns on the investment factor
does not help us say why this is the case nor why monthly value and momentum
go together so well.
Investment is used as a proxy for changes in book equity in the dividend
discount model intuition of Fama and French.
\textcite{kok2017facts} find monthly value helps to identify which high value
stocks will exhibit changes in booke equity in the future.
\textcite{asness2013devil} find that momentum has predictive power for changes
in book equity.
The higher momentum slope of regression 1 over regression 2 is partly
statistical \parencite{fama2015incremental}, but may also be because monthly
value and momentum combine to give a better proxy for changes in book equity
than investment.
With this in mind, I sort on size, monthly value and momentum and compare
characteristics, future growth in book equity in particular, with sorts on
size, annual value and investment.


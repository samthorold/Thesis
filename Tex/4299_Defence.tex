\documentclass{beamer}

\usepackage[backend=biber, style=authoryear]{biblatex}
\usepackage{booktabs}
\usepackage{dcolumn}
\usepackage{graphicx}
\usepackage[utf8]{inputenc}

\usetheme{Madrid}

\addbibresource{references.bib}

\title[Asset-Pricing]{A top-down approach to factor models}
\author[Thorold]{4299 Sam Thorold\\
                 Supervisors: Francisco Santos and Andr\'e Silva}
\institute{NHH and NOVA}
\date[April 2018]{\today}

\begin{document}

\frame{\titlepage}

\section{Intro}

\begin{frame}
  \frametitle{Current state of affairs}
  \begin{itemize}
    \item Characteristic sort cannot be explained by a model
    \item A risk story is attached to the ``anomaly"
    \item Anomaly is made into a factor
  \end{itemize}
\end{frame}

\begin{frame}
  \frametitle{Current state of affairs}
  \begin{itemize}
    \item Quality and prevalence of stock data means factor models are
    particularly open to data-mining
    \item \textcite{harvey2016and} document 300+ factors in the literature
    \item \textcite{mclean2016does} find publication reduces the returns on
    many factors
    \item \textcite{linnainmaa2016history} find ``most" factors are spurious
    out of sample
  \end{itemize}
\end{frame}

\begin{frame}
  \frametitle{Barillas and Shanken to the rescue}
  \begin{itemize}
    \item \textbf{Problem}:
    no way to test models before turning to performance in anomaly sorts
    -- bottom-up approach
    \item \textbf{Solution}:
    \textcite{barillas2016alpha} show that we only need to compare the Sharpe
    ratio of the factors in a model to compare mis-pricing for all assets
    -- top-down approach
  \end{itemize}
\end{frame}

\begin{frame}
\frametitle{Maximizing Sharpe ratio minimizes mis-pricing for all assets}
\begin{itemize}
    \item We want our asset pricing models to minimize pricing errors for
    all assets
    \item When we price returns, the pricing error is given by
    \[
    \text{Sh}^2(a) = \text{Sh}^2(R, f) - \text{Sh}^2(f)
    \]
    \item If $R$ is all assets then $\text{Sh}^2(R, f) = \text{Sh}^2(R)$
    \[
    \text{Sh}^2(a) = \text{Sh}^2(R) - \text{Sh}^2(f)
    \]
    \item Model with the highest Sharpe ratio of the factors is best
\end{itemize}
\end{frame}

\begin{frame}
  \frametitle{Choose factors before addressing anomalies}
  \begin{itemize}
    \item ICAPM of \textcite{merton1973intertemporal}
    says investors care about their payoff in the future but
    also what they can do with it
    \item Market + state variables
    \item We do not observe the state variables but we can make proxies
    \item Investors want to identify \emph{cheap} and \emph{profitable} stocks
  \end{itemize}
\end{frame}

\begin{frame}
\frametitle{Proxies for cheap and profitable}
\begin{itemize}
    \item Cheap $\rightarrow$ Book-to-Market (BM)
    \parencite{fama1993common}
    \item Some problems here as we want to identify changes in BM due to
    changes in market equity rather than book equity
    \item Monthly value + momentum
    \parencite{asness2013devil, kok2017facts}
    \item Profitable $\rightarrow$ Operating profit adjusted for accruals
    \parencite{ball2016accruals}
\end{itemize}
\end{frame}

\begin{frame}
  \frametitle{Top-down approach}
  \begin{itemize}
    \item Given the best model based on Sharpe ratio evidence,
    we can investigate the returns to different anomaly sorts
    \item The model then provides a description of these anomalies
    \item Problem sorts do not necessarily lead to new factors even if they
    have significant alpha in time-series regressions
    \item New factors must reliably improve the Sharpe ratio of the factors
  \end{itemize}
\end{frame}

\begin{frame}
  \frametitle{Preview of results}
  \begin{itemize}
    \item Highest Sharpe ratio:
    \begin{equation}
      R_t^i=a^i + b^iR_t^M + s^iSMB_t + v^iHML_t^m + m^iWML_t + p^iPMU_t
    \end{equation}
    \item Investment factor does not add to the Sharpe ratio nor do sorts on
    investment pose any problems
    \item Sorts on momentum and volatility or those sorts that constrain value
    are a problem
    \item The returns on the ``lethal combination" \parencite{fama2015five,
    fama2016dissecting} of small, unprofitable stocks that somehow invest
    aggressively behave like small, unprofitable stocks with poor recent
    returns -- disappearance of value
  \end{itemize}
\end{frame}

\section{Literature}

\begin{frame}
  \frametitle{Similar research}
  \begin{itemize}
    \item \textcite{fama2016choosing} rely heaviy on
    \textcite{barillas2016alpha} to show the accrual-adjusted profitability
    factor of \textcite{ball2016accruals} improves the Sharpe ratio of their
    original five-factor model \parencite{fama2015five}
    \begin{equation}
      R_t^i=a^i + b^iR_t^M + s^iSMB_t + v^iHML_t + p^iPMU_t + i^iCMA_t
    \end{equation}
    \item \textcite{fama2016dissecting} follow the advice of
    \textcite{lewellen2010skeptical} and test their original five-factor
    model's performance in anomaly sorts constructed from variables not present
    in the model
    \item \textcite{ball2016accruals} present a model similar to the one I
    propose to show an accruals factor does not improve the Sharpe ratio of a
    five-factor model including their accrual-adjusted profitability factor
  \end{itemize}
\end{frame}

\section{Data}

\begin{frame}
  \frametitle{Data}
  \begin{itemize}
    \item Factors: Ken French all except $\text{HML}^m$ which is from AQR
    \item Sorts: Ken French all except size, value, investment/momentum
    \item PMU factor and all characteristic sorts: CRSP and COMPUSTAT
    \item U.S. stocks -- July 1963 through December 2017
    \item I include financial firms and do not winsorize variables
  \end{itemize}
\end{frame}

\section{Results}

\begin{frame}
\frametitle{Sharpe ratio}
\resizebox{\linewidth}{!}{
\begin{tabular}{lcccccc}
  \toprule
  \multicolumn{7}{l}{Panel A: Model factors and maximum squared Sharpe ratios} \\
        & \multicolumn{2}{l}{Name}                         & \multicolumn{3}{l}{Factors}                                   & Sh2   \\
  1     & \multicolumn{2}{l}{Model 1}                      & \multicolumn{3}{l}{$R^M$, $SMB$, $HML^m$, $PMU$, $WML$}        & 0.316 \\
  2     & \multicolumn{2}{l}{Model 1 + $CMA$}              & \multicolumn{3}{l}{$R^M$, $SMB$, $HML^m$, $PMU$, $WML$, $CMA$} & 0.316 \\
  3     & \multicolumn{2}{l}{Fama and French (2017)}       & \multicolumn{3}{l}{$R^M$, $SMB$, $HML$, $PMU$, $WML$, $CMA$}   & 0.240 \\
  4     & \multicolumn{2}{l}{Model 2}                      & \multicolumn{3}{l}{$R^M$, $SMB$, $HML$, $PMU$, $CMA$}          & 0.225 \\
  5     & \multicolumn{2}{l}{Fama and French (2015)}       & \multicolumn{3}{l}{$R^M$, $SMB$, $HML$, $PMU^{06}$, $CMA$}     & 0.099 \\
  6     & \multicolumn{2}{l}{Carhart (1997)}               & \multicolumn{3}{l}{$R^M$, $SMB$, $HML$, $WML$}                 & 0.090 \\
  7     & \multicolumn{2}{l}{Carhart (1997) + $HML^m$}     & \multicolumn{3}{l}{$R^M$, $SMB$, $HML^m$, $WML$}               & 0.136 \\
  \midrule
  \multicolumn{7}{l}{Panel B: 90\% confidence interval for distributions of Sh2(Row) - Sh2(Column)} \\
        &        2         &        3         &        4         &        5         &        6         &        7         \\
  1     & (-0.008,  0.000) & ( 0.039,  0.112) & ( 0.055,  0.138) & ( 0.156,  0.291) & ( 0.169,  0.299) & ( 0.127,  0.248) \\
  2     &                  & ( 0.043,  0.115) & ( 0.058,  0.140) & ( 0.158,  0.294) & ( 0.170,  0.302) & ( 0.129,  0.250) \\
  3     &                  &                  & ( 0.002,  0.048) & ( 0.095,  0.204) & ( 0.107,  0.212) & ( 0.052,  0.172) \\
  4     &                  &                  &                  & ( 0.081,  0.176) & ( 0.080,  0.197) & ( 0.028,  0.155) \\
  5     &                  &                  &                  &                  & (-0.033,  0.052) & (-0.085,  0.012) \\
  6     &                  &                  &                  &                  &                  & (-0.071, -0.025) \\
  \bottomrule
\end{tabular}

}
\end{frame}

\begin{frame}
\frametitle{Contributions to Sharpe ratio}
\begin{center}
\resizebox*{!}{\dimexpr\textheight-1.3cm\relax}{
    \input{Tables/contributions_pres_tbl}
    }
\end{center}
\end{frame}

\begin{frame}
\frametitle{Value and momentum subsume investment}
\begin{center}
\resizebox*{!}{\dimexpr\textheight-1.3cm\relax}{
    \input{Tables/redundant_inv_pres_tbl}
    }
\end{center}
\end{frame}

\begin{frame}
\frametitle{Value and momentum proxy for changes in BE}
\begin{itemize}
    \item Investment factor is subsumed by value and momentum
    \item $ME_t =
    \sum_{s=1}^\infty E\left( \text{Profit}_{t+s}-
    \Delta\text{BE}_{t+s}\right) /R^s$
    \item Investment does not describe future $\Delta$BE for small, growth
    stocks while sorts on value and momentum describe investment and future
    $\Delta$BE
\end{itemize}
\end{frame}

\begin{frame}
\frametitle{Size-$\text{BM}^m$-Prior sorts}
\begin{center}
\resizebox*{!}{\dimexpr\textheight-1.3cm\relax}{
    
\begin{tabular}{lrrrrrrrr}
  \toprule
     & \multicolumn{4}{c}{Small} & \multicolumn{4}{c}{Big}  \\
     \cmidrule(r){2-5} \cmidrule(r){6-9}
    $\text{BM}^m$ $\rightarrow$ & Low & 2 & 3 & High & Low & 2 & 3 & High  \\ 
  \midrule
  
  
    % & \multicolumn{8}{c}{$\text{r}^i$}  \\
    %  \cmidrule(r){2-5} \cmidrule(r){6-9}
    % Low Prior  & 0.09  & 0.76  & 0.98  & 1.01  & 0.55  & 0.75  & 0.95  & 0.96   \\
    % 2  & 0.69  & 1.07  & 1.24  & 1.51  & 0.64  & 0.82  & 0.90  & 1.10   \\
    % 3  & 0.96  & 1.26  & 1.55  & 1.71  & 0.90  & 0.87  & 0.99  & 1.23   \\
    % High Prior  & 1.47  & 1.58  & 1.78  & 2.04  & 1.21  & 1.11  & 1.14  & 1.42   \\
    
  
    % & \multicolumn{8}{c}{Future Var}  \\
    %  \cmidrule(r){2-5} \cmidrule(r){6-9}
    % Low Prior  & 0.20  & 0.13  & 0.13  & 0.19  & 0.05  & 0.05  & 0.04  & 0.05   \\
    % 2  & 0.10  & 0.07  & 0.07  & 0.09  & 0.03  & 0.03  & 0.03  & 0.03   \\
    % 3  & 0.08  & 0.06  & 0.06  & 0.08  & 0.03  & 0.03  & 0.03  & 0.03   \\
    % High Prior  & 0.11  & 0.08  & 0.08  & 0.12  & 0.05  & 0.04  & 0.04  & 0.04   \\
    
  
    & \multicolumn{8}{c}{Future Inv}  \\
     \cmidrule(r){2-5} \cmidrule(r){6-9}
    Low Prior  & \textbf{0.39}  & 0.25  & 0.19  & 0.20  & 0.22  & 0.16  & 0.14  & 0.20   \\
    2  & \textbf{0.27}  & 0.16  & 0.11  & 0.09  & 0.17  & 0.13  & 0.11  & 0.10   \\
    3  & \textbf{0.24}  & 0.13  & 0.09  & 0.07  & 0.17  & 0.13  & 0.11  & 0.09   \\
    High Prior  & \textbf{0.25}  & 0.12  & 0.10  & 0.06  & 0.23  & 0.14  & 0.12  & 0.09   \\
    
  
    & \multicolumn{8}{c}{Future $\Delta\text{BE}$}  \\
     \cmidrule(r){2-5} \cmidrule(r){6-9}
    Low Prior  & \textbf{0.31}  & 0.07  & 0.01  & -0.10  & 0.42  & 0.11  & 0.07  & 0.02   \\
    2  & \textbf{0.30}  & 0.09  & 0.04  & -0.02  & 0.25  & 0.11  & 0.08  & 0.04   \\
    3  & \textbf{0.34}  & 0.10  & 0.05  & -0.00  & 0.27  & 0.11  & 0.08  & 0.05   \\
    High Prior  & \textbf{0.61}  & 0.10  & 0.04  & -0.03  & 0.53  & 0.14  & 0.10  & 0.04   \\
    
  
  \bottomrule
\end{tabular}

    }
\end{center}
\end{frame}

\begin{frame}
\frametitle{Size-BM-Inv sorts}
\begin{center}
\resizebox*{!}{\dimexpr\textheight-1.3cm\relax}{
    
\begin{tabular}{lrrrrrrrr}
  \toprule
     & \multicolumn{4}{c}{Small} & \multicolumn{4}{c}{Big}  \\
     \cmidrule(r){2-5} \cmidrule(r){6-9}
    BM $\rightarrow$ & Low & 2 & 3 & High & Low & 2 & 3 & High  \\ 
  \midrule
  
  
    & \multicolumn{8}{c}{$\text{R}^i$}  \\
     \cmidrule(r){2-5} \cmidrule(r){6-9}
    Low Inv  & 1.01  & 1.33  & 1.45  & 1.46  & 0.98  & 1.04  & 1.05  & 1.14   \\
    2  & 1.22  & 1.31  & 1.3  & 1.48  & 0.92  & 0.93  & 0.95  & 0.99   \\
    3  & 1.25  & 1.29  & 1.41  & 1.28  & 0.92  & 0.94  & 0.93  & 1.16   \\
    High Inv  & 0.94  & 1.17  & 1.27  & 1.31  & 0.98  & 0.8  & 0.9  & 1.06   \\
    
  
    & \multicolumn{8}{c}{Future Var}  \\
     \cmidrule(r){2-5} \cmidrule(r){6-9}
    Low Inv  & 0.16  & 0.11  & 0.1  & 0.14  & 0.04  & 0.03  & 0.03  & 0.04   \\
    2  & 0.09  & 0.07  & 0.07  & 0.09  & 0.03  & 0.03  & 0.03  & 0.03   \\
    3  & 0.08  & 0.06  & 0.06  & 0.09  & 0.03  & 0.03  & 0.03  & 0.04   \\
    High Inv  & 0.11  & 0.09  & 0.08  & 0.11  & 0.05  & 0.04  & 0.04  & 0.04   \\
    
  
    & \multicolumn{8}{c}{Future Inv}  \\
     \cmidrule(r){2-5} \cmidrule(r){6-9}
    Low Inv  & -0.1  & -0.08  & -0.08  & -0.09  & -0.02  & -0.02  & -0.01  & -0.02   \\
    2  & 0.04  & 0.03  & 0.03  & 0.02  & 0.06  & 0.06  & 0.06  & 0.06   \\
    3  & 0.11  & 0.1  & 0.1  & 0.1  & 0.12  & 0.12  & 0.12  & 0.11   \\
    High Inv  & 0.48  & 0.39  & 0.37  & 0.47  & 0.37  & 0.38  & 0.37  & 0.52   \\
    
  
    & \multicolumn{8}{c}{Future $\Delta\text{BE}$}  \\
     \cmidrule(r){2-5} \cmidrule(r){6-9}
    Low Inv  & 0.59  & 0.06  & 0.02  & -0.03  & 0.28  & 0.07  & 0.06  & 0.03   \\
    2  & 0.44  & 0.09  & 0.05  & 0.01  & 0.2  & 0.1  & 0.07  & 0.05   \\
    3  & 0.27  & 0.11  & 0.08  & 0.03  & 0.16  & 0.12  & 0.12  & 0.08   \\
    High Inv  & 0.4  & 0.16  & 0.11  & 0.06  & 0.38  & 0.17  & 0.13  & 0.1   \\
    
  
  \bottomrule
\end{tabular}

    }
\end{center}
\end{frame}

\begin{frame}
\frametitle{GRS statistic}
\begin{center}
\resizebox*{!}{\dimexpr\textheight-1.3cm\relax}{
    
\begin{tabular}{lrrrrrrr}
  \toprule
     & \rotatebox{90}{Carhart (1997)} &
       \rotatebox{90}{Carhart (1997) + $\text{HML}^m$} &
       \rotatebox{90}{Fama and French (2015)} &
       \rotatebox{90}{Model 2} &
       \rotatebox{90}{Model 2 + WML} &
       \rotatebox{90}{Model 1} &
       \rotatebox{90}{Model 1 + CMA} \\
  \midrule
  
    \multicolumn{8}{l}{Value constrained} \\
    
    Size-BM  & 3.28  & 4.19  & 3.19  & 2.19  & 2.08  & 2.58  & 2.60  \\
    
  
    
    
    Size-BM-Inv  & 2.58  & 3.44  & 2.32  & 1.56  & 1.49  & 2.13  & 2.13  \\
    
  
    
    
    Size-BM-$\text{OP}^{06}$  & 2.22  & 3.19  & 2.15  & 1.15  & 1.15  & 1.57  & 1.67  \\
    
  
    
    
    BM constrained  & 2.31  & 2.69  & 2.14  & 1.62  & 1.58  & 1.81  & 1.82  \\
    [1em]
  
    
    
    Size-$\text{BM}^m$-Prior  & 7.26  & 5.91  & 7.55  & 6.52  & 6.51  & 4.85  & 4.85  \\
    
  
    
    
    Value constrained  & 3.95  & 3.67  & 3.74  & 3.21  & 3.22  & 2.83  & 2.83  \\
    [1em]
  
    
    \multicolumn{8}{l}{Selected anomalies} \\
    Size-$\text{OP}^{06}$  & 2.27  & 2.75  & 2.21  & 2.12  & 2.03  & 2.05  & 2.05  \\
    
  
    
    
    Size-Inv  & 3.82  & 3.40  & 3.45  & 2.43  & 2.31  & 2.02  & 2.02  \\
    
  
    
    
    Size-$\text{OP}^{06}$-Inv  & 3.53  & 3.31  & 3.19  & 2.30  & 2.18  & 1.80  & 2.01  \\
    
  
    
    
    Size-Acc  & 3.26  & 3.10  & 3.89  & 2.33  & 2.21  & 2.00  & 2.07  \\
    
  
    
    
    Size-$\beta$  & 1.51  & 1.22  & 1.83  & 1.63  & 1.52  & 1.15  & 1.15  \\
    
  
    
    
    Size-NI  & 3.67  & 3.42  & 3.35  & 2.26  & 2.13  & 1.84  & 1.84  \\
    
  
    
    
    Size-Prior  & 3.96  & 3.89  & 4.43  & 3.94  & 3.59  & 3.62  & 3.66  \\
    
  
    
    
    Size-RVar  & 5.79  & 5.84  & 5.53  & 4.69  & 4.75  & 4.46  & 4.48  \\
    
  
    
    
    Size-Var  & 5.11  & 5.09  & 5.02  & 4.38  & 4.44  & 4.13  & 4.14  \\
    
  
    
    
    Selected anomalies  & 2.84  & 2.78  & 2.80  & 2.37  & 2.32  & 2.23  & 2.22  \\
    [1em]
  
    
    
    All  & 2.85  & 2.70  & 2.80  & 2.44  & 2.40  & 2.23  & 2.22  \\
    
  
  \bottomrule
\end{tabular}


    }
\end{center}
\end{frame}

\begin{frame}
\frametitle{Momentum and variance sorts}
\begin{itemize}
    \item Momentum factor helps, but not much, in sorts on momentum
    \item Something going on with momentum returns that is not to do with
    momentum characteristics
    \item Profit slopes are negative in the extremes of momentum
    \item Defensive equity $\rightarrow$ beta and volatility
    \item Sorts on beta pose the least problems
    \item Sorts on variance are a big problem
    \item Small, high-variance alpha of -0.66\% per month
\end{itemize}
\end{frame}

\begin{frame}
\frametitle{Momentum and variance sorts}
\begin{itemize}
    \item To take advantage of this alpha we need to sell small,
    unprofitable stocks with poor recent returns
    \item That's a tough sell $\rightarrow$ Liquidity
    \item But, liquidity hard to approach from a cross-section perspective
    \item Institutional ownership, Short-term reversal factor as a proxy
    for market-maker's returns $\rightarrow$ market microstructure
\end{itemize}
\end{frame}

\begin{frame}
\frametitle{Momentum and variance sorts}
\begin{itemize}
    \item Cross-section perspective
    \item High-volatility firms are much smaller
    \item Small says high returns while high-volatility says low returns
    \item Value disappears in the problem momentum and variance sorts
    \item Normally, value and profitability do most of the heavy lifting
\end{itemize}
\end{frame}

\section{Conclusions}

\begin{frame}
\frametitle{Conclusions}
\begin{itemize}
    \item Mkt, size, value, momentum and profitability factors have the
    highest Sharpe ratio
    \item Value and momentum subsume investment
    \item ``Lethal combination" of small, unprofitable stocks that somehow
    invest aggressively $\rightarrow$ small, unprofitable firms with poor
    recent returns
\end{itemize}
\end{frame}

\section{Omissions}

\begin{frame}
\frametitle{Omissions}
\begin{itemize}
    \item BAB factor -- small contribution to $\text{Sh}^2$ (0.02) due to
    unexplained volatility
    \item QMJ factor -- small contribution to $\text{Sh}^2$ (0.05) due to low
    unexplained average return
    \item Stance on risk vs behavioural stories
\end{itemize}
\end{frame}

\section{References}

\begin{frame}[allowframebreaks]
  \frametitle{References}
  \printbibliography
\end{frame}


\end{document}


\documentclass[notes]{beamer}  % remove "[notes]" to print only slides

\usepackage{amsmath}
\usepackage[backend=biber, style=authoryear]{biblatex}
\usepackage{booktabs}
\usepackage{dcolumn}
\usepackage{graphicx}
\usepackage{helvet}
\usepackage[utf8]{inputenc}
\usepackage[normalem]{ulem}

% \usetheme{Hannover}
% \usefonttheme{serif}

\addbibresource{references.bib}

\title[Asset-Pricing]{A top-down approach to factor models}
\author[Thorold]{4299 Sam Thorold\\
                 Supervisors: Francisco Santos and Andr\'e Silva}
\institute{NHH and Nova SBE}
\date[June 2018]{\today}

\begin{document}

\frame{\titlepage}

\section{Intro}

\begin{frame} \frametitle{Goal}
  Minimize mispricing for all portfolios with a few portfolios
\end{frame}

% \note[itemize]{
%   \item The goal is not to \emph{generate} alpha, pricing $\neq$ investing -- the model is not a strategy
% }

\begin{frame} \frametitle{Problems}
  \begin{enumerate}
    \item No direct test for all portfolios
    \item Mispricing for subsets of portfolios $\rightarrow$ data-mining
  \end{enumerate}
\end{frame}

\note[itemize]{
  \item Stock data is rich, a researcher can find some subset of stocks that a given model cannot price (an ``anomaly") if they look hard enough
  \item The anomaly then provides the empirical justification for a new factor
  % \item If we are lucky a risk-story is attached later
  \item \textcite{harvey2016and} document 300+ factors in the literature
  % \item \textcite{mclean2016does} find publication reduces the returns on many factors
  \item \textcite{linnainmaa2016history} claim ``most" factors are spurious out of sample
}

\begin{frame} \frametitle{``Mispricing"}
  \textcite{gibbons1989test}
  \[\text{Mispricing} \propto \text{Sh}^2(\text{test portfolios and factors}) - \text{Sh}^2(\text{factors})\]
  \textcite{barillas2016alpha}
  \[\text{Sh}^2(\text{all portfolios and factors}) = \text{Sh}^2(\text{all portfolios})\]
  \[\text{Mispricing} \propto \text{Sh}^2(\text{all portfolios}) - \text{Sh}^2(\text{factors})\]
\end{frame}

% \note[itemize]{
%   \item I focus on $\text{Sh}^2(\text{factors})$ but the ultimate aim is to minimize $\text{Sh}^2(\text{alphas}$)
% }

\begin{frame} \frametitle{Solution: Top-down approach}
  \begin{enumerate}
    \item Maximize $\text{Sh}^2(\text{factors})$  %  to minimize mispricing for all portfolios -- no need identify all portfolios
    \item Model gives description of anomalies  % , rather than description of anomalies driving choice of factors -- reduces data-mining
  \end{enumerate}
\end{frame}

\note[itemize]{
  \item Choice of factors not driven by anomalies
  \item How to choose factors then?
  \item Investors care about their payoff and what they can do with it \parencite{merton1973intertemporal, fama1996multifactor}
  \item Factors are long-short portfolios designed to capture unobserved state variables priced by investors
  \item Identify \emph{cheap} and \emph{profitable} stocks \parencite{graham1934security}
}

\begin{frame} \frametitle{Model}
  \begin{equation} \label{eq:B16}
    R_t^i = a^i+b^iR^M+s^iSMB_t+\underbrace{v^iHML_t^m+m^iWML_t}_\text{cheap}+\underbrace{p^iPMU_t}_\text{profitable}
  \end{equation}
\end{frame}

\note[itemize]{
  % \item Chars used to construct factors are simply sorting variables
  % \item Profitable $\rightarrow$ Operating profit adjusted for accruals \parencite{ball2016accruals}
  \item S for small, B for big, etc.
}

\section{Prior research}

\begin{frame} \frametitle{Prior research}
  \begin{itemize}
    \item Fama and French (2015, 2016a, 2016b)
    \begin{equation} \label{eq:F16}
      R_t^i=a^i + b^iR_t^M + s^iSMB_t + v^iHML_t + p^iPMU_t + i^iCMA_t
    \end{equation}
    \item \textcite{fama2006profitability}
    \[ME_t = \sum_{s=1}^\infty E\left( \text{Profit}_{t+s}-\Delta\text{BE}_{t+s}\right) /R^s\]
    \[\Delta\text{Assets}\approx \Delta\text{BE}\]
  \end{itemize}
\end{frame}

\note[itemize]{
  \item \parencite{fama2015five, fama2016choosing, fama2016dissecting}
  \item $CMA$: Investment factor long conservatively and short aggressively investing stocks
  \item ``Lethal combination" $\rightarrow$ small, unprofitable stocks that somehow invest aggressively
  \item \textcite{ball2016accruals}
}

\section{Data}

\begin{frame}
  \frametitle{Data}
  \begin{itemize}
    \item $PMU$ and summary characteristics constructed from CRSP and COMPUSTAT
    \item Remainder available from Dr. Ken French and AQR Capital Management
    \item U.S. common stocks, 1963-07 -- 2017-12 (654 months)
    % \item I include financial firms and do not winsorize/trim variables
  \end{itemize}
\end{frame}

\section{Results}

\begin{frame}
\frametitle{Sharpe ratio}
\resizebox{\linewidth}{!}{
\begin{tabular}{lcccc}
  \toprule
  \multicolumn{5}{l}{Panel A: Model factors and maximum squared Sharpe ratios} \\
        & \multicolumn{1}{l}{Name}                         & \multicolumn{2}{l}{Factors}                                    & $\text{Sh}^2$   \\
  1     & \multicolumn{1}{l}{\textbf{Model 1}}             & \multicolumn{2}{l}{\footnotesize{$R^M$, $SMB$, $HML^m$, $PMU$, $WML$}}        & \textbf{0.316} \\
  2     & \multicolumn{1}{l}{Model 1*}              & \multicolumn{2}{l}{\footnotesize{$R^M$, $SMB$, $HML^m$, $PMU$, $WML$, $CMA$}} & 0.316 \\
  % 3     & \multicolumn{2}{l}{Fama and French (2017)}       & \multicolumn{3}{l}{$R^M$, $SMB$, $HML$, $PMU$, $WML$, $CMA$}   & 0.240 \\
  4     & \multicolumn{1}{l}{\textbf{Model 2}}             & \multicolumn{2}{l}{\footnotesize{$R^M$, $SMB$, $HML$, $PMU$, $CMA$}}          & \textbf{0.225} \\
  % 5     & \multicolumn{2}{l}{Fama French (2015)}       & \multicolumn{3}{l}{$R^M$, $SMB$, $HML$, $PMU^{06}$, $CMA$}     & 0.099 \\
  6     & \multicolumn{1}{l}{Carhart '97}               & \multicolumn{2}{l}{\footnotesize{$R^M$, $SMB$, $HML$, $WML$}}                 & 0.090 \\
  7     & \multicolumn{1}{l}{Carhart '97*}     & \multicolumn{2}{l}{\footnotesize{$R^M$, $SMB$, $HML^m$, $WML$}}               & 0.136 \\
  \midrule
  \multicolumn{5}{l}{Panel B: 90\% confidence interval for $\text{Sh}^2(\text{Row})$ - $\text{Sh}^2(\text{Column})$} \\
        &        2                &        4         &        6                &        7         \\
  1     & \textbf{(-0.01,  0.00)} & ( 0.06,  0.14)   & ( 0.17,  0.30)          & ( 0.13,  0.25) \\
  2     &                         & ( 0.06,  0.14)   & ( 0.17,  0.30)          & ( 0.13,  0.25) \\
% 3     &                         & ( 0.00,  0.05)   & ( 0.11,  0.21)          & ( 0.05,  0.17) \\
  4     &                         &                  & ( 0.08,  0.20)          & ( 0.03,  0.16) \\
% 5     &                         &                  & \textbf{(-0.03,  0.05)} & \textbf{(-0.09,  0.01)} \\
  6     &                         &                  &                         & (-0.07, -0.03) \\
  \bottomrule
\end{tabular}
}
\end{frame}

\note[itemize]{
  \item Model 1 has the highest Sh2. Adding CMA does not increase the Sh2
  \item Further, replacing annual value with monthly value in the Carhart model increases the Sh2
  \item Bootstrap: 1 simulation is 654 draws w/ replacement\\
  100K simulations gives 100K values of Sh2(row)-Sh2(col)\\
  (5th percentile, 95th percentile) so 90\% of values in this range
  \item Model 1 not reliably improved by including CMA
  \item Monthly value reliably better than annual value for the Carhart model
  \item Annual or monthly value and momentum subsume investment (not shown)
  \item Why? I sort on size-value-momentum
}

\begin{frame} \frametitle{$\Delta$Assets poor proxy for $\Delta$BE} \framesubtitle{Size-$\text{BM}^m$-Prior characteristics}
\begin{center}
\resizebox{\linewidth}{!}{
\begin{tabular}{lrrrrrrrr}
  \toprule
     & \multicolumn{4}{c}{Small} & \multicolumn{4}{c}{Big}  \\
     \cmidrule(r){2-5} \cmidrule(r){6-9}
    $\text{BM}^m$ $\rightarrow$ & Low & 2 & 3 & High & Low & 2 & 3 & High  \\ 
  \midrule
  
  
    & \multicolumn{8}{c}{$\text{R}^i$}  \\
     \cmidrule(r){2-5} \cmidrule(r){6-9}
    Low Prior  & 0.09  & 0.76  & 0.98  & 1.01  & 0.55  & 0.75  & 0.95  & 0.96   \\
    2  & 0.69  & 1.07  & 1.24  & 1.51  & 0.64  & 0.82  & 0.9  & 1.1   \\
    3  & 0.96  & 1.26  & 1.55  & 1.71  & 0.9  & 0.87  & 0.99  & 1.23   \\
    High Prior  & 1.47  & 1.58  & 1.78  & 2.04  & 1.21  & 1.11  & 1.14  & 1.42   \\
    
  
    & \multicolumn{8}{c}{Future Var}  \\
     \cmidrule(r){2-5} \cmidrule(r){6-9}
    Low Prior  & 0.2  & 0.13  & 0.13  & 0.19  & 0.05  & 0.05  & 0.04  & 0.05   \\
    2  & 0.1  & 0.07  & 0.07  & 0.09  & 0.03  & 0.03  & 0.03  & 0.03   \\
    3  & 0.08  & 0.06  & 0.06  & 0.08  & 0.03  & 0.03  & 0.03  & 0.03   \\
    High Prior  & 0.11  & 0.08  & 0.08  & 0.12  & 0.05  & 0.04  & 0.04  & 0.04   \\
    
  
    & \multicolumn{8}{c}{Future Inv}  \\
     \cmidrule(r){2-5} \cmidrule(r){6-9}
    Low Prior  & 0.39  & 0.24  & 0.19  & 0.2  & 0.21  & 0.16  & 0.14  & 0.2   \\
    2  & 0.27  & 0.16  & 0.11  & 0.09  & 0.17  & 0.13  & 0.11  & 0.1   \\
    3  & 0.24  & 0.13  & 0.09  & 0.07  & 0.17  & 0.13  & 0.11  & 0.09   \\
    High Prior  & 0.25  & 0.12  & 0.1  & 0.06  & 0.23  & 0.14  & 0.12  & 0.09   \\
    
  
    & \multicolumn{8}{c}{Future $\Delta\text{BE}$}  \\
     \cmidrule(r){2-5} \cmidrule(r){6-9}
    Low Prior  & 0.31  & 0.07  & 0.01  & -0.1  & 0.42  & 0.11  & 0.07  & 0.02   \\
    2  & 0.3  & 0.09  & 0.04  & -0.02  & 0.25  & 0.11  & 0.08  & 0.04   \\
    3  & 0.34  & 0.1  & 0.05  & -0.0  & 0.27  & 0.11  & 0.08  & 0.05   \\
    High Prior  & 0.61  & 0.1  & 0.04  & -0.03  & 0.53  & 0.14  & 0.1  & 0.04   \\
    
  
  \bottomrule
\end{tabular}
}
\end{center}
\end{frame}

\note[itemize]{
  \item Split stocks into two buckets based on the NYSE median size
  \item w/in the size buckets, split the stocks into four value and four momentum buckets
  \item the intersections of these buckets give 32 portfolios
  \item Avg monthly value-weighted chars
  \item Future investment and future changes in BE backwards for small, growth stocks
  \item Monthly value and investment capture variation on both
  \item I sort on size-value-investment to confirm
}

\begin{frame} \frametitle{$\Delta$Assets poor proxy for $\Delta$BE} \framesubtitle{Size-BM-Inv characteristics}
\begin{center}
\resizebox{\linewidth}{!}{
\begin{tabular}{lrrrrrrrr}
  \toprule
     & \multicolumn{4}{c}{Small} & \multicolumn{4}{c}{Big}  \\
     \cmidrule(r){2-5} \cmidrule(r){6-9}
    BM $\rightarrow$ & Low & 2 & 3 & High & Low & 2 & 3 & High  \\ 
  \midrule
  
  
    % & \multicolumn{8}{c}{$\text{r}^i$}  \\
    %  \cmidrule(r){2-5} \cmidrule(r){6-9}
    % Low Inv  & 1.01  & 1.33  & 1.45  & 1.46  & 0.98  & 1.04  & 1.05  & 1.14   \\
    % 2  & 1.22  & 1.31  & 1.30  & 1.48  & 0.92  & 0.93  & 0.95  & 0.99   \\
    % 3  & 1.25  & 1.29  & 1.41  & 1.28  & 0.92  & 0.94  & 0.93  & 1.16   \\
    % High Inv  & 0.94  & 1.17  & 1.27  & 1.31  & 0.98  & 0.80  & 0.90  & 1.06   \\
    
  
    % & \multicolumn{8}{c}{Future Var}  \\
    %  \cmidrule(r){2-5} \cmidrule(r){6-9}
    % Low Inv  & 0.16  & 0.11  & 0.10  & 0.14  & 0.04  & 0.03  & 0.03  & 0.04   \\
    % 2  & 0.09  & 0.07  & 0.07  & 0.09  & 0.03  & 0.03  & 0.03  & 0.03   \\
    % 3  & 0.08  & 0.06  & 0.06  & 0.09  & 0.03  & 0.03  & 0.03  & 0.04   \\
    % High Inv  & 0.11  & 0.09  & 0.08  & 0.11  & 0.05  & 0.04  & 0.04  & 0.04   \\
    
  
    & \multicolumn{8}{c}{Future Inv}  \\
     \cmidrule(r){2-5} \cmidrule(r){6-9}
    Low Inv  & \textbf{-0.10}  & -0.08  & -0.08  & -0.09  & -0.02  & -0.02  & -0.01  & -0.02   \\
    2  & \textbf{0.04}  & 0.03  & 0.03  & 0.02  & 0.06  & 0.06  & 0.06  & 0.06   \\
    3  & \textbf{0.11}  & 0.10  & 0.10  & 0.10  & 0.12  & 0.12  & 0.12  & 0.11   \\
    High Inv  & \textbf{0.48}  & 0.39  & 0.37  & 0.47  & 0.37  & 0.38  & 0.37  & 0.52   \\
    
  
    & \multicolumn{8}{c}{Future $\Delta\text{BE}$}  \\
     \cmidrule(r){2-5} \cmidrule(r){6-9}
    Low Inv  & \textbf{0.59}  & 0.06  & 0.02  & -0.03  & 0.28  & 0.07  & 0.06  & 0.03   \\
    2  & \textbf{0.44}  & 0.09  & 0.05  & 0.01  & 0.20  & 0.10  & 0.07  & 0.05   \\
    3  & \textbf{0.27}  & 0.11  & 0.08  & 0.03  & 0.16  & 0.12  & 0.12  & 0.08   \\
    High Inv  & \textbf{0.40}  & 0.16  & 0.11  & 0.06  & 0.38  & 0.17  & 0.13  & 0.10   \\
    
  
  \bottomrule
\end{tabular}
}
\end{center}
\end{frame}

\note[itemize]{
  \item Future investment and future changes in BE backwards for small, growth stocks
  \item Problem because investment is a factor and so will not capture small, growth stock's true relationship with future changes in BE
}

\begin{frame} \frametitle{Part 1 of top-down approach summary}
  \begin{itemize}
    \item Model 1 $\text{Sh}^2$ 0.316 $>$ competing models
    \item Value and momentum (\sout{investment})
    \item Part 2: model 1 description of anomalies
  \end{itemize}
\end{frame}

\begin{frame} \frametitle{GRS statistic}
\begin{center}
\resizebox*{!}{\dimexpr\textheight-1.3cm\relax}{
\begin{tabular}{lrrr}
  \toprule
     & \rotatebox{90}{Model 2} &
       \rotatebox{90}{Model 1} &
       \rotatebox{90}{Model 1*} \\
  \midrule

    Size-BM-Inv         & 1.56  & 2.13  & 2.13  \\
    BM constrained      & 1.62  & 1.81  & 1.82  \\
    [1em]

    \multicolumn{4}{l}{Selected anomalies} \\
    Size-Inv            & 2.43  & 2.02  & 2.02  \\
    Size-$\beta$        & 1.63  & 1.15  & 1.15  \\
    Size-Prior          & 3.94  & 3.62  & 3.66  \\
    Size-Var            & 4.38  & 4.13  & 4.14  \\
    Selected anomalies  & 2.37  & 2.23  & 2.22  \\
    [1em]
    All                 & 2.44  & 2.23  & 2.22  \\

  \bottomrule
\end{tabular}

}
\end{center}
\end{frame}

\note[itemize]{
  % \item Anomalies from Ken French website
  \item \textbf{Main message}: Momentum and volatility biggest problems
  \item Annual value a problem -- unwanted variation in BM within BMm buckets (not shown)
  \item All other sorts a wrap for Model 1 (lower GRS so lower mispricing)
  \item Investment factor adds nothing to subsets of stocks
  \item Model 1 description of momentum and volatility
}

\begin{frame} \frametitle{Size-momentum returns}
  \begin{center}
  \resizebox*{!}{\dimexpr\textheight-1.3cm\relax}{
\begin{tabular}{lrrrrrrrrrr}
  \toprule
    
    Prior $\rightarrow$ & Low & 2 & 3 & 4 & High & Low & 2 & 3 & 4 & High  \\ 
  \midrule
  \multicolumn{11}{l}{$R^i=a^i+\beta^iMkt+s^iSize+v^iVal+m^iMom+p^iProf$}  \\
  
     & \multicolumn{5}{c}{a} & \multicolumn{5}{c}{t(a)}   \\
     \cmidrule(r){2-6} \cmidrule(r){7-11} 
    Small  & \textbf{-0.37}  & -0.03  & 0.12  & 0.18  & \textbf{0.38}  & -3.52  & -0.51  & 1.81  & 2.61  & 4.44   \\
    2  & -0.16  & 0.00  & 0.04  & 0.08  & \textbf{0.24}  & -2.09  & 0.05  & 0.69  & 1.34  & 3.54   \\
    3  & 0.09  & 0.07  & -0.00  & -0.12  & \textbf{0.19}  & 0.99  & 1.04  & -0.04  & -1.67  & 2.81   \\
    4  & 0.09  & 0.13  & 0.03  & 0.04  & 0.10  & 0.90  & 1.77  & 0.43  & 0.62  & 1.24   \\
    Big  & 0.13  & 0.23  & -0.01  & -0.13  & -0.11  & 1.23  & 3.19  & -0.16  & -2.06  & -1.45   \\
    
  
     & \multicolumn{5}{c}{v} & \multicolumn{5}{c}{p}   \\
     \cmidrule(r){2-6} \cmidrule(r){7-11} 
    Small  & 0.12  & 0.32  & 0.32  & 0.25  & \textbf{0.03}  & \textbf{-0.19}  & -0.04  & 0.01  & 0.02  & \textbf{-0.20}   \\
    2  & -0.01  & 0.24  & 0.25  & 0.25  & -0.07  & \textbf{-0.18}  & 0.12  & 0.06  & 0.07  & \textbf{-0.17}   \\
    3  & -0.03  & 0.19  & 0.28  & 0.29  & -0.07  & \textbf{-0.19}  & 0.06  & 0.11  & 0.10  & -0.09   \\
    4  & 0.02  & 0.19  & 0.24  & 0.20  & -0.04  & \textbf{-0.18}  & 0.03  & 0.07  & 0.02  & -0.07   \\
    Big  & -0.06  & 0.10  & 0.10  & 0.15  & -0.00  & 0.06  & 0.10  & 0.02  & 0.08  & 0.00   \\
    
  
    %  & \multicolumn{5}{c}{m} & \multicolumn{5}{c}{t(m)}   \\
    %  \cmidrule(r){2-6} \cmidrule(r){7-11} 
    % Small  & -0.62  & -0.12  & 0.05  & 0.18  & 0.31  & -20.98  & -6.31  & 2.51  & 9.08  & 12.92   \\
    % 2  & -0.71  & -0.22  & 0.04  & 0.18  & 0.33  & -31.89  & -12.01  & 2.07  & 10.80  & 17.44   \\
    % 3  & -0.76  & -0.26  & -0.03  & 0.21  & 0.38  & -28.07  & -13.18  & -1.49  & 10.21  & 19.59   \\
    % 4  & -0.78  & -0.30  & -0.05  & 0.17  & 0.43  & -26.26  & -14.24  & -2.48  & 8.19  & 19.73   \\
    % Big  & -0.79  & -0.40  & -0.07  & 0.23  & 0.48  & -25.78  & -19.12  & -3.42  & 12.61  & 23.01   \\
    
  
    %  & \multicolumn{5}{c}{p} & \multicolumn{5}{c}{t(p)}   \\
    %  \cmidrule(r){2-6} \cmidrule(r){7-11} 
    % Small  & \textbf{-0.19}  & -0.04  & 0.01  & 0.02  & \textbf{-0.20}  & -2.43  & -0.75  & 0.10  & 0.30  & -3.12   \\
    % 2  & \textbf{-0.18}  & 0.12  & 0.06  & 0.07  & \textbf{-0.17}  & -3.07  & 2.58  & 1.16  & 1.65  & -3.43   \\
    % 3  & \textbf{-0.19}  & 0.06  & 0.11  & 0.10  & -0.09  & -2.65  & 1.15  & 2.11  & 1.85  & -1.71   \\
    % 4  & \textbf{-0.18}  & 0.03  & 0.07  & 0.02  & -0.07  & -2.34  & 0.48  & 1.39  & 0.40  & -1.24   \\
    % Big  & 0.06  & 0.10  & 0.02  & 0.08  & 0.00  & 0.72  & 1.92  & 0.45  & 1.74  & 0.02   \\
    

  \midrule
  \multicolumn{11}{l}{$R^i=a^i+\beta^iMkt+s^iSize+v^iVal+p^iProf+i^iInv$}  \\
  
     & \multicolumn{5}{c}{a} & \multicolumn{5}{c}{t(a)}   \\
     \cmidrule(r){2-6} \cmidrule(r){7-11} 
    Small  & \textbf{-0.55}  & -0.06  & 0.15  & 0.26  & \textbf{0.54}  & -3.54  & -0.81  & 2.31  & 3.89  & 5.55   \\
    2  & \textbf{-0.44}  & -0.07  & 0.06  & 0.17  & \textbf{0.41}  & -3.11  & -0.88  & 0.95  & 2.90  & 4.55   \\
    3  & -0.22  & -0.06  & -0.01  & -0.03  & \textbf{0.40}  & -1.41  & -0.65  & -0.17  & -0.41  & 4.10   \\
    4  & -0.25  & -0.02  & -0.01  & 0.13  & \textbf{0.32}  & -1.47  & -0.21  & -0.13  & 1.80  & 2.98   \\
    Big  & -0.27  & 0.01  & -0.07  & -0.01  & 0.20  & -1.60  & 0.08  & -1.00  & -0.19  & 1.83   \\

  \bottomrule
\end{tabular}
}
  \end{center}
\end{frame}

\note[itemize]{
  \item Tables of this style show mispricing and selected factor slopes
  \item Portfolios are independent sorts on size and another characteristic
  \item Interested in the extreme buckets of the characteristic, what description does the model give these portfolios
  \item In particular, where there is mispricing
  \item Alphas for models 1 and 2: momentum helps but not completely
  \item Extremes of momentum: value dies and negative prof slopes
  \item Further, Beta smile -- high beta stocks in the extremes of momentum
  \item Negative profitability slopes and high beta slopes suggest defensive equity, beta and vol \parencite{novy2014understanding}
}

\begin{frame} \frametitle{Size-beta returns}
  \begin{center}
  \resizebox*{!}{\dimexpr\textheight-1.3cm\relax}{
\begin{tabular}{lrrrrrrrrrr}
  \toprule
    
    Beta $\rightarrow$ & Low & 2 & 3 & 4 & High & Low & 2 & 3 & 4 & High  \\ 
  \midrule
  \multicolumn{11}{l}{$R^i=a^i+\beta^iMkt+s^iSize+v^iVal+m^iMom+p^iProf$}  \\
  
     & \multicolumn{5}{c}{a} & \multicolumn{5}{c}{t(a)}   \\
     \cmidrule(r){2-6} \cmidrule(r){7-11} 
    Small  & 0.10  & 0.08  & -0.00  & 0.08  & -0.07  & 1.34  & 1.13  & -0.02  & 1.26  & -0.73   \\
    2  & 0.01  & 0.04  & 0.07  & -0.04  & -0.09  & 0.15  & 0.61  & 1.06  & -0.67  & -1.12   \\
    3  & 0.08  & 0.15  & 0.00  & -0.06  & -0.02  & 0.97  & 2.19  & 0.03  & -0.83  & -0.22   \\
    4  & 0.10  & 0.08  & -0.01  & -0.12  & 0.03  & 1.11  & 1.09  & -0.08  & -1.48  & 0.28   \\
    Big  & -0.03  & -0.07  & -0.08  & -0.14  & 0.01  & -0.34  & -1.21  & -1.31  & -1.63  & 0.10   \\
    
  
    %  & \multicolumn{5}{c}{b} & \multicolumn{5}{c}{t(b)}   \\
    %  \cmidrule(r){2-6} \cmidrule(r){7-11} 
    % Small  & 0.65  & 0.78  & 0.88  & 1.01  & 1.16  & 39.89  & 48.20  & 54.71  & 69.21  & 53.20   \\
    % 2  & 0.71  & 0.83  & 0.97  & 1.08  & 1.29  & 39.66  & 55.60  & 65.15  & 73.05  & 71.39   \\
    % 3  & 0.72  & 0.88  & 1.00  & 1.10  & 1.32  & 39.04  & 58.04  & 63.45  & 62.80  & 61.41   \\
    % 4  & 0.77  & 0.96  & 1.06  & 1.17  & 1.38  & 38.52  & 58.34  & 62.82  & 61.26  & 51.53   \\
    % Big  & 0.78  & 0.97  & 1.08  & 1.20  & 1.33  & 45.15  & 70.31  & 73.67  & 61.03  & 40.37   \\
    
  
    %  & \multicolumn{5}{c}{v} & \multicolumn{5}{c}{t(v)}   \\
    %  \cmidrule(r){2-6} \cmidrule(r){7-11} 
    % Small  & 0.28  & 0.31  & 0.27  & 0.20  & -0.06  & 10.72  & 11.84  & 10.45  & 8.38  & -1.74   \\
    % 2  & 0.33  & 0.31  & 0.26  & 0.21  & -0.11  & 11.18  & 12.76  & 10.83  & 8.61  & -3.66   \\
    % 3  & 0.35  & 0.20  & 0.23  & 0.17  & -0.16  & 11.55  & 8.21  & 9.06  & 5.89  & -4.48   \\
    % 4  & 0.33  & 0.24  & 0.17  & 0.12  & -0.14  & 10.25  & 8.78  & 6.11  & 3.78  & -3.28   \\
    % Big  & 0.24  & 0.12  & 0.03  & 0.00  & -0.28  & 8.48  & 5.22  & 1.10  & 0.12  & -5.18   \\
    
  
     & \multicolumn{5}{c}{m} & \multicolumn{5}{c}{p}   \\
     \cmidrule(r){2-6} \cmidrule(r){7-11} 
    Small  & \textbf{0.14}  & 0.15  & 0.15  & 0.07  & -0.10             & \textbf{-0.13}  & -0.02  & -0.01  & -0.07  & -0.16   \\
    2      & \textbf{0.19}  & 0.14  & 0.10  & 0.06  & -0.18             & \textbf{-0.12}  & 0.09  & 0.10  & 0.10  & -0.06   \\
    3      & \textbf{0.20}  & 0.10  & 0.09  & 0.01  & -0.19             & \textbf{-0.15}  & 0.03  & 0.15  & 0.13  & -0.01   \\
    4      & \textbf{0.20}  & 0.09  & 0.05  & -0.03  & -0.20            & \textbf{-0.16}  & 0.06  & 0.11  & 0.05  & 0.04   \\
    Big    & \textbf{0.15}  & 0.06  & -0.05  & -0.08  & -0.24           & 0.08  & 0.23  & 0.23  & 0.15  & -0.16   \\
    
  
    %  & \multicolumn{5}{c}{p} & \multicolumn{5}{c}{t(p)}   \\
    %  \cmidrule(r){2-6} \cmidrule(r){7-11} 
    % Small  & \textbf{-0.13}  & -0.02  & -0.01  & -0.07  & -0.16  & -2.50  & -0.45  & -0.28  & -1.42  & -2.17   \\
    % 2  & \textbf{-0.12}  & 0.09  & 0.10  & 0.10  & -0.06  & -2.04  & 1.82  & 2.10  & 2.15  & -1.09   \\
    % 3  & \textbf{-0.15}  & 0.03  & 0.15  & 0.13  & -0.01  & -2.50  & 0.66  & 2.84  & 2.26  & -0.13   \\
    % 4  & \textbf{-0.16}  & 0.06  & 0.11  & 0.05  & 0.04  & -2.44  & 1.13  & 2.03  & 0.76  & 0.41   \\
    % Big  & 0.08  & 0.23  & 0.23  & 0.15  & -0.16  & 1.49  & 5.05  & 4.86  & 2.31  & -1.44   \\
    

  \midrule
  \multicolumn{11}{l}{$R^i=a^i+\beta^iMkt+s^iSize+v^iVal+p^iProf+i^iInv$}  \\
  
     & \multicolumn{5}{c}{a} & \multicolumn{5}{c}{t(a)}   \\
     \cmidrule(r){2-6} \cmidrule(r){7-11} 
    Small  & \textbf{0.16}  & 0.13  & 0.04  & 0.11  & -0.10  & 2.35  & 1.96  & 0.65  & 1.84  & -1.05   \\
    2  & \textbf{0.10}  & 0.08  & 0.11  & -0.03  & -0.17  & 1.34  & 1.33  & 1.75  & -0.43  & -2.16   \\
    3  & \textbf{0.18}  & 0.16  & 0.03  & -0.09  & -0.09  & 2.34  & 2.59  & 0.48  & -1.17  & -0.98   \\
    4  & \textbf{0.20}  & 0.10  & -0.01  & -0.16  & -0.04  & 2.36  & 1.42  & -0.10  & -1.94  & -0.38   \\
    Big  & 0.01  & -0.06  & -0.12  & -0.16  & -0.10  & 0.20  & -1.03  & -1.85  & -1.89  & -0.72   \\

  \bottomrule
\end{tabular}
}
  \end{center}
\end{frame}

\note[itemize]{
  \item No problems (except for random size3-beta2 portfolio)
  \item Momentum makes the difference here
  \item Previously problematic Low-beta stocks have a new description, unprofitable stocks with strong recent returns
  \item High-beta behave like growth stocks -- volatile behave like growth stocks
  \item Beta and volatility linked because stock vol drives covariance with the market
  \item But beta not a problem -- volatility associated with the market is not a problem
  \item In fact, GRS showed beta caused the least problems of all}

\begin{frame} \frametitle{Size-variance returns}
  \begin{center}
  \resizebox*{!}{\dimexpr\textheight-1.3cm\relax}{
\begin{tabular}{lrrrrrrrrrr}
  \toprule
    
    Var $\rightarrow$ & Low & 2 & 3 & 4 & High & Low & 2 & 3 & 4 & High  \\ 
  \midrule
  \multicolumn{11}{l}{$R^i=a^i+\beta^iMkt+s^iSize+v^iVal+m^iMom+p^iProf$}  \\
  
     & \multicolumn{5}{c}{a} & \multicolumn{5}{c}{t(a)}   \\
     \cmidrule(r){2-6} \cmidrule(r){7-11} 
    Small  & 0.31  & 0.25  & 0.18  & 0.01  & \textbf{-0.75}  & 4.12  & 3.26  & 2.47  & 0.10  & -4.63   \\
    2  & 0.14  & 0.10  & 0.12  & 0.02  & -0.26  & 1.98  & 1.26  & 1.50  & 0.24  & -2.22   \\
    3  & 0.07  & 0.02  & 0.06  & -0.04  & -0.05  & 0.85  & 0.30  & 0.67  & -0.44  & -0.46   \\
    4  & 0.10  & 0.08  & -0.03  & 0.01  & 0.02  & 1.10  & 0.97  & -0.40  & 0.14  & 0.18   \\
    Big  & 0.01  & -0.05  & -0.07  & -0.10  & 0.04  & 0.16  & -0.78  & -1.11  & -1.55  & 0.34   \\
    
  
    %  & \multicolumn{5}{c}{b} & \multicolumn{5}{c}{t(b)}   \\
    %  \cmidrule(r){2-6} \cmidrule(r){7-11} 
    % Small  & 0.65  & 0.94  & 1.05  & 1.14  & 1.19  & 38.78  & 53.50  & 62.64  & 56.35  & 32.19   \\
    % 2  & 0.72  & 0.95  & 1.06  & 1.20  & 1.34  & 43.97  & 53.41  & 60.54  & 65.18  & 51.27   \\
    % 3  & 0.71  & 0.94  & 1.05  & 1.18  & 1.32  & 40.94  & 55.44  & 55.89  & 60.96  & 53.40   \\
    % 4  & 0.73  & 0.92  & 1.07  & 1.18  & 1.35  & 36.48  & 48.29  & 57.07  & 59.49  & 52.78   \\
    % Big  & 0.72  & 0.91  & 1.01  & 1.13  & 1.30  & 41.58  & 62.31  & 70.22  & 75.62  & 51.21   \\
    
  
     & \multicolumn{5}{c}{s} & \multicolumn{5}{c}{v}   \\
     \cmidrule(r){2-6} \cmidrule(r){7-11} 
    Small  & 0.61  & 0.86  & 1.01  & 1.20  & \textbf{1.51}  & 0.36  & 0.42  & 0.35  & 0.17  & \textbf{-0.00}   \\
    2  & 0.49  & 0.66  & 0.75  & 0.88  & 1.20  & 0.37  & 0.40  & 0.33  & 0.16  & -0.30   \\
    3  & 0.23  & 0.40  & 0.49  & 0.62  & 0.91  & 0.37  & 0.37  & 0.31  & 0.24  & -0.38   \\
    4  & 0.01  & 0.11  & 0.16  & 0.25  & 0.60  & 0.37  & 0.29  & 0.29  & 0.17  & -0.34   \\
    Big  & -0.30  & -0.29  & -0.20  & -0.19  & 0.10  & 0.20  & 0.16  & 0.10  & 0.03  & -0.29   \\
    
  
    %  & \multicolumn{5}{c}{v} & \multicolumn{5}{c}{t(v)}   \\
    %  \cmidrule(r){2-6} \cmidrule(r){7-11} 
    % Small  & 0.36  & 0.42  & 0.35  & 0.17  & \textbf{-0.00}  & 13.27  & 14.70  & 12.72  & 5.22  & -0.08   \\
    % 2  & 0.37  & 0.40  & 0.33  & 0.16  & -0.30  & 13.85  & 13.82  & 11.52  & 5.30  & -6.95   \\
    % 3  & 0.37  & 0.37  & 0.31  & 0.24  & -0.38  & 13.01  & 13.39  & 10.23  & 7.68  & -9.32   \\
    % 4  & 0.37  & 0.29  & 0.29  & 0.17  & -0.34  & 11.29  & 9.38  & 9.30  & 5.22  & -8.19   \\
    % Big  & 0.20  & 0.16  & 0.10  & 0.03  & -0.29  & 7.20  & 6.90  & 4.44  & 1.09  & -6.97   \\
    
  
     & \multicolumn{5}{c}{m} & \multicolumn{5}{c}{p}   \\
     \cmidrule(r){2-6} \cmidrule(r){7-11} 
    Small  & 0.08  & 0.05  & -0.06  & -0.18  & \textbf{-0.36}  & 0.17  & 0.11  & 0.04  & -0.19  & \textbf{-0.50}   \\
    2  & 0.13  & 0.12  & 0.05  & -0.05  & -0.32  & 0.19  & 0.19  & 0.12  & 0.08  & -0.47   \\
    3  & 0.15  & 0.08  & 0.08  & 0.01  & -0.27  & 0.14  & 0.17  & 0.22  & 0.16  & -0.40   \\
    4  & 0.15  & 0.06  & 0.06  & 0.01  & -0.26  & 0.03  & 0.11  & 0.22  & 0.02  & -0.33   \\
    Big  & 0.12  & 0.10  & 0.07  & 0.01  & -0.17  & -0.03  & 0.25  & 0.16  & 0.04  & -0.17   \\
    
  
    %  & \multicolumn{5}{c}{p} & \multicolumn{5}{c}{t(p)}   \\
    %  \cmidrule(r){2-6} \cmidrule(r){7-11} 
    % Small  & 0.17  & 0.11  & 0.04  & -0.19  & \textbf{-0.50}  & 3.02  & 1.97  & 0.65  & -2.88  & -4.16   \\
    % 2  & 0.19  & 0.19  & 0.12  & 0.08  & -0.47  & 3.45  & 3.31  & 2.13  & 1.31  & -5.47   \\
    % 3  & 0.14  & 0.17  & 0.22  & 0.16  & -0.40  & 2.51  & 3.02  & 3.53  & 2.58  & -4.88   \\
    % 4  & 0.03  & 0.11  & 0.22  & 0.02  & -0.33  & 0.41  & 1.83  & 3.49  & 0.36  & -3.95   \\
    % Big  & -0.03  & 0.25  & 0.16  & 0.04  & -0.17  & -0.46  & 5.13  & 3.39  & 0.87  & -2.06   \\

  \midrule
  \multicolumn{11}{l}{$R^i=a^i+\beta^iMkt+s^iSize+v^iVal+p^iProf+i^iInv$}  \\
  
     & \multicolumn{5}{c}{a} & \multicolumn{5}{c}{t(a)}   \\
     \cmidrule(r){2-6} \cmidrule(r){7-11} 
    Small  & 0.33  & 0.30  & 0.21  & -0.01  & \textbf{-0.80}  & 4.82  & 3.94  & 2.54  & -0.11  & -4.76   \\
    2  & 0.19  & 0.14  & 0.15  & 0.00  & -0.34  & 2.87  & 1.94  & 2.06  & 0.01  & -3.02   \\
    3  & 0.12  & 0.06  & 0.09  & -0.03  & -0.14  & 1.69  & 0.84  & 1.13  & -0.35  & -1.37   \\
    4  & 0.16  & 0.08  & -0.02  & 0.01  & -0.06  & 1.95  & 0.98  & -0.29  & 0.13  & -0.58   \\
    Big  & 0.03  & -0.02  & -0.05  & -0.10  & 0.02  & 0.49  & -0.31  & -0.87  & -1.56  & 0.23   \\
  
  \bottomrule
\end{tabular}
}
  \end{center}
\end{frame}

\note[itemize]{
  \item Large size effects within size buckets
  \item ``Lethal" combo new description $\rightarrow$ unprofitable stocks with poor recent returns AND value disappears
  \item Size slopes pull fitted returns in the opposite direction to other factors
  \item Small, high-variance bucket a big problem, -0.75\% per month alpha}

\begin{frame} \frametitle{Momentum and variance} \framesubtitle{Selected ``open items" from descriptions of problem sorts}
  \begin{itemize}
    \item Size-Prior: -ve profit slopes $\rightarrow$ beta and volatility
    \item Size-Beta: no alpha $\rightarrow$ not aggregate volatility
    \item Size-Var: small, high-var $\rightarrow$ high size slope and value dies
    \item \textbf{Unifying description}: small, unprofitable stocks with poor recent returns
  \end{itemize}
\end{frame}

\begin{frame} \frametitle{Volatility, size, and value}
  \begin{itemize}
    \item Value and changes in firm size \parencite{gerakos2017decomposing}
    \item Removing size factor drives the alpha to 0
    \item Ang et al. (2006) find illiquidity measures ineffective in the cross-section
  \end{itemize}
\end{frame}

\note[itemize]{
  \item Much of the value premium is to do with changes in firm size
  \item This $HML^s$ is much more related to volatility than the component of value orthogonal to $HML^s$
  \item Remove size factor: profitability slope $\sim$ -1.5 (highly unprofitable) and value slope $\sim$ -0.3 (growth) $\rightarrow$ just like high-beta
  \item Top-down approach: do not remove size factor!
  \begin{itemize}
    \item Model provides description of anomalies but choice of factors is not driven by anomalies
  \end{itemize}}

\section{Contributions}

\begin{frame} \frametitle{Contributions}
  Top-down approach: choose factors based on $\text{Sh}^2$ then describe anomalies
  \[R_t^i = a^i+b^iR^M+s^iSMB_t+\underbrace{v^iHML_t^m+m^iWML_t}_\text{cheap}+\underbrace{p^iPMU_t}_\text{profitable}\]
  \begin{itemize}
    \item Model 1 $\text{Sh}^2$ 0.316 $>$ competing models
    \item Value and momentum (\sout{investment})
    \item Unifying description $\rightarrow$ small, unprofitable stocks with poor recent returns
    \item Volatility $\rightarrow$ size and value (\sout{illiquidity})
  \end{itemize}
\end{frame}

\note[itemize]{
  \item New description for the ``lethal combination" $\rightarrow$ small, unprofitable firms with poor recent returns (and value disappears)
  \item New direction for mispricing in high-volatility portfolios beyond liquidity constraints $\rightarrow$ size and value story
  \item Beta anomaly dead, low-beta are unprofitable firms with strong recent returns
  \item Net-issues and accruals anomalies reduced}

\begin{frame} \frametitle{Volatility discussion} \framesubtitle{Prior Research: Liquidity}
  \begin{itemize}
    \item Sell small, unprofitable stocks with poor recent returns $\rightarrow$ Liquidity \parencite{nagel2005short, nagel2012evaporating}
    \item Institutional ownership, short-term reversal factor (+VIX) $\rightarrow$ market microstructure
    \item Liquidity hard to approach from a cross-section perspective
    \item Ultimately, ineffective \parencite{ang2006cross}
  \end{itemize}
\end{frame}

\begin{frame} \frametitle{How to test models/choose factors discussion}
  \begin{itemize}
    \item Bayesian \parencite{barillas2015comparing}
    \begin{itemize}
      \item Data-mining issues not solved
      \item Monthly updating best but no mention of \textbf{why}
    \end{itemize}
    \item Pure PCA \parencite{kozak2017interpreting}
    \begin{itemize}
      \item Statistical factors as good as ``reduced-form" factors
      \item Interpretation is the magic
    \end{itemize}
    \item Behavioural stories
    \begin{itemize}
      \item Covariance? \parencite{cochrane2011presidential}
    \end{itemize}
  \end{itemize}
\end{frame}

\note[itemize]{
  \item You can't point to PC1 and say ``oh look, there is the size factor" without the size factor to compare to
}

\begin{frame} \frametitle{Cheapness, value, and momentum}
  \begin{itemize}
    \item Cheap $\rightarrow$ book-to-market (BM) \parencite{fama1993common}
    \item Identify $\Delta$BM due to $\Delta$M rather than $\Delta$B
    \item Monthly value + momentum \parencite{asness2013devil, kok2017facts}
  \end{itemize}
\end{frame}

\section{References}

\begin{frame}[allowframebreaks]
  \frametitle{References}
  \printbibliography
\end{frame}


\end{document}

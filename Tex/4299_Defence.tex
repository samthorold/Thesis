\documentclass[notes]{beamer}  % remove "[notes]" to print only slides

\usepackage[backend=biber, style=authoryear]{biblatex}
\usepackage{booktabs}
\usepackage{dcolumn}
\usepackage{graphicx}
\usepackage[utf8]{inputenc}

\usetheme{Madrid}

\addbibresource{references.bib}

\title[Asset-Pricing]{A top-down approach to factor models}
\author[Thorold]{4299 Sam Thorold\\
                 Supervisors: Francisco Santos and Andr\'e Silva}
\institute{NHH and Nova SBE}
\date[June 2018]{\today}

\begin{document}

\frame{\titlepage}

\section{Intro}

\begin{frame} \frametitle{Goal}  % \framesubtitle{Goal}
  \begin{itemize}
    \item Price all stocks with some model
    \item Describe the returns on all portfolios with a linear combination of
    fewer portfolios (``factors")
  \end{itemize}
\end{frame}

\note{
  \begin{itemize}
    \item Factors are long-short portfolios designed to
    capture unobserved state variables priced by investors (ICAPM)
    \item Characteristics used to construct factors are simply sorting
    variables
    \item Factor models price the returns to equity but the factors themselves
    could be anything that creates a spread in the
    returns to equity when stocks are sorted on that characteristic
  \end{itemize}
  What the goal is not;
  \begin{itemize}
    \item Joint-hypothesis -- are markets inefficient or is the model wrong?
    \item Generate alpha
    \item ``Up and to the right" investment opportunity
    \item Pricing $\neq$ investing
  \end{itemize}
}

\begin{frame} \frametitle{Problem} \framesubtitle{Definition of ``all"}
  \begin{itemize}
    \item We do not know what ``all" portfolios are
    \item Hard to choose to factors to minimize mis-pricing for all portfolios
    \item Past research tests models within subsets of portfolios
  \end{itemize}
\end{frame}

\begin{frame} \frametitle{Problem} \framesubtitle{Data-mining}
  \begin{itemize}
    \item The prevalence and quality of stock data can quickly lead to
    data-mining in the search for mis-priced subsets of portfolios
    \item \textcite{harvey2016and} document 300+ factors in the literature
    \item \textcite{mclean2016does} find publication reduces the returns on
    many factors
    \item \textcite{linnainmaa2016history} claim ``most" factors are spurious
    out of sample
  \end{itemize}
\end{frame}

\note{
  Lies, damned lies, and statistics -- If you want to find correlation, you can
  \begin{itemize}
    \item Characteristic sort cannot be explained by a model
    \item A risk story is attached to the ``anomaly"
    \item Anomaly is made into a factor
  \end{itemize}
}

\begin{frame} \frametitle{Solution}
  \begin{itemize}
    \item \textbf{Problem}:
    no way to test models before turning to performance in anomaly sorts
    -- bottom-up approach
    \item \textbf{Solution}:
    \textcite{barillas2016alpha} show that we only need to compare the Sharpe
    ratio of the factors in models to compare mis-pricing for all portfolios
    -- top-down approach
  \end{itemize}
\end{frame}

\begin{frame} \frametitle{Top-down approach}
  \begin{itemize}
    \item Given the best model based on Sharpe ratio evidence,
    we can investigate the returns to different anomaly sorts
    \item The model provides a description of these anomalies
    \item Problem sorts do not necessarily lead to new factors even if they
    have significant alpha in time-series regressions
    \item New factors must reliably improve the Sharpe ratio of the factors
  \end{itemize}
\end{frame}

\begin{frame}
  \frametitle{Choose factors before addressing anomalies}
  \begin{itemize}
    \item ICAPM of \textcite{merton1973intertemporal}
    says investors care about their payoff in the future but
    also what they can do with it
    \item Market + state variables
    \item We do not observe the state variables but we can make proxies
    \item Investors want to identify \emph{cheap} and \emph{profitable} stocks
    \parencite{graham1934security}
  \end{itemize}
\end{frame}

\begin{frame}
\frametitle{Proxies for cheap and profitable}
\begin{itemize}
    \item Cheap $\rightarrow$ Book-to-Market (BM)
    \parencite{fama1993common}
    \item Some problems here as we want to identify changes in BM due to
    changes in market equity rather than book equity
    \item Monthly value + momentum
    \parencite{asness2013devil, kok2017facts}
    \item Profitable $\rightarrow$ Operating profit adjusted for accruals
    \parencite{ball2016accruals}
\end{itemize}
\end{frame}

\begin{frame}
  \frametitle{Preview of results}
  \begin{itemize}
    \item Highest Sharpe ratio:
    \begin{equation}
      R_t^i=a^i + b^iR_t^M + s^iSMB_t + v^iHML_t^m + m^iWML_t + p^iPMU_t
    \end{equation}
    \item Investment factor does not add to the Sharpe ratio nor do sorts on
    investment pose any problems
    \item Sorts on momentum and volatility or those sorts that constrain value
    are a problem
    \item The returns on the ``lethal combination" \parencite{fama2015five,
    fama2016dissecting} of small, unprofitable stocks that somehow invest
    aggressively behave like small, unprofitable stocks with poor recent
    returns -- disappearance of value
  \end{itemize}
\end{frame}

\section{Prior research}

\begin{frame} \frametitle{Prior research} \framesubtitle{Sharpe ratio intution}
\begin{itemize}
    \item \textcite{barillas2016alpha}
    \item We want our asset pricing models to minimize pricing errors for
    all assets
    \item When we price returns, the pricing error is given by
    \[
    \text{Sh}^2(a) = \text{Sh}^2(R, f) - \text{Sh}^2(f)
    \]
    \item If $R$ is all assets then $\text{Sh}^2(R, f) = \text{Sh}^2(R)$
    \[
    \text{Sh}^2(a) = \text{Sh}^2(R) - \text{Sh}^2(f)
    \]
    \item Model with the highest Sharpe ratio of the factors is best
\end{itemize}
\end{frame}

\begin{frame} \frametitle{Prior research} \framesubtitle{Sharpe ratio results}
  \begin{itemize}
    \item \textcite{fama2016choosing} rely heaviy on
    \textcite{barillas2016alpha} to show the accrual-adjusted profitability
    factor of \textcite{ball2016accruals} improves the Sharpe ratio of their
    original five-factor model \parencite{fama2015five}
    \begin{equation}
      R_t^i=a^i + b^iR_t^M + s^iSMB_t + v^iHML_t + p^iPMU_t + i^iCMA_t
    \end{equation}
    \item \textcite{ball2016accruals} present a model similar to the one I
    propose to show an accruals factor does not improve the Sharpe ratio of a
    five-factor model including their accrual-adjusted profitability factor
  \end{itemize}
\end{frame}

\begin{frame} \frametitle{Prior research} \framesubtitle{Anomaly results}
  \begin{itemize}
    \item \textcite{fama2016dissecting} follow the advice of
    \textcite{lewellen2010skeptical} and test their original five-factor
    model's performance in anomaly sorts constructed from variables not present
    in the model
    \item Including profit and investment factors is motivated by the dividend
    discount model \parencite{fama2006profitability}
    \[ME_t =
    \sum_{s=1}^\infty E\left( \text{Profit}_{t+s}-
    \Delta\text{BE}_{t+s}\right) /R^s\]
    \item Change in assets is used instead of change in BE
  \end{itemize}
\end{frame}

\section{Data}

\begin{frame}
  \frametitle{Data}
  \begin{itemize}
    \item Factors: Ken French all except $\text{HML}^m$ which is from AQR
    and PMU
    \item Sorts: Ken French all except size, value, investment/momentum
    \item PMU factor and all characteristics: CRSP and COMPUSTAT
    \item U.S. common stocks -- July 1963 through December 2017
    \item I include financial firms and do not winsorize variables
  \end{itemize}
\end{frame}

\section{Results}

\begin{frame}
\frametitle{Sharpe ratio}
\resizebox{\linewidth}{!}{
\begin{tabular}{lcccccc}
  \toprule
  \multicolumn{7}{l}{Panel A: Model factors and maximum squared Sharpe ratios} \\
        & \multicolumn{2}{l}{Name}                         & \multicolumn{3}{l}{Factors}                                   & Sh2   \\
  1     & \multicolumn{2}{l}{Model 1}                      & \multicolumn{3}{l}{$R^M$, $SMB$, $HML^m$, $PMU$, $WML$}        & 0.316 \\
  2     & \multicolumn{2}{l}{Model 1 + $CMA$}              & \multicolumn{3}{l}{$R^M$, $SMB$, $HML^m$, $PMU$, $WML$, $CMA$} & 0.316 \\
  3     & \multicolumn{2}{l}{Fama and French (2017)}       & \multicolumn{3}{l}{$R^M$, $SMB$, $HML$, $PMU$, $WML$, $CMA$}   & 0.240 \\
  4     & \multicolumn{2}{l}{Model 2}                      & \multicolumn{3}{l}{$R^M$, $SMB$, $HML$, $PMU$, $CMA$}          & 0.225 \\
  5     & \multicolumn{2}{l}{Fama and French (2015)}       & \multicolumn{3}{l}{$R^M$, $SMB$, $HML$, $PMU^{06}$, $CMA$}     & 0.099 \\
  6     & \multicolumn{2}{l}{Carhart (1997)}               & \multicolumn{3}{l}{$R^M$, $SMB$, $HML$, $WML$}                 & 0.090 \\
  7     & \multicolumn{2}{l}{Carhart (1997) + $HML^m$}     & \multicolumn{3}{l}{$R^M$, $SMB$, $HML^m$, $WML$}               & 0.136 \\
  \midrule
  \multicolumn{7}{l}{Panel B: 90\% confidence interval for distributions of Sh2(Row) - Sh2(Column)} \\
        &        2         &        3         &        4         &        5         &        6         &        7         \\
  1     & (-0.008,  0.000) & ( 0.039,  0.112) & ( 0.055,  0.138) & ( 0.156,  0.291) & ( 0.169,  0.299) & ( 0.127,  0.248) \\
  2     &                  & ( 0.043,  0.115) & ( 0.058,  0.140) & ( 0.158,  0.294) & ( 0.170,  0.302) & ( 0.129,  0.250) \\
  3     &                  &                  & ( 0.002,  0.048) & ( 0.095,  0.204) & ( 0.107,  0.212) & ( 0.052,  0.172) \\
  4     &                  &                  &                  & ( 0.081,  0.176) & ( 0.080,  0.197) & ( 0.028,  0.155) \\
  5     &                  &                  &                  &                  & (-0.033,  0.052) & (-0.085,  0.012) \\
  6     &                  &                  &                  &                  &                  & (-0.071, -0.025) \\
  \bottomrule
\end{tabular}

}
\end{frame}

\note[itemize]{
  \item Investment factor does not improve Sharpe ratio of model 1
  \item Model 1 has higher Sharpe ratio than model 2 (+ momentum)
  \item Original five-factor model does not have a reliably higher Sharpe ratio
  than the four-factor model of Carhart
  \item Interesting since profitability and investment are credited with
  reducing the maginitude and number of descriptions of many common anomalies
  \item Investment adds nothing, but which factors subsume it?
}

\begin{frame}
\frametitle{Value and momentum subsume investment}
\begin{center}
\resizebox*{!}{\dimexpr\textheight-1.3cm\relax}{
    
\begin{tabular}{lD{.}{.}{2.4}D{.}{.}{2.4}D{.}{.}{2.4}D{.}{.}{2.4}D{.}{.}{2.4}D{.}{.}{2.4}D{.}{.}{2.4}D{.}{.}{2.4}}
\toprule
          &\multicolumn{1}{c}{(1)}&\multicolumn{1}{c}{(2)}&\multicolumn{1}{c}{(3)}&\multicolumn{1}{c}{(4)}\\
\midrule
$a$       &    -0.04&     0.08&     0.01&     0.12 \\
          &  (-0.70)&   (1.32)&   (0.08)&   (2.01) \\
[1em]
$R^M$     &         &         &    -0.09&    -0.09 \\
          &         &         &  (-6.31)&   (-6.23)\\
[1em]
$SMB$     &         &         &     0.00&     0.01 \\
          &         &         &   (0.99)&    (0.53)\\
[1em]
$HML$     &         &     0.51&         &     0.48 \\
          &         &  (25.32)&         &  (22.49) \\
[1em]
$HML^m$   &     0.52&         &     0.49&          \\
          &  (23.24)&         &  (20.72)&          \\
[1em]
$WML$     &     0.26&     0.05&     0.22&     0.03 \\
          &  (14.45)&   (3.90)&  (12.42)&    (2.23)\\
[1em]
$PMU$     &         &         &     0.10&     0.07 \\
          &         &         &   (2.03)&    (1.56)\\
\midrule
adj. \(R^{2}\)&    0.45&    0.50&    0.50&    0.53 \\
\bottomrule
\multicolumn{5}{r}{\footnotesize \textit{t} statistics in parentheses}\\
\end{tabular}


    }
\end{center}
\end{frame}

\note[itemize]{
  \item Annual or monthly momentum subsume investment
  \item Monthly value and momentum more robust to including other factors
  \item Why do value and momentum subsume investment?
}

\begin{frame}
\frametitle{Size-$\text{BM}^m$-Prior sorts}
\begin{center}
\resizebox*{!}{\dimexpr\textheight-1.3cm\relax}{
    
\begin{tabular}{lrrrrrrrr}
  \toprule
     & \multicolumn{4}{c}{Small} & \multicolumn{4}{c}{Big}  \\
     \cmidrule(r){2-5} \cmidrule(r){6-9}
    $\text{BM}^m$ $\rightarrow$ & Low & 2 & 3 & High & Low & 2 & 3 & High  \\ 
  \midrule
  
  
    & \multicolumn{8}{c}{$\text{R}^i$}  \\
     \cmidrule(r){2-5} \cmidrule(r){6-9}
    Low Prior  & 0.09  & 0.76  & 0.98  & 1.01  & 0.55  & 0.75  & 0.95  & 0.96   \\
    2  & 0.69  & 1.07  & 1.24  & 1.51  & 0.64  & 0.82  & 0.9  & 1.1   \\
    3  & 0.96  & 1.26  & 1.55  & 1.71  & 0.9  & 0.87  & 0.99  & 1.23   \\
    High Prior  & 1.47  & 1.58  & 1.78  & 2.04  & 1.21  & 1.11  & 1.14  & 1.42   \\
    
  
    & \multicolumn{8}{c}{Future Var}  \\
     \cmidrule(r){2-5} \cmidrule(r){6-9}
    Low Prior  & 0.2  & 0.13  & 0.13  & 0.19  & 0.05  & 0.05  & 0.04  & 0.05   \\
    2  & 0.1  & 0.07  & 0.07  & 0.09  & 0.03  & 0.03  & 0.03  & 0.03   \\
    3  & 0.08  & 0.06  & 0.06  & 0.08  & 0.03  & 0.03  & 0.03  & 0.03   \\
    High Prior  & 0.11  & 0.08  & 0.08  & 0.12  & 0.05  & 0.04  & 0.04  & 0.04   \\
    
  
    & \multicolumn{8}{c}{Future Inv}  \\
     \cmidrule(r){2-5} \cmidrule(r){6-9}
    Low Prior  & 0.39  & 0.24  & 0.19  & 0.2  & 0.21  & 0.16  & 0.14  & 0.2   \\
    2  & 0.27  & 0.16  & 0.11  & 0.09  & 0.17  & 0.13  & 0.11  & 0.1   \\
    3  & 0.24  & 0.13  & 0.09  & 0.07  & 0.17  & 0.13  & 0.11  & 0.09   \\
    High Prior  & 0.25  & 0.12  & 0.1  & 0.06  & 0.23  & 0.14  & 0.12  & 0.09   \\
    
  
    & \multicolumn{8}{c}{Future $\Delta\text{BE}$}  \\
     \cmidrule(r){2-5} \cmidrule(r){6-9}
    Low Prior  & 0.31  & 0.07  & 0.01  & -0.1  & 0.42  & 0.11  & 0.07  & 0.02   \\
    2  & 0.3  & 0.09  & 0.04  & -0.02  & 0.25  & 0.11  & 0.08  & 0.04   \\
    3  & 0.34  & 0.1  & 0.05  & -0.0  & 0.27  & 0.11  & 0.08  & 0.05   \\
    High Prior  & 0.61  & 0.1  & 0.04  & -0.03  & 0.53  & 0.14  & 0.1  & 0.04   \\
    
  
  \bottomrule
\end{tabular}

    }
\end{center}
\end{frame}

\note[itemize]{
  \item Large dispursion in returns -- potential to describe a lot of variation
  \item Little variation in variance
  \item Future investment and future changes in BE backwards for small, growth
  stocks
  \item Monthly value and investment capture variation on both
}

\begin{frame}
\frametitle{Size-BM-Inv sorts}
\begin{center}
\resizebox*{!}{\dimexpr\textheight-1.3cm\relax}{
    
\begin{tabular}{lrrrrrrrr}
  \toprule
     & \multicolumn{4}{c}{Small} & \multicolumn{4}{c}{Big}  \\
     \cmidrule(r){2-5} \cmidrule(r){6-9}
    BM $\rightarrow$ & Low & 2 & 3 & High & Low & 2 & 3 & High  \\ 
  \midrule
  
  
    % & \multicolumn{8}{c}{$\text{r}^i$}  \\
    %  \cmidrule(r){2-5} \cmidrule(r){6-9}
    % Low Inv  & 1.01  & 1.33  & 1.45  & 1.46  & 0.98  & 1.04  & 1.05  & 1.14   \\
    % 2  & 1.22  & 1.31  & 1.30  & 1.48  & 0.92  & 0.93  & 0.95  & 0.99   \\
    % 3  & 1.25  & 1.29  & 1.41  & 1.28  & 0.92  & 0.94  & 0.93  & 1.16   \\
    % High Inv  & 0.94  & 1.17  & 1.27  & 1.31  & 0.98  & 0.80  & 0.90  & 1.06   \\
    
  
    % & \multicolumn{8}{c}{Future Var}  \\
    %  \cmidrule(r){2-5} \cmidrule(r){6-9}
    % Low Inv  & 0.16  & 0.11  & 0.10  & 0.14  & 0.04  & 0.03  & 0.03  & 0.04   \\
    % 2  & 0.09  & 0.07  & 0.07  & 0.09  & 0.03  & 0.03  & 0.03  & 0.03   \\
    % 3  & 0.08  & 0.06  & 0.06  & 0.09  & 0.03  & 0.03  & 0.03  & 0.04   \\
    % High Inv  & 0.11  & 0.09  & 0.08  & 0.11  & 0.05  & 0.04  & 0.04  & 0.04   \\
    
  
    & \multicolumn{8}{c}{Future Inv}  \\
     \cmidrule(r){2-5} \cmidrule(r){6-9}
    Low Inv  & \textbf{-0.10}  & -0.08  & -0.08  & -0.09  & -0.02  & -0.02  & -0.01  & -0.02   \\
    2  & \textbf{0.04}  & 0.03  & 0.03  & 0.02  & 0.06  & 0.06  & 0.06  & 0.06   \\
    3  & \textbf{0.11}  & 0.10  & 0.10  & 0.10  & 0.12  & 0.12  & 0.12  & 0.11   \\
    High Inv  & \textbf{0.48}  & 0.39  & 0.37  & 0.47  & 0.37  & 0.38  & 0.37  & 0.52   \\
    
  
    & \multicolumn{8}{c}{Future $\Delta\text{BE}$}  \\
     \cmidrule(r){2-5} \cmidrule(r){6-9}
    Low Inv  & \textbf{0.59}  & 0.06  & 0.02  & -0.03  & 0.28  & 0.07  & 0.06  & 0.03   \\
    2  & \textbf{0.44}  & 0.09  & 0.05  & 0.01  & 0.20  & 0.10  & 0.07  & 0.05   \\
    3  & \textbf{0.27}  & 0.11  & 0.08  & 0.03  & 0.16  & 0.12  & 0.12  & 0.08   \\
    High Inv  & \textbf{0.40}  & 0.16  & 0.11  & 0.06  & 0.38  & 0.17  & 0.13  & 0.10   \\
    
  
  \bottomrule
\end{tabular}

    }
\end{center}
\end{frame}

\note[itemize]{
  \item Less dispursion in returns
  \item Little variation in variance
  \item Future investment and future changes in BE backwards for small, growth
  stocks
  \item Problem because investment is a factor and so will not capture small,
  growth stock's true relationship with future changes in BE
}

\begin{frame}
  \begin{itemize}
    \item Model 1 has the highest Sharpe ratio and so minimizes mis-pricing for
    all portfolios
    \item Value and momentum's relationship with future changes in BE makes the
    investment factor redundant
    \item Model 1 may be best for all portfolios but which portfolios cause
    problems?
  \end{itemize}
\end{frame}

\begin{frame}
\frametitle{GRS statistic}
\begin{center}
\resizebox*{!}{\dimexpr\textheight-1.3cm\relax}{
    
\begin{tabular}{lrrrrrrr}
  \toprule
     & \rotatebox{90}{\footnotesize{Carhart (1997)}} &
       \rotatebox{90}{\footnotesize{Carhart (1997) + $\text{HML}^m$}} &
       \rotatebox{90}{\footnotesize{Fama and French (2015)}} &
       \rotatebox{90}{\textbf{\footnotesize{Model 2}}} &
       \rotatebox{90}{\footnotesize{Model 2 + WML}} &
       \rotatebox{90}{\textbf{\footnotesize{Model 1}}} &
       \rotatebox{90}{\footnotesize{Model 1 + CMA}} \\
  \midrule

    % \multicolumn{8}{l}{Value constrained} \\
    Size-BM-Inv  & 2.58  & 3.44  & 2.32  & 1.56  & 1.49  & 2.13  & 2.13  \\
    BM constrained  & 2.31  & 2.69  & 2.14  & \textbf{1.62}  & 1.58  & \textbf{1.81}  & 1.82  \\
    % [1em]
    % Size-$\text{BM}^m$-Prior  & 7.26  & 5.91  & 7.55  & 6.52  & 6.51  & 4.85  & 4.85  \\
    % Value constrained  & 3.95  & 3.67  & 3.74  & 3.21  & 3.22  & 2.83  & 2.83  \\
    [1em]

    \multicolumn{8}{l}{Selected anomalies} \\
    Size-Inv  & 3.82  & 3.40  & 3.45  & \textbf{2.43}  & 2.31  & \textbf{2.02}  & 2.02  \\
    % Size-Acc  & 3.26  & 3.10  & 3.89  & 2.33  & 2.21  & 2.00  & 2.07  \\
    Size-$\beta$  & 1.51  & \textbf{1.22}  & \textbf{1.83}  & \textbf{1.63}  & 1.52  & \textbf{1.15}  & 1.15  \\
    % Size-NI  & 3.67  & 3.42  & 3.35  & 2.26  & 2.13  & 1.84  & 1.84  \\
    Size-Prior  & 3.96  & \textbf{3.89}  & \textbf{4.43}  & \textbf{3.94}  & 3.59  & \textbf{3.62}  & 3.66  \\
    Size-Var  & 5.11  & 5.09  & 5.02  & \textbf{4.38}  & 4.44  & \textbf{4.13}  & 4.14  \\
    Selected anomalies  & 2.84  & 2.78  & 2.80  & 2.37  & 2.32  & 2.23  & 2.22  \\
    [1em]
    All  & 2.85  & 2.70  & 2.80  & \textbf{2.44}  & 2.40  & \textbf{2.23}  & 2.22  \\

  \bottomrule
\end{tabular}


    }
\end{center}
\end{frame}

\note[itemize]{
  \item Annual value a problem -- unwanted variation in BM within BMm buckets
  \item All other sorts a wrap for Model 1
  \item Investment factor adds nothing to subsets of stocks
  \item Momentum and volatility biggest problems
}

\begin{frame}
  \frametitle{Size-investment sorts}
  \begin{center}
  \resizebox*{!}{\dimexpr\textheight-1.3cm\relax}{
    

\begin{tabular}{lrrrrrrrrrr}
  \toprule
    
    Inv $\rightarrow$ & Low & 2 & 3 & 4 & High & Low & 2 & 3 & 4 & High  \\ 
  \midrule
  \multicolumn{11}{l}{$R^i=a^i+\beta^iMkt+s^iSize+v^iVal+m^iMom+p^iProf$}  \\
  
     & \multicolumn{5}{c}{a} & \multicolumn{5}{c}{t(a)}   \\
     \cmidrule(r){2-6} \cmidrule(r){7-11} 
    Small  & 0.12  & 0.09  & 0.10  & 0.08  & -0.24  & 1.18  & 1.49  & 1.53  & 1.19  & -3.33   \\
    2  & -0.04  & 0.01  & 0.10  & 0.03  & -0.07  & -0.54  & 0.11  & 1.68  & 0.57  & -1.09   \\
    3  & -0.01  & 0.08  & 0.02  & 0.03  & 0.05  & -0.17  & 1.23  & 0.39  & 0.50  & 0.70   \\
    4  & -0.04  & -0.03  & 0.04  & 0.11  & 0.12  & -0.48  & -0.39  & 0.70  & 1.54  & 1.45   \\
    Big  & 0.08  & -0.04  & -0.10  & 0.00  & 0.13  & 0.86  & -0.61  & -1.97  & 0.05  & 1.70   \\
    
  
     & \multicolumn{5}{c}{v} & \multicolumn{5}{c}{t(v)}   \\
     \cmidrule(r){2-6} \cmidrule(r){7-11} 
    Small  & 0.08  & 0.25  & 0.24  & 0.12  & -0.18  & 2.24  & 11.09  & 9.86  & 4.87  & -6.80   \\
    2  & 0.25  & 0.32  & 0.22  & 0.19  & -0.32  & 9.91  & 12.88  & 9.88  & 8.76  & -14.09   \\
    3  & 0.28  & 0.34  & 0.27  & 0.11  & -0.33  & 9.07  & 13.52  & 11.44  & 4.53  & -12.27   \\
    4  & 0.33  & 0.38  & 0.25  & 0.07  & -0.37  & 10.97  & 14.83  & 10.96  & 2.75  & -12.24   \\
    Big  & 0.26  & 0.22  & 0.18  & -0.03  & -0.37  & 8.01  & 10.05  & 9.97  & -1.60  & -12.93   \\
    
  
     & \multicolumn{5}{c}{m} & \multicolumn{5}{c}{t(m)}   \\
     \cmidrule(r){2-6} \cmidrule(r){7-11} 
    Small  & 0.04  & 0.11  & 0.12  & 0.07  & -0.10  & 1.59  & 6.19  & 6.34  & 3.97  & -4.87   \\
    2  & 0.09  & 0.12  & 0.10  & 0.03  & -0.16  & 4.49  & 6.26  & 5.81  & 1.97  & -9.09   \\
    3  & 0.06  & 0.15  & 0.10  & 0.05  & -0.17  & 2.70  & 7.99  & 5.32  & 2.72  & -8.11   \\
    4  & 0.08  & 0.08  & 0.11  & 0.06  & -0.14  & 3.44  & 4.15  & 6.04  & 2.97  & -6.23   \\
    Big  & 0.06  & 0.12  & 0.07  & 0.00  & -0.17  & 2.45  & 6.99  & 5.25  & 0.13  & -7.68   \\

  \midrule
  \multicolumn{11}{l}{$R^i=a^i+\beta^iMkt+s^iSize+v^iVal+p^iProf+i^iInv+m^iMom$}  \\
  
     & \multicolumn{5}{c}{a} & \multicolumn{5}{c}{t(a)}   \\
     \cmidrule(r){2-6} \cmidrule(r){7-11} 
    Small  & 0.12  & 0.11  & 0.13  & 0.07  & -0.24  & 1.34  & 1.96  & 2.05  & 1.22  & -3.48   \\
    2  & -0.02  & 0.04  & 0.12  & 0.07  & -0.09  & -0.39  & 0.58  & 2.04  & 1.35  & -1.57   \\
    3  & 0.00  & 0.12  & 0.07  & 0.05  & 0.04  & 0.05  & 1.93  & 1.10  & 0.86  & 0.57   \\
    4  & -0.03  & 0.04  & 0.09  & 0.11  & 0.08  & -0.37  & 0.60  & 1.48  & 1.64  & 1.14   \\
    Big  & 0.05  & -0.04  & -0.07  & 0.00  & 0.14  & 0.68  & -0.76  & -1.45  & 0.01  & 2.35   \\

  \bottomrule
\end{tabular}

    }
  \end{center}
\end{frame}

\note[itemize]{
  \item Value picks up investment slopes
}

\begin{frame}
  \frametitle{Size-BM-investment sorts}
  \begin{center}
  \resizebox*{!}{\dimexpr\textheight-1.3cm\relax}{
    
\begin{tabular}{lrrrrrrrrrrrrrrrr}
  \toprule
     & \multicolumn{8}{c}{Small} & \multicolumn{8}{c}{Big}  \\
     \cmidrule(r){2-9} \cmidrule(r){10-17}
    BM $\rightarrow$ & Low & 2 & 3 & High & Low & 2 & 3 & High & Low & 2 & 3 & High & Low & 2 & 3 & High  \\ 
  \midrule
  \multicolumn{17}{l}{$R^i=a^i+\beta^iMkt+s^iSize+v^iVal+m^iMom+p^iProf$}  \\
  
     & \multicolumn{4}{c}{a} & \multicolumn{4}{c}{t(a)}  & \multicolumn{4}{c}{a} & \multicolumn{4}{c}{t(a)}   \\
     \cmidrule(r){2-5} \cmidrule(r){6-9}  \cmidrule(r){10-13} \cmidrule(r){14-17} 
    Low Inv  & 0.05  & 0.11  & 0.16  & -0.12  & 0.50  & 1.36  & 1.92  & -1.25  & 0.13  & 0.03  & -0.08  & -0.21  & 1.25  & 0.36  & -0.83  & -2.39   \\
    2  & 0.15  & 0.02  & -0.15  & -0.12  & 2.06  & 0.30  & -2.46  & -1.65  & 0.08  & -0.03  & -0.09  & -0.31  & 0.88  & -0.35  & -1.12  & -3.41   \\
    3  & 0.17  & -0.04  & 0.03  & -0.24  & 2.96  & -0.62  & 0.41  & -2.63  & 0.05  & -0.11  & -0.12  & -0.10  & 0.61  & -1.36  & -1.36  & -0.98   \\
    High Inv  & -0.05  & -0.10  & -0.15  & -0.21  & -0.79  & -1.48  & -1.87  & -1.71  & 0.32  & -0.10  & -0.18  & -0.14  & 3.76  & -0.94  & -1.66  & -1.22   \\
    
  
     & \multicolumn{4}{c}{v} & \multicolumn{4}{c}{t(v)}  & \multicolumn{4}{c}{v} & \multicolumn{4}{c}{t(v)}   \\
     \cmidrule(r){2-5} \cmidrule(r){6-9}  \cmidrule(r){10-13} \cmidrule(r){14-17} 
    Low Inv  & -0.28  & 0.18  & 0.37  & 0.66  & -7.65  & 5.88  & 12.12  & 19.49  & -0.10  & 0.17  & 0.38  & 0.67  & -2.64  & 5.07  & 11.31  & 21.33   \\
    2  & -0.04  & 0.35  & 0.50  & 0.72  & -1.65  & 14.40  & 22.08  & 27.13  & -0.10  & 0.10  & 0.40  & 0.66  & -3.04  & 3.22  & 13.26  & 20.34   \\
    3  & -0.11  & 0.33  & 0.50  & 0.66  & -5.53  & 13.83  & 21.59  & 19.89  & -0.16  & 0.12  & 0.41  & 0.73  & -5.44  & 4.15  & 12.81  & 19.72   \\
    High Inv  & -0.46  & 0.14  & 0.41  & 0.45  & -20.40  & 5.88  & 14.42  & 10.19  & -0.66  & 0.08  & 0.33  & 0.47  & -21.50  & 2.18  & 8.51  & 11.52   \\
    
  
     & \multicolumn{4}{c}{m} & \multicolumn{4}{c}{t(m)}  & \multicolumn{4}{c}{m} & \multicolumn{4}{c}{t(m)}   \\
     \cmidrule(r){2-5} \cmidrule(r){6-9}  \cmidrule(r){10-13} \cmidrule(r){14-17} 
    Low Inv  & -0.06  & 0.04  & 0.16  & 0.19  & -2.11  & 1.84  & 6.66  & 7.09  & -0.01  & 0.06  & 0.14  & 0.20  & -0.25  & 2.15  & 5.36  & 8.34   \\
    2  & -0.02  & 0.12  & 0.22  & 0.24  & -1.19  & 6.56  & 12.41  & 11.76  & -0.06  & 0.06  & 0.17  & 0.25  & -2.48  & 2.72  & 7.13  & 10.06   \\
    3  & -0.03  & 0.12  & 0.19  & 0.18  & -2.05  & 6.27  & 10.86  & 7.04  & -0.08  & 0.06  & 0.19  & 0.23  & -3.47  & 2.71  & 7.80  & 8.02   \\
    High Inv  & -0.22  & 0.03  & 0.15  & 0.15  & -12.66  & 1.81  & 7.00  & 4.42  & -0.27  & 0.00  & 0.06  & 0.13  & -11.20  & 0.11  & 2.03  & 4.28   \\
    
  
  \midrule
  \multicolumn{17}{l}{$R^i=a^i+\beta^iMkt+s^iSize+v^iVal+p^iProf+i^iInv$}  \\
  
     & \multicolumn{4}{c}{a} & \multicolumn{4}{c}{t(a)}  & \multicolumn{4}{c}{a} & \multicolumn{4}{c}{t(a)}   \\
     \cmidrule(r){2-5} \cmidrule(r){6-9}  \cmidrule(r){10-13} \cmidrule(r){14-17} 
    Low Inv  & -0.09  & 0.09  & 0.18  & -0.04  & -1.00  & 1.16  & 2.48  & -0.49  & 0.05  & -0.01  & -0.06  & -0.11  & 0.52  & -0.12  & -0.75  & -1.47   \\
    2  & 0.12  & 0.08  & -0.03  & 0.07  & 1.77  & 1.29  & -0.56  & 0.96  & 0.03  & -0.02  & 0.02  & -0.12  & 0.30  & -0.30  & 0.28  & -1.51   \\
    3  & 0.14  & 0.05  & 0.15  & -0.09  & 2.56  & 0.80  & 2.69  & -1.00  & 0.02  & -0.06  & 0.02  & 0.10  & 0.30  & -0.78  & 0.24  & 1.11   \\
    High Inv  & -0.13  & -0.05  & -0.03  & -0.11  & -2.49  & -0.80  & -0.47  & -0.98  & 0.22  & -0.02  & -0.03  & -0.01  & 3.19  & -0.26  & -0.29  & -0.07   \\
    
  
     & \multicolumn{4}{c}{v} & \multicolumn{4}{c}{t(v)}  & \multicolumn{4}{c}{v} & \multicolumn{4}{c}{t(v)}   \\
     \cmidrule(r){2-5} \cmidrule(r){6-9}  \cmidrule(r){10-13} \cmidrule(r){14-17} 
    Low Inv  & -0.57  & -0.08  & 0.21  & 0.52  & -13.31  & -2.24  & 6.14  & 13.36  & -0.40  & -0.16  & 0.05  & 0.59  & -9.07  & -4.21  & 1.43  & 16.36   \\
    2  & -0.18  & 0.24  & 0.40  & 0.70  & -5.67  & 8.35  & 15.03  & 21.94  & -0.19  & -0.11  & 0.25  & 0.70  & -4.64  & -3.04  & 6.62  & 18.01   \\
    3  & -0.15  & 0.34  & 0.53  & 0.69  & -6.03  & 11.42  & 20.01  & 16.69  & -0.10  & 0.09  & 0.32  & 0.96  & -2.93  & 2.38  & 7.81  & 22.58   \\
    High Inv  & -0.31  & 0.28  & 0.55  & 0.74  & -12.89  & 9.90  & 16.40  & 14.39  & -0.35  & 0.24  & 0.50  & 0.64  & -10.90  & 5.34  & 10.66  & 13.07   \\
    

     & \multicolumn{4}{c}{i} & \multicolumn{4}{c}{t(i)}  & \multicolumn{4}{c}{i} & \multicolumn{4}{c}{t(i)}   \\
     \cmidrule(r){2-5} \cmidrule(r){6-9}  \cmidrule(r){10-13} \cmidrule(r){14-17} 
    Low Inv  & 0.61  & 0.53  & 0.42  & 0.38  & 10.17  & 10.88  & 8.75  & 6.93  & 0.60  & 0.68  & 0.68  & 0.27  & 9.65  & 12.98  & 12.91  & 5.41   \\
    2  & 0.24  & 0.23  & 0.20  & 0.02  & 5.41  & 5.60  & 5.45  & 0.40  & 0.17  & 0.39  & 0.23  & -0.06  & 3.03  & 7.43  & 4.25  & -1.12   \\
    3  & 0.07  & -0.02  & -0.03  & -0.04  & 1.87  & -0.50  & -0.81  & -0.61  & -0.13  & 0.03  & 0.11  & -0.38  & -2.59  & 0.56  & 1.97  & -6.41   \\
    High Inv  & -0.37  & -0.26  & -0.21  & -0.41  & -11.07  & -6.39  & -4.56  & -5.74  & -0.65  & -0.36  & -0.44  & -0.30  & -14.33  & -5.81  & -6.59  & -4.36   \\
    

  \bottomrule
\end{tabular}


    }
  \end{center}
\end{frame}

\note[itemize]{
  \item Constraining value results in worse performance (less variation within
  investment buckets)
  \item Model fails predictably
}

\begin{frame}
  \frametitle{Size-momentum sorts: Model 1}
  \begin{center}
  \resizebox*{!}{\dimexpr\textheight-1.3cm\relax}{
    
\begin{tabular}{lrrrrrrrrrr}
  \toprule
    
    Prior $\rightarrow$ & Low & 2 & 3 & 4 & High & Low & 2 & 3 & 4 & High  \\ 
  \midrule
  \multicolumn{11}{l}{$R^i=a^i+\beta^iMkt+s^iSize+v^iVal+m^iMom+p^iProf$}  \\
  
     & \multicolumn{5}{c}{a} & \multicolumn{5}{c}{t(a)}   \\
     \cmidrule(r){2-6} \cmidrule(r){7-11} 
    Small  & -0.33  & -0.07  & 0.06  & 0.13  & 0.38  & -3.13  & -1.09  & 0.95  & 1.91  & 4.42   \\
    2  & -0.11  & -0.01  & -0.01  & 0.03  & 0.26  & -1.45  & -0.22  & -0.11  & 0.45  & 3.84   \\
    3  & 0.15  & 0.04  & -0.05  & -0.17  & 0.21  & 1.59  & 0.61  & -0.83  & -2.48  & 3.00   \\
    4  & 0.15  & 0.10  & -0.01  & -0.01  & 0.11  & 1.41  & 1.29  & -0.18  & -0.08  & 1.43   \\
    Big  & 0.12  & 0.20  & -0.03  & -0.15  & -0.10  & 1.06  & 2.75  & -0.50  & -2.43  & -1.30   \\
    
  
     & \multicolumn{5}{c}{b} & \multicolumn{5}{c}{t(b)}   \\
     \cmidrule(r){2-6} \cmidrule(r){7-11} 
    Small  & 1.05  & 0.90  & 0.88  & 0.89  & 1.02  & 43.53  & 59.04  & 58.29  & 57.30  & 52.22   \\
    2  & 1.16  & 0.99  & 0.95  & 0.98  & 1.12  & 64.60  & 66.97  & 66.89  & 74.40  & 72.40   \\
    3  & 1.13  & 1.01  & 0.97  & 1.00  & 1.12  & 51.65  & 64.95  & 64.91  & 62.90  & 70.41   \\
    4  & 1.16  & 1.06  & 1.01  & 1.02  & 1.11  & 47.93  & 62.51  & 61.74  & 63.20  & 62.57   \\
    Big  & 1.16  & 0.95  & 0.95  & 0.99  & 1.09  & 46.49  & 56.81  & 60.95  & 67.68  & 64.46   \\
    
  
     & \multicolumn{5}{c}{s} & \multicolumn{5}{c}{t(s)}   \\
     \cmidrule(r){2-6} \cmidrule(r){7-11} 
    Small  & 1.22  & 0.98  & 0.91  & 0.93  & 1.13  & 36.78  & 47.15  & 44.24  & 43.71  & 41.98   \\
    2  & 0.93  & 0.80  & 0.70  & 0.78  & 0.93  & 37.97  & 39.51  & 36.15  & 43.56  & 44.05   \\
    3  & 0.58  & 0.49  & 0.50  & 0.47  & 0.70  & 19.45  & 23.01  & 24.16  & 21.64  & 32.41   \\
    4  & 0.29  & 0.19  & 0.20  & 0.18  & 0.43  & 8.78  & 8.32  & 8.80  & 8.38  & 17.76   \\
    Big  & -0.11  & -0.17  & -0.19  & -0.21  & -0.03  & -3.29  & -7.65  & -8.91  & -10.31  & -1.32   \\
    
  
     & \multicolumn{5}{c}{v} & \multicolumn{5}{c}{t(v)}   \\
     \cmidrule(r){2-6} \cmidrule(r){7-11} 
    Small  & 0.08  & 0.33  & 0.36  & 0.29  & 0.02  & 2.06  & 13.88  & 14.94  & 11.64  & 0.77   \\
    2  & -0.06  & 0.24  & 0.29  & 0.29  & -0.09  & -2.14  & 10.44  & 12.81  & 13.98  & -3.67   \\
    3  & -0.09  & 0.20  & 0.31  & 0.32  & -0.08  & -2.49  & 8.33  & 13.23  & 12.81  & -3.17   \\
    4  & -0.03  & 0.22  & 0.27  & 0.24  & -0.05  & -0.73  & 8.05  & 10.47  & 9.45  & -1.91   \\
    Big  & -0.04  & 0.12  & 0.11  & 0.17  & -0.01  & -1.08  & 4.65  & 4.58  & 7.30  & -0.43   \\
    
  
     & \multicolumn{5}{c}{m} & \multicolumn{5}{c}{t(m)}   \\
     \cmidrule(r){2-6} \cmidrule(r){7-11} 
    Small  & -0.64  & -0.11  & 0.07  & 0.19  & 0.31  & -21.82  & -6.08  & 3.63  & 10.24  & 13.01   \\
    2  & -0.73  & -0.22  & 0.05  & 0.20  & 0.33  & -33.36  & -12.16  & 3.05  & 12.40  & 17.30   \\
    3  & -0.79  & -0.25  & -0.01  & 0.22  & 0.38  & -29.42  & -13.03  & -0.70  & 11.44  & 19.61   \\
    4  & -0.81  & -0.29  & -0.04  & 0.18  & 0.43  & -27.40  & -14.04  & -1.91  & 9.31  & 19.68   \\
    Big  & -0.78  & -0.38  & -0.06  & 0.23  & 0.47  & -25.67  & -18.86  & -3.05  & 13.17  & 23.01   \\
    
  
     & \multicolumn{5}{c}{p} & \multicolumn{5}{c}{t(p)}   \\
     \cmidrule(r){2-6} \cmidrule(r){7-11} 
    Small  & -0.22  & -0.04  & 0.02  & 0.03  & -0.20  & -2.84  & -0.75  & 0.42  & 0.57  & -3.18   \\
    2  & -0.21  & 0.12  & 0.07  & 0.09  & -0.18  & -3.60  & 2.49  & 1.49  & 2.09  & -3.61   \\
    3  & -0.22  & 0.06  & 0.12  & 0.11  & -0.09  & -3.13  & 1.28  & 2.45  & 2.17  & -1.80   \\
    4  & -0.22  & 0.04  & 0.09  & 0.04  & -0.08  & -2.80  & 0.65  & 1.62  & 0.73  & -1.36   \\
    Big  & 0.07  & 0.12  & 0.03  & 0.09  & -0.01  & 0.92  & 2.18  & 0.61  & 1.84  & -0.10   \\
    
  
  \bottomrule
\end{tabular}


    }
  \end{center}
\end{frame}

\note[itemize]{
  \item Negative profitability slopes suggest defensive equity
}

\begin{frame}
  \frametitle{Size-momentum sorts: Model 2}
  \begin{center}
  \resizebox*{!}{\dimexpr\textheight-1.3cm\relax}{
    
\begin{tabular}{lrrrrrrrrrr}
  \toprule
    
    Prior $\rightarrow$ & Low & 2 & 3 & 4 & High & Low & 2 & 3 & 4 & High  \\ 
  \midrule
  \multicolumn{11}{l}{$R^i=a^i+\beta^iMkt+s^iSize+v^iVal+p^iProf+i^iInv$}  \\
  
     & \multicolumn{5}{c}{a} & \multicolumn{5}{c}{t(a)}   \\
     \cmidrule(r){2-6} \cmidrule(r){7-11} 
    Small  & -0.55  & -0.06  & 0.15  & 0.26  & 0.54  & -3.54  & -0.81  & 2.31  & 3.89  & 5.55   \\
    2  & -0.44  & -0.07  & 0.06  & 0.17  & 0.41  & -3.11  & -0.88  & 0.95  & 2.90  & 4.55   \\
    3  & -0.22  & -0.06  & -0.01  & -0.03  & 0.40  & -1.41  & -0.65  & -0.17  & -0.41  & 4.10   \\
    4  & -0.25  & -0.02  & -0.01  & 0.13  & 0.32  & -1.47  & -0.21  & -0.13  & 1.80  & 2.98   \\
    Big  & -0.27  & 0.01  & -0.07  & -0.01  & 0.20  & -1.60  & 0.08  & -1.00  & -0.19  & 1.83   \\
    
  
     & \multicolumn{5}{c}{b} & \multicolumn{5}{c}{t(b)}   \\
     \cmidrule(r){2-6} \cmidrule(r){7-11} 
    Small  & 1.08  & 0.91  & 0.87  & 0.88  & 1.00  & 29.01  & 48.01  & 55.29  & 54.40  & 42.58   \\
    2  & 1.21  & 1.01  & 0.95  & 0.96  & 1.08  & 35.33  & 50.33  & 64.96  & 69.21  & 49.86   \\
    3  & 1.19  & 1.03  & 0.97  & 0.98  & 1.07  & 31.32  & 50.07  & 59.64  & 59.89  & 45.79   \\
    4  & 1.23  & 1.10  & 1.02  & 1.00  & 1.06  & 30.05  & 46.90  & 58.45  & 59.41  & 41.26   \\
    Big  & 1.23  & 0.99  & 0.96  & 0.97  & 1.01  & 30.43  & 40.52  & 59.63  & 59.33  & 39.07   \\
    
  
     & \multicolumn{5}{c}{s} & \multicolumn{5}{c}{t(s)}   \\
     \cmidrule(r){2-6} \cmidrule(r){7-11} 
    Small  & 1.14  & 0.95  & 0.89  & 0.93  & 1.15  & 22.72  & 37.29  & 42.14  & 42.72  & 36.62   \\
    2  & 0.87  & 0.76  & 0.69  & 0.78  & 0.97  & 18.79  & 28.34  & 35.17  & 41.93  & 33.14   \\
    3  & 0.51  & 0.45  & 0.47  & 0.47  & 0.75  & 10.02  & 16.35  & 21.58  & 21.44  & 23.64   \\
    4  & 0.22  & 0.15  & 0.18  & 0.19  & 0.47  & 3.92  & 4.73  & 7.55  & 8.17  & 13.69   \\
    Big  & -0.18  & -0.22  & -0.20  & -0.20  & 0.01  & -3.34  & -6.60  & -9.19  & -8.94  & 0.39   \\
    
  
     & \multicolumn{5}{c}{v} & \multicolumn{5}{c}{t(v)}   \\
     \cmidrule(r){2-6} \cmidrule(r){7-11} 
    Small  & 0.25  & 0.38  & 0.36  & 0.21  & -0.11  & 3.48  & 10.20  & 11.57  & 6.63  & -2.44   \\
    2  & 0.23  & 0.35  & 0.31  & 0.24  & -0.17  & 3.46  & 9.06  & 10.80  & 8.73  & -4.07   \\
    3  & 0.22  & 0.34  & 0.35  & 0.30  & -0.18  & 2.98  & 8.44  & 10.96  & 9.46  & -4.01   \\
    4  & 0.28  & 0.27  & 0.32  & 0.17  & -0.18  & 3.50  & 5.82  & 9.27  & 5.17  & -3.67   \\
    Big  & 0.30  & 0.26  & 0.19  & 0.05  & -0.14  & 3.84  & 5.36  & 6.08  & 1.43  & -2.71   \\
    
  
     & \multicolumn{5}{c}{p} & \multicolumn{5}{c}{t(p)}   \\
     \cmidrule(r){2-6} \cmidrule(r){7-11} 
    Small  & -0.74  & -0.24  & -0.05  & 0.06  & -0.01  & -6.40  & -4.03  & -1.07  & 1.20  & -0.13   \\
    2  & -0.69  & -0.11  & 0.02  & 0.13  & 0.08  & -6.46  & -1.78  & 0.41  & 2.92  & 1.19   \\
    3  & -0.72  & -0.15  & 0.01  & 0.17  & 0.19  & -6.11  & -2.41  & 0.18  & 3.42  & 2.67   \\
    4  & -0.74  & -0.25  & -0.01  & 0.07  & 0.23  & -5.86  & -3.45  & -0.28  & 1.38  & 2.88   \\
    Big  & -0.42  & -0.17  & -0.02  & 0.17  & 0.30  & -3.34  & -2.28  & -0.35  & 3.37  & 3.73   \\
    
  
     & \multicolumn{5}{c}{i} & \multicolumn{5}{c}{t(i)}   \\
     \cmidrule(r){2-6} \cmidrule(r){7-11} 
    Small  & -0.26  & -0.07  & -0.03  & 0.07  & 0.11  & -2.56  & -1.34  & -0.75  & 1.63  & 1.73   \\
    2  & -0.32  & -0.14  & -0.04  & 0.02  & 0.03  & -3.38  & -2.47  & -1.04  & 0.50  & 0.47   \\
    3  & -0.29  & -0.13  & -0.06  & -0.02  & 0.02  & -2.83  & -2.25  & -1.42  & -0.36  & 0.30   \\
    4  & -0.24  & 0.02  & -0.03  & 0.04  & 0.07  & -2.18  & 0.25  & -0.57  & 0.84  & 0.94   \\
    Big  & -0.32  & -0.05  & -0.06  & 0.11  & -0.07  & -2.88  & -0.75  & -1.41  & 2.55  & -0.99   \\
    
  
  \bottomrule
\end{tabular}


    }
  \end{center}
\end{frame}

\begin{frame}
  \frametitle{Size-$\beta$ sorts}
  \begin{center}
  \resizebox*{!}{\dimexpr\textheight-1.3cm\relax}{
    
\begin{tabular}{lrrrrrrrrrr}
  \toprule
    
    Beta $\rightarrow$ & Low & 2 & 3 & 4 & High & Low & 2 & 3 & 4 & High  \\ 
  \midrule
  \multicolumn{11}{l}{$R^i=a^i+\beta^iMkt+s^iSize+v^iVal+m^iMom+p^iProf$}  \\
  
     & \multicolumn{5}{c}{a} & \multicolumn{5}{c}{t(a)}   \\
     \cmidrule(r){2-6} \cmidrule(r){7-11} 
    Small  & 0.10  & 0.08  & -0.00  & 0.08  & -0.07  & 1.34  & 1.13  & -0.02  & 1.26  & -0.73   \\
    2  & 0.01  & 0.04  & 0.07  & -0.04  & -0.09  & 0.15  & 0.61  & 1.06  & -0.67  & -1.12   \\
    3  & 0.08  & 0.15  & 0.00  & -0.06  & -0.02  & 0.97  & 2.19  & 0.03  & -0.83  & -0.22   \\
    4  & 0.10  & 0.08  & -0.01  & -0.12  & 0.03  & 1.11  & 1.09  & -0.08  & -1.48  & 0.28   \\
    Big  & -0.03  & -0.07  & -0.08  & -0.14  & 0.01  & -0.34  & -1.21  & -1.31  & -1.63  & 0.10   \\
    
  
    %  & \multicolumn{5}{c}{b} & \multicolumn{5}{c}{t(b)}   \\
    %  \cmidrule(r){2-6} \cmidrule(r){7-11} 
    % Small  & 0.65  & 0.78  & 0.88  & 1.01  & 1.16  & 39.89  & 48.20  & 54.71  & 69.21  & 53.20   \\
    % 2  & 0.71  & 0.83  & 0.97  & 1.08  & 1.29  & 39.66  & 55.60  & 65.15  & 73.05  & 71.39   \\
    % 3  & 0.72  & 0.88  & 1.00  & 1.10  & 1.32  & 39.04  & 58.04  & 63.45  & 62.80  & 61.41   \\
    % 4  & 0.77  & 0.96  & 1.06  & 1.17  & 1.38  & 38.52  & 58.34  & 62.82  & 61.26  & 51.53   \\
    % Big  & 0.78  & 0.97  & 1.08  & 1.20  & 1.33  & 45.15  & 70.31  & 73.67  & 61.03  & 40.37   \\
    
  
    %  & \multicolumn{5}{c}{v} & \multicolumn{5}{c}{t(v)}   \\
    %  \cmidrule(r){2-6} \cmidrule(r){7-11} 
    % Small  & 0.28  & 0.31  & 0.27  & 0.20  & -0.06  & 10.72  & 11.84  & 10.45  & 8.38  & -1.74   \\
    % 2  & 0.33  & 0.31  & 0.26  & 0.21  & -0.11  & 11.18  & 12.76  & 10.83  & 8.61  & -3.66   \\
    % 3  & 0.35  & 0.20  & 0.23  & 0.17  & -0.16  & 11.55  & 8.21  & 9.06  & 5.89  & -4.48   \\
    % 4  & 0.33  & 0.24  & 0.17  & 0.12  & -0.14  & 10.25  & 8.78  & 6.11  & 3.78  & -3.28   \\
    % Big  & 0.24  & 0.12  & 0.03  & 0.00  & -0.28  & 8.48  & 5.22  & 1.10  & 0.12  & -5.18   \\
    
  
     & \multicolumn{5}{c}{m} & \multicolumn{5}{c}{t(m)}   \\
     \cmidrule(r){2-6} \cmidrule(r){7-11} 
    Small  & 0.14  & 0.15  & 0.15  & 0.07  & -0.10  & 6.70  & 7.40  & 7.23  & 3.79  & -3.78   \\
    2  & 0.19  & 0.14  & 0.10  & 0.06  & -0.18  & 8.57  & 7.40  & 5.31  & 3.29  & -7.76   \\
    3  & 0.20  & 0.10  & 0.09  & 0.01  & -0.19  & 8.62  & 5.46  & 4.45  & 0.59  & -6.90   \\
    4  & 0.20  & 0.09  & 0.05  & -0.03  & -0.20  & 8.11  & 4.41  & 2.44  & -1.25  & -5.87   \\
    Big  & 0.15  & 0.06  & -0.05  & -0.08  & -0.24  & 7.06  & 3.72  & -2.76  & -3.32  & -5.91   \\
    
  
     & \multicolumn{5}{c}{p} & \multicolumn{5}{c}{t(p)}   \\
     \cmidrule(r){2-6} \cmidrule(r){7-11} 
    Small  & -0.13  & -0.02  & -0.01  & -0.07  & -0.16  & -2.50  & -0.45  & -0.28  & -1.42  & -2.17   \\
    2  & -0.12  & 0.09  & 0.10  & 0.10  & -0.06  & -2.04  & 1.82  & 2.10  & 2.15  & -1.09   \\
    3  & -0.15  & 0.03  & 0.15  & 0.13  & -0.01  & -2.50  & 0.66  & 2.84  & 2.26  & -0.13   \\
    4  & -0.16  & 0.06  & 0.11  & 0.05  & 0.04  & -2.44  & 1.13  & 2.03  & 0.76  & 0.41   \\
    Big  & 0.08  & 0.23  & 0.23  & 0.15  & -0.16  & 1.49  & 5.05  & 4.86  & 2.31  & -1.44   \\
    

  \midrule
  \multicolumn{11}{l}{$R^i=a^i+\beta^iMkt+s^iSize+v^iVal+p^iProf+i^iInv$}  \\
  
     & \multicolumn{5}{c}{a} & \multicolumn{5}{c}{t(a)}   \\
     \cmidrule(r){2-6} \cmidrule(r){7-11} 
    Small  & 0.16  & 0.13  & 0.04  & 0.11  & -0.10  & 2.35  & 1.96  & 0.65  & 1.84  & -1.05   \\
    2  & 0.10  & 0.08  & 0.11  & -0.03  & -0.17  & 1.34  & 1.33  & 1.75  & -0.43  & -2.16   \\
    3  & 0.18  & 0.16  & 0.03  & -0.09  & -0.09  & 2.34  & 2.59  & 0.48  & -1.17  & -0.98   \\
    4  & 0.20  & 0.10  & -0.01  & -0.16  & -0.04  & 2.36  & 1.42  & -0.10  & -1.94  & -0.38   \\
    Big  & 0.01  & -0.06  & -0.12  & -0.16  & -0.10  & 0.20  & -1.03  & -1.85  & -1.89  & -0.72   \\

  \bottomrule
\end{tabular}

    }
  \end{center}
\end{frame}

\note[itemize]{
  \item But beta not a problem
}

\begin{frame}
  \frametitle{Size-Var sorts: Model 1}
  \begin{center}
  \resizebox*{!}{\dimexpr\textheight-1.3cm\relax}{
    
\begin{tabular}{lrrrrrrrrrr}
  \toprule
    
    Var $\rightarrow$ & Low & 2 & 3 & 4 & High & Low & 2 & 3 & 4 & High  \\ 
  \midrule
  \multicolumn{11}{l}{$R^i=a^i+\beta^iMkt+s^iSize+v^iVal+m^iMom+p^iProf$}  \\
  
     & \multicolumn{5}{c}{a} & \multicolumn{5}{c}{t(a)}   \\
     \cmidrule(r){2-6} \cmidrule(r){7-11} 
    Small  & 0.22  & 0.18  & 0.14  & 0.01  & -0.66  & 3.18  & 2.40  & 1.93  & 0.10  & -4.03   \\
    2  & 0.07  & 0.01  & 0.06  & -0.04  & -0.15  & 0.99  & 0.08  & 0.85  & -0.49  & -1.36   \\
    3  & -0.01  & -0.04  & -0.01  & -0.08  & 0.03  & -0.15  & -0.60  & -0.19  & -1.01  & 0.33   \\
    4  & 0.03  & 0.01  & -0.09  & -0.02  & 0.11  & 0.33  & 0.18  & -1.08  & -0.23  & 0.97   \\
    Big  & -0.02  & -0.09  & -0.10  & -0.11  & 0.10  & -0.29  & -1.45  & -1.55  & -1.65  & 0.89   \\
    
  
     & \multicolumn{5}{c}{b} & \multicolumn{5}{c}{t(b)}   \\
     \cmidrule(r){2-6} \cmidrule(r){7-11} 
    Small  & 0.68  & 0.96  & 1.06  & 1.15  & 1.17  & 41.89  & 56.81  & 63.98  & 55.63  & 31.26   \\
    2  & 0.74  & 0.98  & 1.08  & 1.22  & 1.31  & 47.15  & 57.72  & 62.20  & 66.23  & 50.89   \\
    3  & 0.73  & 0.96  & 1.06  & 1.19  & 1.30  & 43.70  & 58.22  & 58.07  & 61.82  & 53.16   \\
    4  & 0.75  & 0.94  & 1.09  & 1.19  & 1.32  & 38.15  & 50.00  & 58.45  & 59.62  & 52.43   \\
    Big  & 0.73  & 0.92  & 1.02  & 1.13  & 1.29  & 42.08  & 63.36  & 70.36  & 74.81  & 50.55   \\
    
  
     & \multicolumn{5}{c}{s} & \multicolumn{5}{c}{t(s)}   \\
     \cmidrule(r){2-6} \cmidrule(r){7-11} 
    Small  & 0.63  & 0.88  & 1.03  & 1.20  & 1.50  & 28.36  & 38.19  & 45.26  & 42.68  & 29.36   \\
    2  & 0.51  & 0.69  & 0.77  & 0.89  & 1.18  & 23.69  & 29.68  & 32.45  & 35.35  & 33.56   \\
    3  & 0.25  & 0.42  & 0.51  & 0.63  & 0.89  & 11.05  & 18.65  & 20.17  & 23.99  & 26.58   \\
    4  & 0.03  & 0.13  & 0.18  & 0.26  & 0.58  & 1.08  & 4.95  & 7.07  & 9.40  & 16.82   \\
    Big  & -0.29  & -0.28  & -0.20  & -0.19  & 0.08  & -12.18  & -13.81  & -10.02  & -9.00  & 2.31   \\
    
  
     & \multicolumn{5}{c}{v} & \multicolumn{5}{c}{t(v)}   \\
     \cmidrule(r){2-6} \cmidrule(r){7-11} 
    Small  & 0.42  & 0.47  & 0.37  & 0.16  & -0.10  & 16.50  & 17.49  & 14.01  & 4.99  & -1.61   \\
    2  & 0.42  & 0.47  & 0.36  & 0.21  & -0.38  & 16.90  & 17.44  & 13.15  & 7.10  & -9.23   \\
    3  & 0.42  & 0.41  & 0.36  & 0.27  & -0.44  & 15.84  & 15.83  & 12.45  & 8.98  & -11.29   \\
    4  & 0.41  & 0.34  & 0.32  & 0.19  & -0.40  & 13.27  & 11.39  & 10.92  & 6.05  & -10.12   \\
    Big  & 0.22  & 0.20  & 0.12  & 0.03  & -0.33  & 8.18  & 8.47  & 5.46  & 1.33  & -8.21   \\
    
  
     & \multicolumn{5}{c}{m} & \multicolumn{5}{c}{t(m)}   \\
     \cmidrule(r){2-6} \cmidrule(r){7-11} 
    Small  & 0.10  & 0.07  & -0.05  & -0.19  & -0.40  & 5.23  & 3.48  & -2.36  & -7.38  & -8.89   \\
    2  & 0.15  & 0.15  & 0.06  & -0.03  & -0.36  & 8.01  & 7.38  & 2.92  & -1.44  & -11.35   \\
    3  & 0.17  & 0.10  & 0.10  & 0.03  & -0.30  & 8.52  & 4.95  & 4.44  & 1.16  & -10.06   \\
    4  & 0.17  & 0.08  & 0.08  & 0.02  & -0.29  & 7.17  & 3.54  & 3.47  & 0.83  & -9.39   \\
    Big  & 0.13  & 0.12  & 0.08  & 0.01  & -0.19  & 6.30  & 6.58  & 4.37  & 0.51  & -6.17   \\
    
  
     & \multicolumn{5}{c}{p} & \multicolumn{5}{c}{t(p)}   \\
     \cmidrule(r){2-6} \cmidrule(r){7-11} 
    Small  & 0.19  & 0.13  & 0.04  & -0.20  & -0.56  & 3.68  & 2.39  & 0.70  & -3.08  & -4.68   \\
    2  & 0.21  & 0.22  & 0.13  & 0.10  & -0.51  & 4.06  & 4.08  & 2.36  & 1.76  & -6.15   \\
    3  & 0.16  & 0.18  & 0.24  & 0.18  & -0.42  & 3.04  & 3.45  & 4.03  & 2.84  & -5.38   \\
    4  & 0.04  & 0.13  & 0.23  & 0.03  & -0.36  & 0.68  & 2.23  & 3.83  & 0.50  & -4.43   \\
    Big  & -0.02  & 0.26  & 0.17  & 0.05  & -0.19  & -0.34  & 5.55  & 3.65  & 0.93  & -2.31   \\
    
  
  \bottomrule
\end{tabular}


    }
  \end{center}
\end{frame}

\note[itemize]{
  \item Small, high-variance bucket a big problem
}

\begin{frame}
  \frametitle{Size-Var sorts: Model 2}
  \begin{center}
  \resizebox*{!}{\dimexpr\textheight-1.3cm\relax}{
    
\begin{tabular}{lrrrrrrrrrr}
  \toprule
    
    Var $\rightarrow$ & Low & 2 & 3 & 4 & High & Low & 2 & 3 & 4 & High  \\ 
  \midrule
  \multicolumn{11}{l}{$R^i=a^i+\beta^iMkt+s^iSize+v^iVal+p^iProf+i^iInv$}  \\
  
     & \multicolumn{5}{c}{a} & \multicolumn{5}{c}{t(a)}   \\
     \cmidrule(r){2-6} \cmidrule(r){7-11} 
    Small  & 0.33  & 0.30  & 0.21  & -0.01  & -0.80  & 4.82  & 3.94  & 2.54  & -0.11  & -4.76   \\
    2  & 0.19  & 0.14  & 0.15  & 0.00  & -0.34  & 2.87  & 1.94  & 2.06  & 0.01  & -3.02   \\
    3  & 0.12  & 0.06  & 0.09  & -0.03  & -0.14  & 1.69  & 0.84  & 1.13  & -0.35  & -1.37   \\
    4  & 0.16  & 0.08  & -0.02  & 0.01  & -0.06  & 1.95  & 0.98  & -0.29  & 0.13  & -0.58   \\
    Big  & 0.03  & -0.02  & -0.05  & -0.10  & 0.02  & 0.49  & -0.31  & -0.87  & -1.56  & 0.23   \\
    
  
     & \multicolumn{5}{c}{b} & \multicolumn{5}{c}{t(b)}   \\
     \cmidrule(r){2-6} \cmidrule(r){7-11} 
    Small  & 0.67  & 0.95  & 1.06  & 1.15  & 1.18  & 40.07  & 51.24  & 54.48  & 48.75  & 29.12   \\
    2  & 0.74  & 0.97  & 1.07  & 1.21  & 1.32  & 46.12  & 56.05  & 59.20  & 62.61  & 48.82   \\
    3  & 0.73  & 0.95  & 1.05  & 1.19  & 1.31  & 42.68  & 56.54  & 56.61  & 59.81  & 52.13   \\
    4  & 0.74  & 0.94  & 1.09  & 1.19  & 1.33  & 37.10  & 49.87  & 57.95  & 58.67  & 51.61   \\
    Big  & 0.74  & 0.92  & 1.01  & 1.13  & 1.26  & 42.78  & 62.19  & 69.43  & 73.91  & 50.25   \\
    
  
     & \multicolumn{5}{c}{s} & \multicolumn{5}{c}{t(s)}   \\
     \cmidrule(r){2-6} \cmidrule(r){7-11} 
    Small  & 0.61  & 0.86  & 1.00  & 1.17  & 1.46  & 27.08  & 34.34  & 38.04  & 36.94  & 26.71   \\
    2  & 0.50  & 0.67  & 0.75  & 0.87  & 1.17  & 23.14  & 28.86  & 30.86  & 33.51  & 32.08   \\
    3  & 0.24  & 0.40  & 0.49  & 0.62  & 0.89  & 10.54  & 17.84  & 19.69  & 23.06  & 26.30   \\
    4  & 0.02  & 0.11  & 0.17  & 0.25  & 0.58  & 0.62  & 4.46  & 6.63  & 9.03  & 16.59   \\
    Big  & -0.29  & -0.28  & -0.20  & -0.19  & 0.09  & -12.50  & -13.91  & -10.05  & -9.10  & 2.55   \\
    
  
     & \multicolumn{5}{c}{v} & \multicolumn{5}{c}{t(v)}   \\
     \cmidrule(r){2-6} \cmidrule(r){7-11} 
    Small  & 0.42  & 0.44  & 0.36  & 0.21  & -0.02  & 12.71  & 12.11  & 9.41  & 4.63  & -0.25   \\
    2  & 0.38  & 0.44  & 0.36  & 0.25  & -0.26  & 12.11  & 12.97  & 10.25  & 6.59  & -4.81   \\
    3  & 0.34  & 0.43  & 0.39  & 0.28  & -0.31  & 10.29  & 13.00  & 10.59  & 7.28  & -6.25   \\
    4  & 0.32  & 0.32  & 0.32  & 0.20  & -0.32  & 8.11  & 8.54  & 8.57  & 5.05  & -6.34   \\
    Big  & 0.13  & 0.14  & 0.12  & 0.03  & -0.13  & 3.92  & 4.76  & 4.29  & 0.93  & -2.69   \\
    
  
     & \multicolumn{5}{c}{p} & \multicolumn{5}{c}{t(p)}   \\
     \cmidrule(r){2-6} \cmidrule(r){7-11} 
    Small  & 0.13  & 0.00  & -0.15  & -0.42  & -0.86  & 2.43  & 0.05  & -2.49  & -5.68  & -6.81   \\
    2  & 0.17  & 0.18  & 0.05  & 0.01  & -0.64  & 3.51  & 3.27  & 0.93  & 0.13  & -7.64   \\
    3  & 0.14  & 0.13  & 0.20  & 0.10  & -0.48  & 2.61  & 2.43  & 3.45  & 1.69  & -6.18   \\
    4  & 0.01  & 0.09  & 0.19  & -0.01  & -0.44  & 0.15  & 1.49  & 3.21  & -0.19  & -5.44   \\
    Big  & -0.00  & 0.27  & 0.19  & 0.04  & -0.19  & -0.03  & 5.92  & 4.24  & 0.84  & -2.47   \\
    
  
     & \multicolumn{5}{c}{i} & \multicolumn{5}{c}{t(i)}   \\
     \cmidrule(r){2-6} \cmidrule(r){7-11} 
    Small  & -0.02  & -0.03  & -0.05  & -0.12  & -0.14  & -0.43  & -0.55  & -0.88  & -1.92  & -1.25   \\
    2  & 0.05  & 0.01  & -0.04  & -0.11  & -0.17  & 1.03  & 0.29  & -0.81  & -2.11  & -2.27   \\
    3  & 0.10  & -0.04  & -0.07  & -0.03  & -0.18  & 2.04  & -0.96  & -1.41  & -0.56  & -2.56   \\
    4  & 0.11  & 0.06  & 0.02  & -0.01  & -0.11  & 2.03  & 1.20  & 0.48  & -0.17  & -1.50   \\
    Big  & 0.18  & 0.08  & 0.00  & 0.00  & -0.38  & 3.90  & 1.92  & 0.05  & 0.04  & -5.55   \\
    
  
  \bottomrule
\end{tabular}


    }
  \end{center}
\end{frame}

\begin{frame}
\frametitle{Momentum and variance sorts}
\begin{itemize}
    \item Momentum factor helps, but not much, in sorts on momentum
    \item Something going on with momentum returns that is not to do with
    momentum characteristics
    \item Profit slopes are negative in the extremes of momentum
    \item Defensive equity $\rightarrow$ beta and volatility
    \item Sorts on beta pose the least problems
    \item Sorts on variance are a big problem
    \item Small, high-variance alpha of -0.66\% per month
\end{itemize}
\end{frame}

\begin{frame}
\frametitle{Momentum and variance sorts}
\begin{itemize}
    \item To take advantage of this alpha we need to sell small,
    unprofitable stocks with poor recent returns
    \item That's a tough sell $\rightarrow$ Liquidity
    \item But, liquidity hard to approach from a cross-section perspective
    \item Institutional ownership, Short-term reversal factor as a proxy
    for market-maker's returns $\rightarrow$ market microstructure
\end{itemize}
\end{frame}

\begin{frame}
\frametitle{Momentum and variance sorts}
\begin{itemize}
    \item Cross-section perspective
    \item High-volatility firms are much smaller
    \item Small says high returns while high-volatility says low returns
    \item Value disappears in the problem momentum and variance sorts
    \item Normally, value and profitability do most of the heavy lifting
\end{itemize}
\end{frame}

\section{Conclusions}

\begin{frame}
\frametitle{Conclusions}
\begin{itemize}
    \item Mkt, size, value, momentum and profitability factors have the
    highest Sharpe ratio
    \item Value and momentum subsume investment
    \item ``Lethal combination" of small, unprofitable stocks that somehow
    invest aggressively $\rightarrow$ small, unprofitable firms with poor
    recent returns
\end{itemize}
\end{frame}

\section{Omissions}

\begin{frame}
\frametitle{Omissions}
\begin{itemize}
    \item BAB factor -- small contribution to $\text{Sh}^2$ (0.02) due to
    unexplained volatility
    \item QMJ factor -- small contribution to $\text{Sh}^2$ (0.05) due to low
    unexplained average return
    \item Stance on risk vs behavioural stories
\end{itemize}
\end{frame}

\section{References}

\begin{frame}[allowframebreaks]
  \frametitle{References}
  \printbibliography
\end{frame}


\end{document}


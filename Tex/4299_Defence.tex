\documentclass{beamer}

\usepackage[backend=biber, style=authoryear]{biblatex}
\usepackage{booktabs}
\usepackage{dcolumn}
\usepackage{graphicx}
\usepackage[utf8]{inputenc}

\usetheme{Madrid}

\addbibresource{references.bib}

\title[Asset-Pricing]{Top-down approach to factor models}
\author[Thorold]{4299 Sam Thorold\\
                 Supervisors: Francisco Santos and Andr\'e Silva}
\institute{NHH and NOVA}
\date[April 2018]{\today}

\begin{document}

\frame{\titlepage}

%\begin{frame}
%\frametitle{What's on the cards}
%\tableofcontents
%\end{frame}


\section{Intro}

\begin{frame}
  \frametitle{Current state of affairs}
  \begin{itemize}
    \item Characteristic sort cannot be explained by a model
    \small{$<$insert Fama-French iteration here$>$}
    \item A risk story is attached to the ``anomaly"
    \item Anomaly is made into a factor
  \end{itemize}
\end{frame}

\begin{frame}
  \frametitle{Current state of affairs}
  \begin{itemize}
    \item Quality and prevalence of stock data means factor models are
    particularly open to data-mining
    \item Problem: no way to test models before turning to performance in
    anomaly sorts
  \end{itemize}
\end{frame}

\begin{frame}
  \frametitle{Current state of affairs}
  \begin{itemize}
    \item \cite{harvey2016and}
  \end{itemize}
\end{frame}

\section{Background}

\begin{frame}
\frametitle{What makes a good model?}
\begin{itemize}
    \item<1-> We want our asset pricing models to minimize pricing errors for
    all assets
    \item<1-> When we price returns, the pricing error is given by
    \[
    \text{Sh}^2(a) = \text{Sh}^2(R, f) - \text{Sh}^2(f)
    \]
    \item<2-> If $R$ is all assets then $\text{Sh}^2(R, f) = \text{Sh}^2(R)$
    \[
    \text{Sh}^2(a) = \text{Sh}^2(R) - \text{Sh}^2(f)
    \]
    \item<2->Model with the highest Sharpe ratio of the factors is best
\end{itemize}
\end{frame}

\begin{frame}
\frametitle{Which factors?}
\begin{itemize}
    \item<1-> ICAPM says investors care about their payoff in the future but
    also what they can do with it
    \item<1-> Market + state variables
    \item<1-> We do not know these state variables but we can make proxies
    \item<2-> Investors want to identify \emph{cheap} and \emph{profitable} stocks
\end{itemize}
\end{frame}

\begin{frame}
\frametitle{Proxies for cheap and profitable}
\begin{itemize}
    \item<1-> Cheap $\rightarrow$ Book-to-Market (BM)
    \item<1-> Some problems here as we want to identify changes in BM due to
    changes in market equity rather than book equity
    \item<1-> Monthly value + momentum
    \item<2-> Profitable $\rightarrow$ Operating profit adjusted for accruals
\end{itemize}
\end{frame}

\begin{frame}
\frametitle{Testing the model}
\begin{itemize}
    \item<1-> $R_t^i = a^i + b^iR_t^M + s^iSMB_t + v^iHML_t^m + m^iWML_t + p^iPMU_t$
    \item<2-> Max Sharpe ratio of factors
    \item<3-> Gibbons, Ross, Shanken (GRS) statistic that the Sharpe ratio
    can be improved by investing in the test assets
    \item<4-> Individual time series regressions of test assets on the factors
\end{itemize}
\end{frame}

\section{Results}

\begin{frame}
\frametitle{Sharpe ratio results: Which model is best? (Why?)}
\begin{itemize}
    \item<1-> $\text{Sh}^2(R^M + SMB + HML^m + WML + PMU)=0.316$
    \item<1-> $\text{Sh}^2(R^M + SMB + HML   + CMA + PMU)=0.225$
    \item<2-> Investment factor is subsumed by value and momentum
    \item<3-> $ME_t = \sum_{s=1}^\infty E\left( \text{Profit}_{t+s} - \Delta\text{BE}_{t+s}\right) /R^s$
    \item<3-> Investment does not describe future $\Delta$BE for small, growth
    stocks while sorts on value and momentum describe investment and future
    $\Delta$BE
\end{itemize}
\end{frame}

\begin{frame}
\frametitle{GRS results: Which sorts cause problems?}
\begin{itemize}
    \item<1-> Matches Sharpe ratio results broadly
    \item<2-> Problems for my proposed model in sorts on BM compared to other
    models (various Fama and French iterations)
    \item<2-> Better in all other sorts compared to other models
    \item<3-> Problems in sorts on momentum and variation for all models
\end{itemize}
\end{frame}

\begin{frame}
\frametitle{Momentum and variation sorts: Why problems?}
\begin{itemize}
    \item<1-> Momentum factor helps, but not much, in sorts on momentum
    \item<1-> Something going on with momentum returns that is not to do with
    momentum characteristics
    \item<2-> Profit slopes are negative in the extremes of momentum
    \item<2-> Defensive equity $\rightarrow$ beta and volatility
    \item<3-> Sorts on beta pose the least problems
    \item<4-> Sorts on variance are a big problem
    \item<4-> Small, high-variance alpha of -0.66\% per month
\end{itemize}
\end{frame}

\begin{frame}
\frametitle{Momentum and variation sorts: Why problems?}
\begin{itemize}
    \item<1-> To take advantage of this alpha we need to sell small,
    unprofitable stocks with poor recent returns
    \item<2-> That's a tough sell $\rightarrow$ Liquidity
    \item<3-> But, liquidity hard to approach from a cross-section perspective
    \item<3-> Institutional ownership, Short-term reversal factor as a proxy
    for market-maker's returns $\rightarrow$ market microstructure
\end{itemize}
\end{frame}

\begin{frame}
\frametitle{Momentum and variation sorts: Why problems?}
\begin{itemize}
    \item<1-> Cross-section perspective
    \item<1-> High-volatility firms are much smaller
    \item<1-> Small says high returns while high-volatility says low returns
    \item<2-> Value disappears in the problem momentum and variance sorts
    \item<2-> Normally, value and profitability do most of the heavy lifting
\end{itemize}
\end{frame}

\section{Conclusions}

\begin{frame}
\frametitle{Conclusions}
\begin{itemize}
    \item<1-> Mkt, size, value, momentum and profitability factors have the
    highest Sharpe ratio
    \item<2-> Value and momentum subsume investment
    \item<3-> ``Lethal combination" of small, unprofitable stocks that somehow
    invest aggressively $\rightarrow$ small, unprofitable firms with poor
    recent returns
\end{itemize}
\end{frame}

\section{Tables}

\begin{frame}
\frametitle{Highest Sharpe ratio}
\resizebox{\linewidth}{!}{
\begin{tabular}{lcccccc}
  \toprule
  \multicolumn{7}{l}{Panel A: Model factors and maximum squared Sharpe ratios} \\
        & \multicolumn{2}{l}{Name}                         & \multicolumn{3}{l}{Factors}                                   & Sh2   \\
  1     & \multicolumn{2}{l}{Model 1}                      & \multicolumn{3}{l}{$R^M$, $SMB$, $HML^m$, $PMU$, $WML$}        & 0.316 \\
  2     & \multicolumn{2}{l}{Model 1 + $CMA$}              & \multicolumn{3}{l}{$R^M$, $SMB$, $HML^m$, $PMU$, $WML$, $CMA$} & 0.316 \\
  3     & \multicolumn{2}{l}{Fama and French (2017)}       & \multicolumn{3}{l}{$R^M$, $SMB$, $HML$, $PMU$, $WML$, $CMA$}   & 0.240 \\
  4     & \multicolumn{2}{l}{Model 2}                      & \multicolumn{3}{l}{$R^M$, $SMB$, $HML$, $PMU$, $CMA$}          & 0.225 \\
  5     & \multicolumn{2}{l}{Fama and French (2015)}       & \multicolumn{3}{l}{$R^M$, $SMB$, $HML$, $PMU^{06}$, $CMA$}     & 0.099 \\
  6     & \multicolumn{2}{l}{Carhart (1997)}               & \multicolumn{3}{l}{$R^M$, $SMB$, $HML$, $WML$}                 & 0.090 \\
  7     & \multicolumn{2}{l}{Carhart (1997) + $HML^m$}     & \multicolumn{3}{l}{$R^M$, $SMB$, $HML^m$, $WML$}               & 0.136 \\
  \midrule
  \multicolumn{7}{l}{Panel B: 90\% confidence interval for distributions of Sh2(Row) - Sh2(Column)} \\
        &        2         &        3         &        4         &        5         &        6         &        7         \\
  1     & (-0.008,  0.000) & ( 0.039,  0.112) & ( 0.055,  0.138) & ( 0.156,  0.291) & ( 0.169,  0.299) & ( 0.127,  0.248) \\
  2     &                  & ( 0.043,  0.115) & ( 0.058,  0.140) & ( 0.158,  0.294) & ( 0.170,  0.302) & ( 0.129,  0.250) \\
  3     &                  &                  & ( 0.002,  0.048) & ( 0.095,  0.204) & ( 0.107,  0.212) & ( 0.052,  0.172) \\
  4     &                  &                  &                  & ( 0.081,  0.176) & ( 0.080,  0.197) & ( 0.028,  0.155) \\
  5     &                  &                  &                  &                  & (-0.033,  0.052) & (-0.085,  0.012) \\
  6     &                  &                  &                  &                  &                  & (-0.071, -0.025) \\
  \bottomrule
\end{tabular}

}
\end{frame}

\begin{frame}
\frametitle{Contributions to Sharpe ratio}
\begin{center}
\resizebox*{!}{\dimexpr\textheight-1.3cm\relax}{
    
\begin{tabular}{lrrrrrr}
  \toprule
  \multicolumn{7}{l}{Model 1}  \\
                                &  $R^M$ &  $SMB$ &$HML^m$ &  $WML$ &  $PMU$ &   \\
  $a$                           &  1.22  &  0.44  &  1.00  &  1.04  &  0.50  &   \\
  $\sigma_e$                    &  3.85  &  2.79  &  2.36  &  3.07  &  1.17  &   \\
  $\left( a/\sigma_e\right) ^2$ &  0.10  &  0.03  &  0.18  &  0.11  &  0.18  &   \\
[1em]
  \multicolumn{7}{l}{Model 1 + CMA}  \\
                                &  $R^M$ &  $SMB$ &$HML^m$ &  $WML$ &  $PMU$ &  $CMA$ \\
  $a$                           &  1.15  &  0.44  &  0.59  &  0.83  &  0.49  &  0.01  \\
  $\sigma_e$                    &  3.73  &  2.79  &  1.83  &  2.76  &  1.17  &  1.41  \\
  $\left( a/\sigma_e\right) ^2$ &  0.10  &  0.03  &  0.11  &  0.09  &  0.18  &  0.00  \\
[1em]
  \multicolumn{7}{l}{Model 2 + WML}  \\
                                &  $R^M$ &  $SMB$ &  $HML$ &  $WML$ &  $PMU$ &  $CMA$ \\
  $a$                           &  1.08  &  0.43  &  0.33  &  0.49  &  0.44  &  0.12  \\
  $\sigma_e$                    &  3.74  &  2.78  &  1.89  &  3.94  &  1.18  &  1.37  \\
  $\left( a/\sigma_e\right) ^2$ &  0.08  &  0.02  &  0.03  &  0.02  &  0.14  &  0.01  \\
  \bottomrule
\end{tabular}


    }
\end{center}
\end{frame}

\begin{frame}
\frametitle{Value and momentum subsume investment}
\begin{center}
\resizebox*{!}{\dimexpr\textheight-1.3cm\relax}{
    
\begin{tabular}{lD{.}{.}{2.4}D{.}{.}{2.4}D{.}{.}{2.4}D{.}{.}{2.4}D{.}{.}{2.4}D{.}{.}{2.4}D{.}{.}{2.4}D{.}{.}{2.4}}
\toprule
          &\multicolumn{1}{c}{(1)}&\multicolumn{1}{c}{(2)}&\multicolumn{1}{c}{(3)}&\multicolumn{1}{c}{(4)}\\
\midrule
$a$       &    -0.04&     0.08&     0.01&     0.12 \\
          &  (-0.70)&   (1.32)&   (0.08)&   (2.01) \\
[1em]
$R^M$     &         &         &    -0.09&    -0.09 \\
          &         &         &  (-6.31)&   (-6.23)\\
[1em]
$SMB$     &         &         &     0.00&     0.01 \\
          &         &         &   (0.99)&    (0.53)\\
[1em]
$HML$     &         &     0.51&         &     0.48 \\
          &         &  (25.32)&         &  (22.49) \\
[1em]
$HML^m$   &     0.52&         &     0.49&          \\
          &  (23.24)&         &  (20.72)&          \\
[1em]
$WML$     &     0.26&     0.05&     0.22&     0.03 \\
          &  (14.45)&   (3.90)&  (12.42)&    (2.23)\\
[1em]
$PMU$     &         &         &     0.10&     0.07 \\
          &         &         &   (2.03)&    (1.56)\\
\midrule
adj. \(R^{2}\)&    0.45&    0.50&    0.50&    0.53 \\
\bottomrule
\multicolumn{5}{r}{\footnotesize \textit{t} statistics in parentheses}\\
\end{tabular}


    }
\end{center}
\end{frame}

\begin{frame}
\frametitle{Size-BM-Inv sorts}
\begin{center}
\resizebox*{!}{\dimexpr\textheight-1.3cm\relax}{
    
\begin{tabular}{lrrrrrrrr}
  \toprule
     & \multicolumn{4}{c}{Small} & \multicolumn{4}{c}{Big}  \\
     \cmidrule(r){2-5} \cmidrule(r){6-9}
    BM $\rightarrow$ & Low & 2 & 3 & High & Low & 2 & 3 & High  \\ 
  \midrule
  
  
    % & \multicolumn{8}{c}{$\text{r}^i$}  \\
    %  \cmidrule(r){2-5} \cmidrule(r){6-9}
    % Low Inv  & 1.01  & 1.33  & 1.45  & 1.46  & 0.98  & 1.04  & 1.05  & 1.14   \\
    % 2  & 1.22  & 1.31  & 1.30  & 1.48  & 0.92  & 0.93  & 0.95  & 0.99   \\
    % 3  & 1.25  & 1.29  & 1.41  & 1.28  & 0.92  & 0.94  & 0.93  & 1.16   \\
    % High Inv  & 0.94  & 1.17  & 1.27  & 1.31  & 0.98  & 0.80  & 0.90  & 1.06   \\
    
  
    % & \multicolumn{8}{c}{Future Var}  \\
    %  \cmidrule(r){2-5} \cmidrule(r){6-9}
    % Low Inv  & 0.16  & 0.11  & 0.10  & 0.14  & 0.04  & 0.03  & 0.03  & 0.04   \\
    % 2  & 0.09  & 0.07  & 0.07  & 0.09  & 0.03  & 0.03  & 0.03  & 0.03   \\
    % 3  & 0.08  & 0.06  & 0.06  & 0.09  & 0.03  & 0.03  & 0.03  & 0.04   \\
    % High Inv  & 0.11  & 0.09  & 0.08  & 0.11  & 0.05  & 0.04  & 0.04  & 0.04   \\
    
  
    & \multicolumn{8}{c}{Future Inv}  \\
     \cmidrule(r){2-5} \cmidrule(r){6-9}
    Low Inv  & \textbf{-0.10}  & -0.08  & -0.08  & -0.09  & -0.02  & -0.02  & -0.01  & -0.02   \\
    2  & \textbf{0.04}  & 0.03  & 0.03  & 0.02  & 0.06  & 0.06  & 0.06  & 0.06   \\
    3  & \textbf{0.11}  & 0.10  & 0.10  & 0.10  & 0.12  & 0.12  & 0.12  & 0.11   \\
    High Inv  & \textbf{0.48}  & 0.39  & 0.37  & 0.47  & 0.37  & 0.38  & 0.37  & 0.52   \\
    
  
    & \multicolumn{8}{c}{Future $\Delta\text{BE}$}  \\
     \cmidrule(r){2-5} \cmidrule(r){6-9}
    Low Inv  & \textbf{0.59}  & 0.06  & 0.02  & -0.03  & 0.28  & 0.07  & 0.06  & 0.03   \\
    2  & \textbf{0.44}  & 0.09  & 0.05  & 0.01  & 0.20  & 0.10  & 0.07  & 0.05   \\
    3  & \textbf{0.27}  & 0.11  & 0.08  & 0.03  & 0.16  & 0.12  & 0.12  & 0.08   \\
    High Inv  & \textbf{0.40}  & 0.16  & 0.11  & 0.06  & 0.38  & 0.17  & 0.13  & 0.10   \\
    
  
  \bottomrule
\end{tabular}

    }
\end{center}
\end{frame}
\begin{frame}
\frametitle{Size-$\text{BM}^m$-Prior sorts}
\begin{center}
\resizebox*{!}{\dimexpr\textheight-1.3cm\relax}{
    
\begin{tabular}{lrrrrrrrr}
  \toprule
     & \multicolumn{4}{c}{Small} & \multicolumn{4}{c}{Big}  \\
     \cmidrule(r){2-5} \cmidrule(r){6-9}
    $\text{BM}^m$ $\rightarrow$ & Low & 2 & 3 & High & Low & 2 & 3 & High  \\ 
  \midrule
  
  
    & \multicolumn{8}{c}{$\text{R}^i$}  \\
     \cmidrule(r){2-5} \cmidrule(r){6-9}
    Low Prior  & 0.09  & 0.76  & 0.98  & 1.01  & 0.55  & 0.75  & 0.95  & 0.96   \\
    2  & 0.69  & 1.07  & 1.24  & 1.51  & 0.64  & 0.82  & 0.9  & 1.1   \\
    3  & 0.96  & 1.26  & 1.55  & 1.71  & 0.9  & 0.87  & 0.99  & 1.23   \\
    High Prior  & 1.47  & 1.58  & 1.78  & 2.04  & 1.21  & 1.11  & 1.14  & 1.42   \\
    
  
    & \multicolumn{8}{c}{Future Var}  \\
     \cmidrule(r){2-5} \cmidrule(r){6-9}
    Low Prior  & 0.2  & 0.13  & 0.13  & 0.19  & 0.05  & 0.05  & 0.04  & 0.05   \\
    2  & 0.1  & 0.07  & 0.07  & 0.09  & 0.03  & 0.03  & 0.03  & 0.03   \\
    3  & 0.08  & 0.06  & 0.06  & 0.08  & 0.03  & 0.03  & 0.03  & 0.03   \\
    High Prior  & 0.11  & 0.08  & 0.08  & 0.12  & 0.05  & 0.04  & 0.04  & 0.04   \\
    
  
    & \multicolumn{8}{c}{Future Inv}  \\
     \cmidrule(r){2-5} \cmidrule(r){6-9}
    Low Prior  & 0.39  & 0.24  & 0.19  & 0.2  & 0.21  & 0.16  & 0.14  & 0.2   \\
    2  & 0.27  & 0.16  & 0.11  & 0.09  & 0.17  & 0.13  & 0.11  & 0.1   \\
    3  & 0.24  & 0.13  & 0.09  & 0.07  & 0.17  & 0.13  & 0.11  & 0.09   \\
    High Prior  & 0.25  & 0.12  & 0.1  & 0.06  & 0.23  & 0.14  & 0.12  & 0.09   \\
    
  
    & \multicolumn{8}{c}{Future $\Delta\text{BE}$}  \\
     \cmidrule(r){2-5} \cmidrule(r){6-9}
    Low Prior  & 0.31  & 0.07  & 0.01  & -0.1  & 0.42  & 0.11  & 0.07  & 0.02   \\
    2  & 0.3  & 0.09  & 0.04  & -0.02  & 0.25  & 0.11  & 0.08  & 0.04   \\
    3  & 0.34  & 0.1  & 0.05  & -0.0  & 0.27  & 0.11  & 0.08  & 0.05   \\
    High Prior  & 0.61  & 0.1  & 0.04  & -0.03  & 0.53  & 0.14  & 0.1  & 0.04   \\
    
  
  \bottomrule
\end{tabular}

    }
\end{center}
\end{frame}

\begin{frame}
\frametitle{GRS statistic}
\begin{center}
\resizebox*{!}{\dimexpr\textheight-1.3cm\relax}{
    
\begin{tabular}{lrrrrrrr}
  \toprule
     & \rotatebox{90}{\footnotesize{Carhart (1997)}} &
       \rotatebox{90}{\footnotesize{Carhart (1997) + $\text{HML}^m$}} &
       \rotatebox{90}{\footnotesize{Fama and French (2015)}} &
       \rotatebox{90}{\textbf{\footnotesize{Model 2}}} &
       \rotatebox{90}{\footnotesize{Model 2 + WML}} &
       \rotatebox{90}{\textbf{\footnotesize{Model 1}}} &
       \rotatebox{90}{\footnotesize{Model 1 + CMA}} \\
  \midrule

    % \multicolumn{8}{l}{Value constrained} \\
    Size-BM-Inv  & 2.58  & 3.44  & 2.32  & 1.56  & 1.49  & 2.13  & 2.13  \\
    BM constrained  & 2.31  & 2.69  & 2.14  & \textbf{1.62}  & 1.58  & \textbf{1.81}  & 1.82  \\
    % [1em]
    % Size-$\text{BM}^m$-Prior  & 7.26  & 5.91  & 7.55  & 6.52  & 6.51  & 4.85  & 4.85  \\
    % Value constrained  & 3.95  & 3.67  & 3.74  & 3.21  & 3.22  & 2.83  & 2.83  \\
    [1em]

    \multicolumn{8}{l}{Selected anomalies} \\
    Size-Inv  & 3.82  & 3.40  & 3.45  & \textbf{2.43}  & 2.31  & \textbf{2.02}  & 2.02  \\
    % Size-Acc  & 3.26  & 3.10  & 3.89  & 2.33  & 2.21  & 2.00  & 2.07  \\
    Size-$\beta$  & 1.51  & \textbf{1.22}  & \textbf{1.83}  & \textbf{1.63}  & 1.52  & \textbf{1.15}  & 1.15  \\
    % Size-NI  & 3.67  & 3.42  & 3.35  & 2.26  & 2.13  & 1.84  & 1.84  \\
    Size-Prior  & 3.96  & \textbf{3.89}  & \textbf{4.43}  & \textbf{3.94}  & 3.59  & \textbf{3.62}  & 3.66  \\
    Size-Var  & 5.11  & 5.09  & 5.02  & \textbf{4.38}  & 4.44  & \textbf{4.13}  & 4.14  \\
    Selected anomalies  & 2.84  & 2.78  & 2.80  & 2.37  & 2.32  & 2.23  & 2.22  \\
    [1em]
    All  & 2.85  & 2.70  & 2.80  & \textbf{2.44}  & 2.40  & \textbf{2.23}  & 2.22  \\

  \bottomrule
\end{tabular}


    }
\end{center}
\end{frame}

\section{Omissions}

\begin{frame}
\frametitle{``Big" Omissions}
\begin{itemize}
    \item BAB factor -- small contribution to $\text{Sh}^2$ (0.02) due to
    unexplained volatility
    \item QMJ factor -- small contribution to $\text{Sh}^2$ (0.05) due to low
    unexplained average return
    \item Stance on risk vs behavioural stories
\end{itemize}
\end{frame}

\section{Data}

\begin{frame}
\frametitle{Data}
\begin{itemize}
    \item Factors: Ken French all except $\text{HML}^m$ which is from AQR
    \item Sorts: Ken French all except size, value, investment/momentum
    \item CRSP: PMU factor and all characteristics
\end{itemize}
\end{frame}

\section{References}

\begin{frame}[allowframebreaks]
  \printbibliography
\end{frame}


\end{document}


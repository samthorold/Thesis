\documentclass[notes]{beamer}  % remove "[notes]" to print only slides

\usepackage{amsmath}
\usepackage[backend=biber, style=authoryear]{biblatex}
\usepackage{booktabs}
\usepackage{dcolumn}
\usepackage{graphicx}
\usepackage[utf8]{inputenc}

% \usetheme{Madrid}

\addbibresource{references.bib}

\title[Asset-Pricing]{A top-down approach to factor models}
\author[Thorold]{4299 Sam Thorold\\
                 Supervisors: Francisco Santos and Andr\'e Silva}
\institute{NHH and Nova SBE}
\date[June 2018]{\today}

\begin{document}

\frame{\titlepage}

\section{Intro}

\begin{frame} \frametitle{Problem}
  \begin{itemize}
    \item \textbf{Goal:} Minimize mispricing for the returns on all portfolios with a linear combination of the returns on fewer portfolios (``factors")
    \item \textbf{Problem 1:} Cannot test mispricing for all portfolios without identifying all portfolios, which leads to
    \item \textbf{Problem 2:} Relying on mispricing for subsets of portfolios (``anomalies") has given rise to significant data-mining issues
  \end{itemize}
\end{frame}

\note[itemize]{
  \item The goal is not to \emph{generate} alpha,
  pricing $\neq$ investing -- the model is not a strategy
  \item \textcite{harvey2016and} document 300+ factors in the literature
  \item \textcite{mclean2016does} find publication reduces the returns on many factors
  \item \textcite{linnainmaa2016history} claim ``most" factors are spurious out of sample
  \item Lies, damned lies, and statistics -- If you want to find correlation, you can
}

\begin{frame} \frametitle{Mispricing Definition}
  \begin{itemize}
    \item GRS test of \textcite{gibbons1989test} -- mispricing is given by the quadratic form of the alphas
      \begin{align*}
        \text{Mispricing} &= \text{Sh}^2(\text{alphas})\\
                          &= \text{Sh}^2(\text{test portfolios and factors}) - \text{Sh}^2(\text{factors})
      \end{align*}
    \item \textcite{barillas2016alpha} explain the factors are contained in the set of all portfolios so
    $$
      \text{Sh}^2(\text{all portfolios and factors}) = \text{Sh}^2(\text{all portfolios})
    $$
    and
    $$
      \text{Mispricing} = \text{Sh}^2(\text{all portfolios}) - \text{Sh}^2(\text{factors})
    $$
    \item I focus on $\text{Sh}^2(\text{factors})$ but the ultimate aim is to minimize $\text{Sh}^2(\text{alphas}$)
  \end{itemize}
\end{frame}

\begin{frame} \frametitle{Solution: Top-down approach}
  \begin{enumerate}
    \item Maximize $\text{Sh}^2(\text{factors})$ to minimize mispricing for all portfolios -- no need identify all portfolios\\
    % When the model is on the positive portion of the efficient frontier this is equivalent to maximizing the Sharpe ratio of the factors
    \item Choice of factors gives description of anomalies, rather than description of anomalies driving choice of factors -- reduces data-mining
  \end{enumerate}
\end{frame}

\begin{frame} \frametitle{Model}
  \begin{equation} \label{eq:B16}
    R_t^i = a^i+b^iR^M+s^iSMB_t+
    \underbrace{v^iHML_t^m+m^iWML_t}_\text{cheap}+
    \underbrace{p^iPMU_t}_\text{profitable}
  \end{equation}
  \begin{itemize}
    \item $\text{R}^M$: Excess return on the market
    \item $SMB$: Size factor long small and short big stocks
    \item $HML^m$: Value factor long cheap and short expensive stocks
    \item $WML$: Momentum factor long recent winners and short recent losers
    \item $PMU$: Profitability factor long profitable and short unprofitable stocks
  \end{itemize}
\end{frame}

\note[itemize]{
  \item Factors are long-short portfolios designed to capture unobserved state variables priced by investors (ICAPM of \textcite{merton1973intertemporal})
  \item Characteristics used to construct factors are simply sorting variables -- not necessarily to do with ROE
  \item Investors want to identify \emph{cheap} and \emph{profitable} stocks \parencite{graham1934security}
  \item Cheap $\rightarrow$ Book-to-Market (BM) \parencite{fama1993common}
  \item Some problems here as we want to identify changes in BM due to changes in market equity rather than book equity
  \item Monthly value + momentum \parencite{asness2013devil, kok2017facts}
  \item Profitable $\rightarrow$ Operating profit adjusted for accruals \parencite{ball2016accruals}
}

\section{Prior research}

\begin{frame} \frametitle{Prior research}
  \begin{itemize}
    \item Fama and French (2015, 2016a, 2016b)
    \begin{equation} \label{eq:F16}
      R_t^i=a^i + b^iR_t^M + s^iSMB_t + v^iHML_t + p^iPMU_t + i^iCMA_t
    \end{equation}
    $CMA$: Investment factor long conservatively and short aggressively investing stocks
    \item ``lethal combination": small, unprofitable stocks that somehow invest aggressively
    \item \textcite{fama2006profitability} motivate profitability and investment factors with the dividend discount model
    \[ME_t =
    \sum_{s=1}^\infty E\left( \text{Profit}_{t+s}-
    \Delta\text{BE}_{t+s}\right) /R^s\]
    Change in assets is used instead of change in BE
  \end{itemize}
\end{frame}

\note[itemize]{
  \item \scriptsize{\parencite{fama2015five, fama2016choosing, fama2016dissecting}}
  \item \textcite{ball2016accruals} present a model similar to the one I propose to show an accruals factor does not improve the Sharpe ratio of a five-factor model including their accrual-adjusted profitability factor
}

\section{Data}

\begin{frame}
  \frametitle{Data}
  \begin{itemize}
    \item Factors: Ken French all except $\text{HML}^m$, which is from AQR, and PMU
    \item Sorts: Ken French all except size, value, investment/momentum
    \item PMU factor and all characteristics: CRSP and COMPUSTAT
    \item U.S. common stocks -- July 1963 through December 2017
    \item I include financial firms and do not winsorize variables
  \end{itemize}
\end{frame}

\section{Results}

\begin{frame}
\frametitle{Sharpe ratio}
\resizebox{\linewidth}{!}{
\begin{tabular}{lcccccc}
  \toprule
  \multicolumn{7}{l}{Panel A: Model factors and maximum squared Sharpe ratios} \\
        & \multicolumn{2}{l}{Name}                         & \multicolumn{3}{l}{Factors}                                    & $\text{Sh}^2$   \\
  1     & \multicolumn{2}{l}{\textbf{Model 1}}             & \multicolumn{3}{l}{$R^M$, $SMB$, $HML^m$, $PMU$, $WML$}        & \textbf{0.316} \\
  2     & \multicolumn{2}{l}{Model 1 + $CMA$}              & \multicolumn{3}{l}{$R^M$, $SMB$, $HML^m$, $PMU$, $WML$, $CMA$} & 0.316 \\
  3     & \multicolumn{2}{l}{Fama and French (2017)}       & \multicolumn{3}{l}{$R^M$, $SMB$, $HML$, $PMU$, $WML$, $CMA$}   & 0.240 \\
  4     & \multicolumn{2}{l}{\textbf{Model 2}}             & \multicolumn{3}{l}{$R^M$, $SMB$, $HML$, $PMU$, $CMA$}          & \textbf{0.225} \\
  5     & \multicolumn{2}{l}{Fama and French (2015)}       & \multicolumn{3}{l}{$R^M$, $SMB$, $HML$, $PMU^{06}$, $CMA$}     & 0.099 \\
  6     & \multicolumn{2}{l}{Carhart (1997)}               & \multicolumn{3}{l}{$R^M$, $SMB$, $HML$, $WML$}                 & 0.090 \\
  7     & \multicolumn{2}{l}{Carhart (1997) + $HML^m$}     & \multicolumn{3}{l}{$R^M$, $SMB$, $HML^m$, $WML$}               & 0.136 \\
  \midrule
  \multicolumn{7}{l}{Panel B: 90\% confidence interval for distributions of $\text{Sh}^2(\text{Row})$ - $\text{Sh}^2(\text{Column})$} \\
        &        2         &        3         &        4         &        5         &        6         &        7         \\
  1     & \textbf{(-0.01,  0.00)} & ( 0.04,  0.11) & ( 0.06,  0.14) & ( 0.16,  0.29) & ( 0.17,  0.30) & ( 0.13,  0.25) \\
  2     &                  & ( 0.04,  0.12) & ( 0.06,  0.14) & ( 0.16,  0.29) & ( 0.17,  0.30) & ( 0.13,  0.25) \\
  3     &                  &                  & ( 0.00,  0.05) & ( 0.10,  0.20) & ( 0.11,  0.21) & ( 0.05,  0.17) \\
  4     &                  &                  &                  & ( 0.08,  0.18) & ( 0.08,  0.20) & ( 0.03,  0.16) \\
  5     &                  &                  &                  &                  & \textbf{(-0.03,  0.05)} & \textbf{(-0.09,  0.01)} \\
  6     &                  &                  &                  &                  &                  & (-0.07, -0.03) \\
  \bottomrule
\end{tabular}
}
\end{frame}

\note[itemize]{
  \item Bootstrap: 1 simulation is 654 draws w/ replacement\\
  100K simulations gives 100K values of Sh2(row)-Sh2(col)\\
  (5th percentile, 95th percentile) so 90\% of values in this range
  \item Investment factor does not improve Sharpe ratio of model 1
  \item Model 1 has higher Sharpe ratio than model 2 (+ momentum)
  \item Original five-factor model does not have a reliably higher Sharpe ratio than the four-factor model of Carhart
  \item Profitability and investment are credited with reducing the maginitude and number of descriptions of many common anomalies
  \item Annual or monthly value and momentum subsume investment (not shown)
  \item Why do value and momentum subsume investment?
}

\begin{frame} \frametitle{Inv poor proxy for $\Delta$BE: Size-$\text{BM}^m$-Prior characteristics}
\begin{center}
\resizebox*{!}{\dimexpr\textheight-1.3cm\relax}{
\begin{tabular}{lrrrrrrrr}
  \toprule
     & \multicolumn{4}{c}{Small} & \multicolumn{4}{c}{Big}  \\
     \cmidrule(r){2-5} \cmidrule(r){6-9}
    $\text{BM}^m$ $\rightarrow$ & Low & 2 & 3 & High & Low & 2 & 3 & High  \\ 
  \midrule
  
  
    & \multicolumn{8}{c}{$\text{R}^i$}  \\
     \cmidrule(r){2-5} \cmidrule(r){6-9}
    Low Prior  & 0.09  & 0.76  & 0.98  & 1.01  & 0.55  & 0.75  & 0.95  & 0.96   \\
    2  & 0.69  & 1.07  & 1.24  & 1.51  & 0.64  & 0.82  & 0.9  & 1.1   \\
    3  & 0.96  & 1.26  & 1.55  & 1.71  & 0.9  & 0.87  & 0.99  & 1.23   \\
    High Prior  & 1.47  & 1.58  & 1.78  & 2.04  & 1.21  & 1.11  & 1.14  & 1.42   \\
    
  
    & \multicolumn{8}{c}{Future Var}  \\
     \cmidrule(r){2-5} \cmidrule(r){6-9}
    Low Prior  & 0.2  & 0.13  & 0.13  & 0.19  & 0.05  & 0.05  & 0.04  & 0.05   \\
    2  & 0.1  & 0.07  & 0.07  & 0.09  & 0.03  & 0.03  & 0.03  & 0.03   \\
    3  & 0.08  & 0.06  & 0.06  & 0.08  & 0.03  & 0.03  & 0.03  & 0.03   \\
    High Prior  & 0.11  & 0.08  & 0.08  & 0.12  & 0.05  & 0.04  & 0.04  & 0.04   \\
    
  
    & \multicolumn{8}{c}{Future Inv}  \\
     \cmidrule(r){2-5} \cmidrule(r){6-9}
    Low Prior  & 0.39  & 0.24  & 0.19  & 0.2  & 0.21  & 0.16  & 0.14  & 0.2   \\
    2  & 0.27  & 0.16  & 0.11  & 0.09  & 0.17  & 0.13  & 0.11  & 0.1   \\
    3  & 0.24  & 0.13  & 0.09  & 0.07  & 0.17  & 0.13  & 0.11  & 0.09   \\
    High Prior  & 0.25  & 0.12  & 0.1  & 0.06  & 0.23  & 0.14  & 0.12  & 0.09   \\
    
  
    & \multicolumn{8}{c}{Future $\Delta\text{BE}$}  \\
     \cmidrule(r){2-5} \cmidrule(r){6-9}
    Low Prior  & 0.31  & 0.07  & 0.01  & -0.1  & 0.42  & 0.11  & 0.07  & 0.02   \\
    2  & 0.3  & 0.09  & 0.04  & -0.02  & 0.25  & 0.11  & 0.08  & 0.04   \\
    3  & 0.34  & 0.1  & 0.05  & -0.0  & 0.27  & 0.11  & 0.08  & 0.05   \\
    High Prior  & 0.61  & 0.1  & 0.04  & -0.03  & 0.53  & 0.14  & 0.1  & 0.04   \\
    
  
  \bottomrule
\end{tabular}
}
\end{center}
\end{frame}

\note[itemize]{
  \item Avg monthly value-weighted chars
  \item Large dispursion in returns -- potential to describe a lot of variation
  \item Little variation in variance
  \item Future investment and future changes in BE backwards for small, growth stocks
  \item Monthly value and investment capture variation on both
}

\begin{frame} \frametitle{Inv poor proxy for $\Delta$BE: Size-BM-Inv characteristics}
\begin{center}
\resizebox*{!}{\dimexpr\textheight-1.3cm\relax}{
\begin{tabular}{lrrrrrrrr}
  \toprule
     & \multicolumn{4}{c}{Small} & \multicolumn{4}{c}{Big}  \\
     \cmidrule(r){2-5} \cmidrule(r){6-9}
    BM $\rightarrow$ & Low & 2 & 3 & High & Low & 2 & 3 & High  \\ 
  \midrule
  
  
    % & \multicolumn{8}{c}{$\text{r}^i$}  \\
    %  \cmidrule(r){2-5} \cmidrule(r){6-9}
    % Low Inv  & 1.01  & 1.33  & 1.45  & 1.46  & 0.98  & 1.04  & 1.05  & 1.14   \\
    % 2  & 1.22  & 1.31  & 1.30  & 1.48  & 0.92  & 0.93  & 0.95  & 0.99   \\
    % 3  & 1.25  & 1.29  & 1.41  & 1.28  & 0.92  & 0.94  & 0.93  & 1.16   \\
    % High Inv  & 0.94  & 1.17  & 1.27  & 1.31  & 0.98  & 0.80  & 0.90  & 1.06   \\
    
  
    % & \multicolumn{8}{c}{Future Var}  \\
    %  \cmidrule(r){2-5} \cmidrule(r){6-9}
    % Low Inv  & 0.16  & 0.11  & 0.10  & 0.14  & 0.04  & 0.03  & 0.03  & 0.04   \\
    % 2  & 0.09  & 0.07  & 0.07  & 0.09  & 0.03  & 0.03  & 0.03  & 0.03   \\
    % 3  & 0.08  & 0.06  & 0.06  & 0.09  & 0.03  & 0.03  & 0.03  & 0.04   \\
    % High Inv  & 0.11  & 0.09  & 0.08  & 0.11  & 0.05  & 0.04  & 0.04  & 0.04   \\
    
  
    & \multicolumn{8}{c}{Future Inv}  \\
     \cmidrule(r){2-5} \cmidrule(r){6-9}
    Low Inv  & \textbf{-0.10}  & -0.08  & -0.08  & -0.09  & -0.02  & -0.02  & -0.01  & -0.02   \\
    2  & \textbf{0.04}  & 0.03  & 0.03  & 0.02  & 0.06  & 0.06  & 0.06  & 0.06   \\
    3  & \textbf{0.11}  & 0.10  & 0.10  & 0.10  & 0.12  & 0.12  & 0.12  & 0.11   \\
    High Inv  & \textbf{0.48}  & 0.39  & 0.37  & 0.47  & 0.37  & 0.38  & 0.37  & 0.52   \\
    
  
    & \multicolumn{8}{c}{Future $\Delta\text{BE}$}  \\
     \cmidrule(r){2-5} \cmidrule(r){6-9}
    Low Inv  & \textbf{0.59}  & 0.06  & 0.02  & -0.03  & 0.28  & 0.07  & 0.06  & 0.03   \\
    2  & \textbf{0.44}  & 0.09  & 0.05  & 0.01  & 0.20  & 0.10  & 0.07  & 0.05   \\
    3  & \textbf{0.27}  & 0.11  & 0.08  & 0.03  & 0.16  & 0.12  & 0.12  & 0.08   \\
    High Inv  & \textbf{0.40}  & 0.16  & 0.11  & 0.06  & 0.38  & 0.17  & 0.13  & 0.10   \\
    
  
  \bottomrule
\end{tabular}
}
\end{center}
\end{frame}

\note[itemize]{
  \item Avg monthly value-weighted chars
  \item Less dispursion in returns
  \item Little variation in variance
  \item Future investment and future changes in BE backwards for small, growth stocks
  \item Problem because investment is a factor and so will not capture small, growth stock's true relationship with future changes in BE
  \item Value and momentum created variation in Inv and change in BE
}

\begin{frame} \frametitle{Part 1 of top-down approach summary}
  \begin{itemize}
    \item Model 1 has the highest $\text{Sh}^2$, 0.316, and so leaves less mispricing for all portfolios than competing models
    \item Value and momentum's relationship with future changes in BE makes the investment factor redundant
    \item How does model 1 describe common anomalies? (Part 2 of top-down approach)
  \end{itemize}
\end{frame}

\begin{frame} \frametitle{GRS statistic}
\begin{center}
\resizebox*{!}{\dimexpr\textheight-1.3cm\relax}{
\begin{tabular}{lrrrrrrr}
  \toprule
     & \rotatebox{90}{\footnotesize{Carhart (1997)}} &
       \rotatebox{90}{\footnotesize{Carhart (1997) + $\text{HML}^m$}} &
       \rotatebox{90}{\footnotesize{Fama and French (2015)}} &
       \rotatebox{90}{\textbf{\footnotesize{Model 2}}} &
       \rotatebox{90}{\footnotesize{Model 2 + WML}} &
       \rotatebox{90}{\textbf{\footnotesize{Model 1}}} &
       \rotatebox{90}{\footnotesize{Model 1 + CMA}} \\
  \midrule

    % \multicolumn{8}{l}{Value constrained} \\
    Size-BM-Inv  & 2.58  & 3.44  & 2.32  & 1.56  & 1.49  & 2.13  & 2.13  \\
    BM constrained  & 2.31  & 2.69  & 2.14  & \textbf{1.62}  & 1.58  & \textbf{1.81}  & 1.82  \\
    % [1em]
    % Size-$\text{BM}^m$-Prior  & 7.26  & 5.91  & 7.55  & 6.52  & 6.51  & 4.85  & 4.85  \\
    % Value constrained  & 3.95  & 3.67  & 3.74  & 3.21  & 3.22  & 2.83  & 2.83  \\
    [1em]

    \multicolumn{8}{l}{Selected anomalies} \\
    Size-Inv  & 3.82  & 3.40  & 3.45  & \textbf{2.43}  & 2.31  & \textbf{2.02}  & 2.02  \\
    % Size-Acc  & 3.26  & 3.10  & 3.89  & 2.33  & 2.21  & 2.00  & 2.07  \\
    Size-$\beta$  & 1.51  & \textbf{1.22}  & \textbf{1.83}  & \textbf{1.63}  & 1.52  & \textbf{1.15}  & 1.15  \\
    % Size-NI  & 3.67  & 3.42  & 3.35  & 2.26  & 2.13  & 1.84  & 1.84  \\
    Size-Prior  & 3.96  & \textbf{3.89}  & \textbf{4.43}  & \textbf{3.94}  & 3.59  & \textbf{3.62}  & 3.66  \\
    Size-Var  & 5.11  & 5.09  & 5.02  & \textbf{4.38}  & 4.44  & \textbf{4.13}  & 4.14  \\
    Selected anomalies  & 2.84  & 2.78  & 2.80  & 2.37  & 2.32  & 2.23  & 2.22  \\
    [1em]
    All  & 2.85  & 2.70  & 2.80  & \textbf{2.44}  & 2.40  & \textbf{2.23}  & 2.22  \\

  \bottomrule
\end{tabular}

}
\end{center}
\end{frame}

\note[itemize]{
  \item Anomalies from Ken French website
  \item Annual value a problem -- unwanted variation in BM within BMm buckets (not shown)
  \item All other sorts a wrap for Model 1 (GRS lower -- Sh2 improved less)
  \item Investment factor adds nothing to subsets of stocks
  \item Momentum and volatility biggest problems
}

\begin{frame} \frametitle{Size-momentum returns}
  \begin{center}
  \resizebox*{!}{\dimexpr\textheight-1.3cm\relax}{
\begin{tabular}{lrrrrrrrrrr}
  \toprule
    
    Prior $\rightarrow$ & Low & 2 & 3 & 4 & High & Low & 2 & 3 & 4 & High  \\ 
  \midrule
  \multicolumn{11}{l}{$R^i=a^i+\beta^iMkt+s^iSize+v^iVal+m^iMom+p^iProf$}  \\
  
     & \multicolumn{5}{c}{a} & \multicolumn{5}{c}{t(a)}   \\
     \cmidrule(r){2-6} \cmidrule(r){7-11} 
    Small  & -0.33  & -0.07  & 0.06  & 0.13  & 0.38  & -3.13  & -1.09  & 0.95  & 1.91  & 4.42   \\
    2  & -0.11  & -0.01  & -0.01  & 0.03  & 0.26  & -1.45  & -0.22  & -0.11  & 0.45  & 3.84   \\
    3  & 0.15  & 0.04  & -0.05  & -0.17  & 0.21  & 1.59  & 0.61  & -0.83  & -2.48  & 3.00   \\
    4  & 0.15  & 0.10  & -0.01  & -0.01  & 0.11  & 1.41  & 1.29  & -0.18  & -0.08  & 1.43   \\
    Big  & 0.12  & 0.20  & -0.03  & -0.15  & -0.10  & 1.06  & 2.75  & -0.50  & -2.43  & -1.30   \\
    
  
     & \multicolumn{5}{c}{v} & \multicolumn{5}{c}{t(v)}   \\
     \cmidrule(r){2-6} \cmidrule(r){7-11} 
    Small  & 0.08  & 0.33  & 0.36  & 0.29  & 0.02  & 2.06  & 13.88  & 14.94  & 11.64  & 0.77   \\
    2  & -0.06  & 0.24  & 0.29  & 0.29  & -0.09  & -2.14  & 10.44  & 12.81  & 13.98  & -3.67   \\
    3  & -0.09  & 0.20  & 0.31  & 0.32  & -0.08  & -2.49  & 8.33  & 13.23  & 12.81  & -3.17   \\
    4  & -0.03  & 0.22  & 0.27  & 0.24  & -0.05  & -0.73  & 8.05  & 10.47  & 9.45  & -1.91   \\
    Big  & -0.04  & 0.12  & 0.11  & 0.17  & -0.01  & -1.08  & 4.65  & 4.58  & 7.30  & -0.43   \\
    
  
     & \multicolumn{5}{c}{m} & \multicolumn{5}{c}{t(m)}   \\
     \cmidrule(r){2-6} \cmidrule(r){7-11} 
    Small  & -0.64  & -0.11  & 0.07  & 0.19  & 0.31  & -21.82  & -6.08  & 3.63  & 10.24  & 13.01   \\
    2  & -0.73  & -0.22  & 0.05  & 0.20  & 0.33  & -33.36  & -12.16  & 3.05  & 12.40  & 17.30   \\
    3  & -0.79  & -0.25  & -0.01  & 0.22  & 0.38  & -29.42  & -13.03  & -0.70  & 11.44  & 19.61   \\
    4  & -0.81  & -0.29  & -0.04  & 0.18  & 0.43  & -27.40  & -14.04  & -1.91  & 9.31  & 19.68   \\
    Big  & -0.78  & -0.38  & -0.06  & 0.23  & 0.47  & -25.67  & -18.86  & -3.05  & 13.17  & 23.01   \\
    
  
     & \multicolumn{5}{c}{p} & \multicolumn{5}{c}{t(p)}   \\
     \cmidrule(r){2-6} \cmidrule(r){7-11} 
    Small  & -0.22  & -0.04  & 0.02  & 0.03  & -0.20  & -2.84  & -0.75  & 0.42  & 0.57  & -3.18   \\
    2  & -0.21  & 0.12  & 0.07  & 0.09  & -0.18  & -3.60  & 2.49  & 1.49  & 2.09  & -3.61   \\
    3  & -0.22  & 0.06  & 0.12  & 0.11  & -0.09  & -3.13  & 1.28  & 2.45  & 2.17  & -1.80   \\
    4  & -0.22  & 0.04  & 0.09  & 0.04  & -0.08  & -2.80  & 0.65  & 1.62  & 0.73  & -1.36   \\
    Big  & 0.07  & 0.12  & 0.03  & 0.09  & -0.01  & 0.92  & 2.18  & 0.61  & 1.84  & -0.10   \\


    \midrule
  \multicolumn{11}{l}{$R^i=a^i+\beta^iMkt+s^iSize+v^iVal+p^iProf+i^iInv+m^iMom$}  \\
  
     & \multicolumn{5}{c}{a} & \multicolumn{5}{c}{t(a)}   \\
     \cmidrule(r){2-6} \cmidrule(r){7-11} 
    Small  & -0.21  & 0.06  & 0.19  & 0.22  & 0.39  & -2.08  & 0.89  & 2.92  & 3.36  & 4.72   \\
    2  & -0.10  & 0.08  & 0.09  & 0.13  & 0.23  & -1.26  & 1.31  & 1.45  & 2.25  & 3.52   \\
    3  & 0.15  & 0.10  & 0.06  & -0.08  & 0.19  & 1.60  & 1.60  & 0.88  & -1.13  & 2.91   \\
    4  & 0.14  & 0.17  & 0.06  & 0.08  & 0.10  & 1.40  & 2.32  & 0.90  & 1.22  & 1.30   \\
    Big  & 0.11  & 0.22  & -0.02  & -0.09  & -0.04  & 1.01  & 3.16  & -0.28  & -1.50  & -0.57   \\


     & \multicolumn{5}{c}{v} & \multicolumn{5}{c}{t(v)}   \\
     \cmidrule(r){2-6} \cmidrule(r){7-11} 
    Small  & 0.01  & 0.29  & 0.33  & 0.24  & -0.01  & 0.24  & 9.16  & 10.74  & 7.55  & -0.13   \\
    2  & -0.02  & 0.24  & 0.29  & 0.27  & -0.04  & -0.53  & 8.05  & 10.04  & 9.97  & -1.39   \\
    3  & -0.04  & 0.22  & 0.30  & 0.34  & -0.04  & -1.02  & 7.25  & 9.91  & 10.70  & -1.15   \\
    4  & -0.00  & 0.13  & 0.26  & 0.20  & -0.03  & -0.06  & 3.80  & 8.14  & 6.13  & -0.72   \\
    Big  & 0.03  & 0.10  & 0.16  & 0.10  & 0.03  & 0.65  & 3.09  & 5.05  & 3.51  & 0.99   \\
    
  
     & \multicolumn{5}{c}{p} & \multicolumn{5}{c}{t(p)}   \\
     \cmidrule(r){2-6} \cmidrule(r){7-11} 
    Small  & -0.28  & -0.07  & -0.00  & 0.01  & -0.21  & -3.66  & -1.39  & -0.04  & 0.16  & -3.35   \\
    2  & -0.21  & 0.10  & 0.06  & 0.07  & -0.16  & -3.74  & 2.13  & 1.30  & 1.65  & -3.32   \\
    3  & -0.22  & 0.06  & 0.10  & 0.11  & -0.08  & -3.12  & 1.30  & 2.09  & 2.16  & -1.65   \\
    4  & -0.21  & 0.01  & 0.08  & 0.01  & -0.07  & -2.73  & 0.11  & 1.58  & 0.28  & -1.24   \\
    Big  & 0.09  & 0.12  & 0.05  & 0.06  & -0.02  & 1.16  & 2.21  & 0.99  & 1.30  & -0.40   \\
    
  
     & \multicolumn{5}{c}{i} & \multicolumn{5}{c}{t(i)}   \\
     \cmidrule(r){2-6} \cmidrule(r){7-11} 
    Small  & -0.09  & -0.01  & -0.01  & 0.05  & 0.04  & -1.34  & -0.15  & -0.32  & 1.20  & 0.65   \\
    2  & -0.14  & -0.06  & -0.03  & -0.00  & -0.06  & -2.81  & -1.33  & -0.67  & -0.05  & -1.49   \\
    3  & -0.11  & -0.04  & -0.03  & -0.04  & -0.08  & -1.75  & -1.05  & -0.69  & -0.93  & -1.93   \\
    4  & -0.04  & 0.11  & 0.01  & 0.02  & -0.05  & -0.66  & 2.36  & 0.20  & 0.38  & -0.95   \\
    Big  & -0.13  & 0.06  & -0.04  & 0.07  & -0.19  & -1.81  & 1.27  & -0.87  & 1.78  & -4.14   \\
    
  
     & \multicolumn{5}{c}{m} & \multicolumn{5}{c}{t(m)}   \\
     \cmidrule(r){2-6} \cmidrule(r){7-11} 
    Small  & -0.68  & -0.25  & -0.08  & 0.08  & 0.30  & -29.32  & -16.49  & -5.13  & 5.20  & 15.90   \\
    2  & -0.71  & -0.32  & -0.06  & 0.08  & 0.37  & -41.27  & -22.00  & -4.42  & 6.47  & 24.53   \\
    3  & -0.75  & -0.33  & -0.14  & 0.10  & 0.41  & -35.78  & -21.95  & -9.44  & 6.40  & 27.20   \\
    4  & -0.79  & -0.38  & -0.14  & 0.09  & 0.45  & -34.25  & -23.27  & -9.26  & 5.44  & 26.34   \\
    Big  & -0.76  & -0.43  & -0.10  & 0.16  & 0.48  & -31.73  & -27.15  & -6.65  & 11.56  & 30.08   \\
  
  \bottomrule
\end{tabular}

}
  \end{center}
\end{frame}

\note[itemize]{
  \item Beta smile -- high beta stocks in the extremes of momentum
  \item Value insignificant in the extremes of momentum
  \item Momentum helps but not completely
  \item Negative profitability slopes suggest defensive equity, beta and vol}

\begin{frame} \frametitle{Size-beta returns}
  \begin{center}
  \resizebox*{!}{\dimexpr\textheight-1.3cm\relax}{
\begin{tabular}{lrrrrrrrrrr}
  \toprule
    
    Beta $\rightarrow$ & Low & 2 & 3 & 4 & High & Low & 2 & 3 & 4 & High  \\ 
  \midrule
  \multicolumn{11}{l}{$R^i=a^i+\beta^iMkt+s^iSize+v^iVal+m^iMom+p^iProf$}  \\
  
     & \multicolumn{5}{c}{a} & \multicolumn{5}{c}{t(a)}   \\
     \cmidrule(r){2-6} \cmidrule(r){7-11} 
    Small  & 0.10  & 0.08  & -0.00  & 0.08  & -0.07  & 1.34  & 1.13  & -0.02  & 1.26  & -0.73   \\
    2  & 0.01  & 0.04  & 0.07  & -0.04  & -0.09  & 0.15  & 0.61  & 1.06  & -0.67  & -1.12   \\
    3  & 0.08  & 0.15  & 0.00  & -0.06  & -0.02  & 0.97  & 2.19  & 0.03  & -0.83  & -0.22   \\
    4  & 0.10  & 0.08  & -0.01  & -0.12  & 0.03  & 1.11  & 1.09  & -0.08  & -1.48  & 0.28   \\
    Big  & -0.03  & -0.07  & -0.08  & -0.14  & 0.01  & -0.34  & -1.21  & -1.31  & -1.63  & 0.10   \\
    
  
    %  & \multicolumn{5}{c}{b} & \multicolumn{5}{c}{t(b)}   \\
    %  \cmidrule(r){2-6} \cmidrule(r){7-11} 
    % Small  & 0.65  & 0.78  & 0.88  & 1.01  & 1.16  & 39.89  & 48.20  & 54.71  & 69.21  & 53.20   \\
    % 2  & 0.71  & 0.83  & 0.97  & 1.08  & 1.29  & 39.66  & 55.60  & 65.15  & 73.05  & 71.39   \\
    % 3  & 0.72  & 0.88  & 1.00  & 1.10  & 1.32  & 39.04  & 58.04  & 63.45  & 62.80  & 61.41   \\
    % 4  & 0.77  & 0.96  & 1.06  & 1.17  & 1.38  & 38.52  & 58.34  & 62.82  & 61.26  & 51.53   \\
    % Big  & 0.78  & 0.97  & 1.08  & 1.20  & 1.33  & 45.15  & 70.31  & 73.67  & 61.03  & 40.37   \\
    
  
    %  & \multicolumn{5}{c}{v} & \multicolumn{5}{c}{t(v)}   \\
    %  \cmidrule(r){2-6} \cmidrule(r){7-11} 
    % Small  & 0.28  & 0.31  & 0.27  & 0.20  & -0.06  & 10.72  & 11.84  & 10.45  & 8.38  & -1.74   \\
    % 2  & 0.33  & 0.31  & 0.26  & 0.21  & -0.11  & 11.18  & 12.76  & 10.83  & 8.61  & -3.66   \\
    % 3  & 0.35  & 0.20  & 0.23  & 0.17  & -0.16  & 11.55  & 8.21  & 9.06  & 5.89  & -4.48   \\
    % 4  & 0.33  & 0.24  & 0.17  & 0.12  & -0.14  & 10.25  & 8.78  & 6.11  & 3.78  & -3.28   \\
    % Big  & 0.24  & 0.12  & 0.03  & 0.00  & -0.28  & 8.48  & 5.22  & 1.10  & 0.12  & -5.18   \\
    
  
     & \multicolumn{5}{c}{m} & \multicolumn{5}{c}{t(m)}   \\
     \cmidrule(r){2-6} \cmidrule(r){7-11} 
    Small  & 0.14  & 0.15  & 0.15  & 0.07  & -0.10  & 6.70  & 7.40  & 7.23  & 3.79  & -3.78   \\
    2  & 0.19  & 0.14  & 0.10  & 0.06  & -0.18  & 8.57  & 7.40  & 5.31  & 3.29  & -7.76   \\
    3  & 0.20  & 0.10  & 0.09  & 0.01  & -0.19  & 8.62  & 5.46  & 4.45  & 0.59  & -6.90   \\
    4  & 0.20  & 0.09  & 0.05  & -0.03  & -0.20  & 8.11  & 4.41  & 2.44  & -1.25  & -5.87   \\
    Big  & 0.15  & 0.06  & -0.05  & -0.08  & -0.24  & 7.06  & 3.72  & -2.76  & -3.32  & -5.91   \\
    
  
     & \multicolumn{5}{c}{p} & \multicolumn{5}{c}{t(p)}   \\
     \cmidrule(r){2-6} \cmidrule(r){7-11} 
    Small  & -0.13  & -0.02  & -0.01  & -0.07  & -0.16  & -2.50  & -0.45  & -0.28  & -1.42  & -2.17   \\
    2  & -0.12  & 0.09  & 0.10  & 0.10  & -0.06  & -2.04  & 1.82  & 2.10  & 2.15  & -1.09   \\
    3  & -0.15  & 0.03  & 0.15  & 0.13  & -0.01  & -2.50  & 0.66  & 2.84  & 2.26  & -0.13   \\
    4  & -0.16  & 0.06  & 0.11  & 0.05  & 0.04  & -2.44  & 1.13  & 2.03  & 0.76  & 0.41   \\
    Big  & 0.08  & 0.23  & 0.23  & 0.15  & -0.16  & 1.49  & 5.05  & 4.86  & 2.31  & -1.44   \\
    

  \midrule
  \multicolumn{11}{l}{$R^i=a^i+\beta^iMkt+s^iSize+v^iVal+p^iProf+i^iInv$}  \\
  
     & \multicolumn{5}{c}{a} & \multicolumn{5}{c}{t(a)}   \\
     \cmidrule(r){2-6} \cmidrule(r){7-11} 
    Small  & 0.16  & 0.13  & 0.04  & 0.11  & -0.10  & 2.35  & 1.96  & 0.65  & 1.84  & -1.05   \\
    2  & 0.10  & 0.08  & 0.11  & -0.03  & -0.17  & 1.34  & 1.33  & 1.75  & -0.43  & -2.16   \\
    3  & 0.18  & 0.16  & 0.03  & -0.09  & -0.09  & 2.34  & 2.59  & 0.48  & -1.17  & -0.98   \\
    4  & 0.20  & 0.10  & -0.01  & -0.16  & -0.04  & 2.36  & 1.42  & -0.10  & -1.94  & -0.38   \\
    Big  & 0.01  & -0.06  & -0.12  & -0.16  & -0.10  & 0.20  & -1.03  & -1.85  & -1.89  & -0.72   \\

  \bottomrule
\end{tabular}
}
  \end{center}
\end{frame}

\note[itemize]{
  \item High-beta behave like growth stocks -- volatile behave like growth stocks
  \item Beta and volatility linked because stock vol drives covariance with the market
  \item But beta not a problem -- volatility associated with the market is not a problem
  \item In fact, GRS showed beta caused the least problems of all
}

\begin{frame} \frametitle{Size-variance returns}
  \begin{center}
  \resizebox*{!}{\dimexpr\textheight-1.3cm\relax}{
\begin{tabular}{lrrrrrrrrrr}
  \toprule
    
    Var $\rightarrow$ & Low & 2 & 3 & 4 & High & Low & 2 & 3 & 4 & High  \\ 
  \midrule
  \multicolumn{11}{l}{$R^i=a^i+\beta^iMkt+s^iSize+v^iVal+m^iMom+p^iProf$}  \\
  
     & \multicolumn{5}{c}{a} & \multicolumn{5}{c}{t(a)}   \\
     \cmidrule(r){2-6} \cmidrule(r){7-11} 
    Small  & 0.31  & 0.25  & 0.18  & 0.01  & -0.75  & 4.12  & 3.26  & 2.47  & 0.10  & -4.63   \\
    2  & 0.14  & 0.10  & 0.12  & 0.02  & -0.26  & 1.98  & 1.26  & 1.50  & 0.24  & -2.22   \\
    3  & 0.07  & 0.02  & 0.06  & -0.04  & -0.05  & 0.85  & 0.30  & 0.67  & -0.44  & -0.46   \\
    4  & 0.10  & 0.08  & -0.03  & 0.01  & 0.02  & 1.10  & 0.97  & -0.40  & 0.14  & 0.18   \\
    Big  & 0.01  & -0.05  & -0.07  & -0.10  & 0.04  & 0.16  & -0.78  & -1.11  & -1.55  & 0.34   \\
    
  
    %  & \multicolumn{5}{c}{b} & \multicolumn{5}{c}{t(b)}   \\
    %  \cmidrule(r){2-6} \cmidrule(r){7-11} 
    % Small  & 0.65  & 0.94  & 1.05  & 1.14  & 1.19  & 38.78  & 53.50  & 62.64  & 56.35  & 32.19   \\
    % 2  & 0.72  & 0.95  & 1.06  & 1.20  & 1.34  & 43.97  & 53.41  & 60.54  & 65.18  & 51.27   \\
    % 3  & 0.71  & 0.94  & 1.05  & 1.18  & 1.32  & 40.94  & 55.44  & 55.89  & 60.96  & 53.40   \\
    % 4  & 0.73  & 0.92  & 1.07  & 1.18  & 1.35  & 36.48  & 48.29  & 57.07  & 59.49  & 52.78   \\
    % Big  & 0.72  & 0.91  & 1.01  & 1.13  & 1.30  & 41.58  & 62.31  & 70.22  & 75.62  & 51.21   \\
    
  
     & \multicolumn{5}{c}{s} & \multicolumn{5}{c}{t(s)}   \\
     \cmidrule(r){2-6} \cmidrule(r){7-11} 
    Small  & 0.61  & 0.86  & 1.01  & 1.20  & 1.51  & 26.05  & 35.57  & 43.78  & 42.71  & 29.60   \\
    2  & 0.49  & 0.66  & 0.75  & 0.88  & 1.20  & 21.65  & 27.03  & 31.08  & 34.48  & 33.40   \\
    3  & 0.23  & 0.40  & 0.49  & 0.62  & 0.91  & 9.72  & 17.13  & 18.92  & 23.23  & 26.62   \\
    4  & 0.01  & 0.11  & 0.16  & 0.25  & 0.60  & 0.34  & 4.19  & 6.34  & 9.04  & 17.07   \\
    Big  & -0.30  & -0.29  & -0.20  & -0.19  & 0.10  & -12.55  & -14.12  & -10.31  & -9.11  & 2.75   \\
    
  
     & \multicolumn{5}{c}{v} & \multicolumn{5}{c}{t(v)}   \\
     \cmidrule(r){2-6} \cmidrule(r){7-11} 
    Small  & 0.36  & 0.42  & 0.35  & 0.17  & -0.00  & 13.27  & 14.70  & 12.72  & 5.22  & -0.08   \\
    2  & 0.37  & 0.40  & 0.33  & 0.16  & -0.30  & 13.85  & 13.82  & 11.52  & 5.30  & -6.95   \\
    3  & 0.37  & 0.37  & 0.31  & 0.24  & -0.38  & 13.01  & 13.39  & 10.23  & 7.68  & -9.32   \\
    4  & 0.37  & 0.29  & 0.29  & 0.17  & -0.34  & 11.29  & 9.38  & 9.30  & 5.22  & -8.19   \\
    Big  & 0.20  & 0.16  & 0.10  & 0.03  & -0.29  & 7.20  & 6.90  & 4.44  & 1.09  & -6.97   \\
    
  
     & \multicolumn{5}{c}{m} & \multicolumn{5}{c}{t(m)}   \\
     \cmidrule(r){2-6} \cmidrule(r){7-11} 
    Small  & 0.08  & 0.05  & -0.06  & -0.18  & -0.36  & 3.64  & 2.30  & -2.71  & -7.09  & -7.82   \\
    2  & 0.13  & 0.12  & 0.05  & -0.05  & -0.32  & 6.37  & 5.48  & 2.19  & -2.37  & -9.78   \\
    3  & 0.15  & 0.08  & 0.08  & 0.01  & -0.27  & 6.94  & 3.80  & 3.28  & 0.52  & -8.80   \\
    4  & 0.15  & 0.06  & 0.06  & 0.01  & -0.26  & 6.08  & 2.50  & 2.64  & 0.41  & -8.15   \\
    Big  & 0.12  & 0.10  & 0.07  & 0.01  & -0.17  & 5.72  & 5.63  & 3.75  & 0.37  & -5.43   \\
    
  
     & \multicolumn{5}{c}{p} & \multicolumn{5}{c}{t(p)}   \\
     \cmidrule(r){2-6} \cmidrule(r){7-11} 
    Small  & 0.17  & 0.11  & 0.04  & -0.19  & -0.50  & 3.02  & 1.97  & 0.65  & -2.88  & -4.16   \\
    2  & 0.19  & 0.19  & 0.12  & 0.08  & -0.47  & 3.45  & 3.31  & 2.13  & 1.31  & -5.47   \\
    3  & 0.14  & 0.17  & 0.22  & 0.16  & -0.40  & 2.51  & 3.02  & 3.53  & 2.58  & -4.88   \\
    4  & 0.03  & 0.11  & 0.22  & 0.02  & -0.33  & 0.41  & 1.83  & 3.49  & 0.36  & -3.95   \\
    Big  & -0.03  & 0.25  & 0.16  & 0.04  & -0.17  & -0.46  & 5.13  & 3.39  & 0.87  & -2.06   \\

  % \midrule
  % \multicolumn{11}{l}{$R^i=a^i+\beta^iMkt+s^iSize+v^iVal+p^iProf+i^iInv$}  \\
  
  %    & \multicolumn{5}{c}{a} & \multicolumn{5}{c}{t(a)}   \\
  %    \cmidrule(r){2-6} \cmidrule(r){7-11} 
  %   Small  & 0.33  & 0.30  & 0.21  & -0.01  & -0.80  & 4.82  & 3.94  & 2.54  & -0.11  & -4.76   \\
  %   2  & 0.19  & 0.14  & 0.15  & 0.00  & -0.34  & 2.87  & 1.94  & 2.06  & 0.01  & -3.02   \\
  %   3  & 0.12  & 0.06  & 0.09  & -0.03  & -0.14  & 1.69  & 0.84  & 1.13  & -0.35  & -1.37   \\
  %   4  & 0.16  & 0.08  & -0.02  & 0.01  & -0.06  & 1.95  & 0.98  & -0.29  & 0.13  & -0.58   \\
  %   Big  & 0.03  & -0.02  & -0.05  & -0.10  & 0.02  & 0.49  & -0.31  & -0.87  & -1.56  & 0.23   \\
  
  \bottomrule
\end{tabular}
}
  \end{center}
\end{frame}

\note[itemize]{
  \item ``lethal" combo $\rightarrow$ unprofitable stocks with poor recent returns\\
  AND value disappears
  \item Size slopes pull ``predicted" returns in the opposite direction to other factors
  \item Small, high-variance bucket a big problem, -0.75\% per month alpha
}

\begin{frame} \frametitle{Momentum and variance sorts}
  \begin{itemize}
    \item Momentum factor helps, but not much, in sorts on momentum
    \item Momentum returns $\neq$ momentum characteristics
    \item Profit slopes are negative in the extremes of momentum\\
    Defensive equity $\rightarrow$ beta and volatility \parencite{novy2014understanding}
    \item Sorts on beta pose the least problems\\
    So volatility a problem, but not the part of volatility to do with aggregate volatility (covariance with the market)
    \item Sorts on variance are a big problem\\
    Small, high-variance alpha of -0.75\% per month
  \end{itemize}
\end{frame}

\begin{frame} \frametitle{Volatility problems} \framesubtitle{Prior Research: Liquidity}
  \begin{itemize}
    \item To take advantage of this alpha we need to sell small, unprofitable stocks with poor recent returns
    \item That's a tough sell $\rightarrow$ Liquidity
    \parencite{nagel2005short, nagel2012evaporating}
    \item Institutional ownership, Short-term reversal factor (+VIX) as a proxy for market-maker's returns $\rightarrow$ market microstructure
    \item Liquidity hard to approach from a cross-section perspective
    \item \textcite{ang2006cross} find no liquidity measure helps describe total variance in the cross-section
  \end{itemize}
\end{frame}

\begin{frame} \frametitle{Volatility problems} \framesubtitle{Model Description: Size}
  \begin{itemize}
    % \item Cross-section perspective
    % \item High-volatility firms are much smaller
    \item Size slopes pull fitted return up
    \item Value disappears in the problem sorts (momentum and volatility)
    % \item Normally, value and profitability do most of the heavy lifting
    \item \textcite{gerakos2017decomposing} say that much of the value premium is to do with changes in firm size\\
    Further, the component of value associated with size is more correlated with variance than the component of value not to do with size
    \item Removing the size factor drives the alpha to 0\\
    Profitability slope $\sim$ -1.5 and negative value slope, similar to high-beta sorts that are not a problem
  \end{itemize}
\end{frame}

\section{Conclusions}

\begin{frame} \frametitle{Conclusions}
  \begin{itemize}
    \item I introduce mkt, size, value, momentum and profitability factor model and it has the highest $\text{Sh}^2$, 0.316
    \item Value and momentum subsume popular investment factor through future changes in BE
    \item New description for the ``lethal combination" of small, unprofitable stocks that somehow invest aggressively $\rightarrow$ small, unprofitable firms with poor recent returns
    \item New direction for mispricing in high-volatility portfolios beyond liquidity constraints $\rightarrow$ size and value story
  \end{itemize}
\end{frame}

\section{References}

\begin{frame}[allowframebreaks]
  \frametitle{References}
  \printbibliography
\end{frame}


\end{document}

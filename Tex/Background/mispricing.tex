% !TeX root=../main.tex

\subsection{Measures of Mis-Pricing}

I use three measures of mis-pricing; the maximum squared Sharpe ratio, the GRS statistic
and individual regression alphas. The maximum Sharpe ratio and GRS rely on the quadratic
form of the alphas which emphasizes the importance of covariance of alphas. Individual
regression slopes focus on the magnitude of alphas and help to identify which specific
sorts cause problems leading to a high GRS statistic.

The level of mis-pricing for a set of test assets is given by the quadratic form in the
alphas. This measure gives the amount the squared Sharpe ratio can be improved by
investing in the test assets as well as the factors. This relationship is given by
$a'V_ea=Sh^2\left(R,f\right)-Sh^2\left(f\right)$ where $a$ is the vector of alphas, $V_e$
is the covariance matrix of the residuals, $Sh^2\left(\cdot\right)$ is the maximum squared
Sharpe ratio, $R$ is the matrix of test asset excess returns and $f$ is the matrix of
factor returns. The key insight in \textcite{barillas2016alpha} is that
$Sh^2\left(R,f\right)=Sh^2\left(R\right)$ if $R$ is the returns on all assets. The idea
that test assets are not always relevant to make statements about the quality of an asset-
pricing model is not new, see Fama (1998), but Barillas and Shanken note that we want to
price the returns on \emph{all} assets. This point means we can write the level of mis-
pricing for all assets as $a'V_ea=Sh^2\left(R\right)-Sh^2\left(f\right)$ because our
factors, $f$, are contained in the set of all assets, $R$. We do not need to identify all
the components of $R$ to minimize the mis-pricing given by the quadratic form of the
alphas. We only need to maximize the squared Sharpe ratio of the factors.

The quadratic form in the alphas is a key part of the \textcite{gibbons1989test} (GRS)
statistic of whether a portfolio is mean-variance efficient, i.e. it attains the maximum
Sharpe ratio. The GRS is an F-test that all the alphas from the test assets are jointly
zero. We can use the GRS statistic to rank the ability of different models to price
subsets of assets because the GRS is based on the quadratic form of the alphas. I use the
GRS statistic to highlight where certain factors help to explain the returns on certain
test assets. Comparing the GRS, rather than the average absolute intercept or similar
measures, captures a factor’s ability to explain the variation in returns and not just
average return. GRS evidence does not always align with factor contribution to the overall
Sharpe ratio of a model. I use the divergence of Sharpe ratio and GRS evidence to
investigate where otherwise important factors disappear.

I do not explicitly use the \textcite{hansen1997assessing} (HJ) distance. The HJ distance
gives a measure of model misspecification that indicates how close our proposed stochastic
discount factor (SDF) is to a valid SDF. Given a risk-free asset, \textcite{kan2008model}
show that we can modify the HJ distance to require all SDFs to assign the same price to
the risk-free asset. Their modified HJ distance compares the pricing-errors for excess
returns. With traded factors, the modified HJ distance is the improvement in the squared
Sharpe ratio of the factors given by the quadratic form of the alphas. Maximizing the
Sharpe ratio of factors is the same as minimizing the modified HJ distance for all assets.

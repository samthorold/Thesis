% !TeX root=../main.tex

\subsection{Value, Momentum and Profitability}

I briefly review the literature for value, momentum and profitability as well as discuss
their interaction. A model can capture a great deal of variation across returns if factors
explain returns but are not correlated with each other. Value and momentum are positively
correlated with test asset returns but negatively correlated with each other, while
profitability is positively correlated with test asset returns but has little to do with
value or momentum. This is a strong statistical combination. Moreover, it is based on the
simple intuition of \textcite{graham1934security} - buy cheap, profitable stocks.

Capturing the value premium involves buying underpriced and selling overpriced stocks.
Initial research found value was significant for small and large stocks
\parencite{fama1992cross}. This research culminated in the three-factor model of
\textcite{fama1993common}. In more recent out-of-sample tests, the value premium is
insignificant in large stocks \parencite{asness2015fact}. Risk stories for the value
premium include compensation for the risk of prolonged poor runs, such as before the tech
bubble or during the financial crisis, and compensation for financial distress.
Alternatively, the value premium may be explained by behavioural stories as investors flee
value stocks for more attractive growth stocks. Irrational selling of value and buying of
growth stocks drives the difference in returns \parencite{bondt1985does}.

Stocks that have recently outperformed their peers tend to continue to outperform in the
short term \parencite{jegadeesh1993returns}. This phenomenon has been labeled ``momentum"
and is found in stocks throughout the world \parencite{asness2013value}. Despite strong
historical returns, the momentum factor catches a lot of heat. In a myth-busting survey of
the empirical evidence for momentum, \textcite{asness2014fact} describe some of the most
common falsehoods leveled at the momentum factor. They explain; momentum returns are not
driven solely by the short side, the strength of the factor is equal for small, and large
stocks, and trading costs do not subsume returns. From a risk story perspective, high-
momentum stocks have greater cash-flow risks due to growth prospects. Similarly, high-
momentum stocks could face a higher discount rate due to their investment prospects.
\textcite{moskowitz1999industries} show that industry momentum accounts for much of the
momentum exhibited by individual stocks. Momentum may represent the risk of not
diversifying. Behavioural stories focus on investor underreaction or delayed over-
reaction. Investors may underreact initially to high relative returns due to slow news.
Conversely, investors may have a delayed overreaction as they chase after stocks with high
relative returns.

Stocks with comparatiely high profits have higher returns than stocks with comparatively
lower profits \parencite{fama2006profitability}. The specification of profitability
matters as some income statement captions contain information beyond the pure
profitability of a firm. For example, interest payments contain information about a firm’s
capital structure. This information is important and reflects management’s competence, but
it does not relate to the true profitability of the firm. \textcite{novy2013other}
documents the relationship between value and profitability. Novy-Marx argues that gross
profits is a better measure of true firm profitability than operating profits because
gross profits omits income statement captions such as research and development and
selling, general and administrative costs. \textcite{ball2016accruals} propose removing
the research and development costs as well as accruals from operating profits to give a
measure of cash profitability. Their cash profit factor explains the accrual anomaly.
\textcite{novy2014understanding} demonstrates the important relationship between
volatility and profitability. Those stocks with higher profits tend to be less volatile.
This relationship goes a long way to explaining the unexplained returns to low-volatility
stocks in sorts on size and volatility. Value and momentum are negatively correlated
\parencite{asness1997interaction}. This relationship makes sense because, all else the
same, if an investor observed a positive return they’d expect the book-to-market ratio to
have decreased. \textcite{asness2013devil} suggest altering the specification of value to
take advantage of market price updating more frequently than book equity. Their timelier
measure of value combined with momentum does enough to revive ``disappearing" value in
\textcite{fama2015five}. Furthermore, the relationship between value and momentum is
strong among large stocks \parencite{asness2015fact}. Book-to-market is a noisy proxy for
value. Momentum and profitability are ways of cleaning up our measure of value.
\textcite{kok2017facts} show that high-momentum stocks help to weed out mean-reverting
high-value stocks that will exhibit decreases in book equity instead of the more desirable
increase in price. Profitability offers a measure of quality. We can identify not only
cheap stocks but ``good" cheap stocks. Finding profitable cheap stocks has long been
advocated when seeking individual stocks to buy and hold \parencite{graham1934security}.
This strategy can be applied in a systematic, diversified fashion.

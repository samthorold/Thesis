% !TeX root=../main.tex

\subsection{A Factor Model Spotter's Field Guide}

Factors, except for the market, are long/short spread value-weighted returns from sorts on
size and one other characteristic. The spread, or premium, is created by buying stocks at
the high-return end of the spread and selling stocks at the low-return end of the spread.
By subtracting one gross return from another, we create an excess return as we remove the
risk-free rate implicit in each gross return. Most premiums are weaker for large stocks,
but factors are constructed by averaging the small and large stock spreads. This gives
equal weight to small and large stock spreads and potentially over-weights small stocks.
Factors formed in this way are also referred to as ``traded" factors. We could
theoretically construct these factors with stocks available in the market. We must pay
attention to the plausibility of the factors we create, particularly regarding the short
side of the spread. There are often more constraints on short selling than buying stocks.
If we cannot form factors in real-life we cannot rely on the measures of mis-pricing I
outline below.

When all factors in a model are excess returns, the time series regression slopes of
factors have intuitive meaning and the intercept is called an ``alpha". A positive slope
means test returns behave like the long end of the spread. A negative slope means test
returns behave like the short end of the spread. We must be careful to say test returns
\emph{behave} like the long or short end of factor spreads as characteristics do not
always match regression slopes.

All factor models are extensions of the CAPM. They are linear combinations of the returns
on mimicking portfolios. The oldest factor model that includes more than the market
portfolio is the three-factor model of \textcite{fama1993common}. Based on size and value
anomalies, Fama and French propose a model consisting of the market, size and value
factors. The three-factor model has time series regression form
\[
R_t^i=a^i+b^iR_t^M+s^iSMB_t+h^iHML_t
\]
where $R_t^i$ is the excess return on asset $i$ at time $t$, a is the alpha, $R^M$ is the
excess return on the market, $SMB$ is the return on a portfolio long small stocks and
short big stocks mimicking the size premium and $HML$ is a portfolio long value stocks and
short growth stocks mimicking the value premium. Lower-case letters are regression
coefficients. The four-factor model of \textcite{carhart1997persistence} was not far
behind the three-factor model. Carhart adds a momentum factor to the three-factor model
with time series regression form
\[
R_t^i=a^i+b^iR_t^M+s^iSMB_t+h^iHML_t+w^iWML_t
\]
where $WML$ is a portfolio long past winners and short past losers mimicking the momentum
premium. Not to be outdone, \textcite{fama2015five} propose a five-factor model with the
time series regression form
\[
R_t^i=a^i+b^iR_t^M+s^iSMB_t+h^iHML_t+p^iPMU_t^{06}+c^iCMA_t
\]
where $PMU^{06}$ is a portfolio long profitable stocks and short unprofitable stocks
mimicking the profitability premium and $CMA$ is a portfolio long conservatively investing
stocks and short aggressively investing stocks mimicking the investment premium. The five-
factor model is motivated by the dividend discount model,
$ME_t=\sum_{s=1}^{\infty}{E\left(Y_{t+s}-\Delta BE_{t+s}\right)/R^s}$ where $ME_t$ is the
market value of equity at time $t$, $Y$ is profitability, $\Delta BE$ is the change in
book equity and $R$ is the gross rate of return for the stock. Fama and French use the
growth in assets in place of growth of book equity to proxy for investment. The dividend
discount model shows that, all else the same, increasing profits or decreasing investment
increases the expected return. Fama and French, to my knowledge, have never explicitly
justified their unwillingness to embrace momentum\footnote{I was unable to find a new
paper from Fama and French, cited in \textcite{barillas2017model}, that apparently adds
momentum to their otherwise unaltered five-factor model. Momentum has been known and
researched for decades but only now have Fama and French added it to their model. This is
not to bash Fama and French, but it is curious that they have largely ignored such a
prominent anomaly for so long. While momentum is difficult to fit into a risk story, their
other factors are nakedly based on empirical evidence with risk stories attached later.
Furthermore, to not alter the value factor takes the stance that the interaction between
value and momentum is unimportant.}.

One notable omission is the q-factor model of \textcite{hou2015digesting}. The intuition
for their model comes from investment-based asset pricing. The q-factor model relies on
quarterly data and uses 2x3x3 sorts as opposed to the 2x3 sorts of most other factors.
\textcite{barillas2017model} show that the five-factor model of \textcite{fama2015five}
that updates value with more timely market equity and adds the momentum factor has a
higher Sharpe ratio than the q-factor model that adds momentum.

I alter the specification of \textcite{fama2015five} by updating value every month to take
advantage of the new information in market equity and replacing the investment factor with
momentum. My model has the time series regression form
\[
R_t^i=a^i+b^iR_t^M+s^iSMB_t+h^iHML_t^m+w^iWML_t+p^iPMU_t
\]
I focus on the differences between models rather than a single model’s ability to price a
few subsets of assets. I compare non-nested models to see which factors pick up the slack
when other factors are omitted. Value does much of the heavy lifting, so my model is
vulnerable to sorts which constrain value. I confirm the GRS statistic evidence broadly
matches the maximum squared Sharpe ratio evidence and investigate where the two disagree.
Where value is constrained, the GRS statistic is often larger for my model than that of
Fama and French. I show that the improvement in the Sharpe ratio may be down to value and
momentum acting as a better proxy for changes in book equity than changes in total assets.

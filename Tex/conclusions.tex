% !TeX root=./main.tex

\section*{Conclusions}

Barillas and Shanken’s observation that maximizing the Sharpe ratio of an
asset-pricing model minimizes mispricing for all assets enables a top-down
approach to choosing factors.
Models are tested before turning to subsets of assets which promotes intuition
over statistics.
The best model then provides a description of problem subsets of assets,
rather than subsets of assets driving the choice of factors.
A five-factor model of market, size, value, momentum and profitability factors
has a maximum squared Sharpe ratio of 0.316.
This is not improved by including the investment factor.
Investment is subsumed by vale and momentum because it is an inferior proxy
for future changes in book equity.
GRS evidence for subsets of portfolios confirms Sharpe ratio evidence and
highlights the problems with momentum and volatility sorts.
My proposed model shows the previous ``lethal combination" of Fama and French,
small unprofitable firms that somehow invest aggressively,
are described by small, unprofitable firms with poor recent returns.
Value’s disappearance for these problem sorts coincides with volatility
problems and prior results that liquidity measures do not help price returns
in the cross-section.
The component of value associated with changes in form size is a more
promising avenue for future research than liquidity.

% !TeX root=./main.tex

\section*{Conclusions}

Comparing the maximum squared Sharpe ratio of the factors for asset-pricing
models means we can use a top-down approach to choosing models.
A top-down approach avoids data-mining common in a bottom-up approach and
promotes economic intuition over statistics.
I motivate a five-factor model of market, size, value, momentum and
profitability factors with buying cheap, profitable stocks.
Portfolios constructed from cheap and profitable stocks proxy for
unobserved state-variables priced by investors.
Value, momentum and profitability drive the maximum squared Sharpe ratio of
0.316.
Value and momentum, through their ability to forecast changes in book equity,
make the investment factor redundant because it does not add
to the maximum Sharpe ratio of the other five factors.

GRS evidence corresponds to Sharpe evidence and highlights the problems with
momentum and volatility.
Individual regression slopes show problem sorts share the ``lethal combination"
of small, unprofitable stocks that somehow invest aggressively introduced by
Fama and French.
These sorts behave like small, unprofitable stocks with poor recent returns.
The irrelevance of value here is puzzling as value slopes typically
compensate for the omitted investment factor.
The problems with momentum are linked to the problems with volatility as shown
by the beta and profitability slopes.
Firm size, rather than illiquidity, drives the problems with sorts on volatility as size
slopes inflate the predicted return given by the model.
Future research could investigate the relationship between the size-driven
component of value and variance.

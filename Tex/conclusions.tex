% !TeX root=./main.tex

\section{Omissions}

\begin{itemize}
  \item Bayesian ``learning" framework \textcite{nagel2013empirical} -- to what extent is
    learning captured in the way risk exposures vary through time which is itself captured
    by the varying returns on the factors.
  \item Out-of-sample tests
  \item Bootstrap
  \item Equation \ref{eq:ddm} is just a structural relationship relying on clean surplus
    accounting. It says nothing on the risk vs. behavioural story for returns.
  \item Returns on investment factor more to do with low-investment allowing payouts to
    shareholders than risk from high investment leading to external financing
    \parencite{fama2008dissecting, daniel2006market, pontiff2008share}.


\section{Conclusions}

Nice surveys of anomalies; \textcite{nagel2013empirical}, \textcite{fama2016dissecting}.

A five-factor model of market, size, value, momentum and profitability factors, B2016, has
a maximum Sharpe ratio of 0.6. This is made possible by the interaction between value and
momentum used in conjunction with cash profitability. Value and momentum are negatively
correlated with each other but positively correlated with other portfolios leading to
slope enhancement in regressions. The economics behind their relationship may be to do
with identifying those stocks that will exhibit changes in BE in the future. Value and
momentum make investment reduntant as a factor because it does not add anything to the max
$Sh^2$ of the other five factors. Investment's lack of contribution is driven by explained
average return rather than unexplained variation. In anomaly sorts, the GRS statistic does
not provide as strong evidence in favour of the B2016 model as the max $Sh^2$, although
they are in agreement that B2016 is the best model. Given our desire to price all assets
and the sensitivity of the GRS to the choice of test assets, it is satisfactory that
evidence was not as strong but still supportive of the B2016 model. Individual regression
slopes show that when controlling for value, the value factor is not able to vary across
investment portfolios and therefore the model does not perform as well in these sorts.
Portfolios where value and momentum pull in opposite directions are a problem because
these sorts negate the interaction between value and momentum. These problem sorts share
the ``lethal combination" of returns -- small, unprofitable stocks that somehow invest
aggressively. Like the GRS evidence, I treat the failings of the model in individual
regressions as interesting and significant but not ultimately damning. The max $Sh^2$
provides a clear winner, B2016. The test asset evidence shows we have a lot left to
explain as to why the B2016 model deos not perform as strongly in anomaly sorts.

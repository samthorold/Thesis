% !TeX root=./main.tex

\section{Conclusions}

I motivate a five-factor model of market, size, value, momentum and
profitability factors, B16, with buying cheap, profitable stocks. The intuition
of pricing all assets means we should focus on the interaction between factors.
Value and momentum interact strongly. At the same time, they have little
interaction with profitability. Value, momentum and profitability drive the
maximum squared Sharpe ratio of 0.316. Statistically, value and momentum are
negatively correlated with each other but positively correlated with other
portfolios leading to slope enhancement in regressions. The economics behind
their relationship may be to do with identifying those stocks that will exhibit
changes in book equity in the future. Value and momentum make investment
redundant as a factor because it does not add anything to the maximum Sharpe
ratio of the other five factors. Investment's lack of contribution is driven by
explained average return rather than unexplained variation in returns through
time.

In anomaly sorts, the GRS statistic shows the $\text{Sh}^2$ of B16 can be
improved the least compared to a selection of other models by allowing
ivnestment in a number of common anomaly sorts. Individual regression slopes
show that when controlling for annual value is a problem because the value
factor is not able to vary across investment portfolios. Problems also appear
in sorts on momentum and variation. These problem sorts share the ``lethal
combination” of returns -- small, unprofitable stocks that somehow invest
aggressively popularized by Fama and French.
Updating value monthly and replacing investment with momentum shows these
problem sorts behave like small, unprofitable stocks with poor recent returns.
The irrelevance of value here is puzzling as value slopes typically
compensate for the omitted investment factor.

[Omissions include taking a stance on risk vs. behavioural stories, factor
timing and volatility managed portfolios. We can only say factors describe
returns rather than explain them if we do not take a stance on risk vs.
behavioural stories for factors. Arguments are largely statistical without
putting forward the drivers behind factor returns. I suggest future changes in
book equity drive the interaction of value and momentum subsuming investment,
but this does not clear up why value and momentum generate returns in the first
place. Using many factors can be motivated by diversification to protect
against prolonged poor runs for individual factors. In this case an investor
gives up some of the upside of a factor to hedge another. From an investment
perspective, it is preferable to ``time” factors as we might time the market
instead of relying on diversifying across factors. Dropping factors will lower
the maximum Sharpe ratio of the model. Furthermore, timing factors somewhat
misses the point in an asset-pricing context. Assets will have different
exposures to different factors and to drop factors removes the ability to
capture those exposures. Volatility is the biggest asset-pricing anomaly. We
can use volatility when we weight stocks instead of relying on market equity.
This approach avoids the problem and does not explain the drivers behind the
returns to volatility.]

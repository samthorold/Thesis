% !TeX root=./main.tex

\section*{Conclusions}

Comparing the maximum squared Sharpe ratio of the factors in asset-pricing
models means we can use a top-down approach to choosing models.
A top-down approach has the advantage of avoiding some of the data-mining
problems that a bottom-up approach focusing on anomalies are prone to.
I motivate a five-factor model of market, size, value, momentum and
profitability factors with buying cheap, profitable stocks.
Portfolios constructed from cheap and profitable stocks proxy for the
unobserved state-variables priced by investors.
Value, momentum and profitability drive the maximum squared Sharpe ratio of
0.316.
Value and momentum make the investment factor redundant because it does not add
to the maximum Sharpe ratio of the other five factors.
Value and momentum provide a better proxy for future changes in book equity
than value and investment.

GRS evidence corresponds to Sharpe evidence and highlights the problems with
momentum and volatility.
Individual regression slopes show problem sorts share the ``lethal combination"
of small, unprofitable stocks that somehow invest aggressively introduced by
Fama and French.
I show these sorts behave like small, unprofitable stocks with poor recent
returns.
The irrelevance of value here is puzzling as value slopes typically
compensate for the omitted investment factor.
The problems with momentum are linked to the problems with volatility as shown
by the beta and profitability slopes.

[Omissions include taking a stance on risk vs. behavioural stories, factor
timing and volatility managed portfolios. We can only say factors describe
returns rather than explain them if we do not take a stance on risk vs.
behavioural stories for factors. Arguments are largely statistical without
putting forward the drivers behind factor returns. I suggest future changes in
book equity drive the interaction of value and momentum subsuming investment,
but this does not clear up why value and momentum generate returns in the first
place. Using many factors can be motivated by diversification to protect
against prolonged poor runs for individual factors. In this case an investor
gives up some of the upside of a factor to hedge another. From an investment
perspective, it is preferable to ``time” factors as we might time the market
instead of relying on diversifying across factors. Dropping factors will lower
the maximum Sharpe ratio of the model. Furthermore, timing factors somewhat
misses the point in an asset-pricing context. Assets will have different
exposures to different factors and to drop factors removes the ability to
capture those exposures. Volatility is the biggest asset-pricing anomaly. We
can use volatility when we weight stocks instead of relying on market equity.
This approach avoids the problem and does not explain the drivers behind the
returns to volatility.]

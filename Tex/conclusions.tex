% !TeX root=./main.tex

\section*{Conclusions}

Barillas and Shanken’s observation that maximizing the Sharpe ratio of an
asset-pricing model minimizes mispricing for all portfolios enables a top-down
approach to choosing factors.
Comparing the Sharpe ratio of models before testing mispricing for subsets of
portfolios promotes intuition over statistics and helps fight against data-mining.
The model with the least mispricing for all portfolios provides a description
of problem portfolios, rather than problem portfolios driving the choice of factors.

A five-factor model of market, size, value, momentum and profitability factors
has a maximum squared Sharpe ratio of 0.316.
This is not improved by including the investment factor.
Investment is subsumed by value and momentum because it is an inferior proxy
for future changes in book equity.
GRS evidence for subsets of portfolios confirms Sharpe ratio evidence and
highlights the problems with momentum and volatility sorts.
My proposed model shows the previous ``lethal combination" of unprofitable
firms that somehow invest aggressively is described by unprofitable firms with
poor recent returns.
Value’s disappearance for problem sorts coincides with volatility problems.
Size factor slopes, compounded by the size-driven component of value's
relationship with volatility, drive mispricing for volatile stocks
and point to problems with firm size beyond illiquidity.
My findings promote further research into the relationship between value and
volatility.

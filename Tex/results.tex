% !TeX root=./main.tex

\section{Results} \label{sec:results}

\subsection{Sharpe Ratio}

\import{./Tables/}{Table1}

Table \ref{tbl:Table1} panel A shows summary statistics for individual factors as well as the 
BS2015 and FF2016b models.
The market has a high average monthly return, 0.51\%, offset by a high standard deviation, 
4.42, giving a modest Sharpe ratio of 0.12.
Size, value and investment factors had similar average monthly returns and standard deviations 
giving Sharpe ratios from 0.09 to 0.15.
Momentum had the highest average monthly return at 0.66\%.
This does not translate to a significantly higher Sharpe ratio, 0.16, as the standard 
deviation is also high at 4.23.
Cash profitability has the highest Sharpe ratio of the individual factors at 0.27.
This is driven by the low standard deviation of 1.4 rather than a high average return, which 
is modest at 0.37\%.
The models do not improve the average monthly return, 0.42\% for BS2015 and 0.37\% for 
FF2016b, but do lower the standard deviation of returns to 0.69 and 0.75.
This results in significantly improved Sharpe ratios of 0.6 and 0.5.
In the face of unimpressive monthly returns, we should keep in mind that models most 
appropriate for pricing may not be the most appropriate for investing 
\parencite{pastor2000comparing}.
We are concerned with the way our models co-vary with all assets, not whether they 
generate superior returns.

Table \ref{tbl:Table1} panel B shows the intercepts, standard deviation of the residuals, 
contribution to the maximum squared Sharpe ratio and the weight in the models for each factor.
For BS2015, value and profit factors contribute similar amounts, 0.22 and 0.2, to the Sharpe 
ratio.
This is in-line with our intuition that fairly-priced, profitable stocks will generate higher 
returns those relatively more expensive or less profitable.
Value's large contribution is driven primarily by a large intercept, 1.11, while 
profitability's contribution is driven by a low standard deviation of residuals, 1.16.
The market and momentum also have large intercepts, 1.32 and  1.11, but the higher standard 
deviation of residuals, 3.79 and 3.11, inhibit the factors' contributions to the BS2015 model, 
0.11 and 0.13.
Size contributes very little to the Sharpe ratio, 0.04.
Despite value and profitability's similar contributions to the Sharpe ratio of the BS2015 
model, the profit factor has a much higher weight than value, 0.45 compared to 0.23, in the 
portfolio which generates the maximum Sharpe ratio.
Other factor weights are in-line with contributions to Sharpe ratio.
For FF2016b, the market intercept is 1.09 but the high standard deviation of residuals, 3.76, 
drives a low contribution of 0.08.
Despite investment's low standard deviation of residuals, 1.38, its low intercept, 0.14, 
drives a negligible contribution of 0.01.
Size and value contribute similar amounts, 0.03 and 0.04, thanks to modest intercepts, 0.47 
and 0.36, coupled with modest standard deviations of residuals of 2.83 and 1.9.
Profitability contributes the most at 0.19 driven by a low standard deviation of residuals at 
1.19.
The profitability factor has the largest weight, 0.54, in the portfolio which maximizes the 
Sharpe ratio of FF2016b.
Remaining factors are all weighted between 0.09 and 0.15.

Factor contributions given the other four factors need not sum to the total model squared 
Sharpe ratio because $(a/\sigma)^2$ is the amount the squared Sharpe ratio of the four other 
factors is improved when the fifth is included.
A factor's contribution depends on the amount of its mean return, $a$, and variation, 
$\sigma$, that is left unexplained by the other factors.
Unexplained mean return is the traditional alpha from a time series regression while the 
unexplained variation is the standard deviation of the residuals from the same regression.
The unexplained mean return represents the potential to add information to the other factors 
in a model while the unexplained variation dampens the effect of unexplained factor return on 
the model's Sharpe ratio \parencite{fama2016choosing}.






\subsection{Value and Momentum}

Value-weighted characteristics in sorts.
Average characteristics in size sorts.

\subsection{Anomaly Time-Series Regressions}

\import{./Tables/}{25_Size_Inv_196307_201612}
We have established that the BS2015 model has a higher Sharpe ratio than the FF2016b model but 
it is informative to see if we lose any performance in sorts on the excluded investment factor.
Table \ref{tbl:25_Size_Inv} shows the performance of the FF2016b and BS2015 models when 
explaining returns on size and investment sorts.
BS2015 leaves fewer alphas than FF2016b\footnote{BS2015 also leaves fewer alphas than the 
FF2015 and FF2016a models (see table \ref{tbl:25_Size_Inv_b} in appendix 
\ref{sec:addl_results}).}.
FF2016b struggles with the third investment quintile leaving alphas in the first and second 
size quintiles, both 0.14.
BS2015 struggles where the highest investment quintile meets the fourth highest size quintile, 
leaving an alpha of 0.18.
Both models leave an alpha the small-high investment portfolio.
The values are -0.28 for FF2016b and -0.26 for BS2015.

Table \ref{tbl:25_Size_Inv} also shows the regression slopes on the value and investment or 
value and momentum factors.
For the FF2016b model, the investment slopes, $cma$, decrease as the investment quintiles 
increase.
We expect the returns on portfolios with higher investment characteristics to behave like the 
returns on the factor constructed from a more coarse sort on the same variable.
The spreads between the investment slopes increase as the size quintiles increase.
The small quintile sees the slope fall from 0.47 for the lowest investment quintile to -0.24 
for the highest.
The big size quintile has investment slopes from 0.77 for the low quintile to -0.7 for the 
highest.
Value is fairly uniform across size and investment quintiles except for the negative slopes in 
the highest investment quintile.
For the BS2015 model, the value slopes are uniform for the smallest three investment 
quintiles, except for the small-low investment intersection which is not significant.
The slope becomes negative in the largest investment quintile.
For the larger size quintiles this includes a gradual shift through the fourth largest 
investment quintile.
This suggests that returns on firms with high investment behave like those with a low 
book-to-market ratio, growth firms.
Despite momentum's small slopes, the factor is highly significant is almost all portfolios.
Mirroring value, slopes are uniform for all investment quintiles except for the largest, where 
slopes become negative.
Negative momentum slopes suggest the returns on firms with large investments behave like those 
with lower returns over the previous year.


% !TeX root=./main.tex

\section{Results} \label{sec:results}

\import{./Tables/}{Sh2}

Table \ref{tbl:Sh2} shows the maximum squared Sharpe ratio and the factor contributions for 
the models in equation \ref{eq:tsmodel} and the cash-profitability model in 
\textcite{fama2016choosing} (FF2016).
For both models, $Prof$ refers to the cash profitability factor of \textcite{ball2016accruals}.
For (\ref{eq:tsmodel}), $Val$ refers to the monthly, current value of 
\textcite{asness2013devil}.
$Prof$ and monthly, current $Val$ are created manually.
All other factors are as specified in \textcite{fama2016dissecting} and straight from Ken 
French's 
website.
The maximum Sharpe ratio is calculated as $Sh^2(f)=R'V^{-1}R$ where $R$ is the vector of mean 
returns on the factors and $V$ is the covariance matrix for the returns on the factors.
Contributions to the total Sharpe ratio are calculated as $(\alpha_i / \sigma_i)^2$ where 
$\alpha_i$ is the intercept from regressing the factor $i$ on the remaining factors and 
$\sigma_i$ is the standard deviation of the residuals from the same regression.
The Sharpe ratio of (\ref{eq:tsmodel}), 0.348, is higher than that of FF2016, 0.246.
The investment factor contributes only 0.003 to \ref{eq:model} while the momentum factor 
contributes 0.014 to FF2016.

\subsection{Value and Momentum}

Value-weighted characteristics in sorts.
Average characteristics in size sorts.

\subsection{Selected Anomaly Time-Series Regressions}

Investment, Volatility. \textbf{Differences between value-mom and investment loadings}.

(\ref{eq:tsmodel}) performs worst in sorts on volatility, intercepts of 0.8, and large firms 
with high investment, intercepts of 0.6.
The model's poor performance in sorts on large stocks is not a complete surprise given value's 
weak effect among large stocks.
Where value is weak, it is driven by its relationship with momentum.
Value and momentum are negatively correlated but in some sorts move in the same direction.

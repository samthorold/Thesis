% !TeX root=./main.tex

\section{Introduction}

I propose a five-factor model of market, size, value, momentum and profitability factors.
This specification has a higher maximum squared Sharpe ratio of the factors than the
recent five-factor model of \textcite{fama2015five}, including when momentum is added to
the Fama and French specification. I show that the investment factor is made redundant by
the interaction of value and momentum. While the benefits to regression slopes of
negatively correlated factors are well known and intuitive
\parencite{asness1997interaction, fama2015incremental}, I show why economically, rather
than empirically, this may be the case for value and momentum. Value and momentum have an
opposite relationship with future changes in book equity than value and investment.

An asset’s price equals its expected risk adjusted future payoff. The risk adjustment is
driven by the covariance of asset payoffs with marginal utility. Under the Capital Asset
Pricing Model (CAPM) of \textcite{sharpe1964capital}, \textcite{lintner1965valuation} and
\textcite{black1972capital}, an investor would rather have an asset that pays well when
marginal utility is high, when times are tough, than an asset that pays well when marginal
utility is low, when times are good. Assets that do badly when times are tough have low
prices (high returns). Times are tough when an investor’s other assets have done poorly,
such as when the market performs poorly. Prices are low (returns are high) for assets that
covary strongly with the market portfolio relative to assets that covary weakly with the
market portfolio. Return is compensation for risk where risk is measured as covariance
with the market.

The intertemporal CAPM (ICAPM) of \textcite{merton1973intertemporal} says that investors are
not just concerned with the level of payoff in the future but also what consumption and
investment opportunities are available to them when they receive the payoff.
\textcite{fama1996multifactor} explains that ICAPM investors worry about the covariances
of returns with not only the market but with multiple state variables. The market
portfolio still plays an important role, but investors price additional state variables.

Asset-pricing models that focus on stock returns use traded portfolios to mimic the
returns on the unobserved priced state variables of the ICAPM. The factors are risk
premiums. The excess return on the market is the market risk premium and the Sharpe ratio
of the market gives the market price of risk. Portfolios designed to capture additional
priced state variables of interest to ICAPM investors are typically long/short spread
factors, but the intuition is the same as the market risk premium.

Currently, the choice of factors is empirically motivated. A time series of returns not
explained by a set of factors, an ``anomaly”, is discovered and a couple of years later it
becomes a factor. Factors may be subsumed by other factors, but the list just keeps
increasing. The explosion of factors has been so great that \textcite{harvey2016and} call
for an increase in the usual confidence intervals used to class a new factor “discovered”.
We are throwing darts at anomalies and, by focusing on driving individual regression
intercepts to zero, not even throwing at the right board.

The ultimate goal of an asset-pricing model is to minimize mis-pricing for \emph{all}
assets. We are concerned with the covariance of pricing errors (alphas) as well as their
magnitude since the level of mis-pricing is given by the quadratic form of the alphas.
\textcite{barillas2016alpha} explain that to minimize mis-pricing for all assets, we need
only maximize the Sharpe ratio of the factors. If we care about the covariance of alphas
we must care equally about the covariance of factors and not just the magnitude of the
average return left unexplained.

Reducing the universe of assets to common stocks, we can use time-worn investing
intuition, rather than throwing darts at empirical anomalies, to choose variables that are
priced by investors. To make mimicking factors for priced state variables we apply
investing intuition in a systemic way. \textcite{graham1934security} advocate buying
cheap, profitable stocks. I measure “cheapness” or “value” as a stock’s book-to-market
equity (B/M). A high B/M indicates a cheap stock, a “value” stock, while a low B/M
indicates an expensive stock, a “growth” stock. B/M is a noisy measure of value that can
be cleaned-up with momentum \parencite{kok2017facts, asness2013devil}. Momentum captures
recent returns, typically over the previous year omitting the most recent month. I measure
profitability as the accrual-adjusted operating income \parencite{ball2016accruals}. The
returns on portfolios based on value, momentum and profitability characteristics mimic
state variables priced by investors.

While the maximum Sharpe ratio of the factors indicates the best overall model, there may
be sorts creating subsets of assets for which the best overall model performs poorly. The
GRS statistic of \textcite{gibbons1989test} indicates whether a portfolio is mean-variance
efficient in the context of a set of test assets. The null hypothesis of the test is that
the Sharpe ratio of a set of factors cannot be improved by investing in the test assets. I
compare the GRS statistics of different specifications of model across different test
assets to show where a given factor helps or hinders when explaining the absolute size and
covariance of alphas. The GRS evidence does not always match the maximum Sharpe ratio
evidence.

I use individual portfolio regressions for problem sorts identified by the GRS to
investigate the factor loadings that lead to problems. I compare non-nested specifications
of model to show where value and momentum slopes differ from the subsumed investment
slope. Regression slopes do not always match characteristic sorts. The model performs
poorly where value is constrained or where value and momentum cannot utilize their
interaction. The offending sorts share the ``lethal” combination of small, unprofitable
firms that somehow invest aggressively identified in \textcite{fama2015five}.

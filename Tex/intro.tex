% !TeX root=./main.tex

\section*{Introduction}

The goal of an asset-pricing model is to minimize mispricing for
\emph{all} assets.
We seek a few portfolios, ``factors", that fully describe all portfolios.
The problem with this goal is that we cannot directly test all assets.
This problem has led to data-mining as researchers have relied on minimizing
mispricing for subsets of assets.
Stock data is rich and a researcher can always construct some subset of stocks
for which a model performs poorly, an ``anomaly".
This subset of stocks provides the empirical basis for a new factor.
\textcite{harvey2016and} document more than 300 factors and
\textcite{linnainmaa2016history} find that many are spurious out of sample.
% Subsets of stocks that cannot be priced by a set of factors are important for
% academic and professional reasons, but statistical anomalies should not drive
% the choice of factors.

\textcite{barillas2016alpha} elegantly solve both problems,
the inability to test all assets and ensuing data-mining,
by showing that maximizing the Sharpe ratio of factors minimizes
mis-pricing for all assets.
A top-down approach of testing models' performance for all assets before
turning to performance for subsets of assets favours economic intuition over
statistical anomalies.
The model that performs best pricing all assets then provides a description
of subsets of assets that cannot be fully priced,
rather than subsets of assets driving the choice of factors used to price all
assets.

I apply investing intuition in a diversified way to make mimicking factors for
priced state variables.
\textcite{graham1934security} advocate buying cheap, profitable stocks.
I measure how cheap a stock is by the book-to-market ratio (BM).
\textcite{kok2017facts} show that BM is mean-reverting and that some of this
behaviour is driven by changes in book equity (BE) rather than market equity
(ME).
They show that updating BM monthly and using the measure alongside momentum
helps to identify those stocks that will revert to their mean value through a
change in ME.
I measure momentum as the return over the previous year omitting the most
recent month \parencite{fama2016dissecting}.
I measure profitability as the accrual-adjusted operating income scaled by BE
\parencite{ball2016accruals}.

I propose a five-factor model of market, size, value, momentum and
profitability factors with time series regression form
\begin{equation} \label{eq:B16}
R_t^i = a^i+b^iR^M+s^iSMB_t+
\underbrace{v^iHML_t^m+m^iWML_t}_\text{cheap}+
\underbrace{p^iPMU_t}_\text{profitable}
\end{equation}
where $R_t^i$ is the excess return on asset $i$ at time $t$, $a$ is the
regression intercept and unexplained return left by the factors, $R^M$ is the
excess return on the market, $SMB$ is the return on a portfolio long small
stocks and short big stocks mimicking the size premium, $HML$ is a portfolio
long value stocks and short growth stocks mimicking the value premium, $WML$ is
a portfolio long past winners and short past losers mimicking the momentum
premium and $PMU$ is a portfolio long profitable stocks and short unprofitable
stocks mimicking the profitability premium.
Lower-case letters are regression coefficients.
Value and momentum are rebalanced monthly while size and profitability are
rebalanced annually.
This specification has a higher Sharpe ratio of the factors than the recent
five-factor model of \textcite{fama2015five}, including when momentum is added
to the Fama and French specification.
The investment factor is made redundant by the interaction of value and
momentum through their ability to forecast future changes in book equity.
Value and momentum are negatively correlated and this relationship is a
statistical slam dunk in a regression context
\parencite{asness1997interaction, fama2015incremental}.
Their relationship with book equity shows that value and momentum's interaction
is more than just statistical.

While the Sharpe ratio of the factors indicates the best overall model, there
may be subsets of stocks for which the best overall model performs poorly.
The GRS statistic of \textcite{gibbons1989test} indicates how much the Sharpe
ratio of a set of factors can be improved by investing in the test assets.
GRS evidence broadly, but not always, matches Sharpe ratio evidence.
I compare the GRS statistics of different specifications of model across
different test assets to highlight the problems caused by momentum and
volatility.
I use individual portfolio regressions for problem sorts identified by the GRS
statistic to investigate the factor slopes that lead to problems.
The model performs poorly where value is constrained or disappears.
The offending sorts share the ``lethal” combination of small, unprofitable
stocks that somehow invest aggressively identified in
\textcite{fama2015five, fama2016dissecting}.
Problem sorts behave like small, unprofitable stocks with poor recent returns.


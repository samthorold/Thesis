% !TeX root=./main.tex

% \section{Introduction}

A factor-based asset-pricing model’s ability to price all assets is given by the max
squared Sharpe ratio, $Sh^2$, of the factors. \textcite{gibbons1989test} explain that
maximizing the $Sh^2$ of the factors minimizes the $Sh^2$ of the intercepts from time
series regressions of test assets on the factors. This relationship is given by
$a'V_e^{-1}a=Sh^2(R, f) -Sh^2(f)$ where $a$ is the vector of intercepts, $V_e$ is the
covariance matrix of the residuals, $R$ is the matrix of all assets’ excess returns and
$f$ is the matrix of factor returns. The $Sh^2$ of the intercepts is the amount the $Sh^2$
of the factors can be improved by including the test assets. If the $Sh^2$ of the factors
is not improved by including the test assets, the model ``prices" the test assets.
\textcite{barillas2016alpha} add that, since the factors are contained in all assets we
can reduce the relationship to $a'V_e^{-1}a = Sh^2(R)-Sh^2(f)$. Without knowing all
assets, we can compare relative model performance when pricing all assets by comparing the
$Sh^2$ of the factors alone.

Following \textcite{jensen1968performance}, model analysis has typically focused on the
intercepts from time series regressions of subsets of assets on a model’s factors.
Significant intercepts, or alphas, are ``anomalous” in the sense that the model is unable
to explain the average returns on these assets. Our desire to find models that price all
assets means we should prioritize $Sh^2$ over anomaly returns. Alphas are important from
an academic and practitioner perspective, but they should not be the main driver of factor
choices.

I propose a five-factor model with the time series regression form
\begin{equation}
\label{eq:m1}
R_t^i=a^i+b^iR_t^M+s^iSMB_t+h^iHML_t^m+w^iWML_t+p^iPMU_t
\end{equation}
where
$R_t^i$ is the excess return on asset $i$ for time $t$,
$R^M$ is the excess return on the market,
$a$ is the amount of unexplained return or ``alpha",
$SMB$ is a size factor long small and short big stocks,
$HML^m$ is a fundamental value factor long high and short low book-to-market (B/M) stocks,
$WML$ is a momentum factor long past winners and short past losers and
$PMU$ is a cash profitability factor long profitable and short unprofitable stocks.
Size and profitability are rebalanced annually while value and momentum are rebalanced monthly.

My factor choices are motivated by \textcite{graham1934security} who recommend buying
quality stocks at a cheap price. I use a cash profitability factor in the style of
\textcite{ball2016accruals} as an indicator of quality. The combination of fundamental
value and momentum provide a better indicator of truly cheap stocks as opposed to those
with inflated accounting numbers \textcite{kok2017facts}.

\textcite{fama2016choosing} propose a similar model with the time series regression form
\begin{equation}
\label{eq:m2}
R_t^i=a^i+b^iR_t^M+s^iSMB_t+h^iHML_t+p^iPMU_t+c^iCMA_t
\end{equation}
where $CMA$ is an investment factor long conservatively and short aggressively investing
stocks. All other factors are as in (\ref{eq:m1}) except for $HML$ which is rebalanced
annually. They show that replacing the operating profitability factor of
\textcite{fama2015five} with a cash profitability factor increases the $Sh^2$ of their
five-factor model. \textcite{ball2016accruals} find that an accruals factor does not
increase the $Sh^2$ of a five-factor model that contains a cash profitability factor.
Their choice of factors is the same as (\ref{eq:m1}) except they rebalance value annually.

I use the different specifications to make three points; model (\ref{eq:m1}) increases the
$Sh^2$ of (\ref{eq:m2}) by replacing investment with momentum and rebalancing value
monthly, value and momentum interact to make the investment factor unnecessary and
momentum and volatility remain conundrums in the cross-section of stock returns.

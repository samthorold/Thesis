% !TeX root=./main.tex

\section{Introduction}

Prices equal expected discounted payoffs,
\begin{equation}
\label{eq:pricing}
p=E(mx)
\end{equation}
where; $p$ is the asset price, $m$ is the discount factor and $x$ the asset 
payoff.
If we are primarily concerned with asset returns, we can rewrite equation
\ref{eq:pricing} as
\begin{equation}
\label{eq:returns}
1=E(mR^i)
\end{equation}
where $R^i$ is the gross return on asset $i$.
The discount factor and expected returns can vary across assets and with time 
but expected discounted returns always equal 1.
The discount factor, $m$, is the same for all assets.
Asset returns' covariance with $m$ creates discount rates specific to each 
asset.

Factor pricing models are special cases of the consumption-based description 
of the discount factor, $m=\beta\frac{u'(c_{t+1})}{u'(c_t)}$ where $\beta$ is 
impatience and $u'(c)$ is marginal utility of consumption.
Asset-pricing models have turned into an exercise in describing small subsets 
of assets whose returns are difficult to price.
I approach my model specification from the intuition in equation 
\ref{eq:returns} that \emph{all} discounted returns equal 1.
Subsets of assets should not be the main drivers of asset-pricing models.
A model rooted in intuition that performs well when pricing all asset returns 
can shed light on subsets of assets whose returns are difficult to price.

\cite{graham1934security} tell us to buy quality stocks at a cheap price.
A stock's book-to-market ratio tells us how much we have to pay for the book 
assets of the stock.
As we will see later, combining value and momentum gives a superior measure of 
cheapness than value alone.
The quality of a stock is measured by its profitability.
The three factor model of \cite{fama1993common} combines market, size and 
value factors.
In papers too numerable to cite here, the model has been tested, confirmed and 
denied in just about every way possible.
I add a profitability factor and replace the pure book-to-market measure of 
cheapness with the tandem factors of book-to-market and momentum.
This results in a five-factor model of; the market, size, value, profitability 
and momentum.
\begin{equation}
\label{eq:model}
m_t= w_{Mkt} \cdot Mkt + w_{Size} \cdot Size +
\left( w_{Val} \cdot Val + w_{Mtm} \cdot Mtm \right) + 
w_{Prof} \cdot Prof
\end{equation}

We want to minimize the Euler equation error when substituting 
(\ref{eq:model}) for $m$ in (\ref{eq:returns}).
This is not possible if we cannot specify ``all" assets means.
\cite{barillas2016alpha} say that minimizing the Euler error is the same as 
maximizing the Sharpe ratio of the factors since
\begin{equation}
\label{eq:Sh2}
Sh^2(\alpha) = Sh^2(R) - Sh^2(f)
\end{equation}
We can use the weights that maximize the Sharpe ratio of our factors to find a 
time series of discount rates, $[m_0, m_1, \ldots, m_T]$.

The traditional time series regression form of 
(\ref{eq:model}) is
\begin{equation}
\label{eq:tsmodel}
r_t^i - r_t^f= \alpha^i + \beta^i \cdot Mkt + s^i \cdot Size +
\left( v^i \cdot Val + m^i \cdot Mtm \right) +
p^i \cdot Prof
\end{equation}
If our factor specification for $m$ is correct, the $\alpha$ term in 
(\ref{eq:tsmodel}) should be insignificant.
This form of the model allows us to explore the behaviour of troublesome 
portfolios.
In particular, we are interested in factor loadings for those portfolios which 
have significant alpha.
Many problems in factor pricing models have negative loadings on the 
profitability and investment factors of \cite{fama2016dissecting}.
The two negative loadings have been theatrically named the ``lethal" 
combination.
Since Fama and French include an investment factor while I do 
not, it is informative to compare differences in the factor loadings for the 
two models.
 
Work most similar to mine includes that of; \cite{fama2016choosing},
... and ...
My work is different because I use (\ref{eq:pricing}) to focus on pricing 
\emph{all} asset returns from the beginning.
I increase the maximum Sharpe ratio attainable from a set of factors.
I use the different specifications of Fama and French and I to explore why the 
``lethal" loading combination creates such problems.






















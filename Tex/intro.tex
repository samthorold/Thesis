% !TeX root=./main.tex

%\section*{Introduction}

An asset’s price equals its expected risk adjusted future payoff. The risk
adjustment is driven by the covariance of asset payoffs with marginal utility.
The intertemporal CAPM (ICAPM) of \textcite{merton1973intertemporal} says that
investors are not just concerned with the level of payoff in the future but
also what consumption and investment opportunities are available to them when
they receive the payoff. \textcite{fama1996multifactor} explains that ICAPM
investors worry about the covariances of returns with not only the market but
with multiple state variables.

Asset-pricing models that focus on stock returns use portfolios, ``factors", to
mimic returns on the unobserved priced state variables of the ICAPM. The
prevalence and quality of stock data can quickly lead to data-mining. Many
factors are based largely on statistical anomalies rather than economic
intuition. \textcite{harvey2016and} find more than 300 factors in the
literature and call for an increase in the usual confidence intervals used to
class a new factor as ``discovered”. \textcite{mclean2016does} find that the
unexplained return on many factors is reduced following publication. Subsets of
stocks that cannot be priced by a set of factors are important for academic and
professional reasons, but statistical anomalies should not drive the choice of
factors.

The ultimate goal of an asset-pricing model is to minimize mis-pricing for
\emph{all} assets.
\textcite{barillas2016alpha} explain that to minimize mis-pricing for all
assets, we need only maximize the Sharpe ratio of the factors.
We can rank models using the Sharpe ratio of the factors before we begin to
test performance in subsets of stocks. This top-down approach promotes economic
intuition in favour of statistical anomalies.

To make mimicking factors for priced state variables we can apply investing
intuition in a diversified way. \textcite{graham1934security} advocate buying
cheap, profitable stocks. I measure how cheap a stock is by the book-to-market
ratio, BM. \textcite{kok2017facts, asness2013devil} show that updating B/M
monthly and using the measure alongside momentum helps to identify those stocks
that will exhibit a change in market equity, driving returns, rather than book
equity. I measure momentum as the return over the previous year omitting the
most recent month \parencite{fama2016dissecting}. I measure profitability as
the accrual-adjusted operating income \parencite{ball2016accruals}.

I propose a five-factor model of market, size, value, momentum and
profitability factors with time series regression form
\begin{equation} \label{eq:B16}
R_t^i = a^i+b^iR^M+s^iSMB_t+v^iHML_t^m+m^iWML_t+p^iPMU_t 
\end{equation}
where $R_t^i$ is the excess return on test asset $i$ at time $t$,
$a^i$ is the regression intercept and unexplained return left by the factors,
$R^M$ is the excess return on the market,
$SMB$ is the return on a portfolio long small stocks and short big stocks
mimicking the size premium,
$HML$ is a portfolio long value stocks and short growth stocks mimicking the
value premium,
$WML$ is a portfolio long past winners and short past losers mimicking the
momentum premium and
$PMU$ is a portfolio long profitable stocks and short unprofitable stocks
mimicking the profitability premium.
Lower-case letters are regression coefficients.
Value and momentum are rebalanced monthly while size and profitability are
rebalanced annually.
This specification has a higher Sharpe ratio of the factors
than the recent five-factor model of \textcite{fama2015five}, including when
momentum is added to the Fama and French specification.
I show that the investment factor is made redundant by
the interaction of value and momentum through their ability to forecast
future changes in book equity.
Value and momentum are negatively correlated and this relationship is a
statistical slam dunk in a regression context 
\parencite{asness1997interaction, fama2015incremental}.
Their relationship with book equity shows that value and
momentum's interaction is more than just statistical.

While the Sharpe ratio of the factors indicates the best overall model,
there may be subsets of stocks for which the best overall model
performs poorly.
The GRS statistic of \textcite{gibbons1989test} indicates how much the Sharpe
ratio of a set of factors can be improved by investing in the test assets.
GRS evidence broadly, but not always, matches Sharpe ratio evidence.
I compare the GRS statistics of different specifications of model across
different test assets to highlight the problems caused by momentum and
volatility.

I use individual portfolio regressions for problem sorts identified by the GRS
statistic to investigate the factor loadings that lead to problems. I compare
non-nested model specifications model to show where value and momentum slopes
differ from the subsumed investment slope. Regression slopes do not always
match characteristic sorts. The model performs poorly where value is
constrained or disappears. The offending sorts share the ``lethal”
combination of small, unprofitable stocks that somehow invest aggressively
identified in \textcite{fama2015five, fama2016dissecting}. These sorts behave
like small, unprofitable stocks with poor recent returns.

% !TeX root=./main.tex

\section{Introduction}

The pricing equation says all expected discounted returns should equal 1
\begin{equation}
\label{eq:returns}
1=E(mR^i)
\end{equation}
where $m$ is the discount rate and $R^i$ is the gross return on asset $i$.
While $m$ is the same for all assets, covariance with $R^i$ leads to different risk 
adjustments for each asset.
Factor-based models replace $m$ with a linear combination of returns on a small set of 
portfolios. These portfolios may try to capture the returns for 
taking on risk or try to mimic a non-traded factor.
Factor-based models are special cases of the consumption-based description of the discount 
factor, $m=\beta\frac{u'(c_{t+1})}{u'(c_t)}$ where $\beta$ is impatience and $u'(c)$ is 
marginal utility of consumption.

Sets of factors are often chosen in response to some test portfolio that proves tricky to 
price.
This has lead to wide-spread data-mining.
\textcite{harvey2016and} suggest that our obsession with churning out new factors means many 
are insignificant and our thresholds for significance should be increased.
The best defense against data-mining is intuition.
Tricky or anomalous portfolios may have economic significance but their existence should not 
cause us to abandon intuition in an effort to price them.
Asset-pricing models have turned into an exercise in pricing the returns on small subsets of 
assets.

Equation (\ref{eq:returns}) reminds us that \emph{all} discounted returns should equal 1.
\textcite{barillas2016alpha} show that we do not need to know all assets to measure the 
relative performance of different sets of factors because
\begin{equation}
\label{eq:maxsh}
Sh^2(a) = Sh^2(R) - Sh^2(f)
\end{equation}
where; $Sh^2(a)$ is the squared Sharpe ratio of the intercepts when regressing the time series 
of all assets' returns on the factors, $R$ represents the returns on all assets and $f$ 
represents the returns on the factors.
The set of factors with the highest Sharpe ratio is the set that performs best when pricing 
all returns.
Before attaching stories to tricky portfolios, we must first use intuition to describe a set 
of factors that perform well pricing all returns.
Our model can help us identify why the portfolios are so tricky, but its composition should 
not be driven by mis-pricing in small subsets of assets.

\textcite{graham1934security} tell us to buy good stocks at a fair price.
``Good and fair" is typically split into quality and cheapness.
The quality of a stock is measured by its profitability.
A stock's book-to-market ratio tells us how much we have to pay in the stock market for the 
book equity of the stock.
This measure is useful because we can compare the cheapness of different size stocks relative 
to each other.
As we will see later, combining book-to-market and momentum gives a superior measure of 
cheapness than book-to-market alone.
Beginning with \textcite{fama1992cross}, the literature has a history of referring to 
book-to-market as value.
This terminology allows us to use slightly different specifications of value while referring 
to the same concept.

The three factor model of \textcite{fama1993common} combines market, size and value factors.
In papers too numerable to cite here, the model has been tested, confirmed and denied in just 
about every way possible.
The intuition is strong and the factors were chosen before the mass fishing exercise described 
in \textcite{harvey2016and}.
Based on the intuition of \textcite{graham1934security}, I add a profitability factor and 
replace the pure book-to-market measure of value with the tandem factors of book-to-market and 
momentum.
This gives a five-factor model of; the market, size, value, momentum and profitability. A 
linear combination of the factors gives
\begin{equation}
\label{eq:model}
m_t= w_{Mkt}Mkt + w_{Size}Size + \left( w_{Val}Val + w_{Mtm}Mtm \right) + w_{Prof}Prof
\end{equation}
where $w_{XXX}$ is the weight on a particular factor.
The traditional time series regression form of (\ref{eq:model}) is
\begin{equation}
\label{eq:tsmodel}
r_t^i - r_t^f= a^i + \beta^i Mkt + s^iSize + \left( v^iVal + m^iMtm \right) + p^iProf
\end{equation}
where; $r_t^i$ is the return in period $t$ for asset $i$, $r_t^f$ is the risk-free return and 
$\alpha_i$ is the intercept.
If our factor specification for $m$ is correct, the $\alpha$ term in (\ref{eq:tsmodel}) should 
be insignificant.
Regression-based tests allow us to inspect factor loadings for those portfolios which have 
significant alpha, the so-called ``anomaly" portfolios.
Many problems in factor pricing models have negative loadings on the profitability and 
investment factors of \textcite{fama2016dissecting}.
The two negative loadings have been theatrically named the ``lethal" combination.
Since Fama and French include an investment factor while I do not, it is informative to 
compare differences in the factor loadings for the two models.

\textcite{fama2016choosing} use the Sharpe ratio to compare models using the long and short 
sides of factors including a cash-profitability factor.
\textcite{ball2016accruals} add a cash-profitability factor to the four-factor model of the 
market, size, value and momentum.
\textcite{barillas2015comparing} use the Sharpe ratio to find a six-factor model including 
monthly value and momentum gives the best performance.
Previous work does not combine cash-profitability, monthly value and momentum.
I use non-nested models are used to explore why the ``lethal" loading combination creates such 
problems.


% !TeX root=./main.tex

\section*{Introduction}

The goal of an asset-pricing model of stock returns is to
minimize mispricing for all portfolios.
We aim to describe the returns on all portfolios
with a linear combination of a few portfolios known as ``factors".
% We seek a few portfolios, ``factors",
% that describe the returns on all portfolios.
The problem with this goal is that we cannot directly test
mispricing for all portfolios.
This problem has led to data-mining as researchers have relied on minimizing
mispricing for subsets of portfolios.
Stock data is rich and a researcher can always construct some portfolio,
an ``anomaly", that cannot be priced.
An anomaly provides the empirical basis for a new factor.
\textcite{harvey2016and} document more than 300 factors and
\textcite{linnainmaa2016history} find that many are spurious out of sample.

\textcite{barillas2016alpha} elegantly solve both problems,
the inability to test all portfolios and ensuing data-mining,
by showing that maximizing the Sharpe ratio of factors minimizes
mispricing for all portfolios.
A top-down approach of testing a model's performance for all portfolios,
before examining performance for subsets of portfolios,
favours economic intuition over statistical anomalies.
The model that performs best pricing all portfolios then provides a description
of subsets of portfolios that cannot be fully priced,
rather than subsets of portfolios driving
the choice of factors used to price all portfolios.

I propose a five-factor model of market, size, value, momentum and
profitability factors with time series regression form
\begin{equation} \label{eq:B16}
R_t^i = a^i+b^iR^M+s^iSMB_t+
\underbrace{v^iHML_t^m+m^iWML_t}_\text{cheap}+
\underbrace{p^iPMU_t}_\text{profitable}
\end{equation}
where $R_t^i$ is the excess return on asset $i$ at time $t$, $a$ is the
regression intercept and unexplained return left by the factors, $R^M$ is the
excess return on the market, $SMB$ is the return on a portfolio long small
stocks and short big stocks mimicking the size premium, $HML$ is a portfolio
long value stocks and short growth stocks mimicking the value premium, $WML$ is
a portfolio long past winners and short past losers mimicking the momentum
premium and $PMU$ is a portfolio long profitable stocks and short unprofitable
stocks mimicking the profitability premium.
% Lower-case letters are regression coefficients.
My choice of factors is based on intuition from \textcite{graham1934security}
who argue investors should try to identify cheap and profitable stocks.
The combination of value and momentum identifies stocks that are truly ``cheap"
\parencite{kok2017facts}.

Model \ref{eq:B16} has a higher Sharpe ratio of the factors than the
five-factor model of \textcite{fama2016dissecting}.
% including when momentum is added to the Fama and French specification.
The investment factor is made redundant by value and momentum
through their ability to forecast future changes in book equity.
% Value and momentum are negatively correlated and this relationship is a
% statistical slam dunk in a regression context
% \parencite{asness1997interaction, fama2015incremental}.
% Their relationship with book equity shows that value and momentum's interaction
% is more than just statistical.
While the Sharpe ratio of the factors indicates the best overall model,
there are portfolios for which model \ref{eq:B16} performs poorly.
The GRS statistic of \textcite{gibbons1989test} indicates how much the Sharpe
ratio of a set of factors can be improved by investing in the test portfolios
as well as the factors.
GRS evidence broadly, but not always, matches Sharpe ratio evidence.
I compare the GRS statistics of different specifications of model across
different test portfolios to highlight the problems caused by momentum and
volatility.
I use individual portfolio regressions for problem sorts identified by the GRS
statistic to investigate the factor slopes that lead to problems.
Model \ref{eq:B16} performs poorly where value is constrained or disappears.
The offending portfolios share the ``lethal combination" of small, unprofitable
stocks that somehow invest aggressively identified in
\textcite{fama2015five, fama2016dissecting}.
Problem portfolios behave like small, unprofitable stocks with poor recent returns.
This new description of problem sorts implies firm size, rather than liquidity,
drives mispricing for volatile portfolios.

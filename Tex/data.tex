% !TeX root=./main.tex

\section*{Data}

I use the anomaly sorts, except for size-value-momentum, and market, size and
momentum factors from Dr. Kenneth French's website\footnote{
\url{http://mba.tuck.dartmouth.edu/pages/faculty/ken.french/data_library.html}
}. The monthly value factor is from AQR Capital Management's website\footnote{
\url{https://www.aqr.com/library/data-sets}
}.
I construct the missing cash profitability factor and summary characteristics
from Center for Research in Securities Prices (CRSP) and COMPUSTAT data. My
sample includes all common stocks, CRSP share code 10 or 11, on the NYSE, AMEX
or NASDAQ exchanges. All data begins in July 1963 and ends in December 2017. I
do not omit financial firms nor winsorize variables. See the variables appendix
for more complete definitions of variables and the data codes used to create
each characteristic.

The missing profitability factor is created from independent sorts on size and
cash profitability. Stocks are sorted into two size buckets and three
profitability buckets. Size and profitability characteristics are updated at
the end of June. The size for a given month is the market equity from the most
recent June. For example, a firm’s size from July 1999 to June 2000 is the
market equity from June 1999. Cash profitability for a given month is created
using the balance sheet method of \textcite{ball2016accruals}. Balance sheet
values are taken from the previous year. For example, profitability in July
2000 uses balance sheet values from 1999. These values are not updated until
the next July so profitability in June 2001 still uses balance sheet values
from 1999. I use NYSE breakpoints to assign all stocks to buckets. The
breakpoints are the median for size and the 30th and 70th percentiles for
profitability. Stocks must meet the profitability requirements set out in
\textcite{fama2015five} and missing accruals values are replaced with zero. The
intersections of the two size buckets and three profitability buckets gives six
total buckets. Using the small buckets, I find the spread between the high-
profit and low-profit returns at each month giving a time series of returns
representing the profitability risk premium for small stocks. I do the same for
large stocks. The average of the risk premiums for small and large stocks gives
the cash profitability factor.

While value, momentum and investment factors are available online, their
respective summary characteristics are not. To create the timelier value
characteristic, I use the market equity from the previous month and book equity
from the previous year updated each July as with profitability. To create the
momentum characteristic, I follow the instructions on Kenneth French’s website
and sum the returns from twelve months ago to two months ago. To create the
investment characteristic, I follow \textcite{fama2015five} and use the asset
growth from the previous year.

Summary characteristic sorts are three-way sorts on size, value and momentum or
investment. Sorts on value and momentum or investment are independent of each
other but performed within size buckets using NYSE quartile breakpoints to
assign all stocks to buckets. Characteristics are value-weighted. For the size,
timely value and momentum sorts this means using the previous month’s market
equity because these portfolios are rebalanced monthly. For the size, annual
value and investment sorts this means using the market equity from the most
recent June as with the profitability factor because these portfolios are
rebalanced annually.

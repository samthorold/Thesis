% !TeX root=./main.tex

\section*{Data}

Anomaly sorts, except for size-value-momentum, and market, size and
momentum factors are from Dr. Kenneth French's website\footnote{
\url{http://mba.tuck.dartmouth.edu/pages/faculty/ken.french/data_library.html}
}. The monthly value factor is from AQR Capital Management's website\footnote{
\url{https://www.aqr.com/library/data-sets}
}.
I construct the missing cash profitability factor and summary characteristics
from Center for Research in Securities Prices (CRSP) and COMPUSTAT data. My
sample includes all common stocks, CRSP share code 10 or 11, on the NYSE, AMEX
or NASDAQ exchanges. All data begins in July 1963 and ends in December 2017. I
do not omit financial firms nor winsorize variables. See the variables appendix
for more complete definitions of variables and data codes used.

The missing profitability factor is created from independent sorts on size and
cash profitability.
Stocks are sorted into two size buckets and three profitability buckets.
Size and profitability characteristics are updated at the end of June.
%The size for a given month is the market equity from the most recent June.
%For example, a firm’s size from July 1999 through June 2000 is the market
%equity from June 1999.
Cash profitability for a given month is created using the balance sheet method
of \textcite{ball2016accruals}.
Balance sheet values are taken from the previous year. 
I use NYSE breakpoints to assign all stocks to buckets.
Breakpoints are the median for size and the 30th and 70th percentiles for
profitability.
The intersections of the two size buckets and three profitability buckets
gives six total buckets.
The average spread between high-profit and low-profit returns for small
and big stocks gives the return on the profitability factor.

While value, momentum and investment factors are available online, their
respective characteristics are not.
To construct the timelier value characteristic,
I follow intuition in \textcite{asness2013devil}
and use ME from the previous month and BE from the previous year.
% This is not the same method used by AQR to construct the $\text{HML}^m$ factor.
% They use the book-to-price ratio at the stock level with price updated monthly
% but the intuition is the same.
To construct the annual value characteristic,
I follow the instructions on Kenneth French's website and use BE from the
previous year and ME from December of the same year used for BE.
To construct the momentum characteristic, I follow the instructions on Kenneth
French’s website and sum the returns from twelve months ago to two months ago.
To construct the investment characteristic,
I follow \textcite{fama2015five} and use the asset growth from the previous
year.
To construct the profitability characteristic, I follow
\textcite{ball2016accruals} and remove accruals from operating profitability
scaled by BE all from the previous year.
To construct the variance characteristic, I follow
\textcite{moreira2017volatility} and use the variance of daily returns from the
previous month.
% Summary characteristic sorts are three-way sorts on size, value and momentum or
% investment.
% Sorts on value and momentum or investment are independent of each other but
% performed within two size buckets using NYSE quartile breakpoints to assign all
% stocks to buckets.

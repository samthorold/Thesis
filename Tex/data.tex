% !TeX root=./main.tex

\section{Data and Method}

Where possible, data for factors and anomalies are taken from Dr. Kenneth French's
website\footnote{mba.tuck.dartmouth.edu/pages/faculty/ken.french/}. I use data from the
Centre for Research in Securities Prices (CRSP) and COMPUSTAT databases to construct
missing factors. I use all stocks with a share code of 10 or 11 on the NYSE, AMEX or
NASDAQ exchanges. I do not omit financial firms. The data not available from Dr. French's
website are the monthly value and cash profitability factors as well as any sorts
involving these variables. All factors are zero-investment long-short excess return
factors constructed using 2 x 3 sorts on size and one other variable in the style of
\textcite{fama1992cross}. To assign individual stocks into buckets, characteristics are
compared to breakpoints from NYSE stocks only to prevent small stocks dominating results.
The intersections of the two size buckets and three other characteristic buckets create
six portfolios of stocks with similar characteristics. The return on a factor is then the
average of the spreads between low and high characteristics for small and big stocks.
While the portfolios are value-weighted, taking the average of the small and large stock
spreads in effect overweights the importance of the small stock returns compared to value-
weighting.

To create the cash profitability factor I do the following. The market equity, ME, from
June each year is assigned to the following twelve month holding period. I create the
profitability characteristic, CP,  in the style of \textcite{ball2016accruals} (see also
appendix \ref{sec:vars} forCRSP and COMPUSTAT datacodes). I assign the CP from last fiscal
year, COMPUSTAT \texttt{fyear}, to the twelve month holding period from July to June. This
lag in book values avoids look-ahead bias. Using NYSE stocks, I find the median ME as well
as 30\textsuperscript{th} and 70\textsuperscript{th} percentiles for CP. All stocks are
then assigned a bucket based on the NYSE percentile breakpoints. I find the value-weighted
return at each month for the six unique combinations of ME and CP buckets;
Small(S)-Unprofitable(U), S-Medium Profitability(M), S-Profitable(P), Big(B)-U, B-M, B-P.
Then I find the spread between the returns on profitable and unprofitable firms for small
and big stocks, PMU- small is the returns on SP less the returns on SU and PMU-big
likewise for big stocks. The final return on the factor for each month is the average of
the small and big spreads. While the number of shares of each stock is held constant for
the holding period, the weight within a bucket will change from month to month as returns
increase and decrease. The monthly value factor is created in the same way except I sort
on $BM^m$ instead of CP and do not adjust the weight each month as the factor is
rebalanced each month.


% !TeX root=./main.tex

\section{Data}

Where possible, data are taken from previous research. Authors making their data available
online include Clifford Asness, Robert Novy-Marx  and Kenneth French\footnote{
\url{https://www.aqr.com/library/data-sets},
\url{http://rnm.simon.rochester.edu/data_lib/index.html},
\url{http://mba.tuck.dartmouth.edu/pages/faculty/ken.french/data_library.html},
respectively}.
The data not available online are the cash profitability factor and sorts on size, value
and momentum/investment. I use Center for Research in Securities Prices (CRSP) and
COMPUSTAT data to construct missing factors and sorts. My sample includes all common
stocks, CRSP share code 10 or 11, on the NYSE, AMEX or NASDAQ exchanges. Monthly returns
are from CRSP beginning in July 1963 and ending in December 2017. Annual book values are
from COMPUSTAT beginning in 1961 for lagged variables. I do not omit financial firms nor
winsorize variables. See the data appendix for more complete definitions of variables and
the data codes used to create each characteristic.

I outline in my own words how I create the profitability factor below. Superior
instructions are available on Kenneth French’s website. The missing profitability factor
is created from independent sorts on size and cash profitability. Stocks are sorted into
two size buckets and three profitability buckets. Size and profitability characteristics
are updated at the end of June. The size for a given month is the market equity from the
most recent June. For example, a firm’s size in July 1999 and June 2000 is the market
equity from June 1999. Cash profitability for a given month is created using the balance
sheet method of \textcite{ball2016accruals}. Balance sheet values are taken from the
previous year. For example, profitability in July 2000 uses balance sheet values from
1999. These values are not updated until the next July so profitability in June 2001 still
uses balance sheet values from 1999. I use NYSE breakpoints to assign all stocks to
buckets. The breakpoints are the median for size and the 30th and 70th percentiles for
profitability. Stocks must meet the profitability requirements set out in
\textcite{fama2015five} and missing accruals values are replaced with zero. While value,
momentum and investment factors are available online, their respective characteristics for
each stock are not. To create the timelier value characteristic, I use the market equity
from the previous month and book equity from the previous year updated each July as with
profitability. To create the momentum characteristic, I follow the instructions on Kenneth
French’s website and sum the returns from twelve months ago to two months ago. To create
the investment characteristic, I follow \textcite{fama2015five} and use the asset growth
from the previous year. The intersections of the two size buckets and three profitability
buckets gives six total buckets. The returns each month for each bucket are the value-
weighted returns of the stocks assigned to each bucket. The bucket constituents are held
constant for twelve months from July of year t to June of year t+1. Nevertheless, the
weight each stock has within a bucket will change month-to-month as returns differ across
stocks each month. While size is held constant for the entire holding period to keep a
given stock in the correct bucket, the weight is adjusted for returns in the holding
period. Using the small buckets, I find the spread between the high-profit and low-profit
returns at each month giving a time series of returns representing the profitability risk
premium for small stocks. I do the same for large stocks. The average of the risk premiums
for small and large stocks gives the cash profitability factor. Using the average of the
risk premiums for small and large stocks highlights the possible overweighting of small
stocks embedded in this methodology. We take great pains to avoid size bias in the factors
- using NYSE breakpoints, sorting on size, value-weighting returns - only to then take the
simple average of the spreads from small and large stocks. Nevertheless, the time-series
of returns methodology is extremely prevalent in the literature and has some clear
benefits such as zero-investment portfolios and the intuitive interpretation of regression
loadings. Three-way sorts differ slightly. Sorts on value and momentum/investment are
independent of each other but performed within size buckets using NYSE quartile
breakpoints to assign all stocks to buckets. Characteristics are value-weighted. For the
size, timely value and momentum sorts this means using the previous month’s market equity
because these portfolios are rebalanced monthly. For the size, annual value and investment
sorts this means using the market equity from the most recent June as with the
profitability factor because these portfolios are rebalanced annually. Again, for annually
rebalanced portfolios we must adjust the weight for returns individual stocks make each
month throughout the holding period. Stocks must have values for all the variables
involved to be included in sorts.

% !TeX root=./main.tex

\section{Analysis}

I focus largely on the differences between two models, the five-factor model from
\textcite{fama2016dissecting} which I refer to as F16, and a five-factor model similar to
that of \textcite{ball2016accruals} which I refer to as B16. I focus on these models
because they have the highest Sharpe ratio from the models I outlined earlier, and they
are non-nested. Furthermore, F16 and earlier iterations have been studied extensively. The
problems with F16 are well documented and provide a good benchmark to compare findings
against.

I begin by comparing the maximum Sharpe ratios of the factors for each model as well as
the amount each factor contributes to the Sharpe ratio. Value and profitability are the
biggest contributors for both models and this lines up with the intuition of buying cheap,
profitable stocks. Comparing GRS evidence shows where B16 does not live up to the promise
of its superior Sharpe ratio. I investigate where factors contribute little to lowering
the GRS statistic in certain sorts by comparing different versions of B16 and F16. In
isolation, more timely value is more a hindrance than a help in many sorts. Profitability
is not as helpful in sorts on volatility as previous research
\parencite{novy2014understanding} suggests it should be. For particularly disastrous
performances, I investigate which individual sorts cause the most pain and compare the
performance to F16. Where value is constrained, the value factor does not offer much help
and the profitability factor is left to pick up the slack. In the extreme sorts,
profitability alone cannot explain all the average return, although performance is
superior to that of F16. Sorts where value and momentum work against each other are a
problem. Comparing with F16 shows the troublesome sorts share the ``lethal" return
behaviour of small, unprofitable stocks that invest aggressively.

\import{./Analysis/}{replication}

\import{./Analysis/}{bootstrap}

\import{./Analysis/}{contributions}

\import{./Analysis/}{redundant_CMA}

%\import{./Analysis/}{future_BE}

%\import{./Analysis/}{GRS}

%\import{./Analysis/}{anomalies_group}

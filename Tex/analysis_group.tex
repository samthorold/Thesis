% !TeX root=./main.tex

\section*{Analysis}

I begin by showing the maximum $\text{Sh}^2$ for model \ref{eq:B16} as well
other common models. I use a bootstrap procedure to enhance the reliability of
the $\text{Sh}^2$ results and to show investment does not reliably improve the
$\text{Sh}^2$ of the proposed model. Next I use spanning regressions to show
that value and profitability add the most to the $\text{Sh}^2$ of my proposed
model and that value and momentum combine to make investment redundant. Sorts
on size-value- momentum/investment that monthly value and momentum create
variation in investment and changes in book equity characteristics. Investment
moves in the opposite to changes in book equity for small, growth stocks.

I investigate anomaly performance to show that my proposed model reduces the
GRS statistic for many common anomalies but struggles where value is
constrained as well as sorts on momentum and volatility. I compare anomaly
regressions of models \ref{eq:B16} and \ref{eq:F16} because they have two of
the highest $\text{Sh}^2$ and they are non-nested. I show that value behaves
similarly to investment in anomaly sorts. The choice of anomalies is motivated
by high GRS statistics which indicate the Sharpe ratio of the factors can be
improved by investing in the anomaly portfolios. I include sorts on investment,
even though the GRS is comparatively low, to show performance is not list by
omitting the investment factor.

%\import{./Analysis/}{replication}

\subsection*{Pricing All Stocks}

\import{./Analysis/}{bootstrap}
\import{./Analysis/}{contributions}
\import{./Analysis/}{redundant_CMA}
\import{./Analysis/}{future_BE}

\subsection*{Pricing Subsets of Stocks}

\import{./Analysis/}{GRS}
\import{./Analysis/}{anomalies_group}

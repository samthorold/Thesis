% !TeX root=./main.tex

\section{Analysis}

I begin by showing replicated factors are highly correlated with factors available online.
The main analysis first presents the maximum $\text{Sh}^2$ for my proposed five-factor
model as well other common models. I use a bootstrap procedure to enhance the reliability
of the $\text{Sh}^2$ results and to show investment does not reliably improve the
$\text{Sh}^2$ of the proposed model. Next I use spanning regressions to show that value
and profitability add the most to the $\text{Sh}^2$ of my proposed model and that value
and momentum combine to make investment redundant. Sorts on size-value-
momentum/investment show this may be due to monthly value and momentum's ability to
forecast future changes in book equity.

Following the analysis of performance pricing \emph{all} assets and where this superior
performance may stem from, I investigate anomaly performance to show that my proposed
model reduces the GRS statistic for many common anomalies but struggles where value is
constrained or where value and momentum cannot utilize their interaction. I compare
anomaly regressions of my proposed model with the five-factor model of
\textcite{fama2016dissecting} because these models have two of the highest $\text{Sh}^2$
and they are non-nested. I show that value behaves similarly to investment in anomaly
sorts. The choice of anomalies is motivated largely by GRS evidence. I include sorts on
investment to show performance is not list by omitting the investment factor.

\import{./Analysis/}{replication}

\subsection{Pricing All Stocks}

\import{./Analysis/}{bootstrap}

\import{./Analysis/}{contributions}

\import{./Analysis/}{redundant_CMA}

\import{./Analysis/}{future_BE}

\subsection{Pricing Subsets of Stocks}

\import{./Analysis/}{GRS}

%\import{./Analysis/}{anomalies_group}

% !TeX root=./main.tex

\section*{Analysis}

The maximum $\text{Sh}^2$ for model \ref{eq:B16} is higher than other common
models. I use a bootstrap procedure to increase the reliability of the
$\text{Sh}^2$ results and to show adding the investment factor does not improve
model \ref{eq:B16}. I regress each factor on the other factors
in each model to show that value and profitability add the most to the
$\text{Sh}^2$ of model \ref{eq:B16} and that value and momentum combine to make
investment redundant. Sorts on size, value and momentum or investment show that
investment moves in the opposite direction to changes in book equity for small,
growth stocks. Investment is not the best proxy for changes in book equity.
Model \ref{eq:B16} reduces the GRS statistic for many common anomalies but
struggles where value is constrained as well as in sorts on momentum and
volatility. I compare anomaly regressions of models \ref{eq:B16} and
\ref{eq:F16} because they have two of the highest $\text{Sh}^2$ and are
non-nested.
Individual time series regressions for investment, value, momentum and
volatility sorts reveal that value picks up the slack in investment's absence.
Given this additional workload, model \ref{eq:B16} performs poorly where value
provides a poor description of returns. These stocks behave like small,
unprofitable stocks with low recent returns.

\subsection*{Performance pricing all stocks}

\import{./Analysis/}{bootstrap}
\import{./Analysis/}{contributions}
\import{./Analysis/}{redundant_CMA}
\import{./Analysis/}{future_BE}

\subsection*{Performance pricing subsets of stocks}

\import{./Analysis/}{GRS}
\import{./Analysis/}{anomalies_group}

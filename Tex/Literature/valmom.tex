% !TeX root=./main.tex

\subsection{Value and Momentum}

Quantitative factor strategies and research has given ``value" investing a different 
definition than that originally laid out by \textcite{graham1934security}.
Recent quantitative value strategies, like those of \textcite{lakonishok1994contrarian} and 
\textcite{chan2004value}, focus only on fundamental ratios to try and identify cheap stocks.

\textcite{kok2017facts} argue  that fundamental value measures, such as the popular 
book-to-market ratio, are mean-reverting since
$B_{t+1}/p_{t+1} = B_t/p_t \cdot B_{t+1}/B_t \cdot p_t/p_{t+1}$.
Where $B_t$ is a measure of book value at time $t$ and $p$ is market price.
Taking logs gives
\[
log(B_{t+1}/p_{t+1}) = log(B_{t}/p_{t}) + log(B_{t+1}/B_{t}) - log(p_{t+1}/p_{t})
\]
A high book-to-market ratio can revert to its mean through an increase price or
a decrease in book equity.
Investors wish to identify those stocks that will increase in price because
$R_{t+1}=p_{t+1}/p_{t}$.
Book-price measures are polluted by stocks that revert to mean values through
decreases in book equity.
Book-to-market ratios are useful, so long as we can separate stocks that will
experience a decrease in book values from those that will increase in price.
\textcite{kok2017facts} show that decreases in book-equity are lower for high
momentum stocks.
By using a combined fundamental value and momentum strategy one weeds out stocks that recently 
had bad news that is not yet reflected in accounting numbers.
A book-to-market measure is useful \emph{if} one can back out expected changes in book equity.
Unfortunately this requires perfect foresight.
HML-Momentum is a compromise.

\textcite{asness2013devil} show that momentum can forecast future changes in
book equity.
Combining value and momentum identifies high book-to-market stocks that will
revert to their mean fundamental value through an increase in price.
This is true of the monthly, current measure of book-to-market without the
presence of momentum in a multi-factor regression.
\textbf{Why include momentum then?}

Fundamental value has gotten weaker since publication \parencite{mclean2016does}.
Returns to fundamental value is concentrated in small high-growth stocks with short sale 
constraints \parencite{nagel2005short}.
\textcite{asness2015fact}
\begin{quotation}
  There is no strong stand-alone value premium among large caps.
\end{quotation}

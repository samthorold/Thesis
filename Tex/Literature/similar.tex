% !TeX root=./main.tex

\subsection{Similar Work}

\textcite{fama2016choosing} suggest a five-factor model with the time series regression form of
\begin{equation}
\label{eq:tsFF2016}
r_t^i - r_t^f= \alpha^i + \beta^i \cdot Mkt + s^i \cdot Size + v^i \cdot Val + p^i \cdot Prof 
+ i^i \cdot Inv
\end{equation}
where; $Val$ is annually rebalanced using $ME$ from December of last year.
The other factors are as specified in \ref{eq:tsmodel}.
The dividend discount model, $p_t = \sum E(d_t)/R_t$, says that stock prices are the expected 
discounted sum of their dividends.
\textcite{fama2006profitability} explain that by assuming clean surplus accounting and 
dividing each side by book-equity we find
\begin{equation}
\label{eq:ddm}
\frac{ME_t}{BE_t} = \frac{\sum \frac{E(Prof_t-\Delta BE_t)}{R_t}}{BE_t}
\end{equation}
They use this relationship to make three intuitive points.
All else the same;
a lower $ME$ (and higher $BE/ME$) leads to a higher return,
a higher profit leads to a higher return and
a lower change in book-equity leads to a higher return.
While (\ref{eq:ddm}) calls for change in book-equity, asset-growth is used as a proxy because 
Fama and French argue this is a better ``picture" of what the dividend discount model implies 
$\Delta BE$ represents, investment.

\textcite{ball2016accruals} suggest a model with the time series regression form
\begin{equation}
\label{eq:tsBall2016}
r_t^i - r_t^f= \alpha^i + \beta^i \cdot Mkt + s^i \cdot Size + v^i \cdot Val + p^i \cdot Prof 
+ m^i \cdot Mom
\end{equation}


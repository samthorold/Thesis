% !TeX root=../main.tex

\subsection{Momentum helps identify cheap stocks \cite{kok2017facts}}

\emph{Value investing} is buying securities  for a good value.
It is not the quantitative strategies of \cite{lakonishok1994contrarian} nor 
\cite{chan2004value} which focus on ratios of common fundamental metrics such 
as book-to-market or price-earnings.
Quantitative strategy proponents claim to identify temporarily underpriced 
securities. \textbf{Do they? Or do they claim it is risk?}

\textbf{The common fundamentals approach only works for small stocks}.
These stocks tend to be capacity constrained and illiquid.

Instead of identifying under-priced securities, the fundamentals approach 
identifies securities with inflated accounting numbers.

\subsubsection{Brief History}

\cite{graham1934security} said that investors should buy stocks that are 
trading at a significant discount to intrinsic value.
Where intrinsic value is measured as future earnings power.
Multiples began gaining traction in the 1980s.
High book-to-market stocks began being referred to as ``value" stocks in 
\cite{fama1992cross}.
Indexes popped up representing different ``styles" such as the Russel value 
index among others.
Academics and practitioners found that these indexes were not amazing 
(\cite{loughran1997book} and \cite{asness2015fact}).
In fact, fundamental value has gotten weaker since publication 
(\cite{mclean2016does}).

Returns to fundamental value is concentrated in small high-growth stocks with 
short sale constraints (\cite{nagel2005short}).

Construction of common 2x3 factors equal weight small and big halves which 
overweights small stocks. \textbf{Does it? They're zero investment portfolios}.

\cite{asness2015fact}
\begin{quotation}
  There is no strong stand-alone value premium among large caps.
\end{quotation}

Value indexes often under-perform their standard counterparts.

If fundamental value does not identify under-priced securities what does it 
identify then?
\begin{itemize}
  \item Fundamental metric in the numerator represents intrinsic value
  \item The price in the denominator tells us how much we have to pay for that 
  intrinsic value
  \item A high ratio therefore implies a ``cheap" security
\end{itemize}
A high fundamental value measure may revert to intrinsic value for one of two 
reasons;
a price increase (what investors want) or
a decrease in the fundamental numerator (\cite{graham1934security} warning) 
because the security was not cheap.

Fundamental ratios exhibit mean reversion
\[
\frac{F_{t+1}}{P_{t+1}}=\frac{F_t}{P_t}\cdot\frac{F_{t+1}}{F_t}\cdot\frac{P_{t+1}}{P_t}
\]
taking logs

\[
ln\left(\frac{F_{t+1}}{P_{t+1}}\right) =ln\left(\frac{F_t}{P_t}\right) 
+ln\left(\frac{F_{t+1}}{F_t}\right) -ln\left(\frac{P_{t+1}}{P_t}\right)
\]

The ending ratio value equals the beginning value plus the change in the 
fundamental metric minus the change in the price.

\subsubsection{Fundamental Value and Momentum}

Positive momentum stocks exhibit less mean reversion.
That is to say they exhibit smaller reductions in the fundamental metric.
By using a combined fundamental value and momentum strategy one weeds out 
stocks that recently had bad news that is not yet reflected in accounting 
numbers.
A book-to-market measure is useful \emph{if} one can back out expected changes 
in book equity.
Unfortunately this requires perfect foresight.
HML-Momentum is a compromise.

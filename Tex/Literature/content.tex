\section{CAPM}

Markowitz\cite{markowitz1959portfolio} concerns how an optimizing investor
would behave. Under uncertainty, an investor cannot simply choose to invest in
the asset with the highest return. Acting on expected returns alone, an
investor could choose to invest in the asset with the highest expected return.
This is not a good description of the market. We observe diversification which
suggests that investors are not simply choosing the asset with the highest
expected return. Diversification implies investors are considering the risk
associated with each asset and the risk of the combination of assets that make
up a portfolio. Investors will choose portfolios of assets that maximize the
return for a given level of risk.

Sharpe\cite{sharpe1964capital} and Linter\cite{lintner1965valuation} concern
economic equilibrium assuming investors behave as described by Markowitz. This
is the capital asset pricing model, CAPM.

Black, Jensen and Scholes\cite{jensen1972capital} concerns early empirical
tests of the CAPM. They find that low-beta stocks tend to have higher returns
and high-beta stocks tend to have lower returns than theory predicts.

Roll\cite{roll1977critique} concerns what we can cannot test about the CAPM.
Since we do not have access to the returns on all risky assets we cannot truly
test the validity of the model. We are testing whether certain portfolios
behave as an acceptable proxy for the ``market" portfolio. To be more precise,
we are testing if we can reject the hypothesis that the portfolio in question
could be a proxy for the true market portfolio.

\subsection{CAPM Anomalies}

Banz\cite{banz1981relationship} finds the small stocks have higher returns
than predicted by the CAPM.

\section{Three Factor Model}

Fama and French\cite{fama1993common} concerns a three factor model constructed
to attempt to deal with anomalies surrounding the CAPM.

\subsection{Three Factor Model Anomalies}

Accruals, Momentum

\section{Five Factor Model}

Fama and French\cite{fama2016dissecting} adds profitability and investment
factors to their three factor model.

\subsection{Five Factor Model Anomalies}

Accruals, Momentum - still

\section{Factor Comovement}

Asness and Frazzini\cite{asness2013devil} find that value and momentum are
linked.

Novey-Marx\cite{novy2013other} finds that value and gross profitability are
linked.

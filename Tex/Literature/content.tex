\subsection{Univariate Factors}

Tobin (1958\cite{T58}) split an invester's allocation decision into two phases.
First, the investor must choose the optimum combination of risky assets.
Second, the investor allocates their wealth between the optimum risky portfolio
and a riskless asset.
Sharpe (1964\cite{S64}) and Lintner (1965\cite{L65}) both credited with the
invention of the capital asset pricing model (CAPM).
The CAPM suggests that every investor will choose the same optimum risky
portfolio, regardless of their appetite for risk. Risk aversion will dictate
how much of total wealth is allocated to the riskless asset.
Comovement with the market.
Diversification protects investors from idiosyncratic risk.
By holding all available assets an investor will be as diversified as possible.
In practical terms this means holding all stocks available.
Sharpe and Lintner suggested the optimum risky portfolio is the market.
Stocks comovement with the market dictates the size of the return investors
require to hold that asset.

Banz (1981\cite{B81}) identified that the cross section of stock returns varied
with market capitaliztion. Banz also suggested that size could be a proxy for
some true source of risk.

Value (Rosenburg, Reid and Lanstein 1985\cite{RRL85})

Leverage (Bhandari 1988\cite{B88})

Earnings (Basu 1983\cite{B83})
% !TeX root=../main.tex

\subsection{Why Sharpe ratio? \textcite{barillas2016alpha}}

$SMB$, $HML$, etc. are traded factors while Consumption growth is not.
\textcite{breeden2005intertemporal} suggests that non-traded factors can be replaced with 
mimicking traded ones.
Significant $\alpha$ means that the maximum Sharpe ratio of the factors used in the regression 
can be improved by a position in the asset being tested.

The asset pricing challenge is to describe a small number of observable factors that can come 
close to spanning an efficient portfolio.
This is equivalent to correctly pricing returns.
It is interesting to note when tests of subsets of asset returns do not point 
to the same combination of factors as tests involving a more complete set of 
returns.

Model $M_1$ composed of factors $f_1$ is preferable to model $M_2$ composed of 
factors $f_2$ if
\[
Sh^2(f_1, f_2, R) - Sh^2(f_1) < Sh^2(f_1, f_2, R) - Sh^2(f_2)
\]
Where $R$ represents test asset returns.
Since both sides measure the improvement in squared sharpe ratio when adding a 
factor to the set of test asset returns and additional factors,
we can simplify the condition to
\[
Sh^2(f_1, f_2) - Sh^2(f_1) < Sh^2(f_1, f_2) - Sh^2(f_2)
\]
The condition can be simplified further to
\[
Sh^2(f_1)>Sh^2(f_2)
\]
Model comparison can be thought of in terms of investment opportunities.
Whether the model with the highest squared Sharpe ratio is ``good" enough depends on 
performance pricing test assets.

A model, $M$, is a multi-factor linear regression
\[
R = \alpha + \beta\cdot f + \varepsilon
\]
\textcite{gibbons1989test} show that
\[
\alpha ' \Omega^{-1}\alpha = Sh^2(f, R) - Sh^2(f)
\]
where $\Omega$ is the covariance matrix of $\varepsilon$.

It may seem that trying to find the highest maximum Sharpe ratio means adding many factors.
Additional factors do not necessarily improve model performance.
For example, the three factor model of \textcite{fama1993common} explains the returns from 
sorts on size and accruals better than the five factor model outlined in 
\textcite{fama2015five}.
Consider two models; $R_i=\alpha+\beta_MR_M$ and $R_i=\alpha+\beta_MR_M+\beta_SSMB$.
If a test asset's CAPM $\alpha$ is the opposite sign to its loading on $SMB$ 
then the $\alpha$ in the second model will be accentuated.
By conventional measures the second model is then "worse" than the first model.
The maximum Sharpe ratio of the second model is higher than the maximum Sharpe 
ratio for the first.
Conventional asset pricing tests and maximum sharpe ratios do not always agree.

If test asset and excluded-factor evidence contradict one another, we should favour 
excluded-factor evidence.
Excluded-factor evidence considers the returns on all assets rather than a subset of all 
assets.

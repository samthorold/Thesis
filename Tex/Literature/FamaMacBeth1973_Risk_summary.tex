
# @fama1973risk

*Risk, Return, and Equilibrium: Empirical Tests* 

Investors look at individual assets only in terms of covariance with the current portfolio.
Under the assumption that return distributions are normal, the risk of an asset is measured by its standard deviation.
$$
E(r_p) = \sum_{i=1}^N x_iE(r_i)
$$
$$
\sigma(r_p) = \sum_{i=1}^N x_i
\left[ \frac{\sum_{j=1}^N x_j \sigma_{ij}}{\sigma(r_p)} \right] =
\sum_{i=1}^N x_i \frac{Cov(r_i, r_p)}{\sigma(r_p)}
$$
The risk of an asset is proportional to its covariance with the current portfolio over the standard deviation of the current portfolio^[$Var(r_p)=\sigma^2(r_p)=\sum\sum w_iw_j\sigma_{ij}$ therefore $\sigma(r_p)=\sum w_i\frac{\sum w_j\sigma_{ij}}{\sigma(r_p)}$.
].
The risk of an asset is different for each portfolio.
Lagrangian methods say weights of each asset must satisfy
$$
E(r_i) - E(r_p) = S_p
\left[
\frac{\sum_{j=1}^Nx_j\sigma_{ij}}{\sigma(r_p)}-\sigma(r_p)
\right]
$$ {#eq:1}
where $S_p$ is the change in $E(r_p)$ for a change in $\sigma(r_p)$.
It is the slope of the efficient frontier at the point corresponding to portfolio $p$.
$S_p$ is the increase in expected return on the portfolio for an increase in the risk.
The difference between the return on the asset and the return on the portfolio is proportional to the difference between the risk of the asset and the risk of the portfolio, where the risk of the asset is measured as the additional risk to the portfolio.
$$
E(r_i) = E(r_p) - S_p\sigma(r_p) + \beta_iS_p\sigma(r_p)
$$ {#eq:2}
where
$$
\beta_i = \frac{Cov(r_i, r_p)}{\sigma^2(r_p)} =
    \frac{\sum_{j=1}^Nx_j\sigma_{ij}}{\sigma^2(r_p)} =
    \frac{\frac{\sum_{j=1}^Nx_j\sigma_{ij}}{\sigma(r_p)}}{\sigma(r_p)}
$$
$\beta_i$ refers to the risk of asset $i$ in portfolio $p$ relative to the total risk of $p$.
The intercept in +@eq:2 can be written as
$$
E(r_0) = E(r_p) - S_p\sigma(r_p)
$$ {#eq:3}
This is the return on a security whose return is uncorrelated with $r_p$ (a zero-$\beta$ security).
Since a zero-$\beta$ security does not change the risk of $p$, we can say it is riskless.
This does not mean the security has zero variance of return.
Based on +@eq:3 we can say
$$
S_p = \frac{E(r_p) - E(r_0)}{\sigma(r_p)}
$$ {#eq:sr}
and
$$
E(r_i) = E(r_0) + \left[E(r_p) - E(r_0)\right]\beta_i
$$ {#eq:mm}

*@eq:sr is the *Sharpe Ratio* of the portfolio $p$.
*@eq:mm says the expected return on a security is the riskless return, in $p$, plus a premium that is $\beta_i$ times the amount that $p$ exceeds the riskless rate.
There are three testable implications;

- The relationship between the expected return on a security and its risk in an efficient portfolio is linear
- $\beta_i$ is a complete measure of the risk of security $i$ in an efficient portfolio
- In a market of risk-averse investors, higher risk should be associated with higher return

If prices reflect all available information and investors all have access to information for free, efficient markets, it is not crazy to think that all investors would have the same beliefs about future returns and covariances, homogeneous expectations.
Black [-@black1972capital] says that under these conditions the value-weighted market portfolio is always efficient.
Homogeneous expectations allow us to relate ex ante and ex post return distributions.

Monthly data avoids problems with quarterly or annual data discussed in Miller and Scholes (1972) (**could not find this**).

They use an equally-weighted martket portfolio.
This does not seem smart.



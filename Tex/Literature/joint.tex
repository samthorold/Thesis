% !TeX root=./main.tex

\subsection{Mean-Variance and the Joint Hypothesis}

\textcite{markowitz1952portfolio} describes how an investor will maximize their utility in a 
two-parameter world.
Investors only care about the mean and variance of their portfolios and will choose the 
minimum variance portfolio for a given level of return.
\textcite{sharpe1964capital}, \textcite{lintner1965valuation} and \textcite{black1972capital} 
tell us how the market will behave in equilibrium if all investors are ardent followers of 
Markowitz.
Given this market equilibrium, we can describe a relation between the excess return on the 
market and the returns on all other assets:
\[
  R_t^i = \beta^iR_t^M
\]
where $R_t^i$ is the excess return on asset $i$ at time $t$, $R^M$ is the excess return on the 
market portfolio and $\beta=Cov(R^i, R^M)/Var(R^M)$.
This is the capital asset pricing model (CAPM).
The CAPM relates to the consumption-based model through asset return covariance with the 
market.
\textbf{Negative covariance with rainy days or something}.

\textcite{roll1977critique} points out that any test of the CAPM in actually a test of a joint 
hypothesis - that our model is correct \emph{and} that our choice of ``the market" is correct.
Since we can (probably) never know if we have chosen the market portfolio correctly, there are 
limited testable assumptions in the CAPM.
Similarly, \textcite{fama1991efficient} points out that we can only ever test if prices 
reflect ``all" information in the context of an asset-pricing model.
If the model does not accurately price assets we cannot know for certain whether this is due 
to the model or an inefficient market.
This does not mean asset-pricing models are a waste of time.
Models that allow us to describe the time series and cross-sectional behaviour of returns will 
always be of academic and professional interest.


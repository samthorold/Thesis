% !TeX root=./main.tex

\subsection{Pricing Errors}

We want to test whether a set of factors can price the returns on a set of
assets.
\textcite{jensen1968performance} explains that the intercept from a time
series regression model provides a measure of mis-pricing.
Statistically significant intercepts are predominantly referred to as ``alphas"
in the literature. 
Our general model is
\begin{equation}
  \label{eq:gen_reg_model}
  R_t^i = a^i + b^if_t + e_t^i
\end{equation}
where;
$R_t^i$ is the excess return at time $t$ for asset $i$,
$a^i$ is the amount of average excess return left unexplained by the factors $f$,
$b^i$ is the exposure of asset $i$ to the factors and
$e$ is the amount of variation left unexplained by $f$.
The set of factors, $f$, will change from model to model but the mis-pricing
intuition remains the same.

If asset returns are linearly related to and completely explained by the set of factors, we 
expect $a^i$ to be zero for all assets.
\textcite{gibbons1989test} tell us that we can use an F-test to check that
alphas are jointly zero for all test assets.
More than a simple binary check, the test statistic proposed by Gibbons, Ross
and Shanken (GRS) gives a measure of how much our set of factors can be
improved by including the test assets in the set of factors.
This relationship is given by
\[
  a'V_e^{-1}a = Sh^2(R, f) - Sh^2(f)
\]
where $Sh^2(\cdot)$ is the maximum squared Sharpe ratio.
\textcite{barillas2016alpha} use this relationship to show that test assets are
irrelevant when comparing two sets of factors.
\textcite{fama2016choosing} explain that, since the ultimate goal of
asset-pricing models is to price all assets, our test assets should be all
assets.
Unfortunately, we cannot regress the returns on all assets on a set of factors.
This does not matter since any set of factors is included in all assets,
$Sh^2(R, f)=Sh^2(R)$.
To minimize the level of mis-pricing given by $a'V_e^{-1}a$, we need to
maximize the Sharpe ratio of our set of factors.

The quadratic form of the alphas is also the pricing error we wish to minimize when we use the 
method of moments (GMM).
GMM highlights our use of the residual covariance matrix in $a'V_e^{-1}a$.
Other implications of GMM may use the identity matrix here but we are interested in the 
covariance of alphas not just the magnitude.
For our purposes the residual covariance matrix makes more sense than the identity matrix 
because we are concerned with the maximum Sharpe ratio that can be obtained from the errors.

It may seem that trying to find the highest maximum Sharpe ratio means adding many factors.
Additional factors do not necessarily improve model performance.
For example, the three factor model of \textcite{fama1993common} explains the returns from 
sorts on size and accruals better than the five factor model outlined in 
\textcite{fama2015five}.
Consider two models; $R_i=\alpha+\beta_MR_M$ and $R_i=\alpha+\beta_MR_M+\beta_SSMB$.
If a test asset's CAPM $\alpha$ is the opposite sign to its loading on $SMB$ 
then the $\alpha$ in the second model will be accentuated.
By conventional measures the second model is then "worse" than the first model.
The maximum Sharpe ratio of the second model is higher than the maximum Sharpe 
ratio for the first.
Conventional asset pricing tests and maximum Sharpe ratios do not always agree.
If test asset and excluded-factor evidence contradict one another, we should favour 
excluded-factor evidence.
Excluded-factor evidence considers the returns on all assets rather than a subset of all 
assets.

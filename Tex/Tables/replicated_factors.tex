
\begin{table}[!ht]
%\footnotesize
\centering
\caption{
\scriptsize{
Monthly averages, standard deviations, t-statistics and correlations with factors
available from previous research for \emph{replicated} factor returns, July 1963 -
December 2017 (654 months). At the end of each June, NYSE, AMEX, and NASDAQ stocks are
allocated to two Size groups (small and big) using the NYSE median market-cap breakpoints.
Stocks are allocated independently to three B/M groups (low to high), using NYSE
30\textsuperscript{th} and 70\textsuperscript{th} percentile breakpoints. The
intersections of the two sorts produce six value-weight Size-B/M portfolios. In the sort
for June of year t, B is book equity at the end of the fiscal year ending in year t-1 and
M is market cap at the end of December of year t-1. I use Compustat data to compute book
equity, defined as (1) stockholders' equity (or the par value of preferred plus total
common equity or assets minus liabilities, in that order) minus (2) the redemption,
liquidation, or par value of preferred (in that order) plus (3) balance sheet deferred
taxes. HML is the average of the returns on the two high B/M portfolios from the 2x3 sorts
minus the average of the returns on the two low B/M portfolios. The profitability and
investment factors, PMU (profitable minus unprofitable) and CMA (conservative minus
aggressive), are formed in the same way as HML, except the second sort variable is
operating profitability or investment. Operating profitability, used to create $PMU^{06}$,
in the sort for June of year t is measured with accounting data for the fiscal year ending
in year t-1 and is revenue minus the cost of goods sold, minus selling, general, and
administrative expenses, minus interest expense, all divided by book equity. Adjusted
operating profitability, $PMU^{16}$, is created similarly except research and development
costs are added back in before dividing by book equity. Cash profitbality, $PMU$, is
created by subtracting accruals before dividing by book equity. Investment, Inv, is the
change in total assets from the fiscal year ending in year t-2 to the fiscal year ending
in t-1, divided by t-2 total assets. The momentum factor, WML, is defined in the same way
as HML, except the factor is updated monthly rather than annually. To form the six Size-
Prior 2-12 portfolios at the end of month t-1, Size is the market cap of a stock at the
end of t-1 and Prior 2-12 is its cumulative return for the 11 months from t-12 to t-2.
Monthly value, $HML^m$, is defined in the same way as WML except book equity is only
updated at the end of June.
}
}

\begin{tabular}{lrrrrrrr}
  \toprule
  \multicolumn{8}{l}{Replicated Factors} \\
         & $HML$  & $HML^m$&$PMU^{06}$&$PMU^{16}$&$PMU$& $CMA$ & $WML$ \\
  Mean   &  0.31  &  0.35  &  0.28  &  0.34  &  0.36  &  0.22  &  0.58 \\
  SD     &  2.80  &  3.43  &  2.11  &  1.67  &  1.40  &  1.78  &  3.96 \\
  t-Stat &  2.84  &  2.60  &  3.38  &  5.21  &  6.63  &  3.24  &  3.78 \\
[1em]
  \multicolumn{8}{l}{Online Factors} \\
         & $HML$  & $HML^m$&$PMU^{06}$&$PMU^{16}$&$PMU$& $CMA$ & $WML$ \\
  Mean   &  0.35  &  0.30  &  0.25  &    -   &    -   &  0.29  &  0.66 \\
  SD     &  2.81  &  3.41  &  2.21  &    -   &    -   &  2.01  &  4.19 \\
  t-Stat &  3.14  &  2.26  &  2.89  &    -   &    -   &  3.66  &  4.02 \\
[1em]
  Corr   &  0.98  &  0.97  &  0.98  &    -   &    -   &  0.96  &  0.98 \\
  \bottomrule
\end{tabular}
\label{tbl:replicated_factors}
\end{table}


\begin{table}[!ht]
\centering
\caption{
Maximum Sharpe ratio and 90\% confidence intervals of the distributions of $\text{Sh}^2(\text{Row})-\text{Sh}^2(\text{Column})$ from 100,000 bootstrap simulations.
\scriptsize{
Each bootstrap sample is 654 months drawn randomly with replacement from the
654 months of July 1963-December 2017.
In each bootstrap sample, $\text{Sh}^2(f)$ is computed for the seven models
of Panel A.
Panel B shows the matrix of the 5th and 95th percentiles
(the 90\% confidence interval) of the bootstrap distributions of
$\text{Sh}^2(\text{Row})-\text{Sh}^2(\text{Column})$, the difference between
$\text{Sh}^2(f)$ for the row model and $\text{Sh}^2(f)$ for the column model.
When the 5th and 95th percentiles of the bootstrap distribution of
$\text{Sh}^2(\text{Row})-\text{Sh}^2(\text{Column})$ are positive,
$\text{Sh}^2(\text{Row})$ is greater than $\text{Sh}^2(\text{Column})$ in at
least 90\% of simulation runs, and the 1963-2017 actual $\text{Sh}^2(f)$ for
the row model is reliably higher than the 1963-2017 actual $\text{Sh}^2(f)$ for
the column model.
When the 5\textsuperscript{th} and 95\textsuperscript{th} percentiles of the
bootstrap distribution of $\text{Sh}^2(\text{Row})-\text{Sh}^2(\text{Column})$
are negative, $\text{Sh}^2(\text{Row})$ is less than
$\text{Sh}^2(\text{Column})$ in at least 90\% of simulation runs, and the
1963-2017 actual $\text{Sh}^2(f)$ for the row model is reliably less than
$\text{Sh}^2(f)$ for the column model.
The factors are constructed as follows.
At the end of each June, NYSE, AMEX, and NASDAQ stocks are allocated to two
Size groups (small and big) using the NYSE median market-cap breakpoints.
Stocks are allocated independently to three BM groups (low to high), using
NYSE 30\textsuperscript{th} and 70\textsuperscript{th} percentile breakpoints.
The intersections of the two sorts produce six value-weighted Size-BM portfolios.
In the sort for June of year t, B is book equity at the end of the fiscal year
ending in year t-1 and M is market cap at the end of December of year t-1.
I use Compustat data to construct book equity, defined as stockholders' equity
(or the par value of preferred plus total common equity or assets minus
liabilities) minus the redemption (or liquidation or par value) of preferred
stock plus balance sheet deferred taxes.
HML is the average of the returns on the two high BM portfolios from the 2x3
sorts minus the average of the returns on the two low BM portfolios.
The profitability and investment factors, PMU (profitable minus unprofitable)
and CMA (conservative minus aggressive), are formed in the same way as HML,
except the second sort variable is cash profitability or investment.
Cash profitability, used to create PMU, in the sort for June of year t is
measured with accounting data for the fiscal year ending in year t-1 and is
revenue minus the cost of goods sold, minus selling, general, and
administrative expenses, minus interest expense, plus research and development
costs, minus accruals
($\Delta$receivables - $\Delta$inventories - $\Delta$pre-paid expenses +
$\Delta$deferred revenues + $\Delta$trade payables + $\Delta$accrued expenses)
all divided by book equity.
Investment, Inv, is the change in total assets from the fiscal year ending in
year t-2 to the fiscal year ending in t-1, divided by t-2 total assets.
The momentum factor, WML, is defined in the same way as HML, except the factor
is updated monthly rather than annually.
To form the six Size-Prior portfolios at the end of month t-1,
Size is the market cap of a stock at the end of t-1 and Prior is its
cumulative return for the 11 months from t-12 to t-2.
Monthly value, $HML^m$, is defined in the same way as WML except book
equity is only updated at the end of June.
}
}
\begin{tabular}{lcccccc}
  \toprule
  \multicolumn{7}{l}{Panel A: Model factors and maximum squared Sharpe ratios} \\
        & \multicolumn{2}{l}{Name}                         & \multicolumn{3}{l}{Factors}                                   & $\text{Sh}^2$   \\
  1     & \multicolumn{2}{l}{Model 1}                      & \multicolumn{3}{l}{$R^M$, $SMB$, $HML^m$, $PMU$, $WML$}        & 0.316 \\
  2     & \multicolumn{2}{l}{Model 1 + $CMA$}              & \multicolumn{3}{l}{$R^M$, $SMB$, $HML^m$, $PMU$, $WML$, $CMA$} & 0.316 \\
  3     & \multicolumn{2}{l}{Fama and French (2017)}       & \multicolumn{3}{l}{$R^M$, $SMB$, $HML$, $PMU$, $WML$, $CMA$}   & 0.240 \\
  4     & \multicolumn{2}{l}{Model 2}                      & \multicolumn{3}{l}{$R^M$, $SMB$, $HML$, $PMU$, $CMA$}          & 0.225 \\
  5     & \multicolumn{2}{l}{Fama and French (2015)}       & \multicolumn{3}{l}{$R^M$, $SMB$, $HML$, $PMU^{06}$, $CMA$}     & 0.099 \\
  6     & \multicolumn{2}{l}{Carhart (1997)}               & \multicolumn{3}{l}{$R^M$, $SMB$, $HML$, $WML$}                 & 0.090 \\
  7     & \multicolumn{2}{l}{Carhart (1997) + $HML^m$}     & \multicolumn{3}{l}{$R^M$, $SMB$, $HML^m$, $WML$}               & 0.136 \\
  \midrule
  \multicolumn{7}{l}{Panel B: 90\% confidence interval for distributions of $\text{Sh}^2(\text{Row})$ - $\text{Sh}^2(\text{Column})$} \\
        &        2         &        3         &        4         &        5         &        6         &        7         \\
  1     & (-0.008,  0.000) & ( 0.039,  0.112) & ( 0.055,  0.138) & ( 0.156,  0.291) & ( 0.169,  0.299) & ( 0.127,  0.248) \\
  2     &                  & ( 0.043,  0.115) & ( 0.058,  0.140) & ( 0.158,  0.294) & ( 0.170,  0.302) & ( 0.129,  0.250) \\
  3     &                  &                  & ( 0.002,  0.048) & ( 0.095,  0.204) & ( 0.107,  0.212) & ( 0.052,  0.172) \\
  4     &                  &                  &                  & ( 0.081,  0.176) & ( 0.080,  0.197) & ( 0.028,  0.155) \\
  5     &                  &                  &                  &                  & (-0.033,  0.052) & (-0.085,  0.012) \\
  6     &                  &                  &                  &                  &                  & (-0.071, -0.025) \\
  \bottomrule
\end{tabular}
\label{tbl:bootstrap}
\end{table}
